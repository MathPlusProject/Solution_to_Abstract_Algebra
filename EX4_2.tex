\subsection{教材p77}

\begin{problem}
    设群$G = AB$, 其中$A, B$都是$G$的Abel子群(即交换子群),
且$AB = BA$. 令$G^{(1)}$表示$G$的换位子群, 证明:
\begin{enumerate}[(1)]
    \item $\forall a, x \in A$, $b, y \in B$,
总有$[x^{-1}, y^{-1}][a, b][x^{-1}, y^{-1}]^{-1} = [a, b]$;
    \item $G^{(1)}$是Abel群.
\end{enumerate}
\end{problem}

\begin{proof}
    
\end{proof}

\begin{problem}
    证明:
\begin{enumerate}[(1)]
    \item $S_{n} = \left\langle (1\:2),\: (1\:3),\: \cdots,\: (1\:n) \right\rangle$,
即$S_n$由对换$(1\:2),\: (1\:3),\: \cdots,\: (1\:n)$生成;
    \item $S_{n}$可由$(1\:2)$和$(1\:2\:3 \cdots n)$生成, 即
\[
    S_n = \big\langle (1\:2),(1\:2\:3 \cdots n) \big\rangle.
\]
\end{enumerate}
\end{problem}

\begin{proof}
    
\end{proof}

\begin{problem}
    证明:循环$\pi = (1\:2 \cdots n) \in S_n$的$k$次幂$\pi^k$是
$d$个互不相交的循环之积, 每个循环的长度为$q = \frac nd$,
其中$d = (n, k)$是$n$和$k$的最大公因子.
\end{problem}

\begin{proof}
    
\end{proof}

\begin{problem}
    设$A_n= \{\pi \in S_n \mid \varepsilon_\pi = 1\} \subset S_n$,
证明:
\begin{enumerate}[(1)]
    \item $A_n \lhd S_n$ (即$A_n$是$S_n$的正规子群);
    \item $A_n$由$3$-循环生成, 事实上, $A_n = \left\langle (1\:2\:3),\: (1\:2\:4),\: \cdots (1\:2\:n) \right\rangle$.
(提示:利用$(a\:b) \cdot (b\:c) = (a\:b\:c),\, (a\:b) \cdot (c\:d) = (a\:b) \cdot (b\:c) \cdot (b\:c) \cdot (c\:d)$.)
\end{enumerate}
\end{problem}

\begin{proof}
    
\end{proof}

\begin{problem}
    群$G$中的两个元素$x, y$称为在$G$中共轭, 如果存在$a \in G$, 使
$axa^{-1} = y$. 试证明:
\begin{enumerate}[(1)]
    \item $\forall\, \pi \in S_n\, \alpha = (i_1\:i_2 \cdots i_r) \in S_n$有公式
\[
    \pi \cdot \alpha \cdot \pi^{-1} = (\pi(i_1)\:\pi(i_2) \cdots \pi(i_r)).
\]
    \item 所有$3$-循环在$S_n$中相互共轭.
(所以$S_n$中包含$3$-循环的正规子群必包含$A_n$.)
    \item 如果$n \geqslant 5$,则所有$3$-循环在$A_n$中相互共轭, 即对于任意$3$-循环
$x,y \in A_n$, 存在$a \in A_n$, 使$axa^{-1} = y$.
\end{enumerate}
\end{problem}

\begin{proof}
    
\end{proof}

\begin{problem}
    证明:对任意给定整数$n > 0$, 在同构意义下仅有有限个$n$阶群.
(提示:任意$n$阶群均同构于$S_n$的一个子群.)
\end{problem}

\begin{proof}
    
\end{proof}

\begin{problem}
    证明:所有$4$阶群$G$都是交换群. 在同构意义下, $G$要么是循环群, 
要么同构于下述克莱因$4$元群:
\[
    V_4 = \{(1), (12)(34), (13)(24), (14)(23)\} \subseteq S_4.
\]
(提示:如果$x^2 = 1$对$G$中所有元成立, 则$\forall a, b \in G$, 有
$abab = 1 \implies ab = b^{-1}a^{-1} = b(b^{-1})^2 \cdot (a^{-1})^2a = ba.$)
\end{problem}

\begin{proof}
    
\end{proof}

\begin{problem}
    找出交错群$A_4$的所有子群.
\end{problem}

\begin{proof}

\end{proof}