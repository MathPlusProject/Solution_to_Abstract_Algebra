\subsection{教材p77}

\begin{problem}
    设群$G = AB$, 其中$A, B$都是$G$的Abel子群(即交换子群), 且$AB = BA$. 令$G^{(1)}$表示$G$的换位子群, 证明: 
    \begin{enumerate}[(1)]
        \item $\forall a, x \in A$, $b, y \in B$, 总有$[x^{-1}, y^{-1}][a, b][x^{-1}, y^{-1}]^{-1} = [a, b]$;
        \item $G^{(1)}$是Abel群.
    \end{enumerate}
\end{problem}

\begin{proof}
    
\end{proof}

\begin{problem}
    证明: 
    \begin{enumerate}[(1)]
        \item $S_{n} = \left\langle (1\:2),\: (1\:3),\: \cdots,\: (1\:n) \right\rangle$, 即$S_n$由对换$(1\:2),\: (1\:3),\: \cdots,\: (1\:n)$生成;
        \item $S_{n}$可由$(1\:2)$和$(1\:2\:3 \cdots n)$生成, 即
        \[
            S_n = \big\langle (1\:2),(1\:2\:3 \cdots n) \big\rangle.
        \]
    \end{enumerate}
\end{problem}

\begin{proof}
    \begin{enumerate}[(1)]
        \item 这是教材推论4.2.2的直接结果, 任意置换总能写成有限个对换的乘积, 而$(i\:j) = (1\:i)(1\:j)(1\:i)$.
        \item 由(1), 只需证明$(1\:3), \cdots, (1\:n)$都可以被$(1\:2)$和$(1\:2 \cdots n)$生成. 事实上$(1\:n) = (1\:2 \cdots n)^{-1}(1\:2)(1\:2 \cdots n)$, $(1\:i) = (1\:(i+1))(i\:(i + 1))(1\:(i + 1))$, 且$(i\:(i + 1)) = (1\:2 \cdots n)^{-(n - i + 1)}(1\:2)(1\:2 \cdots n)^{n - i + 1}, 2 < i < n$(其实就是把$i$和$i + 1$先移到$1$和$2$的位置上, 用$(1\:2)$对换, 再移回去).
    \end{enumerate}
\end{proof}

\begin{problem}
    证明: 循环$\pi = (1\:2 \cdots n) \in S_n$的$k$次幂$\pi^k$是$d$个互不相交的循环之积, 每个循环的长度为$q = \frac nd$, 其中$d = (n, k)$是$n$和$k$的最大公因子.
\end{problem}

\begin{proof}
    
\end{proof}

\begin{problem}
    设$A_n= \{\pi \in S_n \mid \varepsilon_\pi = 1\} \,\red{\subseteq}\, S_n$, 证明: 
    \begin{enumerate}[(1)]
        \item $A_n \lhd S_n$ (即$A_n$是$S_n$的正规子群);
        \item $A_n$由$3$-循环生成, 事实上, $A_n = \left\langle (1\:2\:3),\: (1\:2\:4),\: \cdots (1\:2\:n) \right\rangle$.(提示: 利用$(a\:b) \cdot (b\:c) = (a\:b\:c),\, (a\:b) \cdot (c\:d) = (a\:b) \cdot (b\:c) \cdot (b\:c) \cdot (c\:d)$.)
    \end{enumerate}
\end{problem}

\begin{proof}
    \begin{enumerate}[(1)]
        \item $\pi \mapsto \varepsilon_\pi$实际上是一个群同态$S_n \to \{1, -1\} \cong \mathbb{Z}/2\mathbb{Z}$. 而$A_n$恰好是这个同态的kernel.
        \item 根据提示有$(a\:b)(c\:d) = (abc)(bcd)$, 因此所有的$3$-循环能生成$A_n$, 只需说明任意$3$-循环在$\left\langle (1\:2\:3),\: (1\:2\:4),\: \cdots (1\:2\:n) \right\rangle$中. 我们可以做拆解, 对$i, j, k \neq 1, 2$, 反复用上面的等式凑出来$(1\:2\:m)$.
        \[
        \begin{aligned}
            (1\:j\:k) &= (k\:1)(1\:j) = (k\:1)(1\:2)(1\:2)(2\:k)(2\:k)(1\:j) = (1\:2\:k)(1\:2\:k)(1\:j)(2\:k)\\
            &= (1\:2\:k)(1\:2\:k)(1\:j)(1\:2)(1\:2)(2\:k) = (1\:2\:k)(1\:2\:k)(1\:2\:j)(1\:2\:k)
        \end{aligned}
        \]
        同样的可以凑出
        \[
            (i\:j\:k) = (i\:j)(1\:j)(1\:j)(j\:k) = (1\:i\:j)(1\:j\:k)
        \]
    \end{enumerate}
\end{proof}

\begin{problem}
    群$G$中的两个元素$x, y$称为在$G$中共轭, 如果存在$a \in G$, 使$axa^{-1} = y$. 试证明: 
    \begin{enumerate}[(1)]
        \item $\forall\, \pi \in S_n\, \alpha = (i_1\:i_2 \cdots i_r) \in S_n$有公式
        \[
            \pi \cdot \alpha \cdot \pi^{-1} = (\pi(i_1)\:\pi(i_2) \cdots \pi(i_r)).
        \]
        \item 所有$3$-循环在$S_n$中相互共轭. (所以$S_n$中包含$3$-循环的正规子群必包含$A_n$.)
        \item 如果$n \geqslant 5$, 则所有$3$-循环在$A_n$中相互共轭, 即对于任意$3$-循环$x,y \in A_n$, 存在$a \in A_n$, 使$axa^{-1} = y$.
    \end{enumerate}
\end{problem}

\begin{proof}
    \begin{enumerate}[(1)]
        \item 按定义验证, 若$\pi \cdot \alpha \cdot \pi^{-1}(i) = \pi(\alpha(\pi^{-1}(i)))$. 若$i \notin \{\pi(i_1), \pi(i_2), \cdots, \pi(i_r)\} \iff \pi^{-1}(i) \notin \{i_1, i_2, \cdots, i_r\}$, 则$\pi(\alpha(\pi^{-1}(i))) = \pi(\pi^{-1}(i)) = i$. 反之$i \in \{\pi(i_1), \pi(i_2), \cdots, \pi(i_r)\}$, 有$\pi(\alpha(\pi^{-1}(i))) = \pi(\alpha(i_k)) = (\pi(i_1)\:\pi(i_2) \cdots \pi(i_r))(i)$.
        \item (1)的推论. 若$\alpha$是$3$-循环, 任意的$\pi \in S_n$, $\pi\alpha\pi^{-1}$仍是$3$-循环. 具体来说, 对两个$3$-循环$\alpha_1 = (a_1\:b_1\:c_1)$和$\alpha_2 = (a_2\:b_2\:c_2)$, 则令
        \(
            \pi = 
            \begin{pmatrix} 
                a_1 & b_1 & c_1 & \cdots \\
                a_2 & b_2 & c_2 & \cdots 
            \end{pmatrix}
        \) 
        即可.
        \item 设$x = (i_1\:i_2\:i_3), y = (j_1\:j_2\:j_3)$. 当$n \geqslant 5$时, 由(2), 考虑
        \[
            a_1 =
            \begin{pmatrix} 
                i_1 & i_2 & i_3 & i_4 & i_5 & \cdots \\
                j_1 & j_2 & j_3 & j_4 & j_5 & \cdots 
            \end{pmatrix},
            a_2 =
            \begin{pmatrix} 
                i_1 & i_2 & i_3 & i_4 & i_5 & \cdots \\
                j_1 & j_2 & j_3 & j_5 & j_4 & \cdots 
            \end{pmatrix}
        \]
        则由(2)可知$k = 1, 2$都满足$a_kxa_k^{-1} = y$, 但是$a_2 = a_1(j_4, j_5)$, 即刚好差一个对换, 那么$a_1, a_2$必然一奇一偶, 因此存在$a \in A_n$使得$axa^{-1} = y$.
    \end{enumerate}
\end{proof}

\begin{problem}
    证明: 对任意给定整数$n > 0$, 在同构意义下仅有有限个$n$阶群. (提示: 任意$n$阶群均同构于$S_n$的一个子群.)
\end{problem}

\begin{proof}
    
\end{proof}

\begin{problem}\label{ex:4.2.7}
    证明: 所有$4$阶群$G$都是交换群. 在同构意义下, $G$要么是循环群, 要么同构于下述克莱因$4$元群: 
    \[
        V_4 = \{(1), (12)(34), (13)(24), (14)(23)\} \subseteq S_4.
    \]
    (提示: 如果$x^2 = 1$对$G$中所有元成立, 则$\forall a, b \in G$, 有$abab = 1 \implies ab = b^{-1}a^{-1} = b(b^{-1})^2 \cdot (a^{-1})^2a = ba.$)
\end{problem}

\begin{proof}
    对于这种阶很小的群, 我们可以直接分析$4$阶群的乘法表, 这其实和数独有点像. 乘法表的每一行或每一列是不能有相同元素的, 因为左乘映射是单的$ga = gb \implies a = b$, 右乘也一样.

    设$G = \{e, a, b, c\}$, $e$是单位元. 那么首先有
    \[
    \begin{array}{c|c c c c}
      & e & a & b & c \\
    \hline
    e & e & a & b & c \\
    a & a &  &  &  \\
    b & b &  &  &  \\
    c & c &  &  &  \\
    \end{array}
    \]
    由\ref{ex:1.3.10}, $G$有$2$阶元, 不妨设$a^2 = e$, 则$ab \neq e, a, b$, 只能是$c$, $ac, ba, ca$同理, 得到
    \[
    \begin{array}{c|c c c c}
      & e & a & b & c \\
    \hline
    e & e & a & b & c \\
    a & a & e & c & b \\
    b & b & c &  &  \\
    c & c & b &  &  \\
    \end{array}
    \]
    此时若$b^2 = e$, 则得到
    \[
    \begin{array}{c|c c c c}
      & e & a & b & c \\
    \hline
    e & e & a & b & c \\
    a & a & e & c & b \\
    b & b & c & e & a \\
    c & c & b & a & e\\
    \end{array}
    \]
    否则$b^2 = a$, 得到
    \[
    \begin{array}{c|c c c c}
      & e & a & b & c \\
    \hline
    e & e & a & b & c \\
    a & a & e & c & b \\
    b & b & c & a & e \\
    c & c & b & e & a\\
    \end{array}
    \]
    第二种实际上是$\mathbb{Z}/4\mathbb{Z}$是循环群, 生成元是$b$或者$c$, 自然是交换群. 第一种就是题干中的克莱因$4$元群.
\end{proof}

\begin{remark}
    对于阶很大的群这种方法便不适用了. 事实上若$|G| = p^2$, 则$G$一定是Abel群, 其中$p$是素数. 这是共轭作用得到的分类公式的直接推论. 即教材引理4.2.2证明的中间结果
    \[
        |G| = C(G) + \sum_{O(x) > 1} |O(x)|,\quad |O(x)| = [G:H_x]
    \]
    这里的$H_x$是在共轭作用$g \cdot x = gxg^{-1}$下的稳定子, 又称做$x$的中心化子(所有和$x$交换的元素构成的子群), 和\ref{ex:1.2.4}是类似, 一般记作$C(x)$. 这个等式的每一项都是$|G|$的因子, 因此若$G$不是Abel群, 即$C(G) \neq G$, 那么只能是$|C(G)| = p$. 但这是不可能的. 因为对$x \notin C(G)$, $|C(x)|$也是$p^2$的因子, 而按定义$C(G)$是严格包含于$C(x)$的, $C(G) \subsetneq C(x)$, 这意味着$|C(x)| > |C(G)| = p$, 那么$|C(x)| = p^2$, 这就矛盾了, 因为$x \notin C(G)$, 所以$C(x)$不可能等于$G$.

    对一般的群作用$G \times X \to X$, $X$是有限集, 教材的引理4.5.2事实上可以表示为分类公式(class formula)
    \[
        |S| = |Z| + \sum_{|O(x)| > 1} [G:\mathrm{stab}(x)] = |Z| + \sum_{|O(x)| > 1} |O(x)|.
    \]
    其中$Z$称为该群作用下的不动点集, $x \in Z \iff \mathrm{stab}(x) = G \iff O(x) = \{x\}$. 若$G$是$p$-群, 则有
    \[
        |Z| \equiv |S| \mod p
    \]
\end{remark}

\begin{problem}
    找出交错群$A_4$的所有子群.
\end{problem}

\begin{remark}
    这题是Sylow定理, 应该放到习题4.5.
\end{remark}

\begin{proof}
    
\end{proof}