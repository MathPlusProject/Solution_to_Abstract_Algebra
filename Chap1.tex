\textbf{约定:}
\begin{enumerate}
    \item 由于教材中的$\subset$符号意义有些歧义, 我们统一用$\subseteq$表示子集, $\subsetneq$表示真子集, 比如\ref{ex:2.1.5}中我对符号进行了修正.
    \item 习题中也有其他错误, 对教材原文修改的地方我用\red{红色}标出
    \item 环都是有$1$的. 如果看到把$n$看作$R$的元素, 请看\ref{ex:1.2.1}最后的注记.
    \item 文中出现的教材指\cite{2022抽象代数}
    \item 对称群的乘法以教材为准, 见\ref{ex:1.3.5}注记.
    \item PID中的命题$(a, b) = ua + vb$统一称为Bézout's Identity. 主要是Bézout's Theorem现在都指代数几何里的一个定理了. 教材里的那个Bézout's Theorem感觉有些太普通了, 还是别叫它定理了吧\dots
    \item 由于$\mathbb{Z}_p$是$p$-adic integers的标准记号, 为防止混淆, 习题中出现的$\mathbb{Z}_p$一律替换为$\mathbb{Z}/p\mathbb{Z}$或$\mathbb{F}_p$. 对一般的$n$我也会替换.
    \item 和教材保持一致, 用$R^*$表示所有非零元$R \setminus \{0\}$, 单位群用$U(R)$或$R^\times$表示. 域的时候两者是一样的, 主要是区分整环的情形.
    \item 有时对域扩张$K \subseteq L$会使用标准记号$L/K$表示.
    \item 课程的参考书是\cite{lang2012algebra}和\cite{2009近世代数引论}.
    \item 教材答案有误的题目用$*$标出.
\end{enumerate}

\section{群环域}
\subsection{教材p8-p9}

\begin{problem}\label{ex:1.1.1}
    设$K$是一个域, 试证明下述结论: 
    
    \begin{enumerate}[(1)]
        \item 如果$a \cdot c = b \cdot c$, $c \neq 0_K$, 则$a = b$ (乘法消去律);
        \item $\forall \, a, b\in K$, 如果$a \cdot b = 0_K$, 则$a = 0_K$或$b = 0_K$;
        \item $(a^{-1})^{-1} = a \quad (\forall \, a \in K ,\, a \neq 0_K)$;
        \item $(a \cdot b)^{-1} = a^{-1} \cdot b^{-1} \quad (a \neq 0_K ,\, b \neq 0_K)$;
        \item $(-a)^{-1} = -a^{-1} \quad (\forall a \neq 0_K)$;
        \item $\forall a \neq 0_K ,\, m, n \in \mathbb{Z}$, 则$a^{m + n} = a^m \cdot a^n ,\, a^{mn} = (a^m)^n$;
    \end{enumerate}
\end{problem}

\begin{solution}
    \begin{enumerate}[(1)]
        \item 由于$c \neq 0$, 故可在原式左右同乘$c^{-1}$, 得 
        \[ 
        \begin{aligned} 
            a \cdot c \cdot c^{-1} &= b \cdot c \cdot c^{-1}\\ 
            \implies a &= b.
        \end{aligned} 
        \]
        这告诉我们逆元的存在性强于乘法消去律, 乘法消去律已经可以保证乘法逆运算是良定的.
        这对加法也是一样的道理, 见\ref{ex:1.2.1}的(1).

        也可以用分配律得到
        \[
            a \cdot c = b \cdot c \implies (a - b) \cdot c = 0_K.
        \]
        要得到$a = b$需要使用(2), 即域$K$是没有零因子(zero-divisor)的. 由于$c \neq 0_K$,
    则$a - b = 0_K$, 即$a = b$.

        注:无零因子的非零交换环称为整环(integral domain), 见教材2.1节p23.
        \item 只需证明当$a \neq 0_K$时有$b = 0_K$, 同(1), 在等式$a \cdot b = 0_K$
    两端左乘$a^{-1}$即可.
        
        这告诉我们域$\implies$整环. 结合(1)知一个环是整环的条件已经可以推出乘法消去律.
        \item 即要证明$a^{-1}$的逆元是$a$, 这是根据定义以及逆元的唯一性得到, 教材在域, 
    环, 群三处定义下的注记都有提及. 事实上只要$a$在某一个幺半群(monoid)中关于这个运算有
    逆元, 该结论都会成立, 如\ref{ex:1.2.1}的(3).
        \item 即要证明$a \cdot b$的逆元是$a^{-1} \cdot b^{-1}$. 此处需要交换律, 因此
    验证半边逆就够了.
    \[
        (a^{-1} \cdot b^{-1}) \cdot (a \cdot b) = (a \cdot a^{-1}) \cdot (b \cdot b^{-1}) = 1_K.
    \]
    非交换的情形为$(ab)^{-1} = b^{-1}a^{-1}$, 见\ref{ex:1.3.2}.
        \item 即要证明$-a$的逆元是$-a^{-1}$. 我们用一下\ref{ex:1.2.1}的(6)
    \[
        (-a)(-a^{-1}) = aa^{-1} = 1_K, \quad (-a^{-1})(-a) = a^{-1}a = 1_K.
    \]
    这样这一条对一个环中的单位都成立.
        \item 首先需要明确定义, 教材关于$a^n$的定义并不清晰, 包括后面\ref{ex:1.2.1}中的$na$也是.
    事实上, 这种和$\mathbb{Z}$有关的东西都应该由递归定义给出, 相对应的证明要用归纳法.
        
        严格来说, 这是定义了一个映射
    \[
        \mathbb{Z} \times K^* \to K^*, (n, a) \mapsto a^n,
    \]
        这里$K^* = K \setminus \{0_K\}$(见\ref{ex:1.3.2}), 自然数的部分应由递归定义给出, 
    \[
        a^0 \defeq 1_K,\, a^{n + 1} \defeq a^n \cdot a,\, n \in \mathbb{N},
    \]
        负整数的部分定义为
    \[
        a^n \defeq (a^{-1})^{-n},\, n < 0. 
    \]
    由该定义可以验证对任意整数$n \in \mathbb{Z}$均有$a^{n + 1} = a^n \cdot a$
    以及$a^{-n} = (a^{-1})^n$, 这样在使用这两个等式的时候不用再区分正负了.

    回到原题, 对任意的$m \in \mathbb{Z}$, 先用归纳法证明$n \in \mathbb{N}$
    的情形, 负整数的情形可以结合定义得到.

    $n = 0$时根据定义左右均为$a^m$, 假设对$n$有$a^{m + n} = a^m \cdot a^n$, 根据定义有
    \[
        a^{m + n + 1} = a^{m + n} \cdot a = a^m \cdot a^n \cdot a = a^m \cdot a^{n + 1}.
    \]
    由归纳法知
    \begin{equation}
        \forall m \in \mathbb{Z}, n \in \mathbb{N} ,\, a^{m + n} = a^m \cdot a^n.
        \tag{*}
        \label{eq:1.1.1.61}
    \end{equation} 
    当$n < 0$时, 则存在$k \in \mathbb{Z}_{>0}$使得$m + kn < 0$,
    则有
    \[
    \begin{aligned}
        a^{m + n} &= a^{m + kn + (-(k - 1)n)}\\
        &\overset{\eqref{eq:1.1.1.61}}= a^{m + kn} \cdot a^{-(k - 1)n}\\
        &= (a^{-1})^{-m - kn} \cdot a^{n - kn}\\
        &\overset{\eqref{eq:1.1.1.61}}= (a^{-1})^{-m} \cdot (a^{-1})^{-kn} \cdot a^n \cdot a^{-kn}\\
        &= a^m \cdot (a^{-1})^{-kn} \cdot (a^{-1})^{-n} \cdot a^{-kn}\\
        &\overset{\eqref{eq:1.1.1.61}}= a^m \cdot (a^{-1})^{-kn - n} \cdot a^{-kn}\\
        &= a^m \cdot a^{(k + 1)n} \cdot a^{-kn}\\
        &\overset{\eqref{eq:1.1.1.61}}= a^m \cdot a^{(k + 1)n - kn} = a^m \cdot a^n.
    \end{aligned}
    \]
    这里我避免使用了乘法交换律, 这样该结论对一般的环也成立.

    同样地, 由于$a^{m(n + 1)} = a^{mn + m} = a^{mn} \cdot a^m = (a^{m})^n \cdot a^m = (a^m)^{n + 1}$,
    对$n$归纳可得
    \begin{equation}
        \forall m \in \mathbb{Z}, n \in \mathbb{N},\, a^{mn} = (a^m)^n.
        \tag{**}
        \label{eq:1.1.1.62}
    \end{equation}
    当$n < 0$时, 
    \[
    \begin{aligned}
        a^{mn} &= a^{-(m \cdot (-n))}\\
        &= (a^{-1})^{m \cdot (-n)}\\
        &\overset{\eqref{eq:1.1.1.62}}= ((a^{-1})^m)^{-n}\\
        &= (a^{-m})^{-n} = ((a^{-m})^{-1})^n
    \end{aligned}
    \]
    由于$a^{-m} \cdot a^m \overset{\eqref{eq:1.1.1.61}}= a^0 = 1_K$, 即括号内确实为$a^m$,
    故上式等于$(a^m)^n$.
    \end{enumerate}
\end{solution}

\begin{problem}
    设$K$是一个域, 证明:$K$的任意一组子域(可以无限多个)的交集仍是子域.
如果$K_i \subset K \, (i \in \mathbb{N})$是满足条件
$K_i \subseteq K_{i+1} \, (i \in \mathbb{N})$的子域, 则它们的并集也是
$K$的子域.
\end{problem}

\begin{solution}
    令
\(
    F = \bigcap_i K_i
\)
    由子域定义, 需要验证
\[
\begin{aligned}
    &\forall a, b \in F ,\, a - b \in F\\
    &\forall a, b \in F^* ,\, ab^{-1} \in F^*,\, F^* = F \setminus \{0\}.
\end{aligned}
\]
由于$K_i$均为子域, 且$a, b \in F \subseteq K_i$, 故
\[
    \forall i \in \mathbb{N}, \, a - b \in K_i.
\]
因此
\[
    a - b \in \bigcap_i K_i = F.
\]
$F^*$的部分同理, 故$F$为子域.

若还满足$\forall i \in \mathbb{N} ,\, K_i \subseteq K_{i + 1}$,
令
\(
    L = \bigcup_i K_i
\),
如果$a, b \in L$, 则存在$K_i$和$K_j$使得$a \in K_i ,\, b \in K_j$
记$r = \max(i, j)$, 则$a, b \in K_r$. 由于$K_r$为子域, 可得
\[
    a - b \in K_r \subseteq L.
\]
$L^*$同理, 故$L$为子域.
\end{solution}

\begin{problem}
    令$\mathbb{Q}[\sqrt{2}, \sqrt{3}]$表示$\mathbb{C}$中包含
$\mathbb{Q}, \sqrt{2}, \sqrt{3}$的最小子域, 证明
$\mathbb{Q}[\sqrt{2}, \sqrt{3}] = \mathbb{Q}[\sqrt{2} + \sqrt{3}]$.
\end{problem}

\begin{solution}
    该题本应该是域扩张的题, 此处我们只用定义来证明.

    由于$\sqrt{2} + \sqrt{3} \in \mathbb{Q}[\sqrt{2}, \sqrt{3}]$,
我们有$\mathbb{Q}[\sqrt{2} + \sqrt{3}] \subseteq \mathbb{Q}[\sqrt{2}, \sqrt{3}]$.
反过来, 
\(
    \frac{1}{\sqrt{2} + \sqrt{3}} = \sqrt{3} - \sqrt{2} \in \mathbb{Q}[\sqrt{2} + \sqrt{3}]
\),
故有
\(
    \sqrt{2} = \frac{(\sqrt{2} + \sqrt{3}) - (\sqrt{3} - \sqrt{2})}{2} \in \mathbb{Q}[\sqrt{2} + \sqrt{3}]
\),
\(
    \sqrt{3} = \frac{(\sqrt{2} + \sqrt{3}) + (\sqrt{3} - \sqrt{2})}{2} \in \mathbb{Q}[\sqrt{2} + \sqrt{3}]
\).
因此
\(
    \mathbb{Q}[\sqrt{2}, \sqrt{3}] \subseteq \mathbb{Q}[\sqrt{2} + \sqrt{3}]
\).
故两者相等.

注: 证明过程给出了一个$\mathbb{Q}$-线性空间的基变换实际上, 从而两者将同构(见\ref{1.4.9}).
\end{solution}

\begin{problem}\label{ex:1.1.4}
    设$\mathbb{N}$是所有正整数的集合, $\mathbb{Q}$是有理数域.因$\mathbb{Q}$是可数集, 
故存在双射$f:\mathbb{N} \to \mathbb{Q}$. 令$f^{-1}:\mathbb{Q} \to \mathbb{N}$表示$f$的
逆映射, 利用有理数的加法和乘法, 可通过双射$f$定义$\mathbb{N}$上的运算如下:
$\forall n, m \in \mathbb{N}$, 
\[
    n \oplus m = f^{-1}(f(n) + f(m)), \quad n \star m = f^{-1}(f(n)f(m)),
\]
试证明:$\mathbb{N} = (\mathbb{N}, \oplus, \star)$是域, 并求它的零元和单位元.
\end{problem}

\begin{solution}
    验证域公理, 加法交换律和乘法交换律易得.

    结合律:$\forall n, m, l \in \mathbb{N}$,
\[
\begin{aligned}
    (n \oplus m) \oplus l &= f^{-1}\biggl(f\bigl(f^{-1}(f(n) + f(m))\bigr)+ f(l)\biggr)\\
    &= f^{-1}(f(n) + f(m) + f(l)) \\
    &\overset{!}= n \oplus (m \oplus l);\\
    (n \star m) \star l &= f^{-1}\biggl(f\bigl(f^{-1}(f(n)f(m))\bigr) \cdot f(l)\biggr)\\
    &= f^{-1}(f(n)f(m)f(l)) \\
    &= n \star (m \star l).
\end{aligned}
\]
其中!处是因为计算出来的结果关于$n, m, l$是轮换对称的, 后面同理.

    零元为$f^{-1}(0)$: $\forall n \in \mathbb{N}$,
\[
\begin{aligned}
    n \oplus f^{-1}(0) &= f^{-1}(f(n) + f(f^{-1}(0)))\\
    &= f^{-1}(f(n) + 0)\\
    &= f^{-1}(f(n)) = n.
\end{aligned}
\]
    $n$的负元为$f^{-1}(-f(n))$:
\[
\begin{aligned}
    n \oplus f^{-1}(-f(n)) &= f^{-1}\biggl(f(n) + f\bigl(f^{-1}(-f(n))\bigr)\biggr)\\
    &= f^{-1}(f(n) - f(n))\\
    &= f^{-1}(0).
\end{aligned}
\]
    单位元为$f^{-1}(1)$: $\forall n \in \mathbb{N}$,
\[
\begin{aligned}
    n \star f^{-1}(1) &= f^{-1}(f(n) \cdot f(f^{-1}(1)))\\
    &= f^{-1}(f(n)) = n.
\end{aligned}
\]
    $n$的逆元为$f^{-1}(\frac1{f(n)})$:
\[
\begin{aligned}
    n \star f^{-1}(\frac{1}{f(n)}) &= f^{-1}\biggl(f(n) \cdot f\bigl(f^{-1}(\frac{1}{f(n)})\bigr)\biggr)\\
    &= f^{-1}(f(n) \cdot \frac{1}{f(n)})\\
    &= f^{-1}(1).
\end{aligned}
\]
    分配律:$\forall n, m, l \in \mathbb{N}$,
\[
\begin{aligned}
    n \star (m \oplus l) &= f^{-1}\biggl(f(n) \cdot f\bigl(f^{-1}(f(m) + f(l))\bigr)\biggr)\\
    &= f^{-1}\bigl(f(n) \cdot (f(m) + f(l))\bigr)\\
    &= f^{-1}\bigl(f(n)f(m) + f(n)f(l)\bigr)\\
    &= f^{-1}(f(n)f(m)) \oplus f^{-1}(f(n)f(l))\\
    &= n \star m \oplus n \star l.
\end{aligned}
\]
\end{solution}

\begin{problem}\label{ex:1.1.5}
    证明:在域的定义中, 加法的交换律可以由其他条件推
出. 提示:按两种方式展开 $(1 + 1) \cdot (a + b)$.
\end{problem}

\begin{solution}
    一方面
\[
\begin{aligned}
    (1 + 1) \cdot (a + b) &= 1 \cdot (a + b) + 1 \cdot (a + b)
    &= a + b + a + b;
\end{aligned}    
\]
    另一方面
\[
\begin{aligned}
    (1 + 1) \cdot (a + b) &= (1 + 1) \cdot a + (1 + 1) \cdot b\\
    &= a + a + b + b.
\end{aligned}
\]
故有$a + b + a + b = a + a + b + b$.
消去两端的一个$a$和一个$b$即得加法交换律.
\end{solution}

\begin{problem}
    设$p > 2$是素数,
$\mathbb{F}_p = \{\bar{0}, \bar{1}, \bar{2}, \cdots, \overline{p-1}\}$
是$\mathbb{Z}$的模$p$剩余类域. 试计算:
\begin{enumerate}[(1)]
    \item $\bar{2}$在$\mathbb{F}_p$中的逆元$\bar{2}^{-1}$;
    \item $\overline{p - 1} \cdot \overline{p - 2}$;
    \item $\overline{p - 2}$在$\mathbb{F}_p$中的逆元$\overline{p-2}^{-1}$.
\end{enumerate}
\end{problem}

\begin{solution}
\begin{enumerate}[(1)]
    \item 只需找到能被$2$整除的$1 + kp(k \in \mathbb{Z})$.
由于素数$p > 2$, $p + 1$即可. i.e. $\overline{2}^{-1} = \overline{\frac12(p + 1)}$.
    \item $\overline{p - 1} \cdot \overline{p - 2} = \overline{-1} \cdot \overline{-2} = \overline{2}$.
    \item 由(1), $\overline{p - 2}^{-1} = \overline{-2}^{-1} = \overline{-\frac12(p + 1)} = \overline{\frac12(p - 1)}$.
\end{enumerate}
\end{solution}
    
\subsection{教材p13-p14}

\begin{problem}\label{ex:1.2.1}
    设$R$是一个环, 试证明下述结论:
    \begin{enumerate}[(1)]
        \item (加法消去律)\quad 如果$a + c = b + c$, 则$a = b$;
        \item $\forall a \in R$, 有$a \cdot 0_R = 0_R$;
        \item $-(-a) = a,\quad a(b - c) = ab - ac \quad (\forall a, b, c \in R)$;
        \item $-(a + b) = (-a) + (-b) \quad (\forall a, b \in R)$;
        \item $a(-b) = (-a)b = -(ab) \quad (\forall a, b \in R)$;
        \item $(-a)(-b) = ab \quad (\forall a, b \in R)$;
        \item $\forall a \in R, m, n \in \mathbb{Z}$, 有$(m + n)a = ma + na, (mn)a = m(na)$;
        \item $\forall a, b \in R, n \in \mathbb{Z}$, 有$n(a + b) = na + nb, n(ab) = a(nb)$;
        \item $\forall a, b \in R, m, n \in \mathbb{Z}$, 有$(ma) \cdot (nb) = mn(a \cdot b) = (mna) \cdot b$;
        \item (二项式定理)\quad $\forall a, b \in R$,设$ab = ba$, $n$是正整数, 则
        \[
            (a + b)^n = \sum_{i = 0}^n \binom{n}{i} a^{n - i}b^i.
        \]
    \end{enumerate}
\end{problem}

\begin{proof}
    \begin{enumerate}[(1)]
        \item 两边同加$-c$.
        \item 由于
        \[
            a \cdot 0_R = a \cdot (0_R + 0_R) = a \cdot 0_R + a \cdot 0_R.
        \]
        再用一下负元消去即可, $0_R \cdot a = 0_R$同理.

        \begin{remark}
            这里需要用到: 分配律, 零元定义, 负元存在. 与之对比, $0_R \cdot 0_R = 0_R$只需要用到分配律, 零元和单位元. 因此在半环(semiring)中(2)是不成立的, 但仍有$0_R \cdot 0_R = 0_R$, 这里半环要求$0$和$1$存在.
        \end{remark}

        \item 前一个为负元定义(教材p9的注记); 后一个先由分配律,
        \[
            a(b - c) = ab + a(-c),
        \]
        又由于
        \[
            a(-c) + ac = a(c + (-c)) = a \cdot 0_R \overset{(2)}= 0
        \]
        得$a(-c) = -ac$, 这也是(5)的证明. 这里要注意仅使用$-a \overset{(*)}= -1_R \cdot a$也无法将负号提到前面, 需要$R$是交换环或者说明$-1_R \cdot a = a \cdot (-1_R) = -a$.

        $(*)$的证明如下
        \[
            -1_R \cdot a + a = -1_R \cdot a + 1_R \cdot a = (-1_R + 1_R) \cdot a = 0_R \cdot a \overset{(2)}= 0_R.
        \]
        右乘$-1_R$同理.
        \item 利用$-a = -1_R \cdot a$和分配律展开即可.
        \item 见(3).
        \item (3)和(5)的推论.
        \item 参考\ref{ex:1.1.1}的(6), 明确定义: 
        \[
            0a \defeq 0_R,\, (n + 1)a \defeq na + a,\, n \in \mathbb{N}
        \]
        以及
        \[
            na \defeq -((-n)a),\, n < 0.
        \]
        一样的, 可以验证对任意整数$n \in \mathbb{Z}$都有$(n + 1)a = na + a$和$na = -((-n)a)$. 先对$n$归纳得
        \begin{equation}
            (m + n)a = ma + na, \quad \forall m \in \mathbb{Z}, n \in \mathbb{N}
            \tag{i}
            \label{eq:1.2.1.7}
        \end{equation}
        然后$n < 0$, 存在$k\in \mathbb{Z}_{>0}$使得$m + kn < 0$,
        \[
        \begin{aligned}
            (m + n)a &= (m + kn - (k - 1)n)a\\
            &\overset{\eqref{eq:1.2.1.7}}= (m + kn)a + (-(k - 1)n)a\\
            &= -(-m - kn)a + (n - kn)a\\
            &\overset{\eqref{eq:1.2.1.7}}= -((-m)a + (-kn)a) + na + (-kn)a\\
            &\overset{(4)}= ma + (kn)a + na + (-kn)a = ma + na.
        \end{aligned}
        \]
        第二个式子可直接利用第一个证明, $m = 0$根据定义左右均为$0_R$, $m > 0$有,
        \[
        \begin{aligned}
            (mn)a &= \left(\sum_{i = 1}^{m} n\right)a\\
            &= \sum_{i = 1}^{m} (na)\\
            &= m(na).
        \end{aligned}
        \]
        $m < 0$利用$mn = (-m)(-n)$, 做同样的操作.
        \item 对$n$归纳, 由于加法有交换律,
        \[
        \begin{aligned}
            (n + 1)(a + b) &= n(a + b) + a + b\\
            &= na + nb + a + b\\
            &= (n + 1)a + (n + 1)b.
        \end{aligned}
        \]
        得
        \[
            n(a + b) = na + nb, \quad \forall n \in \mathbb{N}
        \]
        当$n < 0$有
        \[
            n(a + b) = -(-n(a + b)) = -((-n)a + (-n)b) \overset{(4)}= na + nb.
        \]
        第二个等式使用分配律, $n = 0$根据定义左右均为$0_R$, $n > 0$,
        \[
        n(ab) = \sum_{i = 1}^{n} ab = a\sum_{i = 1}^{n} b = a(nb).
        \]
        $n < 0$, 用$n = -(-n)$, $n(ab) = -a((-n))b \overset{(5)}= a(nb)$. 同样的也会有$n(ab) = (na)b$.
        \item (7)和(8)的推论,
        \[
        \begin{aligned}
            (ma) \cdot (nb) &\overset{(8)}= m(a \cdot (nb))\\
            &\overset{(8)}= m(n(ab))\\
            &\overset{(7)}= mn(ab)\\
            &\overset{(8)}= (mna) \cdot b.
        \end{aligned}
        \]
        \item 对$n$归纳,
        \[
        \begin{aligned}
            (a + b)^n \cdot (a + b) &= \left(\sum_{i = 0}^{n} \binom{n}{i} a^{n - i}b^i\right) \cdot (a + b)\\
            &= \sum_{i = 0}^{n} \binom{n}{i} a^{n - i}b^ia + \sum_{i = 0}^{n} \binom{n}{i} a^{n - i}b^{i + 1}\\
            &\overset{ab = ba}= \sum_{i = 0}^{n} \binom{n}{i} a^{n - i + 1}b^i + \sum_{i = 0}^{n} \binom{n}{i} a^{n - i}b^{i + 1}\\
            &= a^{n + 1} + \sum_{i = 1}^{n} \left(\binom{n}{i} + \binom{n}{i - 1}\right) a^{n - i + 1}b^{i} + b^{n + 1}\\
            &= \sum_{i = 0}^{n + 1} \binom{n + 1}{i} a^{n + 1 - i}b^i.
        \end{aligned}
        \]
    \end{enumerate}
\end{proof}

\begin{remark}
    (7)-(9)中实际上需要用归纳法证明的只有
    \[
    \begin{aligned}
        n(a + b) &= na + nb,\\
        (m + n)a &= ma + na,\\
        (mn)a &= m(na),\\
    \end{aligned}
    \]
    这三条加上$1a = a$, 是在说任何一个Abel群都是$\mathbb{Z}$-模(\ref{ex:5.1.4}), 因为这几条的证明过程并未用到$R$的乘法, 把$R$换成任意的Abel群也是对的. 再反过来看\ref{ex:1.1.1}的(6), 加上$(ab)^n = a^nb^n$, 也是在说$K^*$是$\mathbb{Z}$-模, 证明过程中用到了$K^*$关于域的乘法是Abel群.
    
    另一方面, 可以先定义
    \[
        N: \mathbb{Z} \to R,\quad n \mapsto n1_R
    \]
    这是一个自然的环同态(使用归纳法证明)
    \[
    \begin{aligned}
        N(m + n) &= N(m) + N(n);\\
        N(mn) &= N(m) \cdot N(n).
    \end{aligned}
    \]
    然后利用这个环同态得到(注意用到的$n(ab) = (na)b$的证明是直接使用分配律的, 因此不存在循环论证. $N$表示使用了这个环同态, $dis$表示使用了分配律, $ass$表示使用了结合律):
    \[
    \begin{aligned}
        n(a + b) &= n(1_R(a + b)) = (n1_R)(a + b) \overset{dis}= (n1_R)a + (n1_R)b = na + nb.\\
        (m + n)a &= (m + n)(1_Ra) = ((m + n)1_R)a \overset{N}= (m1_R + n1_R)a \overset{dis}= (m1_R)a + (n1_R)a\\
        &= ma + na.\\
        (mn)a &= (mn)(1_Ra) = (mn1_R)a \overset{N}= (m1_Rn1_R)a \overset{ass}= (m1_R)((n1_R)a)\\
        &= (m1_R)(na) = m(1_R(na)) = m(na).
    \end{aligned}
    \]
    这个同态是唯一的, 因为我们要求环同态要把$1$映到$1$, 因此$\mathbb{Z}$在$\mathsf{Ring}$中是始对象(initial object), $\mathsf{Ring}$表示环范畴. 因此$n$可看作是$R$中的元素$n1_R$. 所以此后在没有歧义的情况下, 默认$0$就指零元, $1$指幺元.
\end{remark}

\begin{problem}
    假设集合$R$上有两个运算, 除加法的交换律外满足环的所有其他公理. 利用分配律证明: 加法是交换的 (从而$R$是环).
\end{problem}

\begin{proof}
    这和\ref{ex:1.1.5}是一道题.
\end{proof}

\begin{problem}\label{ex:1.2.3}
    设$X$是集合, $P(X)$表示$X$的所有子集形成的集合, 在$P(X)$上定义“加法”和“乘法”: $A + B = A \cup B - A \cap B$, $A \cdot B = A \cap B$. 证明: 在这些运算下$P(X)$是一个环, 且$2A = 0 (\forall A \in P(X))$.
\end{problem}

\begin{proof}
    这里$A + B$为对称差, $A + B = A \cup B - A \cap B = (A - B) \cup (B - A)$. 用$A^c$表示$A$的补集. 那么,
    \[
        A + B = (A \cap B^c) \cup (A^c \cap B).
    \]
    \begin{enumerate}[(i)]
        \item $(P(X), +)$是Abel群. 交换律由定义是显然的.
        
        结合律:
        \[
        \begin{aligned}
            (A + B) + C &= (((A \cap B^c) \cup (A^c \cap B)) \cap C^c)\\
            &\cup (((A \cap B^c) \cup (A^c \cap B))^c \cap C)\\
            &= (A \cap B^c \cap C^c) \cup (A^c \cap B \cap C^c) \cup (A^c \cap B^c \cap C)\\
            &\cup (A \cap B \cap C)\\
            &= A + (B + C). \quad \text{(轮换对称, 见\ref{ex:1.1.4}的结合律证明)}
        \end{aligned}
        \]
        零元为$\varnothing$,
        \[
            A + \varnothing = \varnothing + A = A \cup \varnothing - A \cap \varnothing = A.
        \]
        负元为$A$本身,
        \[
            A + A = A \cup A - A \cap A = A - A = \varnothing.
        \]
        即$2A = 0$.
        \item $(P(X), \cdot)$是(交换)幺半群, 单位元是$X$. 由于$\cdot$就是交集$\cap$, 因此这一点是显然的.
        \item 分配律:
        \[
        \begin{aligned}
            (A + B) \cdot C &= ((A \cap B^c) \cup (A^c \cap B)) \cap C\\
            &= (A \cap B^c \cap C) \cup (A^c \cap B \cap C)\\
            A \cdot C + B \cdot C &= (A \cap C \cap (B \cap C)^c) \cup ((A \cap C)^c \cap B \cap C)\\
            &= (A \cap B^c \cap C) \cup (A^c \cap B \cap C).
        \end{aligned}
        \]
        故有$(A + B) \cdot C = A \cdot C + B \cdot C$. 另一部分证明类似.
    \end{enumerate}
    因此$(P(X), +, \cdot)$为一个(交换)环.
\end{proof}

\begin{problem}\label{ex:1.2.4}
    设$R$是一个环, $S \,\red{\subseteq}\, R$是一个非空子集合. 试证明
    \[
        C(S) \defeq \{a \in R \mid ax = xa, \forall x \in S \}
    \]
    是$R$的一个子环.
\end{problem}

\begin{proof}
    该子环称为子集$S$的中心化子(centralizer). 当$S = R$时就是中心(\ref{ex:2.1.11}).

    $\forall a, b \in C(S)$, 需要验证
    \[
        a - b \in C(S), \quad ab \in C(S), \quad 1 \in C(S).
    \]\
    其中$1 \in C(S)$是显然的. 对$\forall x \in S$
    \[
    \begin{gathered}
        (a - b)x = ax + bx = xa + xb = x(a - b),\\
        (ab)x = a(bx) = a(xb) = (ax)b = (xa)b = x(ab).\\
    \end{gathered}
    \]
    因此$a - b, ab \in C(S)$, $C(S)$是子环.
\end{proof}

\begin{problem}\label{ex:1.2.5}
    证明:如果在环$R$中$1 - ab$可逆, 则$1 - ba$也可逆.
\end{problem}

\begin{proof}
    设$1 - ab$的逆为$c$, 考虑形式级数
    \[
        (1 - x)^{-1} = \sum_{i = 0}^{+\infty} x^i
    \]
    则有
    \[
    \begin{aligned}
        (1 - ba)^{-1} &= \sum_{i = 0}^{+\infty} (ba)^i\\
        &= 1 + b\left(\sum_{i = 0}^{+\infty} (ab)^i\right)a\\
        &= 1 + b(1 - ab)^{-1}\\
        &= 1 + bca.
    \end{aligned} 
    \]
    验证$1 + bca$确实是$1 - ba$的逆: 
    \[
    \begin{aligned}
        (1 - ba)(1 + bca) &= 1 - ba + bca - b(abc)a\\
        &= 1 - ba + bca -b(c - 1)a\\
        &= 1 - ba + bca -bca + ba = 1\\
        (1 + bca)(1 - ba) &= 1 + bca - ba - b(cab)a\\
        &= 1 + bca - ba - b(c - 1)a\\
        &= 1.
    \end{aligned}
    \]
\end{proof}

\begin{remark}
    使用形式级数是合理的, 从\ref{ex:2.3.1}可以看到形式级数环是有定义的, 且和多项式环一样是可以赋值的(\ref{ex:2.3.7}).
\end{remark}

\begin{problem}\label{ex:1.2.6}
    如果环$R$满足条件:$\forall x \in R,\quad x^2 = x$,证明$R$是交换环.
\end{problem}

\begin{proof}
    条件$x^2 = x$称为乘法是幂等(idempotent)的. 考虑
    \[
        (x + 1)^2 = x^2 + 2x + 1 = x + 1,
    \]
    或者直接带入$-x$, 得
    \[
        -x = x^2 = x.
    \]
    再考虑
    \[
        (x + y)^2 = x^2 + xy + yx + y^2 = x + xy + yx + y = x + y,
    \]
    得
    \[
        xy = -yx = yx.
    \]
\end{proof}

\begin{problem}[华罗庚恒等式]
    设$a, b$是环$R$中的元素. 如果$a, b, ab - 1$可逆, 证明$a - b^{-1}$, $(a - b^{-1})^{-1} - a^{-1}$也可逆, 且有下列恒等式:
    \[
        \left((a - b^{-1})^{-1} - a^{-1}\right)^{-1} = aba - a.
    \]
\end{problem}

\begin{proof}
    由于$a, b, ab - 1$均可逆, 即$a, b, ab - 1 \in U(R)$. $U(R)$为环$R$的单位群(\ref{ex:1.3.2}). 故
    \[
        a - b^{-1} = (ab - 1)b^{-1} \in U(R),
    \]
    那么只需证明华罗庚恒等式. 直接验证即可. 由\ref{ex:1.2.5}, $(ba - 1)^{-1} = b(ab - 1)^{-1}a - 1$以及\ref{ex:1.3.2}证明的(1).
    \[
    \begin{aligned}
        \left((a - b^{-1})^{-1} - a^{-1}\right)^{-1} &= \left(((ab - 1)b^{-1})^{-1} - a^{-1}\right)^{-1}\\
        &= (b(ab - 1)^{-1} - a^{-1})^{-1}\\
        &= \left((b(ab - 1)^{-1}a - 1)a^{-1}\right)^{-1}\\
        &= a(b(ab - 1)^{-1}a - 1)^{-1}\\
        &= a(ba - 1)\\
        &= aba - a.
    \end{aligned}
    \]
\end{proof}

\begin{problem}[多项式矩阵的带余除法]\label{ex:1.2.8}
    设$A \in M_n(K)$是一个给定的$n$阶矩阵. 对任意多项式矩阵$A(x) \in M_{n \times m}(K[x])$, 证明存在唯一的$B(x) \in M_{n \times m}(K[x])$, $R \in M_{n \times m}(K)$使得$A(x) = (xI_n - A)B(x) + R$.
\end{problem}

\begin{proof}
    先证唯一性, 若存在$B'(x) \in M_{n \times m}(K[x])$和$R' \in M_{n \times m}(K)$也满足条件, 则有
    \[
        (xI_n - A)(B(x) - B'(x)) = R' - R \in M_{n \times m}(K).
    \]
    设
    \[
        B(x) - B'(x) = B_0 + B_1x + B_2x^2 + \cdots + B_kx^k, \quad B_i \in M_{n \times m}(K), 0 \leqslant i \leqslant k.
    \]
    将左边展开得
    \[
    \begin{aligned}
        B_k &= 0,\\
        -AB_k + B_{k - 1} &= 0 \implies B_{k - 1} = 0,\\
        -AB_{k - 1} + B_{k - 2} &= 0 \implies B_{k - 2} = 0,\\
        \vdots\\
        -AB_1 + B_0 &= 0 \implies B_0 = 0,\\
        -AB_0 &= R' - R = 0.
    \end{aligned}
    \]
    再证存在性, 将$A(x)$写成多项式的形式,
    \[
        A(x) = A_0 + A_1x + A_2x^2 + \cdots + A_kx^k, \quad A_i \in M_{n \times m}(K), 0 \leqslant i \leqslant k.
    \]
    我们对$k$归纳, $k = 0$时, $A(x) = A_0$为常数矩阵, 取$B(x) = O_{n \times m}$(零矩阵), $R = A_0$即可.
    
    假设对任意$k$次多项式$A(x)$有$B(x) \in M_{n \times m}(K[x])$, $R \in M_{n \times m}(K)$使得$A(x) = (xI_n - A)B(x) + R$. 考查$k + 1$的情形:
    \[
    \begin{aligned}
        A(x) &= A_0 + x(A_1 + A_2x + \cdots + A_{k + 1}x^k)\\
        &= A_0 + x((xI_n - A)\tilde{B}(x) + \tilde{R})\\
        &= (xI_n - A)x\tilde{B}(x) + xI_n\tilde{R} - A\tilde{R} + A\tilde{R}+ A_0\\
        &= (xI_n - A)(x\tilde{B}(x) + \tilde{R}) + A\tilde{R} + A_0.
    \end{aligned}
    \]
    取$B(x) = x\tilde{B}(x) + \tilde{R} \in M_{n \times m}(K[x])$, $R = A\tilde{R} + A_0$即可.
\end{proof}

\begin{problem}\label{ex:1.2.9}
    设$m > 0$是任意整数, $\red{\mathbb{Z}/m\mathbb{Z}} = \{\overline{0}, \overline{1}, \cdots, \overline{m-1}\}$是$\mathbb{Z}$的模$m$剩余类环. 试证明: $\overline{a} \in \red{\mathbb{Z}/m\mathbb{Z}}$可逆当且仅当$(a, m) = 1$(即: $a$与$m$互素).
\end{problem}

\begin{proof}
    $\overline{a} \in \mathbb{Z}/m\mathbb{Z}$可逆,
    \[
    \begin{aligned}
        &\iff \exists \overline{b} \in \mathbb{Z}, \quad \overline{a}\overline{b} = \overline{1}\\
        &\iff ab = 1 + km, \quad k \in \mathbb{Z},\\
        &\iff (a, m) = 1. \quad \text{(Bézout's Identity)}
    \end{aligned}
    \]
\end{proof}

\begin{remark}
    一般用记号$\mathbb{Z}/m\mathbb{Z}$表示模$m$剩余类环(理想和商环, 教材2.1节p25). 若$(a, m) = 1$, 则$\overline{a}$是加法群$(\mathbb{Z}/m\mathbb{Z}, +)$的生成元, 即$\overline{a}$(在加法群)的阶(教材1.3节, p17)是$m$.
\end{remark}

\begin{problem}
    设$R$是仅有$n$个元素的环, 试证明对任意$a \in R$有
    \[
        na \defeq \underbrace{a + a + \cdots + a}_n = 0.
    \]
\end{problem}
    
\begin{proof}
    该题的证明归结为一句话, 加法群的阶$(R, +)$为$n$, 故$na = 0$.
\end{proof}

\begin{remark}
    有限群$G$内任一元素$a$, 有$|a| \Big| |G|$(教材4.1节p70推论4.1.3), 因此必有$a^{|G|} = e$, 在这道题就是$na = 0$.
\end{remark}

\begin{problem}
    环$R$中非零元$x$称为幂零元(nilpotent), 若存在$n > 0$使$x^n = 0$. 证明:
    \begin{enumerate}[(1)]
        \item 如果$x$是幂零元, 则$1 - x$是可逆元;
        \item $\mathbb{Z}/m\mathbb{Z}$有幂零元当且仅当$m$可以被一个大于$1$的整数的平方整除.
    \end{enumerate}
\end{problem}

\begin{proof}
    \begin{enumerate}[(1)]
        \item 注意到
        \[
            1 = 1 - x^n = (1 - x)(1 + x + x^2 + \cdots + x^{n - 1})
        \]
        \item "$\Rightarrow$": 若$\mathbb{Z}/m\mathbb{Z}$有幂零元$\overline{a}$, 则存在$n > 1(a \neq 0)$使得$\overline{a}^n = \overline{a^n} = \overline{0}$. 即$m \mid a^n$. 取素数$p \mid m$, 则$p \mid a^n$, 故$p \mid a$. 因此, 若$m$的素因数分解为$m = p_1^{e_1}p_2^{e_2}\cdots p_r^{e_r}$, 其中$p_1, p_2, \cdots, p_r$为互异的素数, $e_1, e_2, \cdots, e_r \geqslant 1$, 则$p_i \mid a,\, 1 \leqslant i \leqslant r$, 故有$p_1p_2\cdots p_r \mid a$. 因此有$p_1p_2\cdots p_r \leqslant a < m$, 故必有某个$e_i > 1$, 即$\exists 1 \leqslant i \leqslant r$, $e_i \geqslant 2$, 这样$p_i^2 \mid m$.
        
        "$\Leftarrow$": 反过来, 若$m$可以被某个大于$1$的平方整除, 则上述$e_i$中必有一个大于$1$, 此时取$a = p_1p_2\cdots p_r$, $\overline{a}$为$\mathbb{Z}/m\mathbb{Z}$的幂零元.
    \end{enumerate}
\end{proof}

\begin{problem}
    设$R$是一个环, 如果$(xy)^2 = x^2y^2 (\forall x, y \in R)$, 则$R$是交换环.
\end{problem}

\begin{proof}
    先考虑
    \[
        ((x + 1)y)^2 = (x + 1)^2y^2 \implies\, xy^2 = yxy,
    \]
    再将上式中$y$换成$y + 1$,
    \[
        x(y + 1)^2 = (y + 1)x(y + 1) \implies xy = yx.
    \]
\end{proof}

\begin{problem}
    如果环$R$满足条件: $x^6 = x (\forall x \in R)$. 证明:
    \begin{enumerate}[(1)]
        \item $x^2 = x (\forall x \in R)$;
        \item $R$是一个交换环.
    \end{enumerate}
\end{problem}

\begin{proof}
    \begin{enumerate}[(1)]
        \item 先带入$-x$,
        \[
            -x = (-x)^6 = x^6 = x \implies 2x = 0.
        \]
        考虑$(x + 1)^6$,
        \[
            (x + 1)^6 = x^6 + 6x^5 + 15x^4 + 20x^3 + 15x^2 + 6x + 1 = x + 1,
        \]
        得到
        \[
            6x^5 + 15x^4 + 20x^3 + 15x^2 + 6x = 0.
        \]
        利用$2x = 0$消去含$2x$的项得
        \[
            x^4 + x^2 = 0.
        \]
        两边乘$x^2$得
        \[
            x + x^4 = 0.
        \]
        再相减得$x^2 = x$.
        \item 由(1)和\ref{ex:1.2.6}.
    \end{enumerate}
\end{proof}
\subsection{教材p17-p18}

\begin{problem}\label{ex:1.3.1}
    设$G$是一个群, 对于任意的$a, b \in G$, 证明$ab$的阶和$ba$的阶相等.
\end{problem}

\begin{proof}
    若$|ab| = n < \infty$, 则
    \[
        (ba)^n = b \cdot (ab)^n \cdot b^{-1} = bb^{-1} = e.
    \]
    且对$1 \leqslant k < n$, $(ba)^k = b(ab)^kb^{-1} \neq e$. 因此$|ba| = n$. 反之亦然.
    
    若$|ab| = \infty$, 则
    \[
        \forall n \in \mathbb{Z}_{\geqslant 1}, \quad (ba)^n = b(ab)^nb^{-1} \neq e.
    \]
    故$|ba| = \infty$. 反之亦然.
\end{proof}

\begin{remark}
    事实上, 群$G$内$g$和$h = aga^{-1}$阶相等. $h$称为$g$的一个共轭(conjugate, 教材p77).
    \[
        \sigma_a: G \to G, \quad g \mapsto aga^{-1}
    \]
    是群$G$的一个自同构. 而对一般的群同态$\varphi: G \to G'$, $|g| < \infty \implies |\varphi(g)| < \infty$且$|\varphi(g)| \big| |g|$. 因此若$\varphi$为同构, 则$|g| = |\varphi(g)|$(包括左右为无穷的情况).
\end{remark}

\begin{problem}\label{ex:1.3.2}
    设$R$是一个环, $U(R)$表示$R$中所有可逆元集合, 试证明: $U(R)$关于环$R$的乘法是一个群(称为$R$的单位群).
\end{problem}

\begin{proof}
    \begin{enumerate}[(1)]
        \item 这里首先需要验证运算的封闭性, $\forall a, b \in U(R)$, 有$(b^{-1}a^{-1})(ab) = b^{-1}(a^{-1}a)b = 1$, 故$ab \in U(R)$且$(ab)^{-1} = b^{-1}a^{-1}$.
        \item $1 \in U(R)$, 因为$1 \cdot 1 = 1$的确可逆;
        \item 由于乘法是$R$上的乘法, 故结合律成立;
        \item 若$a \in U(R)$, 则由\ref{ex:1.1.1}的(3), $a^{-1} \in U(R)$且$(a^{-1})^{-1} = a$;
    \end{enumerate}
\end{proof}

\begin{remark}
    一般$U(R)$也记作$R^\times$, 比如$K$是域时, $K^\times = K^* =  K \setminus \{0\}$.
\end{remark}

\begin{problem}
    证明除了单位元之外所有元素的阶都是$2$的群一定是交换群.
\end{problem}

\begin{proof}
    由于任意$a^2 = e$, 故$a = a^{-1}$.

    考虑
    \[
        (ab)^2 = e \implies ab = b^{-1}a^{-1} = ba.
    \]
    或直接验证
    \[
        ab = ab \cdot (ba)^2 = abbaba = ba
    \]
\end{proof}

\begin{problem}
    令$C(\mathbb{R} ) = \left\{\text{所有连续函数: } \mathbb{R} \overset{f}\to \mathbb{R} \right\}$, $\forall f ,\, g \in C(\mathbb{R})$,
    \[
        f + g \in C(\mathbb{R}),\quad f \cdot g \in C(\mathbb{R})
    \]
    定义: $\forall x \in \mathbb{R}, (f + g)(x) = f(x) + g(x), (f \cdot g)(x) = f(g(x))$, 证明$(C(\mathbb{R}), +)$是交换群. $(C(\mathbb{R}), +, \cdot)$是否为环?
\end{problem}

\begin{proof}
    $(C(\mathbb{R}), +)$的零元为零函数$\mathbf{0}: \mathbb{R} \to \mathbb{R},\, x \mapsto 0$, $(f + \mathbf{0})(x) = f(x) + 0 = f(x) = 0 + f(x) = (\mathbf{0} + f)(x),\, \forall x \in \mathbb{R}$.

    $f \in C(\mathbb{R})$的负元为$-f: \mathbb{R} \to \mathbb{R},\, x \mapsto -f(x)$, $(f + (-f))(x) = ((-f) + f)(x) = f(x) - f(x) = 0 = \mathbf{0}(x)$.

    由于$f + g$为逐点定义, 故交换律和结合律依赖于$\mathbb{R}$的加法, 是平凡的. 故$(C(\mathbb{R}), +)$是Abel群.

    若$f$不是$\mathbb{R}$-线性函数, 如$f(x) = x^2$, 则$(f \cdot (g + h))(x) = f((g + h)(x)) = f(g(x) + h(x)) \neq f(g(x)) + f(h(x))$. 故$C(\mathbb{R}, +, \cdot)$不是环.
\end{proof}

\begin{problem}\label{ex:1.3.5}
    写出对称群$S_3$的乘法表.
\end{problem}

\begin{proof}
    记$\mathrm{id}_{S_3} = e$, 令$a = (1\:2)$, $b = (1\:2\:3)$, 有$a^2 = e$, $b^3 = e$, $abab = e \iff ba = ab^2$. 乘法表如下:
    \[
    \begin{array}{c|cccccc}
             & e    & a    & b   & b^2  & ab   & ab^2 \\
        \hline
        e    & e    & a    & b   & b^2  & ab   & ab^2 \\
        a    & a    & e    & ab  & ab^2 & b^2  & b \\
        b    & b    & ab^2 & b^2 & e    & a    & ab \\
        b^2  & b^2  & ab   & e   & b    & ab^2 & a \\
        ab   & ab   & b^2  & a   & ab^2 & e    & b \\
        ab^2 & ab^2 & b    & ab  & a    & b^2  & e \\
    \end{array}
    \]
\end{proof}

\begin{remark}
    可以看到$S_3$, 若取$a = (1\:2),\, b = (1\:2\:3)$, 则$S_3$可以由$a, b$生成, 即考虑所有可能的乘积, 一般可以表示为$S_3 = \langle a, b \rangle,\, a = (1\:2),\, b = (1\:2\:3)$.
    
    若不给$a, b$加任何限制, 便得到一个自由群(free group)$F(\{a, b\})$. 一般地, 任意一个集合$A$都可以生成一个自由群$F(A)$, $A$就是生成元组成的集合. 可以证明任何一个群都同构于某个自由群的商群, 而对应的正规子群便是由生成元满足的某些关系确定(将$A$看成字母表, $\Sigma_A$表示单词的集合, 这些关系可以表示为一些满足$w = e$单词$w \in \Sigma_A$). 把这些$w$组成的集合记为$\mathscr{R}$, $A$和$\mathscr{R}$将唯一确定一个群$G$, $(A \mid \mathscr{R})$称为$G$的一个展示(presentation). 以$S_3$为例, $S_3$的一个展示为$(\{a, b\} \mid a^2, b^3, abab)$. 另外有二面体群(Dihedral Groups)$D_{2n} = (a, b \mid a^2, b^n, abab)$
    
    由于这本教材没有讲自由群, 所以想要了解的话需要查阅别的教材.(可参考\cite{aluffi2009algebra}II.\S5和II.\S8.2)
    
    BTW, 这本教材和很多教材一样, 会把集合$A$对称群$S_A$上的乘法写成$f \cdot g \defeq f \circ g$, 这个其实会有一点不舒服. 正常我们习惯于说: 映射$f:X \to Y$和$g:Y \to Z$的复合是$g \circ f$. 这在范畴的定义也是习惯于这样, 复合会写成这样:
    \[
        \mathrm{Hom}_{\mathcal{C}}(X, Y) \times \mathrm{Hom}_{\mathcal{C}}(Y, Z) \to \mathrm{Hom}_{\mathcal{C}}(X, Z),\, (f, g) \mapsto g \circ f.
    \]
    这样说的好处在于一眼能感觉出这个运算是不交换的. 当然这只是个人感觉, 也有可能是我先入为主了, 因为我最开始接触到的范畴里的复合是这样写的. 如果引入范畴的记号, $S_A$会记作$\mathrm{Aut}_{\mathsf{Set}}(A)$, 其中$\mathsf{Set}$表示集合范畴. 那么$S_A$上的乘法按范畴的定义来写应该是:
    \[
        S_A \times S_A \to S_A,\quad (f, g) \mapsto f \cdot g \defeq g \circ f
    \]
    可以看到和$f \cdot g \defeq f \circ g$刚好是反过来的. 没有使用范畴语言的话就还好, 不会出现前后不自洽的问题, 但如果介绍了范畴语言, 那应该注意$S_A$上乘法的定义要和范畴定义不能冲突, 这一点\cite{lang2012algebra}和\cite{hungerford2003algebra}就做的很好. 它的范畴定义故意反了过来, 它写成$\mathrm{Hom}_{\mathcal{C}}(Y, Z) \times \mathrm{Hom}_{\mathcal{C}}(X, Y) \to \mathrm{Hom}_{\mathcal{C}}(X, Z)$.
    
    那么哪一个才对呢, 事实上都是对的, 你总能验证$S_A$确实时一个群. 原因在于, 当你只考虑所有的同构时, 就得到一个子范畴, 这是一个群胚(groupoid), 它是一个自反范畴, 所以顺序就没区别了. 但我个人认为还是统一一下比较好, 主要是复合是非交换的, $f \circ g$和$g \circ f$一般不等. 为了方便还是按照教材为准吧, 使用$f \cdot g = f \circ g$.(尽管我是有点不习惯的)
\end{remark}

\begin{problem}
    证明: 一个群$G$不会是两个真子群(不等于$G$的子群)的并.
\end{problem}

\begin{proof}
    反证, 假设$H_1, H_2 \lvertneqq G$且$G = H_1 \cup H_2$, 则$\exists h_1 \in G \setminus H_2 \subseteq H_1, h_2 \in G \setminus H_1 \subseteq H_2$, 有$h_1h_2 \in G = H_1 \cup H_2$, 矛盾. (不妨设$h_1h_2 \in H_1 \implies h_2 \in H_1$)
\end{proof}

\ref{ex:1.3.7}-\ref{ex:1.3.9}为群的其他三种定义.

\begin{problem}\label{ex:1.3.7}
    一个非空集合$G$带有满足结合律的“乘法”运算, 我们称之为半群. 如果$G$是一个半群, 且满足如下性质:
    \begin{enumerate}[(1)]
        \item $G$含有右单位元$1_r$(即: $a \cdot 1_r = a$, $\forall a \in G)$;
        \item $G$中的每个元素$a$有右逆(即: 存在$b \in G$, 使得$a \cdot b = 1_r)$.
    \end{enumerate}
    试证明: $G$是一个群.
\end{problem}

\begin{proof}
    先证右逆为逆,
    \[
    \begin{gathered}
        \forall a \in G \, \exists b \in G, ab = 1_r,\\
        \implies \exists c \in G, bc = 1_r,\\
        \implies ba = (ba)1_r = (ba)(bc) = b(ab)c = b1_rc = bc = 1_r.
    \end{gathered}
    \]
    再证右单位为单位,
    \[
        1_ra = (ab)a = a(ba) = a1_r = a.
    \]
\end{proof}

\begin{problem}\label{ex:1.3.8}
    证明: 半群$G$是群的充要条件是: $\forall a, b \in G$, $ax = b$和$ya = b$都有(唯一)解.
\end{problem}

\begin{proof}
\begin{enumerate}[(1)]
    \item "$\impliedby$": 取定一个$a \in G$, 方程$ax = a$的解设为$e_a$. 对$\forall b \in G$, 方程$ya = b$有解$y_b$, 则有
    \[
        be_a = (y_ba)e_a = y_b(ae_a) = y_ba = b.
    \]
    即$e_a$是$G$的右单位, 记为$1_r$, 又因为$\forall a \in G$, 方程$ax = 1_r$有解, 即$a$有右逆, 由\ref{ex:1.3.7}知$G$是群.
    \item "$\implies$": 若$G$是群, 则方程$ax = b$的唯一解为$a^{-1}b$, 方程$ya = b$的唯一解为$ba^{-1}$.
\end{enumerate}
    
\end{proof}

\begin{problem}\label{ex:1.3.9}
    证明:
    \begin{enumerate}[(1)]
        \item 在群中左右消去律都成立: 如果$ax = ay$, 则$x = y$; 如果$xa = ya$,则$x = y$.
        \item 左右消去律都成立的有限半群一定是群.
    \end{enumerate}
\end{problem}

\begin{proof}
    设$G = \{a_1, a_2, \cdots a_n\}$. 对$\forall 1 \leqslant i, j \leqslant n$,
    \[
        a_ia_1, a_ia_2, \cdots, a_ia_n
    \]
    互异, 否则存在$a_k \neq a_l$使得$a_ia_k = a_ia_l$, 由消去律得$a_k = a_l$矛盾. 因此$\exists 1 \leqslant t \leqslant n$, $a_ia_t = a_j$, 即方程$a_ix = a_j$有解. 同理方程$ya_i = a_j$也有解, 由\ref{ex:1.3.8}, $G$是群.
\end{proof}

\begin{problem}\label{ex:1.3.10}
    证明:偶数阶有限群$G$中必有$2$阶元.
\end{problem}

\begin{proof}
    设$|G| = 2n$. 对$e \neq g \in G$, $|g| = 2 \iff g = g^{-1}$. 定义$G$上的一个等价关系
    \[
        g \sim g' \iff g = g' \lor g' = g^{-1}.
    \]
    考虑商集$G/\sim = \{\overline{g} \mid g \in G\}$, 用$\#S$表示集合$S$的元素个数(基数)防止混淆. 若$|g| = 2$或$g = e$, 则$\#\, \overline{g} = 1$, 否则$\#\, \overline{g} = 2$. 因此若$m$为$G$中阶为$2$的元素的个数, 则$2n = m + 1 + 2(\#\, (G/\sim) - m - 1)$, 故$2n - m - 1$为偶数, 因此$m > 0$.
\end{proof}

\begin{remark}
    当然可以用Sylow定理一步到位.
\end{remark}

\begin{problem}
    证明:$GL_2(\mathbb{R})$中的元素
    \(
        x = \begin{pmatrix}
            0 & 1\\
            -1 & 0
        \end{pmatrix},
        y = \begin{pmatrix}
            0 & 1\\
            -1 & -1
        \end{pmatrix}
    \)
    的阶分别是$4$和$3$. 但$xy$是无限阶元.
\end{problem}

\begin{proof}
    用$I_n$表示$n$阶单位阵, 计算可得
    \[
        x^2 = \begin{pmatrix}
            -1 & 0 \\
            0 & -1
        \end{pmatrix},
        x^3 = \begin{pmatrix}
            0 & -1 \\
            1 & 0
        \end{pmatrix},
        x^4 = \begin{pmatrix}
            1 & 0 \\
            0 & 1
        \end{pmatrix} = I_2.
    \]
    故$|x| = 4$, 同理,
    \[
        y^2 = \begin{pmatrix}
            -1 & -1 \\
            1 & 0
        \end{pmatrix},
        y^3 = \begin{pmatrix}
            1 & 0 \\
            0 & 1
        \end{pmatrix} = I_2.
    \]
    $|y| = 3$. 最后是$xy$,
    \[
        xy = \begin{pmatrix}
            -1 & -1 \\
            0 & -1
        \end{pmatrix},
        (xy)^2 = \begin{pmatrix}
            1 & 2 \\
            0 & 1
        \end{pmatrix},
        (xy)^3 = \begin{pmatrix}
            -1 & -3 \\
            0 & -1
        \end{pmatrix}, \cdots
    \]
    可以用归纳法证明
    \[
        (xy)^n = (-1)^n
        \begin{pmatrix}
            1 & n \\
            0 & 1
        \end{pmatrix}
        \neq I_2, \forall n \in \mathbb{Z}_{\geqslant 1}.
    \]
    故$|xy| = \infty$.
\end{proof}
    
\begin{problem}\label{ex:1.3.12}
    证明群的任意多个子群的交仍是子群.
\end{problem}

\begin{proof}
    设$G$是群, 记$I$为指标集, $H_i < G,\, \forall \in I$. 验证\(H = \displaystyle\bigcap_{i \in I} H_i < G\): 首先$e_G \in H$, $H \neq \varnothing$,
    \[
    \begin{gathered}
        \forall a, b \in H = \bigcap_{i \in I} H_i \implies \forall i \in I,\,a, b \in H_i\\
        \implies ab^{-1} \in H_i, \quad \forall i \in I\\
        \implies ab^{-1} \in \bigcap_{i \in I} H_i = H.
    \end{gathered}
    \]
\end{proof}

\begin{remark}
    教材中并未提及这个判断子群的命题, 但其实是最常用的.

    \begin{propstar}[子群的判定]
        设$G$是一个群, $\varnothing \neq S \subseteq G$, 则$S < G$($S$是$G$的子群的记号)当且仅当
        \[
            \forall a, b \in S \iff ab^{-1} \in S.
        \]
    \end{propstar}
    证明可参考\cite{aluffi2009algebra}p79.
\end{remark}
\subsection{教材p21-p22}

\begin{problem}\label{ex:1.4.1}
    设$\varphi:G \to G'$是群同态, 试证明:
\begin{enumerate}[(1)]
    \item $\ker(\varphi) \defeq \{g \in G \mid \varphi(g) = e'\}$ $(e' \in G'$表示的单位元)是$G$的子群(称
为群同态$\varphi$的核);
    \item \[\varphi(G) = \{\varphi(g) \mid \forall g \in G\} \subset G'\]
是$G'$的子群(称为群同态$\varphi$的像).
\end{enumerate}
\end{problem}

\begin{solution}
    教材命题1.4.1的(1)(5)直接使用.
\begin{enumerate}[(1)]
    \item $e \in \ker(\varphi)$非空, 直接验证\[
    \begin{gathered}
        \forall a, b \in \ker(\varphi),\, \varphi(ab^{-1}) = \varphi(a)\varphi(b)^{-1} = e'e' = e'\\
        \implies ab^{-1} \in \ker(\varphi).
    \end{gathered}
    \]
    \item $e' \in \varphi(G)$非空, 直接验证\[
    \begin{gathered}
        \forall x, y \in \varphi(G),\, \exists a, b \in G,\, x = \varphi(a), y = \varphi(b)\\
        \implies xy^{-1} = \varphi(a)(\varphi(b))^{-1} = \varphi(a)\varphi(b^{-1}) = \varphi(ab^{-1}) \in \varphi(G).
    \end{gathered}
    \]
\end{enumerate}
\end{solution}

\begin{problem}
    令$G$是函数$f(x) = \frac1x, g(x) = \frac{x-1}x$关于函数的合成生成的一个群
(即群乘法为函数合成), 证明$G$同构于$S_3$.
\end{problem}

\begin{solution}
    由\ref{ex:1.3.5}的注记, 只需验证$f^2 = \mathrm{id}, g^3 = \mathrm{id}, fgfg = \mathrm{id}$.
\[
\begin{aligned}
    f^2(x) &= f(f(x)) = \frac{1}{\frac{1}{x}} = x.\\
    g^2(x) &= g(g(x)) = 1 - \frac{1}{(1 - \frac{1}{x})} = -\frac{1}{x - 1}\\
    g^3(x) &= g(g^2(x)) = 1 - (\frac{1}{-\frac{1}{x - 1}}) = 1 + x - 1 = x.\\
    (fg)(x) &= f(g(x)) = \frac{x}{x - 1},\\
    (fgfg)(x) &= (fg)^2(x) = 1 + \frac{1}{\frac{x}{x - 1} - 1} = 1 + x - 1 = x.
\end{aligned}
\]

\end{solution}

\begin{problem}
    设$R \overset{\varphi}\to R'$是环同态, 证明集合
$ker(\varphi) = \{x \in R \mid \varphi(x) = 0_{R'}\}$满足:
\begin{enumerate}[(1)]
    \item $\ker(\varphi)$是$(R, +)$的子群;
    \item $\forall a \in \ker (\varphi), x \in R$有$ax \in \ker(\varphi)$, $xa \in \ker (\varphi)$.
$(\ker(\varphi)$称为环同态$\varphi$的核.)
\end{enumerate}
\end{problem}

\begin{solution}
\begin{enumerate}[(1)]
    \item 即\ref{ex:1.4.1}(1);
    \item 直接验证\[
        \forall a \in \ker(\varphi),\, x \in R,\, \varphi(xa) = \varphi(x)\varphi(a) = \varphi(x)0_{R'} = 0_{R'}
    \]
    另一半同理.
\end{enumerate}
注:满足(1)(2)的$R$的子集称为$R$的一个理想(ideal), 教材p25定义2.1.4.
\end{solution}

\begin{problem}
    设$K$是一个域, $\phi:K[x] \to K[x]$是$K$的多项式环之间的环自同态. 
如果对于任意的$k \in K, \phi(k) = k$, 试证明:$\phi$是满同态的充分必要条件是存在
$a, b \in K(a \neq 0)$使得$\phi(x) = ax + b$.
\end{problem}

\begin{solution}
\begin{enumerate}[(1)]
    \item "$\implies$": 记$f(x) = \phi(x)$, 若$\phi$是满的, 则存在$g(x) \in K[x]$
使得$\phi(g(x)) = x$, 则$x = \phi(g(x)) \overset{!}= g(\phi(x)) = g(f(x))$,
!处是根据环同态的定义以及$\phi(k) = k,\, \forall k \in K$得到.
考查次数$1 = \deg(g(f(x))) = \deg(g) \cdot \deg(f)$(域没有零因子). 因此$\deg(f) = \deg(g) = 1$,
i.e. $\phi(x) = f(x) = ax + b,\, \exists a \neq 0, b \in K$.
    \item "$\impliedby$": 若存在$a \neq 0, b \in K$使得$\phi(x) = ax + b$,
则令$y = ax + b$得到$x = a^{-1}(y - b)$. 那么对任意的$f(x) \in K[x]$,
存在$g(x) = f(a^{-1}(x - b)) \in K[x]$使得$\phi(g(x)) = g(\phi(x)) = g(y) = f(a^{-1}(y - b)) = f(x)$.
\end{enumerate}
\end{solution}

\begin{problem}
    证明实数的加法群$(\mathbb{R}, +)$和正实数的乘法群$(\mathbb{R}_{>0}, \cdot)$同构.
\end{problem}

\begin{solution}
    注意到$f: \mathbb{R} \to \mathbb{R}_{>0},\, x \mapsto e^x$
是同构. $f^{-1}(x) = \ln x$.

注: 事实上, 由$f(x + y) = f(x)f(y)$并利用归纳法和同态定义可以直接推出
$f(x) = a^x,\, a = f(1),\, x \in \mathbb{Q}$, 若有连续性则可以延拓到$\mathbb{R}$上.
\end{solution}

\begin{problem}
    证明有理数的加法群$(\mathbb{Q}, +)$和正有理数的乘法群$(\mathbb{Q}_{>0}, \cdot)$不同构.
\end{problem}

\begin{solution}
    反证, 假设存在同构$f: \mathbb{Q} \to \mathbb{Q}_{>0}$,
则设$2 = f(a) = f(\frac{a}{2} + \frac{a}{2}) = f(\frac{a}{2}) \cdot f(\frac{a}{2}) = f(\frac{a}{2})^2$
矛盾.
\end{solution}

\begin{problem}
    证明有理数域$\mathbb{Q}$和实数域$\mathbb{R}$的自同构都只有恒等映射.
\end{problem}

\begin{solution}
    不妨设$\sigma: \mathbb{Q} \to \mathbb{Q}$是同构, 根据定义,
有$\sigma(0) = 0, \sigma(1) = 1, \sigma(-a) = -\sigma(a), \sigma(a^{-1}) = (\sigma(a))^{-1}$.
因此先用归纳法得到$\sigma|_{\mathbb{N}} = \mathrm{id}_{\mathbb{N}}$, 用负元延拓到$\mathbb{Z}$,
再用逆元延拓到$\mathbb{Q}$得$\sigma = \mathrm{id}_{\mathbb{Q}}$. 事实上, 这个推导对于
任何特征$0$的域都是对的, 即$\mathbb{Q}$是特征$0$最小域(环的特征见教材2.1节p27定义2.1.5).

    对$\mathbb{R}$, 首先若$\phi: \mathbb{R} \to \mathbb{R}$是同构, 有上面可知
$\phi|_{\mathbb{Q}} = \mathrm{id}_{\mathbb{Q}}$. 另外, 可以证明$\phi$保序结构,
即$x \geqslant 0 \implies \phi(x) \geqslant 0$. 这是因为对$x > 0$总有
$\phi(x) = \phi(\sqrt{x} \cdot \sqrt{x}) = \phi(\sqrt{x})^2 > 0$. 保序则保极限,
即对单调有界有理数列$\{q_n\}_{n \in \mathbb{N}}$有
$\lim_{n \to \infty} \phi(q_n) = \lim_{n \to \infty} q_n$(实际上保序就可以保持
$\mathbb{R}$上的拓扑结构, $\phi$是连续的). 由于$\mathbb{Q}$在$R$中稠密,
从而$\phi = \mathrm{id}_{\mathbb{R}}$.

    一般情况下子域的自同构是不一定能延拓到扩域上, 比如考虑$\mathbb{Q}(\sqrt{2})$
的共轭自同构(类似复共轭, $\sqrt{2} \mapsto -\sqrt{2}$), 它不能延拓到$\mathbb{R}$上.

    综上可得,
$\mathrm{Aut}_{\mathsf{Ring}}(\mathbb{R}) = \mathrm{Aut}_{\mathbb{Q}}(\mathbb{R})$
是平凡群.(由于$\mathbb{R}/\mathbb{Q}$并不是Galois扩张, 因此没有用符号
$\mathrm{Gal}(\mathbb{R}/\mathbb{Q})$, 另外$\mathsf{Ring}$表示环范畴)
\end{solution}

\begin{problem}
    证明:$\mathbb{Q}[\sqrt 2] = \{a + b\sqrt 2 \mid a, b \in \mathbb{Q}\}$,
$\mathbb{Q}[\sqrt 5] = \{a + b\sqrt 5 \mid a, b \in \mathbb{Q}\}$
都是$\mathbb{R}$的子域. 它们是同构的域吗?
\end{problem}

\begin{solution}
    由教材命题1.4.1的(9), 两个域若存在同态则一定是单同态, 即只有两种可能, 
    一个域为另一个域的扩张或两者同构. 我们断言这两个域之间不存在同态.
    
    假设存在同态
$\varphi: \mathbb{Q}[\sqrt{2}] \to \mathbb{Q}[\sqrt{5}]$,
则设$\varphi(\sqrt{2}) = a + b\sqrt{5},\, a, b \in \mathbb{Q}$.
注意到由同态定义有$\varphi(2) = 2$, 立刻有
\[
    2 = \varphi(2) = \varphi(\sqrt{2})^2 = (a + b\sqrt{5})^2 = a^2 + 5b^2 + 2ab\sqrt{5}
\]
这要求$a^2 + 5b^2 = 2$且$ab = 0$, 这是不可能的, 矛盾.
\end{solution}

\begin{problem}
    设$K, L$是两个域, 如果$L$是$K$的子域, 则$K$称为$L$的扩域,
$K \supset L$称为域扩张, 试证明:
\begin{enumerate}[(1)]
    \item 域的加法和乘法使得$K$是一个$L$-向量空间$([K:L] = \dim_L(K)$称为域
扩张$K \supset L$ 的次数);
    \item 如果$K \supset \mathbb{R}$是一个二次扩张(即$[K:\mathbb{R}] = 2)$,
则$K$必同构于复数域$\mathbb{C}$.
\end{enumerate}
\end{problem}

\begin{solution}
\begin{enumerate}[(1)]
    \item $(K, +)$是一个Abel群, 这一点无需再说明. 乘法在这里可能有些歧义,
    此处是要验证乘法限制在$L \times K$上, 即
    \[
        \cdot: L \times K \to K, \quad (l, k) \mapsto lk
    \]
    是数乘. 即要验证
    \[
    \begin{gathered}
        (l_1l_2)k = l_1(l_2k),\\
        (l_1 + l_2)k = l_1k + l_2k,\\
        l(k_1 + k_2) = lk_1 + lk_2,\\
        1k = k = k1.
    \end{gathered}
    \]
    这些都由域的定义得到.

    这也说明若同态$K_1 \to K_2$保持$L$($K_1, K_2$为$L$的两个扩域),
    则一定是$L$-线性映射.

    事实上, 若有环同态$R \overset{\varphi}\to S$, 则$S$上自动有一个
    $R$-模结构
    \[
        R \times S \to S, \quad (r, s) \mapsto \varphi(r)s
    \]
    这道题对应的同态其实就是包含(inclusion)$L \overset{i}\hookrightarrow K$.
    \item 由(1), 扩域$\mathbb{C}/\mathbb{R}$的自同构一定是$\mathbb{R}$-线性的.
    设同构$f: \mathbb{C} \to \mathbb{C}$, 则有
    $f(x + yi) = x + yf(i),\, x, y \in \mathbb{R}$, 且
    保持乘法, 可得$f(i) = \pm i$. 也就是说$\mathbb{C}/\mathbb{R}$的自同构都只有
    恒等映射和共轭, 即$\mathrm{Gal}(\mathbb{C}/\mathbb{R}) = \mathbb{Z}/2\mathbb{Z}$.

    由线性代数的结论, 可以直接得到$K$和$\mathbb{C}$是作为线性空间同构, 但这是不够的,
    只有上述两个线性映射是域同构, 需要做基变换转为恒等或共轭才能保持乘法. 事实上只要存在
    一个基变换就能变回恒等映射, 恒等映射总是同构,
    但前提是承载集合(underlying set)要一样. 比如$\mathbb{Q}(\sqrt{2})$
    和$\mathbb{Q}(\sqrt{3})$作为$\mathbb{Q}$-线性空间也是同构的, 但他们
    之间没有域同态.

    可取$K$的一组基为$1, \alpha$, 其中$\alpha \in \mathbb{C} \setminus \mathbb{R}$.
    不可避免地要考虑$\alpha^2$的结果, 由于$1, \alpha$是基, 因此$\alpha^2$可以被线性表出,
    即$\alpha^2 = x + y\alpha$. 由于$\alpha \notin \mathbb{R}$, 有$y^2 + 4x < 0$,
    解二次方程得到$\alpha = \frac{y \pm i\sqrt{|y^2 + 4x|}}{2}$.
    故映射
\[
    f: K \to \mathbb{C},\, u + v\alpha \mapsto u + v\frac{y \pm i\sqrt{|y^2 + 4x|}}{2}
\]
    是域同构.
\end{enumerate}
\end{solution}

\begin{problem}
    设$d$是一个非零整数, 且$\sqrt d \notin \mathbb{Q}$. 证明:
\[
    \mathbb{Q}[\sqrt{d}] = \{a + b\sqrt{d} \mid a, b \in \mathbb{Q}\} \supset \mathbb{Q}
\]
是一个二次扩张($d < 0$时, $\mathbb{Q}[\sqrt{d}]$称为虚二次域, $d > 0$时称为实二次域).
\end{problem}

\begin{solution}
    $sqrt{d}$满足多项式$f(x) = x^2 - d \in \mathbb{Q}[x]$,
    因此在$\mathbb{Q}[\sqrt{d}]$中.
\end{solution}

\begin{problem}
    设$L \supset K$是一个域扩张, 证明: 下述集合
\[
\mathrm{Gal}(L/K)=
\left\{L \xrightarrow{\sigma} L \mid \sigma\text{ 是域同构},\text{ 且 } \sigma(a) = a \text{ 对任意 } a \in K \text{ 成立}\right\}
\]
关于映射的合成是一个群(称为域扩张$L\supset K$的伽罗瓦群).
\end{problem}

\begin{solution}
    $\mathrm{Gal}(L/K) \subseteq \mathrm{Aut}(L)$, 只需说明$\mathrm{Gal}(L/K)$
是子群.

    $\forall \varphi, \psi \in \mathrm{Gal}(L/K)$, 由于$\psi|_K = \mathrm{id}_K$,
因此$\psi^{-1}|_K = \mathrm{id}_K$, 故$(\varphi \circ \psi^{-1})|_K = \mathrm{id}_K$,
即$\varphi \circ \psi^{-1} \in \mathrm{Gal}(L/K)$.
\end{solution}

\begin{problem}
    求$\mathrm{Gal}\left(\mathbb{Q}[\sqrt{d}]/\mathbb{Q}\right)$,
此处$d \in \mathbb{Z}, \sqrt{d} \notin \mathbb{Q}$.
\end{problem}

\begin{solution}
    
\end{solution}

\begin{problem}
    设$V = (V, +)$ 是一个加法群, $\mathrm{Hom}(V)$表示它的自同态环.
对任意域$K$, 如果存在一个数乘运算
$K \times V \to V,\, (\lambda,v) \mapsto \lambda \cdot v$,
使得加法群$V = (V, +)$成为一个$K$-线性空间, 则称该数乘运算是加法群
$V = (V, +)$上的一个$K$-线性空间结构. 试证明:
\begin{enumerate}[(1)]
    \item 如果存在一个环同态$\varphi:K \to \mathrm{Hom}(V)$,则数乘运算
\[
    K \times V \to V,\quad (\lambda, v) \mapsto \lambda \cdot v \defeq \varphi(\lambda)(v)
\]
是$V$上的一个$K$-线性空间结构;
    \item 如果在$V$上存在$K$-线性空间结构$\phi:K \times V \to V$,则映射
\[
    \varphi:K \to \mathrm{Hom}(V),\quad \lambda \mapsto \phi(\lambda, \cdot)
\]
是一个环同态, 其中$\phi(\lambda, \cdot):V \to V$定义为
$v \mapsto \phi(\lambda, v) \defeq \lambda \cdot v$;
    \item 对任意域$K$, 整数加法群$\mathbb{Z} = (\mathbb{Z}, +)$上不存在$K$-线性空间结构.
\end{enumerate}
\end{problem}

\begin{solution}
    
\end{solution}

\begin{problem}
    证明:在整数集合$\mathbb{Z}$上存在运算
$\mathbb{Z} \times \mathbb{Z} \to \mathbb{Z}, (a,b) \mapsto a \oplus b$,
使得$(\mathbb{Z}, \oplus)$是一个交换群, 但它与整数加法群
$(\mathbb{Z}, +)$不同构. 提示:利用$\mathbb{Q}$是可数集和上题中的问题(3).
\end{problem}

\begin{solution}
    
\end{solution}