\documentclass{../solutions-cn}

\begin{document}
\section*{第十六周作业参考解答及补充}

\subsection*{作业}

\begin{exercise}[习题4.4.1]
    设$E$是$x^4-2$在$\mathbb{Q}$上的分裂域.
    \begin{enumerate}[(1)]
        \item 试求出$E/\mathbb{Q}$的全部中间域;
        \item 试问哪些中间域是$\mathbb{Q}$的伽罗瓦扩张, 哪些域彼此共轭?
    \end{enumerate}
\end{exercise}

\begin{proof}
    \begin{enumerate}[(1)]
        \item 和\ref{ex:3.3.2}, \ref{ex:3.3.4}类似, $E = \mathbb{Q}[\sqrt[4]{2}, i]$($\zeta_4 = i$), $[E:\mathbb{Q}] = 4 \cdot 2 = 8$. 这是一个Galois扩张, 求中间域等价求$\mathrm{Gal}(E/\mathbb{Q})$的子群, 记
        \[
            \alpha:E \to E,\, \sqrt[4]{2} \mapsto \sqrt[4]{2}i,\, i \mapsto i,\, \beta:E \to E,\, \sqrt[4]{2} \mapsto \sqrt[4]{2},\, i \mapsto -i
        \]
        那么$\mathrm{Gal}(E/\mathbb{Q}) = (\alpha, \beta \mid \alpha^4, \beta^2, \alpha\beta\alpha\beta) = D_8$(\ref{ex:1.3.5}). 根据Sylow定理它有$2$阶和$4$阶子群, $2$阶的有
        \[
            \langle \beta \rangle, \langle \alpha\beta \rangle, \langle \alpha^2 \rangle, \langle \alpha^2\beta \rangle, \langle \alpha^3\beta \rangle
        \]
        $4$阶的有
        \[
            \langle \alpha \rangle, \langle \alpha^2, \beta \rangle, \langle \alpha^2, \alpha\beta \rangle
        \]
        因此共有$8$个中间域, 按上面的顺序计算不动域$E^H$依次为
        \[
            \mathbb{Q}[\sqrt[4]{2}],\, \mathbb{Q}[\sqrt[4]{2}(1 + i)],\, \mathbb{Q}[\sqrt{2}, i],\, \mathbb{Q}[\sqrt[4]{2}i],\, \mathbb{Q}[\sqrt[4]{2}(1 - i)]
        \]
        \[
            \mathbb{Q}[i],\, \mathbb{Q}[\sqrt{2}],\, \mathbb{Q}[\sqrt{2}i]
        \]
        \item $E^H/K$是有限可分扩张(\ref{3.3.6}的注记以及Galois对应), 因此和Galois扩张之间只差正规性, 根据Galois理论的基本定理, $H \lhd \mathrm{Gal}(E/K) \iff E^H/K$正规, 因此只需找出(1)中正规子群对应的不动域$E^H$:
        \[
            E^{\langle \alpha^2 \rangle} = \mathbb{Q}[\sqrt{2}, i], E^{\langle \alpha \rangle} = \mathbb{Q}[i], E^{\langle \alpha^2, \beta \rangle} = \mathbb{Q}[\sqrt{2}], E^{\langle \alpha^2, \alpha\beta \rangle} = \mathbb{Q}[\sqrt{2}i]
        \]
        共轭子群对应共轭域: $\mathbb{Q}[\sqrt[4]{2}(1 + i)]$和$\mathbb{Q}[\sqrt[4]{2}(1 - i)]$, $\mathbb{Q}[\sqrt[4]{2}]$和$\mathbb{Q}[\sqrt[4]{2}i]$. 因为$\beta\langle \alpha\beta \rangle\beta^{-1} = \langle \alpha^3\beta \rangle$, $\alpha\langle \beta \rangle\alpha^{-1} = \langle \alpha^2\beta \rangle$.
    \end{enumerate}
\end{proof}

\begin{exercise}[习题4.4.3]
    设$\xi = e^{\frac{2\pi i}{13}}, \alpha = \xi + \xi^4 + \xi^3 + \xi^{12} + \xi^9 + \xi^{10}$, 证明: 
    \begin{enumerate}[(1)]
        \item $\mathrm{Gal}\left(\mathbb{Q}[\xi]/\mathbb{Q}\right)$同构于乘法群$\mathbb{F}_{13}^* = \mathbb{F}_{13} \setminus \{0\}$.
        \item $\bigl[\mathbb{Q}[\xi]:\mathbb{Q}[\alpha]\bigr] = 6$.
        \item 求$\alpha$在$\mathbb{Q}$上的极小多项式.
    \end{enumerate}
\end{exercise}

\begin{proof}
    \begin{enumerate}[(1)]
        \item 3.4.1.
        \item 由3.4.1, 这是一个Galois扩张, $\mathbb{Q}[\alpha]$是中间域. 记$\sigma_i \in \mathrm{Gal}\left(\mathbb{Q}[\xi]/\mathbb{Q}\right)$是同构$\xi \mapsto \xi^i$. 而$\alpha$的每一项恰好是子群$\langle \sigma_4 \rangle$里的元素, 也就是说$\sigma_4(\alpha) = \alpha$. 那么根据Galois理论基本定理, $\mathbb{Q}[\xi]^{\langle \sigma_4 \rangle} = \mathbb{Q}[\alpha]$, $\bigl[\mathbb{Q}[\xi]:\mathbb{Q}[\alpha]\bigr] = |\langle \sigma_4 \rangle| = 6$.
        \item 由(2)知$\bigl[\mathbb{Q}[\alpha]:\mathbb{Q}\bigr] = 2$, 极小多项式次数为$2$, 因此计算$\alpha^2$.
        \[
        \begin{aligned}
            \alpha^2 &= 3\xi^2 + 3\xi^8 + 3\xi^6 + 3\xi^{11} + 3\xi^5 + 3\xi^7 + 2\xi^4 + 2\xi^{10} + 2\xi^3 + 2\xi + 2\xi^{12}\\
            &+ 2\xi^9 + 6\\
            &= 3(-1 - \alpha) + 2\alpha + 6\\
            &= -\alpha + 3
        \end{aligned}
        \]
        记$\beta = \xi^2 + \xi^5 + \xi^6 + \xi^7 + \xi^8 + \xi^{11}$. 这里使用了
        \[
            0 = \frac{\xi^{13} - 1}{\xi - 1}  = 1 + \xi + \cdots + \xi^{12} = \alpha + 1 + \beta.
        \]
        因此$\alpha$的极小多项式是$x^2 + x - 3$.
    \end{enumerate}
\end{proof}

\begin{exercise}[习题4.4.4]
    设$p > 2$是素数, $\xi_p = e^{\frac{2\pi i}p},\, \xi_{p^2}$为$p^2$次本原单位根.
    \begin{enumerate}[(1)]
        \item 求$\mathbb{Q}(\xi_p)/\mathbb{Q}$的扩张次数, 并证明$\mathrm{Gal}(\mathbb{Q}(\xi_p)/\mathbb{Q}) \cong F_p^*$;
        \item 求$\mathbb{Q}(\xi_{p^2})/\mathbb{Q}$的扩张次数, 并确定$\mathrm{Gal}(\mathbb{Q}(\xi_{p^2})/\mathbb{Q})$(提示: 该群是$(\mathbb{Z}/p^2\mathbb{Z})^*$);
        \item 试确定$\mathbb{Q}(\xi_{p^2})/\mathbb{Q}(\xi_p)$的扩张次数, 并证明这是一个伽罗瓦扩张.
    \end{enumerate}
\end{exercise}

\begin{proof}
    \begin{enumerate}[(1)]
        \item 3.4.1.
        \item 由4.3.3, $\mathrm{Gal}(\mathbb{Q}(\xi_{p^2})/\mathbb{Q}) \cong U(\mathbb{Z}/p^2\mathbb{Z}) = p(p - 1)$.
        \item 由(2), $\bigl[\mathbb{Q}(\xi_{p^2}):\mathbb{Q}(\xi_p)\bigr]\bigl[\mathbb{Q}(\xi_p):\mathbb{Q}\bigr] = \bigl[\mathbb{Q}(\xi_{p^2}):\mathbb{Q}\bigr] = p(p - 1)$. 从而$\bigl[\mathbb{Q}(\xi_{p^2}):\mathbb{Q}(\xi_p)\bigr] = p$. 它是Galois扩张是因为$\mathbb{Q}(\xi_{p^2})/\mathbb{Q}$是Galois扩张.
    \end{enumerate}
\end{proof}

\begin{remark}
    对于Galois扩张$L/K$, 对中间域$E$来说, $L/E$总是Galois扩张, 这是因为$L$是$K$上可分多项式$f(x)$的分裂域, 从而也能看成$f(x) \in E[x]$的分裂域.
\end{remark}

\end{document}