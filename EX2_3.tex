\subsection{教材p41-p42}

\begin{problem}
    设$F$是一个域, $F[[x]]$是系数在$F$中的形式幂级数环, 试证明: 
\begin{enumerate}[(1)]
    \item $f(x) = a_0 + a_1x + a_2x^2 + \cdots$在$F[[x]]$中可逆$\Leftrightarrow a_0 \neq 0$;
    \item $F[[x]]$中任意不可约元$p(x)$均与$x$相伴, 即$p(x) \sim x$;
    \item $F[[x]]$是主理想整环, 它是欧氏整环吗?如果是, 请写出一个欧氏映射.
\end{enumerate}
\end{problem}

\begin{proof}
    \begin{enumerate}[(1)]
        \item 按定义, 存在$g = \sum_{k = 0}^{\infty} b_kx^k$使得$fg = 1$. 则
        \[
        \begin{aligned}
            a_0b_0 &= 1,\\
            a_1b_0 + a_0b_1 &= 0,\\
            a_2b_0 + a_1b_1 + a_0b_2 &= 0,\\
            \vdots
        \end{aligned}
        \]
        因此$a_0 \neq 0$.
        
        反过来, 若$a_0 \neq 0$, 由$F$是域, $a_0$可逆. 即存在$b_0 \in F$, $a_0b_0 = 1$. 我们可以通过上面的无穷个方程组递归的解出$b_k(k \geqslant 1)$.
        \[
        \begin{aligned}
            b_1 &= -a_1b_0^2,\\
            b_2 &= -a_2b_0^2 - a_1b_1b_0,\\
            \vdots\\
            b_k &= -\sum_{i = 1}^{k} a_ib_{k - i}b_0,\\
            \vdots
        \end{aligned}
        \]
        从而存在$g = \sum_{k = 0}^{\infty} b_kx^k$为$f$的逆.
        \item 设$p(x) = \sum_{k = 0}^{\infty} a_kx^k$. $p$不可约则不是可逆元, 由(1), $a_0 = 0$, 因此$p(x) = xp_1(x)$. 又因为不可约, 且$x$不是可逆元, 因此$p_1$是可逆元, 故$p(x) \sim x$.
        \item 由(2), $\forall f(x) \in F[[x]]$, 则有唯一分解$f(x) = x^ng(x)$, 其中$g(x)$是可逆的, 即$g$的常数项非零, 也就是$a_n \neq 0$. 那么$n = \min \{i \in \mathbb{N} \mid a_i \neq 0\}$. 令$\delta(f) = n$, 这就是一个欧氏映射. 带余除法是比较显然的, 设$f_1 = x^ng_1, f_2 = x^mg_2$, 不妨设$n > m$, 由于$g_2$可逆, $f_1 = x^ng_2(g_2^{-1}g_1) = f_2(x^{n - m}g_2^{-1}g_1)$. 因此得到的结论更强, 只要$n > m$, 就有$f_2 \mid f_1$.
    \end{enumerate}
\end{proof}

\begin{remark}
    由(1)(2)知, $f(x) \in F[[x]]$, 要么是单位, 否则一定在理想$(x)$中. 这说明$(x)$是$F[[x]]$唯一的极大理想, 这种环称为局部环(local ring).
    
    请参考\cite{atiyah1994introduction}p4.
\end{remark}

\begin{problem}
    设$F$是一个域, $p(x) \in F[x]$不可约, 令$I = p(x)F[x]$表示由$p(x)$生
成的理想, 试证明: 商环$F[x]/I$是一个域, 且环同态
\[
    \varphi:F[x] \to F[x]/I,\quad f(x) \mapsto \overline{f(x)}
\]
诱导了域嵌入$\varphi|_F: F\hookrightarrow F[x]/I, a \mapsto \bar{a}$
(如果将$F$与它的像等同, 则$\bar{x} \in F[\bar{x}] \defeq F[x]/I$
是$p(x)$在扩域$F[\bar{x}]$中的一个根).
\end{problem}

\begin{proof}
    \ref{ex:2.2.6}注记的最后已经提到过, 这里再详细解释一下. 由于$F$是域, 因此$F[x]$是PID, 因此若$p(x)$是不可约的, 则$I = p(x)F[x]$是极大理想. 因为不可约元按定义在所有主理想中是极大的, 这一点可以参考\ref{ex:2.2.2}的注记, 设$p$是不可约元就能得到
    \[
        (p) \subseteq (p') \implies p' \mid p \implies p' \sim p \text{ 或 } p' \sim 1 \implies (p') = (p) \text{ 或 } (p') = (1)
    \]. 因此由\ref{ex:2.1.6}知$F[x]/I$是域.
    
    所谓的域嵌入(embbeding)在这里实际上就是单同态, 这其实就是同态复合了一下
    \[
        \begin{tikzcd}
            F \arrow[r, hook] & {F[x]} \arrow[r, two heads] & {F[x]/I}
        \end{tikzcd}
    \]
    这是域之间的同态, 因此一定是单的.
\end{proof}

\begin{problem}
    设$F$是一个域, $K \subset F$是一个子域, $f(x), g(x) \in K[x]$.
试证明: $f(x)$, $g(x)$在$K[x]$中互素$\Leftrightarrow f(x)$,
$g(x)$在$F[x]$中互素.
\end{problem}

\begin{proof}
    利用PID上满足Bézout Identity立得.
\end{proof}

\begin{problem}
    设$F$是特征零的域, $f(x) \in F[x]$不可约. 证明$f(x)$与$f'(x)$互素.
\end{problem}

\begin{proof}
    由于$0 \leqslant \deg(f') < \deg(f)$且$f$不可约, 若有非单位的公因式$d(x)$, 则$\deg(f) > \deg(f') \geqslant \deg(d) > 0$且$d(x) \mid f(x)$与不可约矛盾.

    特征零是为了排除$f' = 0$的情况.
\end{proof}

\begin{problem}
    设$\mathbb{F}_2 = \mathbb{Z}/(2) = \{\bar{0}, \bar{1}\}$
是一个二元域. 证明: 
\[
    f(x) = x^n + a_1x^{n - 1} + \cdots + a_{n - 1}x + a_n \in \mathbb{F}_2[x]
\]
没有一次因子(即不被一次多项式整除)
\(
    \Leftrightarrow a_n\left(1 + \sum_{i = 1}^n a_i\right) \neq 0.
\)
写出$\mathbb{F}_2[x]$中所有次数不超过$3$的所有不可约多项式.
\end{problem}

\begin{proof}
    $\mathbb{F}_2$只有两个一次多项式$x$和$x + 1$. 其中比较简单的是
    \[
        x \mid f(x) \iff a_0 = 0,
    \]
    另一个
    \[
        x + 1 \mid f(x) \iff f(x) = (x + 1)g(x)
    \]
    设$g(x) = x^{n - 1} + \cdots + b_{n - 1}$, 对比系数
    \[
        a_n = b_{n - 1},\, a_{n - 1} = b_{n - 1} + b_{n - 2},\, \cdots,\, a_{1} = b_1 + 1
    \]
    由于$\mathbb{F}_2$里$-1 = 1$, 因此可以得到
    \[
        b_1 = a_1 - 1 = a_1 + 1,\, b_2 = a_2 - b_1 = a_2 + a_1 + 1,\, \cdots,\, a_n = b^{n - 1} = 1 + \sum_{k = 0}^{n - 1} a_k
    \]
    因此
    \[
        x + 1 \mid f(x) \iff 1 + \sum_{k = 1}^{n} = 2a_n = 0.
    \]
    不过也可以不这么麻烦, 一次多项式对应$f(x)$的根, 所以$f(x)$无一次因子等价于$f(0) \neq 0$且$f(1) \neq 0$, 即$a_0 \neq 0$和$1 + \sum_{k = 1}^{n} a_k \neq 0$.

    次数不超过$3$的多项式只有有限个, 可以列举出来, 去掉比较明显的可约多项式
    \[
    \begin{aligned}
        &x,\, x + 1,\\
        &x^2 + 1,\, x^2 + x + 1\\
        &x^3 + 1,\, x^3 + x + 1,\, x^3 + x^2 + 1
    \end{aligned}
    \]
    注意$x^2 + 1 = x^2 - 1 = (x - 1)(x + 1) = (x + 1)^2$可约, $x^3 + 1$同理, 其余五个为不可约多项式.
\end{proof}

\begin{problem}
    设$p$是素数, $\mathbb{Z} \to \mathbb{F}_p = \mathbb{Z}/(p)\mathbb{Z},~a \mapsto \bar{a}$,
是商同态. 证明: 
\begin{enumerate}[(1)]
    \item 映射
\[
    \phi_p:\mathbb{Z}[x] \to \mathbb{F}_p[x],\quad f(x) = \sum_{i = 1}^n a_ix^i \mapsto \bar{f}(x) = \sum_{i = 1}^n \bar{a}_ix^i
\]
是环同态;
    \item 对于首项系数为$1$的多项式$f(x) \in \mathbb{Z}[x]$,
如果存在素数$p$使$\bar{f}(x)$在$\mathbb{F}_p[x]$中不可约, 
则$f(x)$在$\mathbb{Z}[x]$中也不可约.
\end{enumerate}
\end{problem}

\begin{proof}
    \begin{enumerate}[(1)]
        \item \ref{ex:2.1.8}的注记或教材引理2.3.2(我才发现教材有写延拓)
        \item 用反证法, 假设$f(x)$可约, $f(x) = g(x)h(x)$, 则$\deg(g), \deg(h) > 0$且$g, h$都是首一的. 那么根据同态有$\bar{f} = \bar{g}\bar{h}$, 且$\bar{g}$和$\bar{h}$还是首一的次数大于$0$的多项式, 这和$\bar{f}$不可约矛盾.
    \end{enumerate}
\end{proof}

\begin{problem}\label{ex:2.3.7}
    设$R, A$是两个环, $C(A) \subset A$是$A$的中心, $\psi:R \to C(A)$是一个环同态. 证明: $\forall u \in A$, 存在唯一环同态$\psi_u:R[x] \to A$满足: 
\[
    \psi_u(x) = u,\quad \psi_u(a) = \psi(a) \quad (\forall a \in R).
\]
所以, $\forall f(x) = a_nx^n + a_{n - 1}x^{n - 1} + \cdots + a_1x + a_0 \in R[x]$,
它在$\psi_u$下的像
\[
    \psi_u(f(x)) = \psi(a_n)u^n + \psi(a_{n - 1})u^{n - 1} + \cdots + \psi(a_1)u + \psi(a_0) \in A
\]
称为$f(x)$在$u \in A$的取值, 记为$f(u) \defeq \psi_u(f(x))$.
\end{problem}

\begin{proof}
    \ref{ex:2.1.8}的注记. 在这里重新阐述的详细一点. 给定环同态$\psi:R \to C(A)$, 我们可以指定一个集合的映射
    \[
        f_u:\{1\} \to A, 1 \mapsto u
    \]
    所谓的自由交换$R$-代数的泛性质是指, 对任意给定的集合映射$f_u$, 存在唯一的同态$\psi_u:R[x] \to A$使得图表交换:
    \[
        \begin{tikzcd}
            {R[x]} \arrow[rr, "\exists!\psi_u"] &                                          & A \\
                                                & \{1\} \arrow[lu, "i"] \arrow[ru, "f_u"'] &  
        \end{tikzcd}
    \]
    其中$i:R \to R[x],\, 1 \mapsto x$.

    因为$\psi(R) \subseteq C(A)$, 因此$\psi_u$才能保持乘法, 这在\ref{ex:2.1.8}的注记里已经证明. 验证了$\psi_u$是环同态就相当于证明了存在性, 而唯一性是根据定义就能得到, $\psi_u$是被给定的$\psi$和$f_u$唯一确定的.
\end{proof}

\begin{remark}
    这里$\{1\}$可以换成任意集合$S$
\[
    \begin{tikzcd}
        {R[S]} \arrow[rr, "\exists!\psi_u"] &                                      & A \\
                                            & S \arrow[lu, "i"] \arrow[ru, "f_u"'] &  
        \end{tikzcd}
\]
\end{remark}

\begin{problem}
    设$R$是一个交换环, $f(x) \in R[x]$. 证明: $f(x)$是环$R[x]$
中的零因子当且仅当存在$0 \neq r \in R$使得$r \cdot f(x) = 0$.
\end{problem}

\begin{proof}
    由于$R \subseteq R[x]$, 只需证"$\implies$"的方向.

    记$f(x) = \sum_{k = 0}^{n} a_kx^k$,设存在$g(x) = \sum_{k = 0}^{m} b_kx^k \neq 0$使得$fg = 0$, 并要求$g(x)$是次数最低的. 考虑最高次项, $a_nb_m = 0$. 那么$a_ng(x)$是一个比$g(x)$次数更小的多项式且$f(x)(a_ng(x)) = a_nf(x)g(x) = 0$. 因此$a_ng(x) = 0$, 从而$a_nb_k = 0, 0 \leqslant k \leqslant m$. 那么此时$n + m - 1$项的系数变为$a_{n - 1}b_m = 0$, 于是可以重复讨论. 根据归纳法最后得到$a_ib_m = 0, \forall i$且$b_m \neq 0$, 因此$b_mf(x) = 0$.
\end{proof}

\begin{problem}
    证明多项式$f(x) = x^4 - 10x^2 + 1$在$\mathbb{Z}[x]$中不可约, 
但是对任意的素数$p$, 它在$\mathbb{F}_p[x]$中总是可约的.
\end{problem}

\begin{proof}
    注意到$f(x)$是关于$x^2$的二次方程且有正实根$x_{1,2} =  5 \pm 2\sqrt{6}$, 而且恰好有$5 \pm 2\sqrt{6} = (\sqrt{2} \pm \sqrt{3})^2$, 记$\alpha_1 = \sqrt{2} + \sqrt{3}$, $\alpha_2 = \sqrt{2} - \sqrt{3}$,
    \[
        x^4 - 10x^2 + 1 = (x + \alpha_1)(x - \alpha_1)(x + \alpha_2)(x - \alpha_2)
    \]
    这是一个$\mathbb{R}[x]$上的唯一分解. 因此可以得到$f(x)$在$\mathbb{Z}$上不可分, 因为无论怎组合都得不到整系数的因式.

    在$\mathbb{F}_p[x]$上, 根据二次剩余的结论(见注记), 可以知道$Q_p = \{a^2 \mid a \in \mathbb{F}_p^*\}$(即所有的非零二次剩余)是一个子群. 且当$p > 2$时, $[\mathbb{F}_p^*:Q_p] = 2$. 那么对$\mathbb{F}_p^*$中任意两个元素$a, b$, $a, b, ab$中必有一个为二次剩余.

    现在将之前的因式分解做组合, 得到三种分解
    \[
    \begin{aligned}
        f(x) &= (x^2 - 5 - 2\sqrt{6})(x^2 - 5 + 2\sqrt{6})\\
        &= ((x - \sqrt{2})^2 - 3)((x + \sqrt{2})^2 - 3)\\
        &= ((x - \sqrt{3})^2 - 2)((x + \sqrt{2})^2 - 2)
    \end{aligned}
    \]
    因此$p > 3$, 取$a = 2, b = 3$, 则上面必有一种是$\mathbb{F}_p[x]$中的分解, 而$p = 2, 3$时$6 = 0$, 取第一种就行.
\end{proof}

\begin{remark}
    若$a \in \mathbb{F}_p$使得同余方程
    \[
        x^2 \equiv a \mod p
    \]
    有解, 则称$a$为一个二次剩余, 其中$0$是平凡的情形. 该方程自然有两个根$x$和$-x$. 当$p > 2$时$p$为奇数, 因此$x \neq -x$. 此时考虑平方映射
    \[
        f: \mathbb{F}_p^* \to \mathbb{F}_p^*,\quad x \mapsto x^2
    \]
    由于乘法交换, 这是一个群同态, $f(\mathbb{F}_p^*) = Q_p < \mathbb{F}_p^*$. 我们考虑映射自带的一个等价关系
    \[
        x \sim y \iff f(x) = f(y)
    \]
    等价类即为$[x] = f^{-1}(f(x)) = \{x, -x\}$. 那么就有$[\mathbb{F}_p^*:Q_p] = 2$.
\end{remark}

\begin{problem}
    设$f(x) \in \mathbb{R}(x)$是一个有理函数. 如果对任意
整数$m \in \mathbb{Z}$必有$f(m) \in \mathbb{Z}$,
试证明$f(x)$必为多项式. 这样的$f(x)$是否必为有理系数多项式?
请证明你的结论.
\end{problem}

\begin{proof}
    根据有理函数的定义, 设$f(x) = \frac{p(x)}{q(x)}$, $p, q \in \mathbb{R}[x],\, q \neq 0,\, (p, q) = 1$. 若$p = 0$结论平凡, 因此只考虑$p \neq 0$的情况.
    
    这题可以用一些分析的想法.
    \begin{remark}
        $q(x)$有整数根的时候, 会存在某个整数使得$f(m)$无定义, 因此本题题目条件默认$q(m) \neq 0,\, \forall m \in \mathbb{Z}$.
    \end{remark}
    我们利用一个简单的结论:
    \[
        \lim_{m \to \infty} f(m) = \lim_{m \to \infty} \frac{p(m)}{q(m)} =
        \begin{cases}
            0 & \deg(p) < \deg(q),\\
            \infty & \deg(p) > \deg(q),\\
            \frac{a_n}{b_n} & \deg(p) = \deg(q) = n.
        \end{cases}
    \]
    这里$a_n$和$b_n$分别是$p(x)$和$q(x)$的首项系数.

    注意$p(x)$最多只有$\deg(p)$个根, 因此最多只有$\deg(p)$个整数使得$f(m) = 0$. 那么利用$\deg(p) < \deg(q)$时的结果, 存在$N \in \mathbb{Z}_{>0}$使得任意$m > N$有$0 < |f(m)| < 1$, 和条件矛盾.

    $\deg(p) = \deg(q)$时, 若$\frac{a_n}{b_n} \notin \mathbb{Z}$, 利用极限的定义可以得到类似的矛盾(即$m$足够大时把$|f(m)|$限制在一个无整数的区间内). 而若$\frac{a_n}{b_n} \in \mathbb{Z}$, 则可以得到$m$足够大时都有$f(m) = \frac{a_n}{b_n} = k \in \mathbb{Z}$, 即
    \[
        f(m) = \frac{a_nm^n + \cdots + a_0}{b_nm^n + \cdots + b_0} = \frac{a_n}{b_n} 
    \]
    则可得$a_nb_i = a_ib_n,\, i = 0, 1, \cdots, n - 1$, 即$\frac{a_i}{b_i} = \frac{a_n}{b_n} = k$. 因此$f(x) = \frac{a_n}{b_n}$是常整数多项式.

    当$\deg(p) > \deg(q)$时只需证明$q(x) \mid p(x)$, 此时需要一些技巧. 先用带余除法令$p(x) = q(x)s(x) + r(x),\, \deg(r) < \deg(q)$, 那么$f(x) = s(x) + \frac{r(x)}{q(x)}$, 需要证明$r = 0$.
    \begin{remark}
        注意此时并不能用之前的结论, $s(m)$是否为整数并不知道.
    \end{remark}
    基本的想法是保持整数的情况下去降次, 利用差分就可以做到这一点. 考虑
    \[
        \Delta_1 f(x) = f(x + 1) - f(x) 
    \]
    若记$\deg(f) = \deg(p) - \deg(q)$, 则有$\deg(\Delta_1 f) < \deg(f)$($s(x)$会消去最高次项, $\frac{r(x)}{q(x)}$的部分指数不会增加), 且对任意整数$m$也有$\Delta_1 f(m) \in \mathbb{Z}$. 记$k = \deg(f)$, 则$\Delta_1^k f$($k$阶差分)化归为第二种情况. 而由$\Delta_1^{k - 1} f(x + 1) - \Delta_1^{k - 1} f(x) = \Delta_1^k f(x)$可知, 若$\Delta_1^k f(x)$是多项式, 则$\Delta_1^{k - 1} f(x)$也是多项式(分式项做差分无法消去), 从而反推得$f(x)$是多项式.

    因此$f(x) \in \mathbb{R}[x]$, 记$f(x) = c_nx^n + \cdots + c_0$, 任意选择$n + 1$个不同整数$m_0, m_1, \cdots, m_n$带入得到一个关于$a_0, a_1, \cdots, a_n$的整系数线性方程组$\sum_{k = 0}^{n} m_i^ka_k = b_i,\, i = 0, 1, \cdots, n$, 系数矩阵的行列式恰为Vandenmonde行列式, 因此不为零, 方程组有唯一有理数解.
\end{proof}