\subsection{教材p41-p42}

\begin{problem}
    设$F$是一个域, $F[[x]]$是系数在$F$中的形式幂级数环, 试证明:
\begin{enumerate}[(1)]
    \item $f(x) = a_0 + a_1x + a_2x^2 + \cdots$在$F[[x]]$中可逆$\Leftrightarrow a_0 \neq 0$;
    \item $F[[x]]$中任意不可约元$p(x)$均与$x$相伴, 即$p(x) \sim x$;
    \item $F[[x]]$是主理想整环, 它是欧氏整环吗?如果是, 请写出一个欧氏映射.
\end{enumerate}
\end{problem}

\begin{solution}
    \begin{enumerate}[(1)]
        \item 按定义, 存在$g = \sum_{k = 0}^{\infty} b_kx^k$使得$fg = 1$. 则
        \[
        \begin{aligned}
            a_0b_0 &= 1,\\
            a_1b_0 + a_0b_1 &= 0,\\
            a_2b_0 + a_1b_1 + a_0b_2 &= 0,\\
            \vdots
        \end{aligned}
        \]
        因此$a_0 \neq 0$.
        
        反过来, 若$a_0 \neq 0$, 由$F$是域, $a_0$可逆. 即存在$b_0 \in F$, $a_0b_0 = 1$. 我们可以通过上面的无穷个方程组递归的解出$b_k(k \geqslant 1)$.
        \[
        \begin{aligned}
            b_1 &= -a_1b_0^2,\\
            b_2 &= -a_2b_0^2 - a_1b_1b_0,\\
            \vdots\\
            b_k &= -\sum_{i = 1}^{k} a_ib_{k - i}b_0,\\
            \vdots
        \end{aligned}
        \]
        从而存在$g = \sum_{k = 0}^{\infty} b_kx^k$为$f$的逆.
        \item 设$p(x) = \sum_{k = 0}^{\infty} a_kx^k$. $p$不可约则不是可逆元, 由(1), $a_0 = 0$, 因此$p(x) = xp_1(x)$. 又因为不可约, 且$x$不是可逆元, 因此$p_1$是可逆元, 故$p(x) \sim x$.
        \item 
    \end{enumerate}
\end{solution}

\begin{problem}
    设$F$是一个域, $p(x) \in F[x]$不可约, 令$I = p(x)F[x]$表示由$p(x)$生
成的理想, 试证明:商环$F[x]/I$是一个域, 且环同态
\[
    \varphi:F[x] \to F[x]/I,\quad f(x) \mapsto \overline{f(x)}
\]
诱导了域嵌入$\varphi|_F: F\hookrightarrow F[x]/I, a \mapsto \bar{a}$
(如果将$F$与它的像等同, 则$\bar{x} \in F[\bar{x}] \defeq F[x]/I$
是$p(x)$在扩域$F[\bar{x}]$中的一个根).
\end{problem}

\begin{solution}
    
\end{solution}

\begin{problem}
    设$F$是一个域, $K \subset F$是一个子域, $f(x), g(x) \in K[x]$.
试证明:$f(x)$, $g(x)$在$K[x]$中互素$\Leftrightarrow f(x)$,
$g(x)$在$F[x]$中互素.
\end{problem}

\begin{solution}
    
\end{solution}

\begin{problem}
    设$F$是特征零的域, $f(x) \in F[x]$不可约. 证明$f(x)$与$f'(x)$互素.
\end{problem}

\begin{solution}
    
\end{solution}

\begin{problem}
    设$\mathbb{F}_2 = \mathbb{Z}/(2) = \{\bar{0}, \bar{1}\}$
是一个二元域. 证明:
\[
    f(x) = x^n + a_1x^{n - 1} + \cdots + a_{n - 1}x + a_n \in \mathbb{F}_2[x]
\]
没有一次因子(即不被一次多项式整除)
\(
    \Leftrightarrow a_n\left(1 + \sum_{i = 1}^n a_i\right) \neq 0.
\)
写出$\mathbb{F}_2[x]$中所有次数不超过$3$的所有不可约多项式.
\end{problem}

\begin{solution}
    
\end{solution}

\begin{problem}
    设$p$是素数, $\mathbb{Z} \to \mathbb{F}_p = \mathbb{Z}/(p)\mathbb{Z},~a \mapsto \bar{a}$,
是商同态. 证明:
\begin{enumerate}[(1)]
    \item 映射
\[
    \phi_p:\mathbb{Z}[x] \to \mathbb{F}_p[x],\quad f(x) = \sum_{i = 1}^n a_ix^i \mapsto \bar{f}(x) = \sum_{i = 1}^n \bar{a}_ix^i
\]
是环同态;
    \item 对于首项系数为$1$的多项式$f(x) \in \mathbb{Z}[x]$,
如果存在素数$p$使$\bar{f}(x)$在$\mathbb{F}_p[x]$中不可约, 
则$f(x)$在$\mathbb{Z}[x]$中也不可约.
\end{enumerate}
\end{problem}

\begin{solution}
    
\end{solution}

\begin{problem}
    设$R, A$是两个环, $C(A) \subset A$是$A$的中心,
$\psi:R \to C(A)$是一个环同态. 证明:$\forall u \in A$,
存在唯一环同态$\psi_u:R[x] \to A$满足:
\[
    \psi_u(x) = u,\quad \psi_u(a) = \psi(a) \quad (\forall a \in R).
\]
所以, $\forall f(x) = a_nx^n + a_{n - 1}x^{n - 1} + \cdots + a_1x + a_0 \in R[x]$,
它在$\psi_u$下的像
\[
    \psi_u(f(x)) = \psi(a_n)u^n + \psi(a_{n - 1})u^{n - 1} + \cdots + \psi(a_1)u + \psi(a_0) \in A
\]
称为$f(x)$在$u \in A$的取值, 记为$f(u) \defeq \psi_u(f(x))$.
\end{problem}

\begin{solution}
    
\end{solution}

\begin{problem}
    设$R$是一个交换环, $f(x) \in R[x]$. 证明:$f(x)$是环$R[x]$
中的零因子当且仅当存在$0 \neq r \in R$使得$r \cdot f(x) = 0$.
\end{problem}

\begin{solution}
    
\end{solution}

\begin{problem}
    证明多项式$f(x) = x^4 - 10x^2 + 1$在$\mathbb{Z}[x]$中不可约, 
但是对任意的素数$p$, 它在$\mathbb{F}_p[x]$中总是可约的.
\end{problem}

\begin{solution}
    
\end{solution}

\begin{problem}
    设$f(x) \in \mathbb{R}(x)$是一个有理函数. 如果对任意
整数$m \in \mathbb{Z}$必有$f(m) \in \mathbb{Z}$,
试证明$f(x)$必为多项式. 这样的$f(x)$是否必为有理系数多项式?
请证明你的结论.
\end{problem}

\begin{solution}
    
\end{solution}