\documentclass{../solutions-cn}

\begin{document}
\section*{第六周作业参考解答及补充}

\subsection*{作业}

\begin{exercise}[习题2.3.2]
    设$F$是一个域, $p(x) \in F[x]$不可约, 令$I = p(x)F[x]$表示由$p(x)$生成的理想, 试证明: 商环$F[x]/I$是一个域, 且环同态
    \[
        \varphi:F[x] \to F[x]/I,\quad f(x) \mapsto \overline{f(x)}
    \]
    诱导了域嵌入$\varphi|_F: F \hookrightarrow F[x]/I, a \mapsto \bar{a}$(如果将$F$与它的像等同, 则$\bar{x} \in F[\bar{x}] \defeq F[x]/I$是$p(x)$在扩域$F[\bar{x}]$中的一个根).
\end{exercise}

\begin{proof}
    2.2.6注记的最后已经提到过, 这里再详细解释一下. 由于$F$是域, 因此$F[x]$是PID, 因此若$p(x)$是不可约的, 则$I = p(x)F[x]$是极大理想. 因为不可约元按定义在所有主理想中是极大的, 这一点可以参考2.2.2的注记, 设$p$是不可约元就能得到
    \[
        (p) \subseteq (p') \implies p' \mid p \implies p' \sim p \text{ 或 } p' \sim 1 \implies (p') = (p) \text{ 或 } (p') = (1)
    \]. 因此由2.1.6知$F[x]/I$是域.
    
    所谓的域嵌入(embbeding)在这里实际上就是单同态, 这其实就是同态复合了一下
    \[
        \begin{tikzcd}
            F \arrow[r, hook] & {F[x]} \arrow[r, two heads] & {F[x]/I}
        \end{tikzcd}
    \]
    这是域之间的同态, 因此一定是单的.
\end{proof}

\begin{exercise}[习题2.3.3]
    设$F$是一个域, $K \,\red{\subseteq}\, F$是一个子域, $f(x), g(x) \in K[x]$. 试证明: $f(x)$, $g(x)$在$K[x]$中互素$\Leftrightarrow f(x)$, $g(x)$在$F[x]$中互素.
\end{exercise}

\begin{proof}
    利用PID上满足Bézout Identity立得. 注意到和2.2.1不同的是, 互素的时候两个条件等价.
\end{proof}

\begin{exercise}[习题2.3.4]
    设$F$是特征零的域, $f(x) \in F[x]$不可约. 证明$f(x)$与$f'(x)$互素.
\end{exercise}

\begin{proof}
    由于$0 \leqslant \deg(f') < \deg(f)$且$f$不可约, 若有非单位的公因式$d(x)$, 则$\deg(f) > \deg(f') \geqslant \deg(d) > 0$且$d(x) \mid f(x)$与不可约矛盾.

    特征零是为了排除$f' = 0$的情况.
\end{proof}

\begin{exercise}[习题2.3.5]
    设$\mathbb{F}_2 = \mathbb{Z}/(2) = \{\bar{0}, \bar{1}\}$是一个二元域. 证明: 
    \[
        f(x) = x^n + a_1x^{n - 1} + \cdots + a_{n - 1}x + a_n \in \mathbb{F}_2[x]
    \]
    没有一次因子(即不被一次多项式整除)
    \(
        \Leftrightarrow a_n\left(1 + \sum_{i = 1}^n a_i\right) \neq 0.
    \)
    写出$\mathbb{F}_2[x]$中所有次数不超过$3$的所有不可约多项式.
\end{exercise}

\begin{proof}
    $\mathbb{F}_2$只有两个一次多项式$x$和$x + 1$. 其中比较简单的是
    \[
        x \mid f(x) \iff a_0 = 0,
    \]
    另一个
    \[
        x + 1 \mid f(x) \iff f(x) = (x + 1)g(x)
    \]
    设$g(x) = x^{n - 1} + \cdots + b_{n - 1}$, 对比系数
    \[
        a_n = b_{n - 1},\, a_{n - 1} = b_{n - 1} + b_{n - 2},\, \cdots,\, a_{1} = b_1 + 1
    \]
    由于$\mathbb{F}_2$里$-1 = 1$, 因此可以得到
    \[
        b_1 = a_1 - 1 = a_1 + 1,\, b_2 = a_2 - b_1 = a_2 + a_1 + 1,\, \cdots,\, a_n = b^{n - 1} = 1 + \sum_{k = 0}^{n - 1} a_k
    \]
    因此
    \[
        x + 1 \mid f(x) \iff 1 + \sum_{k = 1}^{n} = 2a_n = 0.
    \]
    不过也可以不这么麻烦, 一次多项式对应$f(x)$的根, 所以$f(x)$无一次因子等价于$f(0) \neq 0$且$f(1) \neq 0$, 即$a_0 \neq 0$和$1 + \sum_{k = 1}^{n} a_k \neq 0$.

    次数不超过$3$的多项式只有有限个, 可以列举出来, 去掉比较明显的可约多项式
    \[
    \begin{aligned}
        &x,\, x + 1,\\
        &x^2 + 1,\, x^2 + x + 1\\
        &x^3 + 1,\, x^3 + x + 1,\, x^3 + x^2 + 1
    \end{aligned}
    \]
    注意$x^2 + 1 = x^2 - 1 = (x - 1)(x + 1) = (x + 1)^2$可约, $x^3 + 1$同理, 其余五个为不可约多项式.
\end{proof}

\begin{exercise}[习题2.3.6]
    设$p$是素数, $\mathbb{Z} \to \mathbb{F}_p = \mathbb{Z}/(p)\mathbb{Z},~a \mapsto \bar{a}$, 是商同态. 证明: 
    \begin{enumerate}[(1)]
        \item 映射
        \[
            \phi_p:\mathbb{Z}[x] \to \mathbb{F}_p[x],\quad f(x) = \sum_{i = 1}^n a_ix^i \mapsto \bar{f}(x) = \sum_{i = 1}^n \bar{a}_ix^i
        \]
        是环同态;
        \item 对于首项系数为$1$的多项式$f(x) \in \mathbb{Z}[x]$, 如果存在素数$p$使$\bar{f}(x)$在$\mathbb{F}_p[x]$中不可约, 则$f(x)$在$\mathbb{Z}[x]$中也不可约.
    \end{enumerate}
\end{exercise}

\begin{proof}
    \begin{enumerate}[(1)]
        \item 2.1.8的注记或教材引理2.3.2(原来教材有写延拓)
        \item 用反证法, 假设$f(x)$可约, $f(x) = g(x)h(x)$, 则$\deg(g), \deg(h) > 0$且$g, h$都是首一的. 那么根据同态有$\bar{f} = \bar{g}\bar{h}$, 且$\bar{g}$和$\bar{h}$还是首一的次数大于$0$的多项式, 这和$\bar{f}$不可约矛盾.
    \end{enumerate}
\end{proof}

\begin{exercise}[习题2.3.7]
    设$R, A$是两个环, $C(A) \,\red{\subseteq}\, A$是$A$的中心, $\psi:R \to C(A)$是一个环同态. 证明: $\forall u \in A$, 存在唯一环同态$\psi_u:R[x] \to A$满足: 
    \[
        \psi_u(x) = u,\quad \psi_u(a) = \psi(a) \quad (\forall a \in R).
    \]
    所以, $\forall f(x) = a_nx^n + a_{n - 1}x^{n - 1} + \cdots + a_1x + a_0 \in R[x]$, 它在$\psi_u$下的像
    \[
        \psi_u(f(x)) = \psi(a_n)u^n + \psi(a_{n - 1})u^{n - 1} + \cdots + \psi(a_1)u + \psi(a_0) \in A
    \]
    称为$f(x)$在$u \in A$的取值, 记为$f(u) \defeq \psi_u(f(x))$.
\end{exercise}

\begin{proof}
    2.1.8的注记. 在这里重新阐述的详细一点. 给定环同态$\psi:R \to C(A)$, 我们可以指定一个集合的映射
    \[
        f_u:\{1\} \to A, 1 \mapsto u
    \]
    所谓的自由交换$R$-代数的泛性质是指, 对任意给定的集合映射$f_u$, 存在唯一的同态$\psi_u:R[x] \to A$使得图表交换:
    \[
        \begin{tikzcd}
            {R[x]} \arrow[rr, "\exists!\psi_u"] &                                          & A \\
                                                & \{1\} \arrow[lu, "i"] \arrow[ru, "f_u"'] &  
        \end{tikzcd}
    \]
    其中$i:R \to R[x],\, 1 \mapsto x$.

    因为$\psi(R) \subseteq C(A)$, 因此$\psi_u$才能保持乘法, 这在2.1.8的注记里已经证明. 验证了$\psi_u$是环同态就相当于证明了存在性, 而唯一性是根据定义就能得到, $\psi_u$是被给定的$\psi$和$f_u$唯一确定的.
\end{proof}

\begin{remark}
    这里$\{1\}$可以换成任意集合$S$
    \[
        \begin{tikzcd}
        {R[S]} \arrow[rr, "\exists!\psi_u"] &                                      & A \\
                                            & S \arrow[lu, "i"] \arrow[ru, "f_u"'] &  
        \end{tikzcd}
    \]
    $u = (u_s)_{s \in S} \in A^S,\, f_u(s) = u_s$. (2.4.5为$S$是有限集的情形)
\end{remark}

\begin{exercise}[习题2.3.8]
    设$R$是一个交换环, $f(x) \in R[x]$. 证明: $f(x)$是环$R[x]$中的零因子当且仅当存在$0 \neq r \in R$使得$r \cdot f(x) = 0$.
\end{exercise}

\begin{proof}
    由于$R \subseteq R[x]$, 只需证"$\implies$"的方向.

    记$f(x) = \sum_{k = 0}^{n} a_kx^k$,设存在$g(x) = \sum_{k = 0}^{m} b_kx^k \neq 0$使得$fg = 0$, 并要求$g(x)$是次数最低的. 考虑最高次项, $a_nb_m = 0$. 那么$a_ng(x)$是一个比$g(x)$次数更小的多项式且$f(x)(a_ng(x)) = a_nf(x)g(x) = 0$. 因此$a_ng(x) = 0$, 从而$a_nb_k = 0, 0 \leqslant k \leqslant m$. 那么此时$n + m - 1$项的系数变为$a_{n - 1}b_m = 0$, 于是可以重复讨论. 根据归纳法最后得到$a_ib_m = 0, \forall i$且$b_m \neq 0$, 因此$b_mf(x) = 0$.
\end{proof}

\subsection*{课上的补充内容}
    無い, たぶん\dots
\end{document}