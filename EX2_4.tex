\subsection{教材p48-p49}

\begin{problem}
    设$F$是一个域, $R = F[x_1, x_2, \cdots, x_n]$,
令$R_m \subset R$表示所有$m$次齐次多项式的集合(并上零多项式).
证明:$R_m$是域$F$上的$\binom{m + n - 1}{m}$维向量空间.
\end{problem}

\begin{proof}
    
\end{proof}

\begin{problem}
    证明:$f(x_1, x_2, \cdots, x_n) \in F[x_1, x_2, \cdots, x_n]$
是$m$次齐次多项式当且仅当$f(tx_1, tx_2, \cdots, tx_n) = t^mf(x_1, x_2, \cdots, x_n)$,
($t$是一个新的不定元).
\end{problem}

\begin{proof}
    
\end{proof}

\begin{problem}
    设$F$是一个域, $K \supset F$是$F$的一个扩域, 试证明:$a\in K$是多项式
$f(x) \in F[x]$的重根$\Leftrightarrow f(a) = 0, f'(a) = 0$.
\end{problem}

\begin{proof}
    
\end{proof}

\begin{problem}
    设$F$是一个无限域, $f(x_1, x_2, \cdots ,x_n) \in F[x_1, x_2, \cdots,x_n]$
是一非零多项式. 试证明:存在$a_1, a_2, \cdots, a_n \in F$,
使$f(a_1, a_2, \cdots, a_n) \neq 0$.
\end{problem}

\begin{proof}
    
\end{proof}

\begin{problem}
    设$\psi:R \to A$是环同态,
$u = (u_1, u_2, \cdots, u_n) \in A^n$满足:
\[
    u_iu_j = u_ju_i,\quad u_i\psi(a) = \psi(a)u_i \quad (\forall a \in R, 1 \leqslant i, j \leqslant n).
\]
请直接验证取值映射$\psi_u:R[x_1, x_2, \cdots, x_n] \to A$,
\[
    f = \sum_{i_1i_2\cdots i_n} a_{i_1i_2\cdots i_n}x_1^{i_1}x_2^{i_2}\cdots x_n^{i_n}
    \mapsto \psi_u(f) \defeq \sum_{i_1i_2\cdots i_n} \psi(a_{i_1i_2\cdots i_n})u_1^{i_1}u_2^{i_2}\cdots u_n^{i_n},
\]
是一个环同态.
\end{problem}

\begin{proof}
    
\end{proof}

\begin{problem}
    设$K$是一个域, $A = \{(a_{ij})_{n \times n}|a_{ij} \in K[\lambda]\}$
是$n$阶$\lambda$-矩阵环, $u = \lambda \cdot I_n \in A$表示对角线上全为
$\lambda$的矩阵. 试证明:如果$R = M_n(K),\, \psi:R \to R$是恒等映射, 则取值映射
$\psi_u:R[x] \to A$是一个环同构.
\end{problem}

\begin{proof}
    
\end{proof}

\begin{problem}
    设$R$是一个无零因子的非交换环, $\psi:R \to R$是恒等映射. 证明存在
$u \in R$使得$\psi _u:R[x] \to R$, $f(x) \mapsto f(u)$,
不是一个映射.
\end{problem}

\begin{proof}
    
\end{proof}

\begin{problem}
    设$K$是一个域, $M_m(K)$是$m$-阶矩阵环,
$\psi:K \to M_m(K)$定义为$\psi(a) = a \cdot I_{m}$
(对角线元素为$a$的数量矩阵). 令
\[
    u = (A, B) \in M_m(K) \times M_m(K),\quad AB \neq BA,
\]
试证明$\psi_u:K[x_1, x_2] \to M_m(K),\, f(x_1, x_2) \mapsto f(A, B)$,
不是一个映射.
\end{problem}

\begin{proof}
    
\end{proof}