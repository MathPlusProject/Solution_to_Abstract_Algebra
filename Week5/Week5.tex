\documentclass{../solutions-cn}

\begin{document}
\section*{第五周作业参考解答及补充}

\subsection*{作业}

\begin{exercise}[习题2.1.3]
    证明: 只有有限个元素的整环一定是一个域.
\end{exercise}

\begin{proof}
    整环$R$有乘法消去律(习题1.1.1的(1), 证明乘法消去律事实上只需要分配律加无零因子), 而习题1.3.9告诉我们, 满足消去律的有限半群是群. 因此$(R \setminus \{0\}, \cdot)$是群, 即$R$是一个域.
\end{proof}

\begin{exercise}[习题2.1.4]
    证明: 只有有限个理想的整环是一个域.
\end{exercise}

\begin{proof}
    事实上条件可以再减弱一点, 一个Artin整环一定是域.

    设$a \neq 0$, 考虑理想降链
    \[
        (a) \supseteq (a^2) \supseteq \cdots 
    \]
    因此$\exists n \in \mathbb{Z}_{>0},\, (a^n) = (a^{n + 1})$. 即有$a^n \in (a^{n + 1})$, 那么$\exists b \in R,\, a^n = a^{n + 1}b$, 从而$ab = 1_R$.
\end{proof}

\begin{remark}
    Artin环定义为任意理想降链稳定的环, i.e. 若有理想降链
\[
    I_1 \supseteq I_2 \supseteq \cdots 
\]
则存在$n \in \mathbb{Z}_{>0}$使得$\forall m > n,\, I_m = I_n$, 也就是说从某一个$n$开始就稳定了$I_n = I_{n + 1} = \cdots$.
这个条件称为descending chain condition(d.c.c.), 与之对应的是ascending chain condtion(a.c.c.), 满足a.c.c.的正是Noether环.
\end{remark}

\begin{exercise}[习题2.1.9]
    映射$D:R[x] \longrightarrow R[x]$定义如下: 
$\forall f(x) = a_nx^n + \cdots + a_1x + a_0$,
\[
    D(f) = na_nx^{n - 1} + (n - 1)a_{n - 1}x^{n - 2} + \cdots + 2a_2x + a_1.
\]
$\forall$ $a \in R$, $f, g \in R[x]$, 试证明: 
\begin{enumerate}[(1)]
    \item $D(f + g) = D(f) + D(g)$, $D(af) = aD(f)$;
    \item $D(f \cdot g) = D(f) \cdot g + f \cdot D(g)$.
\end{enumerate}
($D(f)$称为$f(x)$的导数. 记为
$f'(x) = D(f),\, f^{(m)}(x) = \overset{m}{\widehat{D \cdots D}}(f)$
称为$f(x)$的$m$次导数).
\end{exercise}

\begin{proof}
    按定义验证. 设$f = a_nx^n + \cdots + a_1x + a_0$, $g = b_mx^m + \cdots + b_1x + b_0$.
    \begin{enumerate}[(1)]
        \item 不妨设$n \geqslant m$, 且令$b_k = 0, k > m$.
        \[
        \begin{aligned}
            D(f + g) &= D\left(\sum_{k = 0}^{n} (a_k + b_k)x^k\right) = \sum_{k = 1}^{n} k(a_k + b_k)x^{k - 1}\\ 
            &= \sum_{k = 1}^{n} ka_kx^{k - 1} + \sum_{k = 1}^{m} kb_kx^{k - 1} = D(f) + D(g).
        \end{aligned}
        \]
        \[
            D(af) = D\left(\sum_{k = 0}^{n} aa_kx^k\right) = \sum_{k = 1}^{n} kaa_kx^{k - 1} = a\sum_{k = 1}^{n} ka_kx^{k - 1} = aD(f).
        \]
        这里能把$a$提出来是因为$k$作为$k1_R$(见习题1.2.1的新增的注记), 有$ka = ak$.
        \item \[
            \begin{aligned}
                D(f \cdot g) &= D\left(\sum_{k = 0}^{n + m} \sum_{i + j = k} a_ib_jx^k\right) = \sum_{k = 1}^{n + m} \sum_{i + j = k} ka_ib_jx^{k - 1}\\
                &= \sum_{k = 1}^{n + m} \sum_{i + j = k} (i + j)a_ib_jx^{i + j - 1}\\
                &= \sum_{k = 1}^{n + m} \sum_{i + j = k} (ia_ix^{i - 1})b_jx^j + a_ix^i(jb_jx^{j - 1})\\
                &= \sum_{k = 0}^{n + m - 1} \sum_{(i - 1) + j = k} (ia_i)b_jx^{k} + a_i(jb_j)x^k\\
                &= D(f) \cdot g + f \cdot D(g).
            \end{aligned}
        \]
    \end{enumerate}
\end{proof}

\begin{exercise}[习题2.1.10]
    如果$F$是特征零的域, 则$f'(x) = 0 \Leftrightarrow \deg(f) = 0$
或$f(x) = 0$(即常数); 如果$F$的特征是$p > 0$, 则
$f'(x) = 0 \Leftrightarrow$存在$g(x) \in F[x]$使得$f(x) = g(x^p)$.
\end{exercise}

\begin{proof}
    $\mathrm{Char}(F) = 0$, 即$\forall n \in \mathbb{Z}_{>0}, n \neq 0$(这里$n$看做$n1_F$, 见习题1.2.1的新增的注记), 那么
\[
    f'(x) = na^{n - 1} + \cdots + a_1 = 0 \implies 1 \leqslant k \leqslant n, ka_k = 0 \implies 1 \leqslant k \leqslant n, a_k = 0
\]
故$f(x) = a_0$, $\deg(f) = 0$或$f = 0$, 反过来是平凡的.

    若$\mathrm{Char}(F) = p$, 则$p = 0$, 那么设$\deg(f) = n = kp + r, 0 \leqslant r < p, k \in \mathbb{N}$,
\[
\begin{aligned}
    f &= a_0 + a_1x + \cdots a_px^p + \cdots + a_{2p}x^{2p} + \cdots + a_{kp}x^{kp} + a_nx^n.\\
    \implies f' &= a_1 + \cdots + pa_px^{p - 1} + \cdots + kpa_{kp}x^{kp - 1} + \cdots na_nx^{n - 1}\\
    &= a_1 + \cdots + (p - 1)a_{p - 1}x^{p - 2} + (p + 1)a_{p + 1}x^p + \cdots (kp - 1)a_{kp - 1}x^{kp - 2}\\
    &+ (kp + 1)a_{kp + 1}x^kp + \cdots + na_nx^{n - 1}.
\end{aligned}
\]
此时$f' = 0$有$f = a_0 + a_px^p + \cdots + a_{kp}x^{kp} = g(x^p)$. 这里$g = a_0 + a_px + \cdots + a_{kp}x^k$. 反过来也是类似的.
\end{proof}

\begin{exercise}[习题2.2.1]
    设$m, n$是两个正整数, 证明它们在$\mathbb{Z}$中的最大公因数
和它们在$\mathbb{Z}[i]$中的最大公因数相同.
\end{exercise}

注意这里的相同指的在相伴的意义下相同.

\begin{proof}
    由于$U(\mathbb{Z}[i]) = \{\pm 1, \pm i\}$, 在相伴的意义下, 可以假设$(m, n)$在$\mathbb{Z}$和$\mathbb{Z}[i]$中都是正整数, 分别记为$d$和$d'$.

    那么PID上Bézout's Identity成立, 有
    \[
        d = mu + nv,\quad d' = m\alpha + n\beta.
    \]
    其中$u, v \in \mathbb{Z},\, \alpha, \beta \in \mathbb{Z}[i]$. 设$\alpha = a_1 + ia_2,\, \beta = b_1 + ib_2$, 由于我们假设的是$d' \in \mathbb{Z}_{>0}$, 故$d' = ma_1 + nb_1$, 从而$d \mid m, d \mid n \implies d \mid d'$. 反过来也有$d' \mid d$, 所以$d = d'$.
\end{proof}

\begin{exercise}[习题2.2.6]
    令$\mathbb{R}, \mathbb{C}$分别表示实数域和复数域, 试证明: 
\begin{enumerate}[(1)]
    \item 若$R$是由关于$\cos t$和$\sin t$的实系数多项式组成的函数环, 
则$R \cong \mathbb{R}[x, y]/(x^2 + y^2 - 1)$;
    \item $\mathbb{C}[x, y]/(x^2 + y^2 - 1)$是唯一分解整环(提示: 证明其为ED);
    \item $\mathbb{R}[x, y]/(x^2 + y^2 - 1)$不是唯一分解整环.
\end{enumerate}
\end{exercise}

\begin{proof}
\begin{enumerate}[(1)]
    \item 考虑同态
    \[
        \varphi: \mathbb{R}[x, y] \to R = \mathbb{R}[\cos t, \sin t],\, x \mapsto \cos t, y \mapsto \sin t,
    \]
    这自然是一个满同态, 由同态基本定理, 关键在于证明
    \[
        \ker(\varphi) = (x^2 + y^2 - 1)
    \]
    若多项式$f(x, y)$满足$\varphi(f) = f(\cos t, \sin t) = 0$, 将$f$看成是关于$y$的多项式
    \[
        f(x, y) = a_0(x) + a_1(x)y + \cdots + a_n(x)y^n,\, a_i(x) \in \mathbb{R}[x],\, 0 \leqslant i \leqslant n
    \]
    由于$x^2 + y^n - 1$关于$y$是首一的, 因此可以做带余除法, 得$f = gq + r$, 其中$r(x, y) = r_0(x) + r_1(x)y$. 带入$x = \cos t, y = \sin t$得$r(\cos t, \sin t) = 0$, 即
    \[
        r_0(\cos t) + r_1(\cos t)\sin t = 0
    \]
    做代换$t \mapsto -t$, 得
    \[
        r_0(\cos t) - r_1(\cos t)\sin t = 0
    \]
    两式相加得$r_0 = 0$, 相减得$r_1 = 0$, 从而$r = 0$. 因此$f \in (x^2 + y^2 - 1)$, 即$\ker(\varphi) \subseteq (x^2 + y^2 - 1)$. 另一方面$x^2 + y^2 - 1 \in \ker(\varphi)$, 故$\ker(\varphi) = (x^2 + y^2 - 1)$
    \item 做基变换$u = x + iy, v = x - iy$, 他有逆变换$x = \frac{u + v}{2}, y = \frac{u - v}{2i}$. 因此有同构$\mathbb{C}[u, v] \cong \mathbb{C}[x, y]$. 从而
    \[
        \mathbb{C}[x, y]/(x^2 + y^2 - 1) \cong \mathbb{C}[u, v]/(uv - 1)
    \]
    而同态
    \[
        \mathbb{C}[u, v] \to \mathbb{C}[u, u^{-1}],\, u \mapsto u, v \mapsto u^{-1}
    \]
    是满的, 且kernel是$(uv - 1)$, 证明类似于(1). 因此
    \[
        \mathbb{C}[u, v]/(uv - 1) \cong \mathbb{C}[u, u^{-1}]
    \]
    这个环称为Laurent多项式环, 这个环上可以做带余除法, 非零多项式的次数定义为最高次数$-$最低次数. 即$f = a_nu^n + a_{n + 1}u^{n + 1} + \cdots + a_mu^m, n, m \in \mathbb{Z}, n < m$的次数为$\deg(f) = m - n$.
    因此这是一个ED, 从而是UFD.
    \item 由(2), $\mathbb{C}[\cos t, \sin t]$是UFD, 用待定系数, 假设
    \[
        \cos t = (a_1\cos t + a_2\sin t + a_3)(b_1\cos t + b_2\sin t + b_3)
    \]
    其中$a_i, b_i \in \mathbb{C}, i = 1, 2, 3$. 我们要忽略掉$a_1 = b_3 = 1$其余都是$0$这种平凡的情况, 左右展开得到
    \[
        \begin{aligned}
            a_1b_1 - a_2b_2 &= 0,\\
            a_1b_2 + a_2b_1 &= 0,\\
            a_1b_1 + a_3b_3 &= 0,\\
            a_1b_3 + a_3b_1 &= 1,\\
            a_2b_3 + a_3b_2 &= 0.
        \end{aligned}
    \]
    由第一个式子得$b_1 = \frac{a_2}{a_1}b_2$, 带入第二个式子得$a_2 = \pm ia_1$, 从而$b_1 = \pm ib_2$.
    
    由一, 三又能得到$a_2b_2 = -a_3b_3$, 类似地, 带入第五个式子, 有$a_3 = \pm a_2$, $b_2 = \pm b_3$.

    再用四, 五得$a_1b_3 = a_3b_1 = \frac{1}{2}$.

    把上述关系带入
    \[
    \begin{aligned}
        \cos t &= a_1b_3(\cos t \pm i\sin t \pm i)(\pm i\cos t \pm \sin t + 1)\\
        &= \frac{1}{2}(\cos t \pm i\sin t \pm i)(\pm i\cos t \pm \sin t + 1)
    \end{aligned}
    \]
    检查正负号, 得到结果
    \[
        \cos t = \frac{1}{2}(\cos t + i\sin t - i)(i\cos t + \sin t + 1)  
    \]
    类似有
    \[
        1 - \sin t = \frac{1}{2}(\cos t + i\sin t - i)(\cos t - i\sin t + i).
    \]
    带入$-t$就是$1 + \sin t$的分解.
    
    但这种方法比较难检查等式右边的因式确实为不可约元, 我们可以利用同构$\mathbb{C}[x, y]/(x^2 + y^2 - 1) \cong \mathbb{C}[u, u^{-1}]$, 那么等式变为
    \[
        x = \frac{1}{2}(u + u^{-1}) = \frac{u^{-1}}{2}(u - i)(u + i)
    \]
    注意到$U(\mathbb{C}[u, u^{-1}]) = \mathbb{C} \cup \{u^n \mid n \in \mathbb{Z}\}$. 右边为两个都是一次的且常数项不为$0$, 容易验证不可逆(注意这里$x = \frac{1}{2}(u + u^{-1})$次数为$2$). 对$1 - \sin t$同理.
    
    因此$\cos t$和$1 \pm \sin t$无法在$\mathbb{R}[\cos t, \sin t]$中分解(分解出的系数中一定带$i$). 这样就有$\cos^2 t = \cos t\cos t = (1 - \sin t)(1 + \sin t)$. 因此不是UFD.
\end{enumerate}
\end{proof}

\begin{remark}
    (2)中若允许正次数到无穷的话, 则该环称为Laurent形式级数域(可以验证确实是一个域).

    另外, 可以说$x^2 + y^2 - 1$是单位圆的"极小多项式". 但这种说法是有些不合理的, 因为这样$a_{ij}x^iy^j$次数将定义成$i + j$, $f(x, y)$的次数定义成单项次数的最大值, 一旦这么定义就无法做带余除法, 就无法得到满足某个点集(一般是代数集, 即某些多项式的共同零点)的多项式是其极小多项式的倍数.
    
    一般地设$k$是一个域, $S \subseteq k[x_1, x_2, \cdots, x_n]$, 那么可以定义$S$中所有多项式的公共零点集
    \[
        Z(S) = \{(a_1, a_2, \cdots, a_n) \in k^n \mid \forall f \in S,\, f(a_1, a_2, \cdots a_n) = 0\}
    \]
    按定义有$S \subseteq S' \implies Z(S') \subseteq Z(S)$. 考虑$S$生成的理想$I = (S)$(见2.1.6的注记), 则有$Z(I) \subseteq Z(S)$. 另一方面, 根据$I = \left\{\sum f_ig_i \Big| f_i \in k[x_1, x_2, \cdots, x_n], g_i \in S\right\}$, 立刻得到$Z(S) \subseteq Z(I)$. 从而$Z(S) = Z(I)$. 我们称$Z(S)$这种点集为代数集(algebraic set), 用$\mathbb{A}_k^n$代替的$k^n$表示将它看作一个代数集(因为按定义$Z(\varnothing) = k^n$), 而Hilbert's Basis Theorem告诉我们$k[x_1, x_2, \cdots, x_n]$是Noether环, 所以理想都是有限生成的, 那么总有$Z(I) = Z(f_1, f_2, \cdots, f_r)$.
    
    反过来, 对$X \subseteq \mathbb{A}_k^n$, 定义
    \[
        \mathscr{I}(X) = \{f \in k[x_1, x_2, \cdots x_n] \mid \forall (a_1, a_2, \cdots a_n) \in X, f(a_1, a_2, \cdots a_n) = 0\}
    \]
    可以验证$\mathscr{I}(X)$是根理想(2.1.1的注记). 当$k$是代数闭域(algebraic closed field)时, 有一一对应
    \[
        \begin{tikzcd}
            \{\mathbb{A}_k^n \text{ 的代数集}\}  \arrow[r, "\mathscr{I}", shift left] & {\{k[x_1, x_2, \cdots x_n] \text{ 的根理想}\}} \arrow[l, "Z", shift left]
        \end{tikzcd}
    \]
    这就是Strong Nullstellensatz.

    那么(1)中$I = (x^2 + y^2 - 1)$, $Z(I) = \{(x, y) \in \mathbb{R}^2 \mid \forall f \in I,\, f(x, y) = 0\} = \{(x, y) \in \mathbb{R}^2 \mid x^2 + y^2 = 1\}$, 恰好是单位圆. 那么(1)的关键在于说明$\mathscr{I}(Z(I)) = I$. 可惜的是一般情况下这个并不成立, 比如还是在$\mathbb{R}[x, y]$上考虑, 记$J = (x^2 + y^2)$, 那么$Z(J) = {(0,0)}$, $\mathscr{I}(Z(J)) = (x, y) \neq J$. 这里$x^2 + y^2$是不可约的, 所以即使是单独一个不可约多项式也不一定可以有这个等式, $x^2 + y^2 - 1$这个不可约多项式还是比较特殊的.

    域扩张中的极小多项式和不可约是一样的, 这是由于$K[x]$是一个PID, 不可约元对应极大理想, 从而对应极小多项式.
\end{remark}

\subsection*{课上的补充内容}
\begin{additional}[Noetherian $\iff$ a.c.c.]
    $R$是诺特环当且仅当$R$满足a.c.c.

    其中a.c.c.指若有环$R$的理想升链
\[
    I_1 \subseteq I_2 \subseteq \cdots
\]
则该链必稳定, 即$\exists n \in \mathbb{Z}_{>0}$
使得$I_n$后的理想都相等, $I_n = I_{n + 1} = \cdots$.
\end{additional}
\end{document}