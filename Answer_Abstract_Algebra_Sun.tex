\documentclass[UTF8,fontset=windows]{ctexart}
\usepackage{amsthm,amssymb,mathtools,mathrsfs,tikz-cd}
\usepackage{hyperref,enumerate,tcolorbox}
\usepackage[nottoc,notlot,notlof]{tocbibind}
\usepackage[top=25mm, bottom=20mm, left=30mm, right=30mm, a4paper]{geometry}
\tcbuselibrary{breakable}
\everymath{\displaystyle}

\hypersetup{
    colorlinks = true,
    linkcolor = blue,
    citecolor = green,
    urlcolor = cyan
}

\ctexset{
    section = {
        format += {\zihao{3}},
        name = {第, 章},
        number = {\arabic{section}}
    },
    subsection={
        format += {\bfseries\raggedright\zihao{-3}},
        name = {习题,}
    }
}

\newcommand{\defeq}{\mathrel{\coloneqq}}
\newcommand{\eqdef}{\mathrel{\eqqcolon}}
\newcommand{\defequiv}{\mathrel{\vcentcolon\Leftrightarrow}}

\newtheoremstyle{dotless}{3pt}{3pt}{\songti}{2em}{\bfseries}{}{\ccwd}{}
\theoremstyle{dotless}
\newtheorem{problem}{}[subsection]

\newtheorem*{propstar}{\bfseries\heiti 命题}

\newenvironment{solution}
{\begin{tcolorbox}[colback=blue!10, colframe=blue!50, title=\textit{proof}, breakable]}
{\end{tcolorbox}}

\title{孙笑涛《抽象代数》习题解答(自制)}
\author{MathPlus}
\date{
    \today
    \\[2em]
    仅供学习交流使用, 本人对此未做出任何学术贡献。
}

\begin{document}
\maketitle
\tableofcontents

\textbf{约定:} \begin{enumerate}
    \item 由于教材中的$\subset$符号意义有些歧义, 我们统一用$\subseteq$表示子集, $\subsetneq$表示真子集, 比如\ref{ex:2.1.5}中我对符号进行了修正.
    \item 习题中也有其他错误, 对教材原文修改的地方我用\red{红色}标出
    \item 环都是有$1$的. 如果看到把$n$看作$R$的元素, 请看\ref{ex:1.2.1}最后的注记.
    \item 文中出现的教材指\cite{2022抽象代数}
\end{enumerate}

\section{群环域}
\subsection{教材p8-p9}

\begin{problem}\label{ex:1.1.1}
    设$K$是一个域, 试证明下述结论: 
    
    \begin{enumerate}[(1)]
        \item 如果$a \cdot c = b \cdot c$, $c \neq 0_K$, 则$a = b$ (乘法消去律);
        \item $\forall \, a, b\in K$, 如果$a \cdot b = 0_K$, 则$a = 0_K$或$b = 0_K$;
        \item $(a^{-1})^{-1} = a \quad (\forall \, a \in K ,\, a \neq 0_K)$;
        \item $(a \cdot b)^{-1} = a^{-1} \cdot b^{-1} \quad (a \neq 0_K ,\, b \neq 0_K)$;
        \item $(-a)^{-1} = -a^{-1} \quad (\forall a \neq 0_K)$;
        \item $\forall a \neq 0_K ,\, m, n \in \mathbb{Z}$, 则$a^{m + n} = a^m \cdot a^n ,\, a^{mn} = (a^m)^n$;
    \end{enumerate}
\end{problem}

\begin{solution}
    \begin{enumerate}[(1)]
        \item 由于$c \neq 0$, 故可在原式左右同乘$c^{-1}$, 得 
        \[ 
        \begin{aligned} 
            a \cdot c \cdot c^{-1} &= b \cdot c \cdot c^{-1}\\ 
            \implies a &= b.
        \end{aligned} 
        \]
        这告诉我们逆元的存在性强于乘法消去律, 乘法消去律已经可以保证乘法逆运算是良定的.
        这对加法也是一样的道理, 见\ref{ex:1.2.1}的(1).

        也可以用分配律得到
        \[
            a \cdot c = b \cdot c \implies (a - b) \cdot c = 0_K.
        \]
        要得到$a = b$需要使用(2), 即域$K$是没有零因子(zero-divisor)的. 由于$c \neq 0_K$,
    则$a - b = 0_K$, 即$a = b$.

        注:无零因子的非零交换环称为整环(integral domain), 见教材2.1节p23.
        \item 只需证明当$a \neq 0_K$时有$b = 0_K$, 同(1), 在等式$a \cdot b = 0_K$
    两端左乘$a^{-1}$即可.
        
        这告诉我们域$\implies$整环. 结合(1)知一个环是整环的条件已经可以推出乘法消去律.
        \item 即要证明$a^{-1}$的逆元是$a$, 这是根据定义以及逆元的唯一性得到, 教材在域, 
    环, 群三处定义下的注记都有提及. 事实上只要$a$在某一个幺半群(monoid)中关于这个运算有
    逆元, 该结论都会成立, 如\ref{ex:1.2.1}的(3).
        \item 即要证明$a \cdot b$的逆元是$a^{-1} \cdot b^{-1}$. 此处需要交换律, 因此
    验证半边逆就够了.
    \[
        (a^{-1} \cdot b^{-1}) \cdot (a \cdot b) = (a \cdot a^{-1}) \cdot (b \cdot b^{-1}) = 1_K.
    \]
    非交换的情形为$(ab)^{-1} = b^{-1}a^{-1}$, 见\ref{ex:1.3.2}.
        \item 即要证明$-a$的逆元是$-a^{-1}$. 我们用一下\ref{ex:1.2.1}的(6)
    \[
        (-a)(-a^{-1}) = aa^{-1} = 1_K, \quad (-a^{-1})(-a) = a^{-1}a = 1_K.
    \]
    这样这一条对一个环中的单位都成立.
        \item 首先需要明确定义, 教材关于$a^n$的定义并不清晰, 包括后面\ref{ex:1.2.1}中的$na$也是.
    事实上, 这种和$\mathbb{Z}$有关的东西都应该由递归定义给出, 相对应的证明要用归纳法.
        
        严格来说, 这是定义了一个映射
    \[
        \mathbb{Z} \times K^* \to K^*, (n, a) \mapsto a^n,
    \]
        这里$K^* = K \setminus \{0_K\}$(见\ref{ex:1.3.2}), 自然数的部分应由递归定义给出, 
    \[
        a^0 \defeq 1_K,\, a^{n + 1} \defeq a^n \cdot a,\, n \in \mathbb{N},
    \]
        负整数的部分定义为
    \[
        a^n \defeq (a^{-1})^{-n},\, n < 0. 
    \]
    由该定义可以验证对任意整数$n \in \mathbb{Z}$均有$a^{n + 1} = a^n \cdot a$
    以及$a^{-n} = (a^{-1})^n$, 这样在使用这两个等式的时候不用再区分正负了.

    回到原题, 对任意的$m \in \mathbb{Z}$, 先用归纳法证明$n \in \mathbb{N}$
    的情形, 负整数的情形可以结合定义得到.

    $n = 0$时根据定义左右均为$a^m$, 假设对$n$有$a^{m + n} = a^m \cdot a^n$, 根据定义有
    \[
        a^{m + n + 1} = a^{m + n} \cdot a = a^m \cdot a^n \cdot a = a^m \cdot a^{n + 1}.
    \]
    由归纳法知
    \begin{equation}
        \forall m \in \mathbb{Z}, n \in \mathbb{N} ,\, a^{m + n} = a^m \cdot a^n.
        \tag{*}
        \label{eq:1.1.1.61}
    \end{equation} 
    当$n < 0$时, 则存在$k \in \mathbb{Z}_{>0}$使得$m + kn < 0$,
    则有
    \[
    \begin{aligned}
        a^{m + n} &= a^{m + kn + (-(k - 1)n)}\\
        &\overset{\eqref{eq:1.1.1.61}}= a^{m + kn} \cdot a^{-(k - 1)n}\\
        &= (a^{-1})^{-m - kn} \cdot a^{n - kn}\\
        &\overset{\eqref{eq:1.1.1.61}}= (a^{-1})^{-m} \cdot (a^{-1})^{-kn} \cdot a^n \cdot a^{-kn}\\
        &= a^m \cdot (a^{-1})^{-kn} \cdot (a^{-1})^{-n} \cdot a^{-kn}\\
        &\overset{\eqref{eq:1.1.1.61}}= a^m \cdot (a^{-1})^{-kn - n} \cdot a^{-kn}\\
        &= a^m \cdot a^{(k + 1)n} \cdot a^{-kn}\\
        &\overset{\eqref{eq:1.1.1.61}}= a^m \cdot a^{(k + 1)n - kn} = a^m \cdot a^n.
    \end{aligned}
    \]
    这里我避免使用了乘法交换律, 这样该结论对一般的环也成立.

    同样地, 由于$a^{m(n + 1)} = a^{mn + m} = a^{mn} \cdot a^m = (a^{m})^n \cdot a^m = (a^m)^{n + 1}$,
    对$n$归纳可得
    \begin{equation}
        \forall m \in \mathbb{Z}, n \in \mathbb{N},\, a^{mn} = (a^m)^n.
        \tag{**}
        \label{eq:1.1.1.62}
    \end{equation}
    当$n < 0$时, 
    \[
    \begin{aligned}
        a^{mn} &= a^{-(m \cdot (-n))}\\
        &= (a^{-1})^{m \cdot (-n)}\\
        &\overset{\eqref{eq:1.1.1.62}}= ((a^{-1})^m)^{-n}\\
        &= (a^{-m})^{-n} = ((a^{-m})^{-1})^n
    \end{aligned}
    \]
    由于$a^{-m} \cdot a^m \overset{\eqref{eq:1.1.1.61}}= a^0 = 1_K$, 即括号内确实为$a^m$,
    故上式等于$(a^m)^n$.
    \end{enumerate}
\end{solution}

\begin{problem}
    设$K$是一个域, 证明:$K$的任意一组子域(可以无限多个)的交集仍是子域.
如果$K_i \subset K \, (i \in \mathbb{N})$是满足条件
$K_i \subseteq K_{i+1} \, (i \in \mathbb{N})$的子域, 则它们的并集也是
$K$的子域.
\end{problem}

\begin{solution}
    令
\(
    F = \bigcap_i K_i
\)
    由子域定义, 需要验证
\[
\begin{aligned}
    &\forall a, b \in F ,\, a - b \in F\\
    &\forall a, b \in F^* ,\, ab^{-1} \in F^*,\, F^* = F \setminus \{0\}.
\end{aligned}
\]
由于$K_i$均为子域, 且$a, b \in F \subseteq K_i$, 故
\[
    \forall i \in \mathbb{N}, \, a - b \in K_i.
\]
因此
\[
    a - b \in \bigcap_i K_i = F.
\]
$F^*$的部分同理, 故$F$为子域.

若还满足$\forall i \in \mathbb{N} ,\, K_i \subseteq K_{i + 1}$,
令
\(
    L = \bigcup_i K_i
\),
如果$a, b \in L$, 则存在$K_i$和$K_j$使得$a \in K_i ,\, b \in K_j$
记$r = \max(i, j)$, 则$a, b \in K_r$. 由于$K_r$为子域, 可得
\[
    a - b \in K_r \subseteq L.
\]
$L^*$同理, 故$L$为子域.
\end{solution}

\begin{problem}
    令$\mathbb{Q}[\sqrt{2}, \sqrt{3}]$表示$\mathbb{C}$中包含
$\mathbb{Q}, \sqrt{2}, \sqrt{3}$的最小子域, 证明
$\mathbb{Q}[\sqrt{2}, \sqrt{3}] = \mathbb{Q}[\sqrt{2} + \sqrt{3}]$.
\end{problem}

\begin{solution}
    该题本应该是域扩张的题, 此处我们只用定义来证明.

    由于$\sqrt{2} + \sqrt{3} \in \mathbb{Q}[\sqrt{2}, \sqrt{3}]$,
我们有$\mathbb{Q}[\sqrt{2} + \sqrt{3}] \subseteq \mathbb{Q}[\sqrt{2}, \sqrt{3}]$.
反过来, 
\(
    \frac{1}{\sqrt{2} + \sqrt{3}} = \sqrt{3} - \sqrt{2} \in \mathbb{Q}[\sqrt{2} + \sqrt{3}]
\),
故有
\(
    \sqrt{2} = \frac{(\sqrt{2} + \sqrt{3}) - (\sqrt{3} - \sqrt{2})}{2} \in \mathbb{Q}[\sqrt{2} + \sqrt{3}]
\),
\(
    \sqrt{3} = \frac{(\sqrt{2} + \sqrt{3}) + (\sqrt{3} - \sqrt{2})}{2} \in \mathbb{Q}[\sqrt{2} + \sqrt{3}]
\).
因此
\(
    \mathbb{Q}[\sqrt{2}, \sqrt{3}] \subseteq \mathbb{Q}[\sqrt{2} + \sqrt{3}]
\).
故两者相等.

注: 证明过程给出了一个$\mathbb{Q}$-线性空间的基变换实际上, 从而两者将同构(见\ref{1.4.9}).
\end{solution}

\begin{problem}\label{ex:1.1.4}
    设$\mathbb{N}$是所有正整数的集合, $\mathbb{Q}$是有理数域.因$\mathbb{Q}$是可数集, 
故存在双射$f:\mathbb{N} \to \mathbb{Q}$. 令$f^{-1}:\mathbb{Q} \to \mathbb{N}$表示$f$的
逆映射, 利用有理数的加法和乘法, 可通过双射$f$定义$\mathbb{N}$上的运算如下:
$\forall n, m \in \mathbb{N}$, 
\[
    n \oplus m = f^{-1}(f(n) + f(m)), \quad n \star m = f^{-1}(f(n)f(m)),
\]
试证明:$\mathbb{N} = (\mathbb{N}, \oplus, \star)$是域, 并求它的零元和单位元.
\end{problem}

\begin{solution}
    验证域公理, 加法交换律和乘法交换律易得.

    结合律:$\forall n, m, l \in \mathbb{N}$,
\[
\begin{aligned}
    (n \oplus m) \oplus l &= f^{-1}\biggl(f\bigl(f^{-1}(f(n) + f(m))\bigr)+ f(l)\biggr)\\
    &= f^{-1}(f(n) + f(m) + f(l)) \\
    &\overset{!}= n \oplus (m \oplus l);\\
    (n \star m) \star l &= f^{-1}\biggl(f\bigl(f^{-1}(f(n)f(m))\bigr) \cdot f(l)\biggr)\\
    &= f^{-1}(f(n)f(m)f(l)) \\
    &= n \star (m \star l).
\end{aligned}
\]
其中!处是因为计算出来的结果关于$n, m, l$是轮换对称的, 后面同理.

    零元为$f^{-1}(0)$: $\forall n \in \mathbb{N}$,
\[
\begin{aligned}
    n \oplus f^{-1}(0) &= f^{-1}(f(n) + f(f^{-1}(0)))\\
    &= f^{-1}(f(n) + 0)\\
    &= f^{-1}(f(n)) = n.
\end{aligned}
\]
    $n$的负元为$f^{-1}(-f(n))$:
\[
\begin{aligned}
    n \oplus f^{-1}(-f(n)) &= f^{-1}\biggl(f(n) + f\bigl(f^{-1}(-f(n))\bigr)\biggr)\\
    &= f^{-1}(f(n) - f(n))\\
    &= f^{-1}(0).
\end{aligned}
\]
    单位元为$f^{-1}(1)$: $\forall n \in \mathbb{N}$,
\[
\begin{aligned}
    n \star f^{-1}(1) &= f^{-1}(f(n) \cdot f(f^{-1}(1)))\\
    &= f^{-1}(f(n)) = n.
\end{aligned}
\]
    $n$的逆元为$f^{-1}(\frac1{f(n)})$:
\[
\begin{aligned}
    n \star f^{-1}(\frac{1}{f(n)}) &= f^{-1}\biggl(f(n) \cdot f\bigl(f^{-1}(\frac{1}{f(n)})\bigr)\biggr)\\
    &= f^{-1}(f(n) \cdot \frac{1}{f(n)})\\
    &= f^{-1}(1).
\end{aligned}
\]
    分配律:$\forall n, m, l \in \mathbb{N}$,
\[
\begin{aligned}
    n \star (m \oplus l) &= f^{-1}\biggl(f(n) \cdot f\bigl(f^{-1}(f(m) + f(l))\bigr)\biggr)\\
    &= f^{-1}\bigl(f(n) \cdot (f(m) + f(l))\bigr)\\
    &= f^{-1}\bigl(f(n)f(m) + f(n)f(l)\bigr)\\
    &= f^{-1}(f(n)f(m)) \oplus f^{-1}(f(n)f(l))\\
    &= n \star m \oplus n \star l.
\end{aligned}
\]
\end{solution}

\begin{problem}\label{ex:1.1.5}
    证明:在域的定义中, 加法的交换律可以由其他条件推
出. 提示:按两种方式展开 $(1 + 1) \cdot (a + b)$.
\end{problem}

\begin{solution}
    一方面
\[
\begin{aligned}
    (1 + 1) \cdot (a + b) &= 1 \cdot (a + b) + 1 \cdot (a + b)
    &= a + b + a + b;
\end{aligned}    
\]
    另一方面
\[
\begin{aligned}
    (1 + 1) \cdot (a + b) &= (1 + 1) \cdot a + (1 + 1) \cdot b\\
    &= a + a + b + b.
\end{aligned}
\]
故有$a + b + a + b = a + a + b + b$.
消去两端的一个$a$和一个$b$即得加法交换律.
\end{solution}

\begin{problem}
    设$p > 2$是素数,
$\mathbb{F}_p = \{\bar{0}, \bar{1}, \bar{2}, \cdots, \overline{p-1}\}$
是$\mathbb{Z}$的模$p$剩余类域. 试计算:
\begin{enumerate}[(1)]
    \item $\bar{2}$在$\mathbb{F}_p$中的逆元$\bar{2}^{-1}$;
    \item $\overline{p - 1} \cdot \overline{p - 2}$;
    \item $\overline{p - 2}$在$\mathbb{F}_p$中的逆元$\overline{p-2}^{-1}$.
\end{enumerate}
\end{problem}

\begin{solution}
\begin{enumerate}[(1)]
    \item 只需找到能被$2$整除的$1 + kp(k \in \mathbb{Z})$.
由于素数$p > 2$, $p + 1$即可. i.e. $\overline{2}^{-1} = \overline{\frac12(p + 1)}$.
    \item $\overline{p - 1} \cdot \overline{p - 2} = \overline{-1} \cdot \overline{-2} = \overline{2}$.
    \item 由(1), $\overline{p - 2}^{-1} = \overline{-2}^{-1} = \overline{-\frac12(p + 1)} = \overline{\frac12(p - 1)}$.
\end{enumerate}
\end{solution}
    
\subsection{教材p13-p14}

\begin{problem}\label{ex:1.2.1}
    设$R$是一个环, 试证明下述结论:
    \begin{enumerate}[(1)]
        \item (加法消去律)\quad 如果$a + c = b + c$, 则$a = b$;
        \item $\forall a \in R$, 有$a \cdot 0_R = 0_R$;
        \item $-(-a) = a,\quad a(b - c) = ab - ac \quad (\forall a, b, c \in R)$;
        \item $-(a + b) = (-a) + (-b) \quad (\forall a, b \in R)$;
        \item $a(-b) = (-a)b = -(ab) \quad (\forall a, b \in R)$;
        \item $(-a)(-b) = ab \quad (\forall a, b \in R)$;
        \item $\forall a \in R, m, n \in \mathbb{Z}$, 有$(m + n)a = ma + na, (mn)a = m(na)$;
        \item $\forall a, b \in R, n \in \mathbb{Z}$, 有$n(a + b) = na + nb, n(ab) = a(nb)$;
        \item $\forall a, b \in R, m, n \in \mathbb{Z}$, 有$(ma) \cdot (nb) = mn(a \cdot b) = (mna) \cdot b$;
        \item (二项式定理)\quad $\forall a, b \in R$,设$ab = ba$, $n$是正整数, 则
        \[
            (a + b)^n = \sum_{i = 0}^n \binom{n}{i} a^{n - i}b^i.
        \]
    \end{enumerate}
\end{problem}

\begin{proof}
    \begin{enumerate}[(1)]
        \item 两边同加$-c$.
        \item 由于
        \[
            a \cdot 0_R = a \cdot (0_R + 0_R) = a \cdot 0_R + a \cdot 0_R.
        \]
        再用一下负元消去即可, $0_R \cdot a = 0_R$同理.

        \begin{remark}
            这里需要用到: 分配律, 零元定义, 负元存在. 与之对比, $0_R \cdot 0_R = 0_R$只需要用到分配律, 零元和单位元. 因此在半环(semiring)中(2)是不成立的, 但仍有$0_R \cdot 0_R = 0_R$, 这里半环要求$0$和$1$存在.
        \end{remark}

        \item 前一个为负元定义(教材p9的注记); 后一个先由分配律,
        \[
            a(b - c) = ab + a(-c),
        \]
        又由于
        \[
            a(-c) + ac = a(c + (-c)) = a \cdot 0_R \overset{(2)}= 0
        \]
        得$a(-c) = -ac$, 这也是(5)的证明. 这里要注意仅使用$-a \overset{(*)}= -1_R \cdot a$也无法将负号提到前面, 需要$R$是交换环或者说明$-1_R \cdot a = a \cdot (-1_R) = -a$.

        $(*)$的证明如下
        \[
            -1_R \cdot a + a = -1_R \cdot a + 1_R \cdot a = (-1_R + 1_R) \cdot a = 0_R \cdot a \overset{(2)}= 0_R.
        \]
        右乘$-1_R$同理.
        \item 利用$-a = -1_R \cdot a$和分配律展开即可.
        \item 见(3).
        \item (3)和(5)的推论.
        \item 参考\ref{ex:1.1.1}的(6), 明确定义: 
        \[
            0a \defeq 0_R,\, (n + 1)a \defeq na + a,\, n \in \mathbb{N}
        \]
        以及
        \[
            na \defeq -((-n)a),\, n < 0.
        \]
        一样的, 可以验证对任意整数$n \in \mathbb{Z}$都有$(n + 1)a = na + a$和$na = -((-n)a)$. 先对$n$归纳得
        \begin{equation}
            (m + n)a = ma + na, \quad \forall m \in \mathbb{Z}, n \in \mathbb{N}
            \tag{i}
            \label{eq:1.2.1.7}
        \end{equation}
        然后$n < 0$, 存在$k\in \mathbb{Z}_{>0}$使得$m + kn < 0$,
        \[
        \begin{aligned}
            (m + n)a &= (m + kn - (k - 1)n)a\\
            &\overset{\eqref{eq:1.2.1.7}}= (m + kn)a + (-(k - 1)n)a\\
            &= -(-m - kn)a + (n - kn)a\\
            &\overset{\eqref{eq:1.2.1.7}}= -((-m)a + (-kn)a) + na + (-kn)a\\
            &\overset{(4)}= ma + (kn)a + na + (-kn)a = ma + na.
        \end{aligned}
        \]
        第二个式子可直接利用第一个证明, $m = 0$根据定义左右均为$0_R$, $m > 0$有,
        \[
        \begin{aligned}
            (mn)a &= \left(\sum_{i = 1}^{m} n\right)a\\
            &= \sum_{i = 1}^{m} (na)\\
            &= m(na).
        \end{aligned}
        \]
        $m < 0$利用$mn = (-m)(-n)$, 做同样的操作.
        \item 对$n$归纳, 由于加法有交换律,
        \[
        \begin{aligned}
            (n + 1)(a + b) &= n(a + b) + a + b\\
            &= na + nb + a + b\\
            &= (n + 1)a + (n + 1)b.
        \end{aligned}
        \]
        得
        \[
            n(a + b) = na + nb, \quad \forall n \in \mathbb{N}
        \]
        当$n < 0$有
        \[
            n(a + b) = -(-n(a + b)) = -((-n)a + (-n)b) \overset{(4)}= na + nb.
        \]
        第二个等式使用分配律, $n = 0$根据定义左右均为$0_R$, $n > 0$,
        \[
        n(ab) = \sum_{i = 1}^{n} ab = a\sum_{i = 1}^{n} b = a(nb).
        \]
        $n < 0$, 用$n = -(-n)$, $n(ab) = -a((-n))b \overset{(5)}= a(nb)$. 同样的也会有$n(ab) = (na)b$.
        \item (7)和(8)的推论,
        \[
        \begin{aligned}
            (ma) \cdot (nb) &\overset{(8)}= m(a \cdot (nb))\\
            &\overset{(8)}= m(n(ab))\\
            &\overset{(7)}= mn(ab)\\
            &\overset{(8)}= (mna) \cdot b.
        \end{aligned}
        \]
        \item 对$n$归纳,
        \[
        \begin{aligned}
            (a + b)^n \cdot (a + b) &= \left(\sum_{i = 0}^{n} \binom{n}{i} a^{n - i}b^i\right) \cdot (a + b)\\
            &= \sum_{i = 0}^{n} \binom{n}{i} a^{n - i}b^ia + \sum_{i = 0}^{n} \binom{n}{i} a^{n - i}b^{i + 1}\\
            &\overset{ab = ba}= \sum_{i = 0}^{n} \binom{n}{i} a^{n - i + 1}b^i + \sum_{i = 0}^{n} \binom{n}{i} a^{n - i}b^{i + 1}\\
            &= a^{n + 1} + \sum_{i = 1}^{n} \left(\binom{n}{i} + \binom{n}{i - 1}\right) a^{n - i + 1}b^{i} + b^{n + 1}\\
            &= \sum_{i = 0}^{n + 1} \binom{n + 1}{i} a^{n + 1 - i}b^i.
        \end{aligned}
        \]
    \end{enumerate}
\end{proof}

\begin{remark}
    (7)-(9)中实际上需要用归纳法证明的只有
    \[
    \begin{aligned}
        n(a + b) &= na + nb,\\
        (m + n)a &= ma + na,\\
        (mn)a &= m(na),\\
    \end{aligned}
    \]
    这三条加上$1a = a$, 是在说任何一个Abel群都是$\mathbb{Z}$-模(\ref{ex:5.1.4}), 因为这几条的证明过程并未用到$R$的乘法, 把$R$换成任意的Abel群也是对的. 再反过来看\ref{ex:1.1.1}的(6), 加上$(ab)^n = a^nb^n$, 也是在说$K^*$是$\mathbb{Z}$-模, 证明过程中用到了$K^*$关于域的乘法是Abel群.
    
    另一方面, 可以先定义
    \[
        N: \mathbb{Z} \to R,\quad n \mapsto n1_R
    \]
    这是一个自然的环同态(使用归纳法证明)
    \[
    \begin{aligned}
        N(m + n) &= N(m) + N(n);\\
        N(mn) &= N(m) \cdot N(n).
    \end{aligned}
    \]
    然后利用这个环同态得到(注意用到的$n(ab) = (na)b$的证明是直接使用分配律的, 因此不存在循环论证. $N$表示使用了这个环同态, $dis$表示使用了分配律, $ass$表示使用了结合律):
    \[
    \begin{aligned}
        n(a + b) &= n(1_R(a + b)) = (n1_R)(a + b) \overset{dis}= (n1_R)a + (n1_R)b = na + nb.\\
        (m + n)a &= (m + n)(1_Ra) = ((m + n)1_R)a \overset{N}= (m1_R + n1_R)a \overset{dis}= (m1_R)a + (n1_R)a\\
        &= ma + na.\\
        (mn)a &= (mn)(1_Ra) = (mn1_R)a \overset{N}= (m1_Rn1_R)a \overset{ass}= (m1_R)((n1_R)a)\\
        &= (m1_R)(na) = m(1_R(na)) = m(na).
    \end{aligned}
    \]
    这个同态是唯一的, 因为我们要求环同态要把$1$映到$1$, 因此$\mathbb{Z}$在$\mathsf{Ring}$中是始对象(initial object), $\mathsf{Ring}$表示环范畴. 因此$n$可看作是$R$中的元素$n1_R$. 所以此后在没有歧义的情况下, 默认$0$就指零元, $1$指幺元.
\end{remark}

\begin{problem}
    假设集合$R$上有两个运算, 除加法的交换律外满足环的所有其他公理. 利用分配律证明: 加法是交换的 (从而$R$是环).
\end{problem}

\begin{proof}
    这和\ref{ex:1.1.5}是一道题.
\end{proof}

\begin{problem}\label{ex:1.2.3}
    设$X$是集合, $P(X)$表示$X$的所有子集形成的集合, 在$P(X)$上定义“加法”和“乘法”: $A + B = A \cup B - A \cap B$, $A \cdot B = A \cap B$. 证明: 在这些运算下$P(X)$是一个环, 且$2A = 0 (\forall A \in P(X))$.
\end{problem}

\begin{proof}
    这里$A + B$为对称差, $A + B = A \cup B - A \cap B = (A - B) \cup (B - A)$. 用$A^c$表示$A$的补集. 那么,
    \[
        A + B = (A \cap B^c) \cup (A^c \cap B).
    \]
    \begin{enumerate}[(i)]
        \item $(P(X), +)$是Abel群. 交换律由定义是显然的.
        
        结合律:
        \[
        \begin{aligned}
            (A + B) + C &= (((A \cap B^c) \cup (A^c \cap B)) \cap C^c)\\
            &\cup (((A \cap B^c) \cup (A^c \cap B))^c \cap C)\\
            &= (A \cap B^c \cap C^c) \cup (A^c \cap B \cap C^c) \cup (A^c \cap B^c \cap C)\\
            &\cup (A \cap B \cap C)\\
            &= A + (B + C). \quad \text{(轮换对称, 见\ref{ex:1.1.4}的结合律证明)}
        \end{aligned}
        \]
        零元为$\varnothing$,
        \[
            A + \varnothing = \varnothing + A = A \cup \varnothing - A \cap \varnothing = A.
        \]
        负元为$A$本身,
        \[
            A + A = A \cup A - A \cap A = A - A = \varnothing.
        \]
        即$2A = 0$.
        \item $(P(X), \cdot)$是(交换)幺半群, 单位元是$X$. 由于$\cdot$就是交集$\cap$, 因此这一点是显然的.
        \item 分配律:
        \[
        \begin{aligned}
            (A + B) \cdot C &= ((A \cap B^c) \cup (A^c \cap B)) \cap C\\
            &= (A \cap B^c \cap C) \cup (A^c \cap B \cap C)\\
            A \cdot C + B \cdot C &= (A \cap C \cap (B \cap C)^c) \cup ((A \cap C)^c \cap B \cap C)\\
            &= (A \cap B^c \cap C) \cup (A^c \cap B \cap C).
        \end{aligned}
        \]
        故有$(A + B) \cdot C = A \cdot C + B \cdot C$. 另一部分证明类似.
    \end{enumerate}
    因此$(P(X), +, \cdot)$为一个(交换)环.
\end{proof}

\begin{problem}\label{ex:1.2.4}
    设$R$是一个环, $S \,\red{\subseteq}\, R$是一个非空子集合. 试证明
    \[
        C(S) \defeq \{a \in R \mid ax = xa, \forall x \in S \}
    \]
    是$R$的一个子环.
\end{problem}

\begin{proof}
    该子环称为子集$S$的中心化子(centralizer). 当$S = R$时就是中心(\ref{ex:2.1.11}).

    $\forall a, b \in C(S)$, 需要验证
    \[
        a - b \in C(S), \quad ab \in C(S), \quad 1 \in C(S).
    \]\
    其中$1 \in C(S)$是显然的. 对$\forall x \in S$
    \[
    \begin{gathered}
        (a - b)x = ax + bx = xa + xb = x(a - b),\\
        (ab)x = a(bx) = a(xb) = (ax)b = (xa)b = x(ab).\\
    \end{gathered}
    \]
    因此$a - b, ab \in C(S)$, $C(S)$是子环.
\end{proof}

\begin{problem}\label{ex:1.2.5}
    证明:如果在环$R$中$1 - ab$可逆, 则$1 - ba$也可逆.
\end{problem}

\begin{proof}
    设$1 - ab$的逆为$c$, 考虑形式级数
    \[
        (1 - x)^{-1} = \sum_{i = 0}^{+\infty} x^i
    \]
    则有
    \[
    \begin{aligned}
        (1 - ba)^{-1} &= \sum_{i = 0}^{+\infty} (ba)^i\\
        &= 1 + b\left(\sum_{i = 0}^{+\infty} (ab)^i\right)a\\
        &= 1 + b(1 - ab)^{-1}\\
        &= 1 + bca.
    \end{aligned} 
    \]
    验证$1 + bca$确实是$1 - ba$的逆: 
    \[
    \begin{aligned}
        (1 - ba)(1 + bca) &= 1 - ba + bca - b(abc)a\\
        &= 1 - ba + bca -b(c - 1)a\\
        &= 1 - ba + bca -bca + ba = 1\\
        (1 + bca)(1 - ba) &= 1 + bca - ba - b(cab)a\\
        &= 1 + bca - ba - b(c - 1)a\\
        &= 1.
    \end{aligned}
    \]
\end{proof}

\begin{remark}
    使用形式级数是合理的, 从\ref{ex:2.3.1}可以看到形式级数环是有定义的, 且和多项式环一样是可以赋值的(\ref{ex:2.3.7}).
\end{remark}

\begin{problem}\label{ex:1.2.6}
    如果环$R$满足条件:$\forall x \in R,\quad x^2 = x$,证明$R$是交换环.
\end{problem}

\begin{proof}
    条件$x^2 = x$称为乘法是幂等(idempotent)的. 考虑
    \[
        (x + 1)^2 = x^2 + 2x + 1 = x + 1,
    \]
    或者直接带入$-x$, 得
    \[
        -x = x^2 = x.
    \]
    再考虑
    \[
        (x + y)^2 = x^2 + xy + yx + y^2 = x + xy + yx + y = x + y,
    \]
    得
    \[
        xy = -yx = yx.
    \]
\end{proof}

\begin{problem}[华罗庚恒等式]
    设$a, b$是环$R$中的元素. 如果$a, b, ab - 1$可逆, 证明$a - b^{-1}$, $(a - b^{-1})^{-1} - a^{-1}$也可逆, 且有下列恒等式:
    \[
        \left((a - b^{-1})^{-1} - a^{-1}\right)^{-1} = aba - a.
    \]
\end{problem}

\begin{proof}
    由于$a, b, ab - 1$均可逆, 即$a, b, ab - 1 \in U(R)$. $U(R)$为环$R$的单位群(\ref{ex:1.3.2}). 故
    \[
        a - b^{-1} = (ab - 1)b^{-1} \in U(R),
    \]
    那么只需证明华罗庚恒等式. 直接验证即可. 由\ref{ex:1.2.5}, $(ba - 1)^{-1} = b(ab - 1)^{-1}a - 1$以及\ref{ex:1.3.2}证明的(1).
    \[
    \begin{aligned}
        \left((a - b^{-1})^{-1} - a^{-1}\right)^{-1} &= \left(((ab - 1)b^{-1})^{-1} - a^{-1}\right)^{-1}\\
        &= (b(ab - 1)^{-1} - a^{-1})^{-1}\\
        &= \left((b(ab - 1)^{-1}a - 1)a^{-1}\right)^{-1}\\
        &= a(b(ab - 1)^{-1}a - 1)^{-1}\\
        &= a(ba - 1)\\
        &= aba - a.
    \end{aligned}
    \]
\end{proof}

\begin{problem}[多项式矩阵的带余除法]\label{ex:1.2.8}
    设$A \in M_n(K)$是一个给定的$n$阶矩阵. 对任意多项式矩阵$A(x) \in M_{n \times m}(K[x])$, 证明存在唯一的$B(x) \in M_{n \times m}(K[x])$, $R \in M_{n \times m}(K)$使得$A(x) = (xI_n - A)B(x) + R$.
\end{problem}

\begin{proof}
    先证唯一性, 若存在$B'(x) \in M_{n \times m}(K[x])$和$R' \in M_{n \times m}(K)$也满足条件, 则有
    \[
        (xI_n - A)(B(x) - B'(x)) = R' - R \in M_{n \times m}(K).
    \]
    设
    \[
        B(x) - B'(x) = B_0 + B_1x + B_2x^2 + \cdots + B_kx^k, \quad B_i \in M_{n \times m}(K), 0 \leqslant i \leqslant k.
    \]
    将左边展开得
    \[
    \begin{aligned}
        B_k &= 0,\\
        -AB_k + B_{k - 1} &= 0 \implies B_{k - 1} = 0,\\
        -AB_{k - 1} + B_{k - 2} &= 0 \implies B_{k - 2} = 0,\\
        \vdots\\
        -AB_1 + B_0 &= 0 \implies B_0 = 0,\\
        -AB_0 &= R' - R = 0.
    \end{aligned}
    \]
    再证存在性, 将$A(x)$写成多项式的形式,
    \[
        A(x) = A_0 + A_1x + A_2x^2 + \cdots + A_kx^k, \quad A_i \in M_{n \times m}(K), 0 \leqslant i \leqslant k.
    \]
    我们对$k$归纳, $k = 0$时, $A(x) = A_0$为常数矩阵, 取$B(x) = O_{n \times m}$(零矩阵), $R = A_0$即可.
    
    假设对任意$k$次多项式$A(x)$有$B(x) \in M_{n \times m}(K[x])$, $R \in M_{n \times m}(K)$使得$A(x) = (xI_n - A)B(x) + R$. 考查$k + 1$的情形:
    \[
    \begin{aligned}
        A(x) &= A_0 + x(A_1 + A_2x + \cdots + A_{k + 1}x^k)\\
        &= A_0 + x((xI_n - A)\tilde{B}(x) + \tilde{R})\\
        &= (xI_n - A)x\tilde{B}(x) + xI_n\tilde{R} - A\tilde{R} + A\tilde{R}+ A_0\\
        &= (xI_n - A)(x\tilde{B}(x) + \tilde{R}) + A\tilde{R} + A_0.
    \end{aligned}
    \]
    取$B(x) = x\tilde{B}(x) + \tilde{R} \in M_{n \times m}(K[x])$, $R = A\tilde{R} + A_0$即可.
\end{proof}

\begin{problem}\label{ex:1.2.9}
    设$m > 0$是任意整数, $\red{\mathbb{Z}/m\mathbb{Z}} = \{\overline{0}, \overline{1}, \cdots, \overline{m-1}\}$是$\mathbb{Z}$的模$m$剩余类环. 试证明: $\overline{a} \in \red{\mathbb{Z}/m\mathbb{Z}}$可逆当且仅当$(a, m) = 1$(即: $a$与$m$互素).
\end{problem}

\begin{proof}
    $\overline{a} \in \mathbb{Z}/m\mathbb{Z}$可逆,
    \[
    \begin{aligned}
        &\iff \exists \overline{b} \in \mathbb{Z}, \quad \overline{a}\overline{b} = \overline{1}\\
        &\iff ab = 1 + km, \quad k \in \mathbb{Z},\\
        &\iff (a, m) = 1. \quad \text{(Bézout's Identity)}
    \end{aligned}
    \]
\end{proof}

\begin{remark}
    一般用记号$\mathbb{Z}/m\mathbb{Z}$表示模$m$剩余类环(理想和商环, 教材2.1节p25). 若$(a, m) = 1$, 则$\overline{a}$是加法群$(\mathbb{Z}/m\mathbb{Z}, +)$的生成元, 即$\overline{a}$(在加法群)的阶(教材1.3节, p17)是$m$.
\end{remark}

\begin{problem}
    设$R$是仅有$n$个元素的环, 试证明对任意$a \in R$有
    \[
        na \defeq \underbrace{a + a + \cdots + a}_n = 0.
    \]
\end{problem}
    
\begin{proof}
    该题的证明归结为一句话, 加法群的阶$(R, +)$为$n$, 故$na = 0$.
\end{proof}

\begin{remark}
    有限群$G$内任一元素$a$, 有$|a| \Big| |G|$(教材4.1节p70推论4.1.3), 因此必有$a^{|G|} = e$, 在这道题就是$na = 0$.
\end{remark}

\begin{problem}
    环$R$中非零元$x$称为幂零元(nilpotent), 若存在$n > 0$使$x^n = 0$. 证明:
    \begin{enumerate}[(1)]
        \item 如果$x$是幂零元, 则$1 - x$是可逆元;
        \item $\mathbb{Z}/m\mathbb{Z}$有幂零元当且仅当$m$可以被一个大于$1$的整数的平方整除.
    \end{enumerate}
\end{problem}

\begin{proof}
    \begin{enumerate}[(1)]
        \item 注意到
        \[
            1 = 1 - x^n = (1 - x)(1 + x + x^2 + \cdots + x^{n - 1})
        \]
        \item "$\Rightarrow$": 若$\mathbb{Z}/m\mathbb{Z}$有幂零元$\overline{a}$, 则存在$n > 1(a \neq 0)$使得$\overline{a}^n = \overline{a^n} = \overline{0}$. 即$m \mid a^n$. 取素数$p \mid m$, 则$p \mid a^n$, 故$p \mid a$. 因此, 若$m$的素因数分解为$m = p_1^{e_1}p_2^{e_2}\cdots p_r^{e_r}$, 其中$p_1, p_2, \cdots, p_r$为互异的素数, $e_1, e_2, \cdots, e_r \geqslant 1$, 则$p_i \mid a,\, 1 \leqslant i \leqslant r$, 故有$p_1p_2\cdots p_r \mid a$. 因此有$p_1p_2\cdots p_r \leqslant a < m$, 故必有某个$e_i > 1$, 即$\exists 1 \leqslant i \leqslant r$, $e_i \geqslant 2$, 这样$p_i^2 \mid m$.
        
        "$\Leftarrow$": 反过来, 若$m$可以被某个大于$1$的平方整除, 则上述$e_i$中必有一个大于$1$, 此时取$a = p_1p_2\cdots p_r$, $\overline{a}$为$\mathbb{Z}/m\mathbb{Z}$的幂零元.
    \end{enumerate}
\end{proof}

\begin{problem}
    设$R$是一个环, 如果$(xy)^2 = x^2y^2 (\forall x, y \in R)$, 则$R$是交换环.
\end{problem}

\begin{proof}
    先考虑
    \[
        ((x + 1)y)^2 = (x + 1)^2y^2 \implies\, xy^2 = yxy,
    \]
    再将上式中$y$换成$y + 1$,
    \[
        x(y + 1)^2 = (y + 1)x(y + 1) \implies xy = yx.
    \]
\end{proof}

\begin{problem}
    如果环$R$满足条件: $x^6 = x (\forall x \in R)$. 证明:
    \begin{enumerate}[(1)]
        \item $x^2 = x (\forall x \in R)$;
        \item $R$是一个交换环.
    \end{enumerate}
\end{problem}

\begin{proof}
    \begin{enumerate}[(1)]
        \item 先带入$-x$,
        \[
            -x = (-x)^6 = x^6 = x \implies 2x = 0.
        \]
        考虑$(x + 1)^6$,
        \[
            (x + 1)^6 = x^6 + 6x^5 + 15x^4 + 20x^3 + 15x^2 + 6x + 1 = x + 1,
        \]
        得到
        \[
            6x^5 + 15x^4 + 20x^3 + 15x^2 + 6x = 0.
        \]
        利用$2x = 0$消去含$2x$的项得
        \[
            x^4 + x^2 = 0.
        \]
        两边乘$x^2$得
        \[
            x + x^4 = 0.
        \]
        再相减得$x^2 = x$.
        \item 由(1)和\ref{ex:1.2.6}.
    \end{enumerate}
\end{proof}
\subsection{教材p17-p18}

\begin{problem}\label{ex:1.3.1}
    设$G$是一个群, 对于任意的$a, b \in G$, 证明$ab$的阶和$ba$的阶相等.
\end{problem}

\begin{proof}
    若$|ab| = n < \infty$, 则
    \[
        (ba)^n = b \cdot (ab)^n \cdot b^{-1} = bb^{-1} = e.
    \]
    且对$1 \leqslant k < n$, $(ba)^k = b(ab)^kb^{-1} \neq e$. 因此$|ba| = n$. 反之亦然.
    
    若$|ab| = \infty$, 则
    \[
        \forall n \in \mathbb{Z}_{\geqslant 1}, \quad (ba)^n = b(ab)^nb^{-1} \neq e.
    \]
    故$|ba| = \infty$. 反之亦然.
\end{proof}

\begin{remark}
    事实上, 群$G$内$g$和$h = aga^{-1}$阶相等. $h$称为$g$的一个共轭(conjugate, 教材p77).
    \[
        \sigma_a: G \to G, \quad g \mapsto aga^{-1}
    \]
    是群$G$的一个自同构. 而对一般的群同态$\varphi: G \to G'$, $|g| < \infty \implies |\varphi(g)| < \infty$且$|\varphi(g)| \big| |g|$. 因此若$\varphi$为同构, 则$|g| = |\varphi(g)|$(包括左右为无穷的情况).
\end{remark}

\begin{problem}\label{ex:1.3.2}
    设$R$是一个环, $U(R)$表示$R$中所有可逆元集合, 试证明: $U(R)$关于环$R$的乘法是一个群(称为$R$的单位群).
\end{problem}

\begin{proof}
    \begin{enumerate}[(1)]
        \item 这里首先需要验证运算的封闭性, $\forall a, b \in U(R)$, 有$(b^{-1}a^{-1})(ab) = b^{-1}(a^{-1}a)b = 1$, 故$ab \in U(R)$且$(ab)^{-1} = b^{-1}a^{-1}$.
        \item $1 \in U(R)$, 因为$1 \cdot 1 = 1$的确可逆;
        \item 由于乘法是$R$上的乘法, 故结合律成立;
        \item 若$a \in U(R)$, 则由\ref{ex:1.1.1}的(3), $a^{-1} \in U(R)$且$(a^{-1})^{-1} = a$;
    \end{enumerate}
\end{proof}

\begin{remark}
    一般$U(R)$也记作$R^\times$, 比如$K$是域时, $K^\times = K^* =  K \setminus \{0\}$.
\end{remark}

\begin{problem}
    证明除了单位元之外所有元素的阶都是$2$的群一定是交换群.
\end{problem}

\begin{proof}
    由于任意$a^2 = e$, 故$a = a^{-1}$.

    考虑
    \[
        (ab)^2 = e \implies ab = b^{-1}a^{-1} = ba.
    \]
    或直接验证
    \[
        ab = ab \cdot (ba)^2 = abbaba = ba
    \]
\end{proof}

\begin{problem}
    令$C(\mathbb{R} ) = \left\{\text{所有连续函数: } \mathbb{R} \overset{f}\to \mathbb{R} \right\}$, $\forall f ,\, g \in C(\mathbb{R})$,
    \[
        f + g \in C(\mathbb{R}),\quad f \cdot g \in C(\mathbb{R})
    \]
    定义: $\forall x \in \mathbb{R}, (f + g)(x) = f(x) + g(x), (f \cdot g)(x) = f(g(x))$, 证明$(C(\mathbb{R}), +)$是交换群. $(C(\mathbb{R}), +, \cdot)$是否为环?
\end{problem}

\begin{proof}
    $(C(\mathbb{R}), +)$的零元为零函数$\mathbf{0}: \mathbb{R} \to \mathbb{R},\, x \mapsto 0$, $(f + \mathbf{0})(x) = f(x) + 0 = f(x) = 0 + f(x) = (\mathbf{0} + f)(x),\, \forall x \in \mathbb{R}$.

    $f \in C(\mathbb{R})$的负元为$-f: \mathbb{R} \to \mathbb{R},\, x \mapsto -f(x)$, $(f + (-f))(x) = ((-f) + f)(x) = f(x) - f(x) = 0 = \mathbf{0}(x)$.

    由于$f + g$为逐点定义, 故交换律和结合律依赖于$\mathbb{R}$的加法, 是平凡的. 故$(C(\mathbb{R}), +)$是Abel群.

    若$f$不是$\mathbb{R}$-线性函数, 如$f(x) = x^2$, 则$(f \cdot (g + h))(x) = f((g + h)(x)) = f(g(x) + h(x)) \neq f(g(x)) + f(h(x))$. 故$C(\mathbb{R}, +, \cdot)$不是环.
\end{proof}

\begin{problem}\label{ex:1.3.5}
    写出对称群$S_3$的乘法表.
\end{problem}

\begin{proof}
    记$\mathrm{id}_{S_3} = e$, 令$a = (1\:2)$, $b = (1\:2\:3)$, 有$a^2 = e$, $b^3 = e$, $abab = e \iff ba = ab^2$. 乘法表如下:
    \[
    \begin{array}{c|cccccc}
             & e    & a    & b   & b^2  & ab   & ab^2 \\
        \hline
        e    & e    & a    & b   & b^2  & ab   & ab^2 \\
        a    & a    & e    & ab  & ab^2 & b^2  & b \\
        b    & b    & ab^2 & b^2 & e    & a    & ab \\
        b^2  & b^2  & ab   & e   & b    & ab^2 & a \\
        ab   & ab   & b^2  & a   & ab^2 & e    & b \\
        ab^2 & ab^2 & b    & ab  & a    & b^2  & e \\
    \end{array}
    \]
\end{proof}

\begin{remark}
    可以看到$S_3$, 若取$a = (1\:2),\, b = (1\:2\:3)$, 则$S_3$可以由$a, b$生成, 即考虑所有可能的乘积, 一般可以表示为$S_3 = \langle a, b \rangle,\, a = (1\:2),\, b = (1\:2\:3)$.
    
    若不给$a, b$加任何限制, 便得到一个自由群(free group)$F(\{a, b\})$. 一般地, 任意一个集合$A$都可以生成一个自由群$F(A)$, $A$就是生成元组成的集合. 可以证明任何一个群都同构于某个自由群的商群, 而对应的正规子群便是由生成元满足的某些关系确定(将$A$看成字母表, $\Sigma_A$表示单词的集合, 这些关系可以表示为一些满足$w = e$单词$w \in \Sigma_A$). 把这些$w$组成的集合记为$\mathscr{R}$, $A$和$\mathscr{R}$将唯一确定一个群$G$, $(A \mid \mathscr{R})$称为$G$的一个展示(presentation). 以$S_3$为例, $S_3$的一个展示为$(\{a, b\} \mid a^2, b^3, abab)$. 另外有二面体群(Dihedral Groups)$D_{2n} = (a, b \mid a^2, b^n, abab)$
    
    由于这本教材没有讲自由群, 所以想要了解的话需要查阅别的教材.(可参考\cite{aluffi2009algebra}II.\S5和II.\S8.2)
    
    BTW, 这本教材和很多教材一样, 会把集合$A$对称群$S_A$上的乘法写成$f \cdot g \defeq f \circ g$, 这个其实会有一点不舒服. 正常我们习惯于说: 映射$f:X \to Y$和$g:Y \to Z$的复合是$g \circ f$. 这在范畴的定义也是习惯于这样, 复合会写成这样:
    \[
        \mathrm{Hom}_{\mathcal{C}}(X, Y) \times \mathrm{Hom}_{\mathcal{C}}(Y, Z) \to \mathrm{Hom}_{\mathcal{C}}(X, Z),\, (f, g) \mapsto g \circ f.
    \]
    这样说的好处在于一眼能感觉出这个运算是不交换的. 当然这只是个人感觉, 也有可能是我先入为主了, 因为我最开始接触到的范畴里的复合是这样写的. 如果引入范畴的记号, $S_A$会记作$\mathrm{Aut}_{\mathsf{Set}}(A)$, 其中$\mathsf{Set}$表示集合范畴. 那么$S_A$上的乘法按范畴的定义来写应该是:
    \[
        S_A \times S_A \to S_A,\quad (f, g) \mapsto f \cdot g \defeq g \circ f
    \]
    可以看到和$f \cdot g \defeq f \circ g$刚好是反过来的. 没有使用范畴语言的话就还好, 不会出现前后不自洽的问题, 但如果介绍了范畴语言, 那应该注意$S_A$上乘法的定义要和范畴定义不能冲突, 这一点\cite{lang2012algebra}和\cite{hungerford2003algebra}就做的很好. 它的范畴定义故意反了过来, 它写成$\mathrm{Hom}_{\mathcal{C}}(Y, Z) \times \mathrm{Hom}_{\mathcal{C}}(X, Y) \to \mathrm{Hom}_{\mathcal{C}}(X, Z)$.
    
    那么哪一个才对呢, 事实上都是对的, 你总能验证$S_A$确实时一个群. 原因在于, 当你只考虑所有的同构时, 就得到一个子范畴, 这是一个群胚(groupoid), 它是一个自反范畴, 所以顺序就没区别了. 但我个人认为还是统一一下比较好, 主要是复合是非交换的, $f \circ g$和$g \circ f$一般不等. 为了方便还是按照教材为准吧, 使用$f \cdot g = f \circ g$.(尽管我是有点不习惯的)
\end{remark}

\begin{problem}
    证明: 一个群$G$不会是两个真子群(不等于$G$的子群)的并.
\end{problem}

\begin{proof}
    反证, 假设$H_1, H_2 \lvertneqq G$且$G = H_1 \cup H_2$, 则$\exists h_1 \in G \setminus H_2 \subseteq H_1, h_2 \in G \setminus H_1 \subseteq H_2$, 有$h_1h_2 \in G = H_1 \cup H_2$, 矛盾. (不妨设$h_1h_2 \in H_1 \implies h_2 \in H_1$)
\end{proof}

\ref{ex:1.3.7}-\ref{ex:1.3.9}为群的其他三种定义.

\begin{problem}\label{ex:1.3.7}
    一个非空集合$G$带有满足结合律的“乘法”运算, 我们称之为半群. 如果$G$是一个半群, 且满足如下性质:
    \begin{enumerate}[(1)]
        \item $G$含有右单位元$1_r$(即: $a \cdot 1_r = a$, $\forall a \in G)$;
        \item $G$中的每个元素$a$有右逆(即: 存在$b \in G$, 使得$a \cdot b = 1_r)$.
    \end{enumerate}
    试证明: $G$是一个群.
\end{problem}

\begin{proof}
    先证右逆为逆,
    \[
    \begin{gathered}
        \forall a \in G \, \exists b \in G, ab = 1_r,\\
        \implies \exists c \in G, bc = 1_r,\\
        \implies ba = (ba)1_r = (ba)(bc) = b(ab)c = b1_rc = bc = 1_r.
    \end{gathered}
    \]
    再证右单位为单位,
    \[
        1_ra = (ab)a = a(ba) = a1_r = a.
    \]
\end{proof}

\begin{problem}\label{ex:1.3.8}
    证明: 半群$G$是群的充要条件是: $\forall a, b \in G$, $ax = b$和$ya = b$都有(唯一)解.
\end{problem}

\begin{proof}
\begin{enumerate}[(1)]
    \item "$\impliedby$": 取定一个$a \in G$, 方程$ax = a$的解设为$e_a$. 对$\forall b \in G$, 方程$ya = b$有解$y_b$, 则有
    \[
        be_a = (y_ba)e_a = y_b(ae_a) = y_ba = b.
    \]
    即$e_a$是$G$的右单位, 记为$1_r$, 又因为$\forall a \in G$, 方程$ax = 1_r$有解, 即$a$有右逆, 由\ref{ex:1.3.7}知$G$是群.
    \item "$\implies$": 若$G$是群, 则方程$ax = b$的唯一解为$a^{-1}b$, 方程$ya = b$的唯一解为$ba^{-1}$.
\end{enumerate}
    
\end{proof}

\begin{problem}\label{ex:1.3.9}
    证明:
    \begin{enumerate}[(1)]
        \item 在群中左右消去律都成立: 如果$ax = ay$, 则$x = y$; 如果$xa = ya$,则$x = y$.
        \item 左右消去律都成立的有限半群一定是群.
    \end{enumerate}
\end{problem}

\begin{proof}
    设$G = \{a_1, a_2, \cdots a_n\}$. 对$\forall 1 \leqslant i, j \leqslant n$,
    \[
        a_ia_1, a_ia_2, \cdots, a_ia_n
    \]
    互异, 否则存在$a_k \neq a_l$使得$a_ia_k = a_ia_l$, 由消去律得$a_k = a_l$矛盾. 因此$\exists 1 \leqslant t \leqslant n$, $a_ia_t = a_j$, 即方程$a_ix = a_j$有解. 同理方程$ya_i = a_j$也有解, 由\ref{ex:1.3.8}, $G$是群.
\end{proof}

\begin{problem}\label{ex:1.3.10}
    证明:偶数阶有限群$G$中必有$2$阶元.
\end{problem}

\begin{proof}
    设$|G| = 2n$. 对$e \neq g \in G$, $|g| = 2 \iff g = g^{-1}$. 定义$G$上的一个等价关系
    \[
        g \sim g' \iff g = g' \lor g' = g^{-1}.
    \]
    考虑商集$G/\sim = \{\overline{g} \mid g \in G\}$, 用$\#S$表示集合$S$的元素个数(基数)防止混淆. 若$|g| = 2$或$g = e$, 则$\#\, \overline{g} = 1$, 否则$\#\, \overline{g} = 2$. 因此若$m$为$G$中阶为$2$的元素的个数, 则$2n = m + 1 + 2(\#\, (G/\sim) - m - 1)$, 故$2n - m - 1$为偶数, 因此$m > 0$.
\end{proof}

\begin{remark}
    当然可以用Sylow定理一步到位.
\end{remark}

\begin{problem}
    证明:$GL_2(\mathbb{R})$中的元素
    \(
        x = \begin{pmatrix}
            0 & 1\\
            -1 & 0
        \end{pmatrix},
        y = \begin{pmatrix}
            0 & 1\\
            -1 & -1
        \end{pmatrix}
    \)
    的阶分别是$4$和$3$. 但$xy$是无限阶元.
\end{problem}

\begin{proof}
    用$I_n$表示$n$阶单位阵, 计算可得
    \[
        x^2 = \begin{pmatrix}
            -1 & 0 \\
            0 & -1
        \end{pmatrix},
        x^3 = \begin{pmatrix}
            0 & -1 \\
            1 & 0
        \end{pmatrix},
        x^4 = \begin{pmatrix}
            1 & 0 \\
            0 & 1
        \end{pmatrix} = I_2.
    \]
    故$|x| = 4$, 同理,
    \[
        y^2 = \begin{pmatrix}
            -1 & -1 \\
            1 & 0
        \end{pmatrix},
        y^3 = \begin{pmatrix}
            1 & 0 \\
            0 & 1
        \end{pmatrix} = I_2.
    \]
    $|y| = 3$. 最后是$xy$,
    \[
        xy = \begin{pmatrix}
            -1 & -1 \\
            0 & -1
        \end{pmatrix},
        (xy)^2 = \begin{pmatrix}
            1 & 2 \\
            0 & 1
        \end{pmatrix},
        (xy)^3 = \begin{pmatrix}
            -1 & -3 \\
            0 & -1
        \end{pmatrix}, \cdots
    \]
    可以用归纳法证明
    \[
        (xy)^n = (-1)^n
        \begin{pmatrix}
            1 & n \\
            0 & 1
        \end{pmatrix}
        \neq I_2, \forall n \in \mathbb{Z}_{\geqslant 1}.
    \]
    故$|xy| = \infty$.
\end{proof}
    
\begin{problem}\label{ex:1.3.12}
    证明群的任意多个子群的交仍是子群.
\end{problem}

\begin{proof}
    设$G$是群, 记$I$为指标集, $H_i < G,\, \forall \in I$. 验证\(H = \displaystyle\bigcap_{i \in I} H_i < G\): 首先$e_G \in H$, $H \neq \varnothing$,
    \[
    \begin{gathered}
        \forall a, b \in H = \bigcap_{i \in I} H_i \implies \forall i \in I,\,a, b \in H_i\\
        \implies ab^{-1} \in H_i, \quad \forall i \in I\\
        \implies ab^{-1} \in \bigcap_{i \in I} H_i = H.
    \end{gathered}
    \]
\end{proof}

\begin{remark}
    教材中并未提及这个判断子群的命题, 但其实是最常用的.

    \begin{propstar}[子群的判定]
        设$G$是一个群, $\varnothing \neq S \subseteq G$, 则$S < G$($S$是$G$的子群的记号)当且仅当
        \[
            \forall a, b \in S \iff ab^{-1} \in S.
        \]
    \end{propstar}
    证明可参考\cite{aluffi2009algebra}p79.
\end{remark}
\subsection{教材p21-p22}

\begin{problem}\label{ex:1.4.1}
    设$\varphi:G \to G'$是群同态, 试证明:
\begin{enumerate}[(1)]
    \item $\ker(\varphi) \defeq \{g \in G \mid \varphi(g) = e'\}$ $(e' \in G'$表示的单位元)是$G$的子群(称
为群同态$\varphi$的核);
    \item \[\varphi(G) = \{\varphi(g) \mid \forall g \in G\} \subset G'\]
是$G'$的子群(称为群同态$\varphi$的像).
\end{enumerate}
\end{problem}

\begin{solution}
    教材命题1.4.1的(1)(5)直接使用.
\begin{enumerate}[(1)]
    \item $e \in \ker(\varphi)$非空, 直接验证\[
    \begin{gathered}
        \forall a, b \in \ker(\varphi),\, \varphi(ab^{-1}) = \varphi(a)\varphi(b)^{-1} = e'e' = e'\\
        \implies ab^{-1} \in \ker(\varphi).
    \end{gathered}
    \]
    \item $e' \in \varphi(G)$非空, 直接验证\[
    \begin{gathered}
        \forall x, y \in \varphi(G),\, \exists a, b \in G,\, x = \varphi(a), y = \varphi(b)\\
        \implies xy^{-1} = \varphi(a)(\varphi(b))^{-1} = \varphi(a)\varphi(b^{-1}) = \varphi(ab^{-1}) \in \varphi(G).
    \end{gathered}
    \]
\end{enumerate}
\end{solution}

\begin{problem}
    令$G$是函数$f(x) = \frac1x, g(x) = \frac{x-1}x$关于函数的合成生成的一个群
(即群乘法为函数合成), 证明$G$同构于$S_3$.
\end{problem}

\begin{solution}
    由\ref{ex:1.3.5}的注记, 只需验证$f^2 = \mathrm{id}, g^3 = \mathrm{id}, fgfg = \mathrm{id}$.
\[
\begin{aligned}
    f^2(x) &= f(f(x)) = \frac{1}{\frac{1}{x}} = x.\\
    g^2(x) &= g(g(x)) = 1 - \frac{1}{(1 - \frac{1}{x})} = -\frac{1}{x - 1}\\
    g^3(x) &= g(g^2(x)) = 1 - (\frac{1}{-\frac{1}{x - 1}}) = 1 + x - 1 = x.\\
    (fg)(x) &= f(g(x)) = \frac{x}{x - 1},\\
    (fgfg)(x) &= (fg)^2(x) = 1 + \frac{1}{\frac{x}{x - 1} - 1} = 1 + x - 1 = x.
\end{aligned}
\]

\end{solution}

\begin{problem}
    设$R \overset{\varphi}\to R'$是环同态, 证明集合
$ker(\varphi) = \{x \in R \mid \varphi(x) = 0_{R'}\}$满足:
\begin{enumerate}[(1)]
    \item $\ker(\varphi)$是$(R, +)$的子群;
    \item $\forall a \in \ker (\varphi), x \in R$有$ax \in \ker(\varphi)$, $xa \in \ker (\varphi)$.
$(\ker(\varphi)$称为环同态$\varphi$的核.)
\end{enumerate}
\end{problem}

\begin{solution}
\begin{enumerate}[(1)]
    \item 即\ref{ex:1.4.1}(1);
    \item 直接验证\[
        \forall a \in \ker(\varphi),\, x \in R,\, \varphi(xa) = \varphi(x)\varphi(a) = \varphi(x)0_{R'} = 0_{R'}
    \]
    另一半同理.
\end{enumerate}
注:满足(1)(2)的$R$的子集称为$R$的一个理想(ideal), 教材p25定义2.1.4.
\end{solution}

\begin{problem}
    设$K$是一个域, $\phi:K[x] \to K[x]$是$K$的多项式环之间的环自同态. 
如果对于任意的$k \in K, \phi(k) = k$, 试证明:$\phi$是满同态的充分必要条件是存在
$a, b \in K(a \neq 0)$使得$\phi(x) = ax + b$.
\end{problem}

\begin{solution}
\begin{enumerate}[(1)]
    \item "$\implies$": 记$f(x) = \phi(x)$, 若$\phi$是满的, 则存在$g(x) \in K[x]$
使得$\phi(g(x)) = x$, 则$x = \phi(g(x)) \overset{!}= g(\phi(x)) = g(f(x))$,
!处是根据环同态的定义以及$\phi(k) = k,\, \forall k \in K$得到.
考查次数$1 = \deg(g(f(x))) = \deg(g) \cdot \deg(f)$(域没有零因子). 因此$\deg(f) = \deg(g) = 1$,
i.e. $\phi(x) = f(x) = ax + b,\, \exists a \neq 0, b \in K$.
    \item "$\impliedby$": 若存在$a \neq 0, b \in K$使得$\phi(x) = ax + b$,
则令$y = ax + b$得到$x = a^{-1}(y - b)$. 那么对任意的$f(x) \in K[x]$,
存在$g(x) = f(a^{-1}(x - b)) \in K[x]$使得$\phi(g(x)) = g(\phi(x)) = g(y) = f(a^{-1}(y - b)) = f(x)$.
\end{enumerate}
\end{solution}

\begin{problem}
    证明实数的加法群$(\mathbb{R}, +)$和正实数的乘法群$(\mathbb{R}_{>0}, \cdot)$同构.
\end{problem}

\begin{solution}
    注意到$f: \mathbb{R} \to \mathbb{R}_{>0},\, x \mapsto e^x$
是同构. $f^{-1}(x) = \ln x$.

注: 事实上, 由$f(x + y) = f(x)f(y)$并利用归纳法和同态定义可以直接推出
$f(x) = a^x,\, a = f(1),\, x \in \mathbb{Q}$, 若有连续性则可以延拓到$\mathbb{R}$上.
\end{solution}

\begin{problem}
    证明有理数的加法群$(\mathbb{Q}, +)$和正有理数的乘法群$(\mathbb{Q}_{>0}, \cdot)$不同构.
\end{problem}

\begin{solution}
    反证, 假设存在同构$f: \mathbb{Q} \to \mathbb{Q}_{>0}$,
则设$2 = f(a) = f(\frac{a}{2} + \frac{a}{2}) = f(\frac{a}{2}) \cdot f(\frac{a}{2}) = f(\frac{a}{2})^2$
矛盾.
\end{solution}

\begin{problem}
    证明有理数域$\mathbb{Q}$和实数域$\mathbb{R}$的自同构都只有恒等映射.
\end{problem}

\begin{solution}
    不妨设$\sigma: \mathbb{Q} \to \mathbb{Q}$是同构, 根据定义,
有$\sigma(0) = 0, \sigma(1) = 1, \sigma(-a) = -\sigma(a), \sigma(a^{-1}) = (\sigma(a))^{-1}$.
因此先用归纳法得到$\sigma|_{\mathbb{N}} = \mathrm{id}_{\mathbb{N}}$, 用负元延拓到$\mathbb{Z}$,
再用逆元延拓到$\mathbb{Q}$得$\sigma = \mathrm{id}_{\mathbb{Q}}$. 事实上, 这个推导对于
任何特征$0$的域都是对的, 即$\mathbb{Q}$是特征$0$最小域(环的特征见教材2.1节p27定义2.1.5).

    对$\mathbb{R}$, 首先若$\phi: \mathbb{R} \to \mathbb{R}$是同构, 有上面可知
$\phi|_{\mathbb{Q}} = \mathrm{id}_{\mathbb{Q}}$. 另外, 可以证明$\phi$保序结构,
即$x \geqslant 0 \implies \phi(x) \geqslant 0$. 这是因为对$x > 0$总有
$\phi(x) = \phi(\sqrt{x} \cdot \sqrt{x}) = \phi(\sqrt{x})^2 > 0$. 保序则保极限,
即对单调有界有理数列$\{q_n\}_{n \in \mathbb{N}}$有
$\lim_{n \to \infty} \phi(q_n) = \lim_{n \to \infty} q_n$(实际上保序就可以保持
$\mathbb{R}$上的拓扑结构, $\phi$是连续的). 由于$\mathbb{Q}$在$R$中稠密,
从而$\phi = \mathrm{id}_{\mathbb{R}}$.

    一般情况下子域的自同构是不一定能延拓到扩域上, 比如考虑$\mathbb{Q}(\sqrt{2})$
的共轭自同构(类似复共轭, $\sqrt{2} \mapsto -\sqrt{2}$), 它不能延拓到$\mathbb{R}$上.

    综上可得,
$\mathrm{Aut}_{\mathsf{Ring}}(\mathbb{R}) = \mathrm{Aut}_{\mathbb{Q}}(\mathbb{R})$
是平凡群.(由于$\mathbb{R}/\mathbb{Q}$并不是Galois扩张, 因此没有用符号
$\mathrm{Gal}(\mathbb{R}/\mathbb{Q})$, 另外$\mathsf{Ring}$表示环范畴)
\end{solution}

\begin{problem}
    证明:$\mathbb{Q}[\sqrt 2] = \{a + b\sqrt 2 \mid a, b \in \mathbb{Q}\}$,
$\mathbb{Q}[\sqrt 5] = \{a + b\sqrt 5 \mid a, b \in \mathbb{Q}\}$
都是$\mathbb{R}$的子域. 它们是同构的域吗?
\end{problem}

\begin{solution}
    由教材命题1.4.1的(9), 两个域若存在同态则一定是单同态, 即只有两种可能, 
    一个域为另一个域的扩张或两者同构. 我们断言这两个域之间不存在同态.
    
    假设存在同态
$\varphi: \mathbb{Q}[\sqrt{2}] \to \mathbb{Q}[\sqrt{5}]$,
则设$\varphi(\sqrt{2}) = a + b\sqrt{5},\, a, b \in \mathbb{Q}$.
注意到由同态定义有$\varphi(2) = 2$, 立刻有
\[
    2 = \varphi(2) = \varphi(\sqrt{2})^2 = (a + b\sqrt{5})^2 = a^2 + 5b^2 + 2ab\sqrt{5}
\]
这要求$a^2 + 5b^2 = 2$且$ab = 0$, 这是不可能的, 矛盾.
\end{solution}

\begin{problem}
    设$K, L$是两个域, 如果$L$是$K$的子域, 则$K$称为$L$的扩域,
$K \supset L$称为域扩张, 试证明:
\begin{enumerate}[(1)]
    \item 域的加法和乘法使得$K$是一个$L$-向量空间$([K:L] = \dim_L(K)$称为域
扩张$K \supset L$ 的次数);
    \item 如果$K \supset \mathbb{R}$是一个二次扩张(即$[K:\mathbb{R}] = 2)$,
则$K$必同构于复数域$\mathbb{C}$.
\end{enumerate}
\end{problem}

\begin{solution}
\begin{enumerate}[(1)]
    \item $(K, +)$是一个Abel群, 这一点无需再说明. 乘法在这里可能有些歧义,
    此处是要验证乘法限制在$L \times K$上, 即
    \[
        \cdot: L \times K \to K, \quad (l, k) \mapsto lk
    \]
    是数乘. 即要验证
    \[
    \begin{gathered}
        (l_1l_2)k = l_1(l_2k),\\
        (l_1 + l_2)k = l_1k + l_2k,\\
        l(k_1 + k_2) = lk_1 + lk_2,\\
        1k = k = k1.
    \end{gathered}
    \]
    这些都由域的定义得到.

    这也说明若同态$K_1 \to K_2$保持$L$($K_1, K_2$为$L$的两个扩域),
    则一定是$L$-线性映射.

    事实上, 若有环同态$R \overset{\varphi}\to S$, 则$S$上自动有一个
    $R$-模结构
    \[
        R \times S \to S, \quad (r, s) \mapsto \varphi(r)s
    \]
    这道题对应的同态其实就是包含(inclusion)$L \overset{i}\hookrightarrow K$.
    \item 由(1), 扩域$\mathbb{C}/\mathbb{R}$的自同构一定是$\mathbb{R}$-线性的.
    设同构$f: \mathbb{C} \to \mathbb{C}$, 则有
    $f(x + yi) = x + yf(i),\, x, y \in \mathbb{R}$, 且
    保持乘法, 可得$f(i) = \pm i$. 也就是说$\mathbb{C}/\mathbb{R}$的自同构都只有
    恒等映射和共轭, 即$\mathrm{Gal}(\mathbb{C}/\mathbb{R}) = \mathbb{Z}/2\mathbb{Z}$.

    由线性代数的结论, 可以直接得到$K$和$\mathbb{C}$是作为线性空间同构, 但这是不够的,
    只有上述两个线性映射是域同构, 需要做基变换转为恒等或共轭才能保持乘法. 事实上只要存在
    一个基变换就能变回恒等映射, 恒等映射总是同构,
    但前提是承载集合(underlying set)要一样. 比如$\mathbb{Q}(\sqrt{2})$
    和$\mathbb{Q}(\sqrt{3})$作为$\mathbb{Q}$-线性空间也是同构的, 但他们
    之间没有域同态.

    可取$K$的一组基为$1, \alpha$, 其中$\alpha \in \mathbb{C} \setminus \mathbb{R}$.
    不可避免地要考虑$\alpha^2$的结果, 由于$1, \alpha$是基, 因此$\alpha^2$可以被线性表出,
    即$\alpha^2 = x + y\alpha$. 由于$\alpha \notin \mathbb{R}$, 有$y^2 + 4x < 0$,
    解二次方程得到$\alpha = \frac{y \pm i\sqrt{|y^2 + 4x|}}{2}$.
    故映射
\[
    f: K \to \mathbb{C},\, u + v\alpha \mapsto u + v\frac{y \pm i\sqrt{|y^2 + 4x|}}{2}
\]
    是域同构.
\end{enumerate}
\end{solution}

\begin{problem}
    设$d$是一个非零整数, 且$\sqrt d \notin \mathbb{Q}$. 证明:
\[
    \mathbb{Q}[\sqrt{d}] = \{a + b\sqrt{d} \mid a, b \in \mathbb{Q}\} \supset \mathbb{Q}
\]
是一个二次扩张($d < 0$时, $\mathbb{Q}[\sqrt{d}]$称为虚二次域, $d > 0$时称为实二次域).
\end{problem}

\begin{solution}
    $sqrt{d}$满足多项式$f(x) = x^2 - d \in \mathbb{Q}[x]$,
    因此在$\mathbb{Q}[\sqrt{d}]$中.
\end{solution}

\begin{problem}
    设$L \supset K$是一个域扩张, 证明: 下述集合
\[
\mathrm{Gal}(L/K)=
\left\{L \xrightarrow{\sigma} L \mid \sigma\text{ 是域同构},\text{ 且 } \sigma(a) = a \text{ 对任意 } a \in K \text{ 成立}\right\}
\]
关于映射的合成是一个群(称为域扩张$L\supset K$的伽罗瓦群).
\end{problem}

\begin{solution}
    $\mathrm{Gal}(L/K) \subseteq \mathrm{Aut}(L)$, 只需说明$\mathrm{Gal}(L/K)$
是子群.

    $\forall \varphi, \psi \in \mathrm{Gal}(L/K)$, 由于$\psi|_K = \mathrm{id}_K$,
因此$\psi^{-1}|_K = \mathrm{id}_K$, 故$(\varphi \circ \psi^{-1})|_K = \mathrm{id}_K$,
即$\varphi \circ \psi^{-1} \in \mathrm{Gal}(L/K)$.
\end{solution}

\begin{problem}
    求$\mathrm{Gal}\left(\mathbb{Q}[\sqrt{d}]/\mathbb{Q}\right)$,
此处$d \in \mathbb{Z}, \sqrt{d} \notin \mathbb{Q}$.
\end{problem}

\begin{solution}
    
\end{solution}

\begin{problem}
    设$V = (V, +)$ 是一个加法群, $\mathrm{Hom}(V)$表示它的自同态环.
对任意域$K$, 如果存在一个数乘运算
$K \times V \to V,\, (\lambda,v) \mapsto \lambda \cdot v$,
使得加法群$V = (V, +)$成为一个$K$-线性空间, 则称该数乘运算是加法群
$V = (V, +)$上的一个$K$-线性空间结构. 试证明:
\begin{enumerate}[(1)]
    \item 如果存在一个环同态$\varphi:K \to \mathrm{Hom}(V)$,则数乘运算
\[
    K \times V \to V,\quad (\lambda, v) \mapsto \lambda \cdot v \defeq \varphi(\lambda)(v)
\]
是$V$上的一个$K$-线性空间结构;
    \item 如果在$V$上存在$K$-线性空间结构$\phi:K \times V \to V$,则映射
\[
    \varphi:K \to \mathrm{Hom}(V),\quad \lambda \mapsto \phi(\lambda, \cdot)
\]
是一个环同态, 其中$\phi(\lambda, \cdot):V \to V$定义为
$v \mapsto \phi(\lambda, v) \defeq \lambda \cdot v$;
    \item 对任意域$K$, 整数加法群$\mathbb{Z} = (\mathbb{Z}, +)$上不存在$K$-线性空间结构.
\end{enumerate}
\end{problem}

\begin{solution}
    
\end{solution}

\begin{problem}
    证明:在整数集合$\mathbb{Z}$上存在运算
$\mathbb{Z} \times \mathbb{Z} \to \mathbb{Z}, (a,b) \mapsto a \oplus b$,
使得$(\mathbb{Z}, \oplus)$是一个交换群, 但它与整数加法群
$(\mathbb{Z}, +)$不同构. 提示:利用$\mathbb{Q}$是可数集和上题中的问题(3).
\end{problem}

\begin{solution}
    
\end{solution}
\clearpage
\section{唯一分解整环}
\subsection{教材p28-p29}

\begin{problem}\label{ex:2.1.1}
    设$R$是一个交换环, $I \,\red{\subsetneq}\, R$是一个理想. 证明
    \[
        \sqrt{I} = \{r \in R \mid \exists m \in \red{\mathbb{N}} \text{ 使得 } r^m \in I\}
    \]
    也是$R$的理想(称为理想$I$的根).
\end{problem}

\begin{remark}
    这题的理想的根定义有误, 应是$\mathbb{N}$而不是$\mathbb{Z}$. 一旦出现负整数意味着有可逆元, 从而$\sqrt{I}$是单位理想了.
\end{remark}

\begin{proof}
    先验证加法子群, 
    \[
    \begin{gathered}
        \forall a, b \in \sqrt{I},\, \exists m, n \in \mathbb{N},\, a^m, b^n \in I,\\
        \implies (a - b)^{m + n - 1} \in I
    \end{gathered}
    \]
    这是因为单项$a^ib^j$的指数$i + j = m + n - 1$, 故$i < m$和$j < n$不能同时成立, 即$i \geqslant m$或$j \geqslant n$, i.e. $a^i \in I$或$b^j \in I$.
    从而$(a - b)^{m + n} \in I$, $a - b \in \sqrt{I}$.

    再验证吸收律(交换验证单边即可),
    \[
        \forall a \in \sqrt{I}, r \in R,\, \exists m \in \mathbb{N},\, a^m \in I \implies (ar)^m = a^mr^m \in I
    \]
    因此$ar \in \sqrt{I}$.
\end{proof}

\begin{remark}
    零理想的根$\sqrt{\{0\}} = \{x \in R \mid \exists n \in \mathbb{N}, x^n = 0\}$是所有幂零元(nilpotent)组成的理想, 叫做$R$的幂零根(nilradical), 一般记作$\mathfrak{N}(R)$. 可以证明$\mathfrak{N}(R) = \bigcap_{\mathfrak{p} \text{ 是素理想}} \mathfrak{p}$.(可以参考\cite{atiyah1994introduction}p5)

    对任何的理想$I$可以清楚地看出$I \subseteq \sqrt{I}$. 若$\sqrt{I} = I$, 我们称$I$是一个根理想(radical ideal). 任何的素理想(\ref{ex:2.1.5})都是根理想.
\end{remark}

\begin{problem}\label{ex:2.1.2}
    设$R$是一个交换环, $p > 0$是一个素数. 如果$p \cdot x = 0 (\forall x\in R)$. 试证明: $(x + y)^{p^m} = x^{p^m}+y^{p^m} (\forall x, y \in R, m > 0)$
\end{problem}

\begin{proof}
    事实上, 这个$p$就是环$R$的特征. 若$\mathrm{Char}(R) \neq p$, 则由$p = 0$, $\mathrm{Char}(R) < p$. 那么$(p, \mathrm{Char}(R)) = 1$, 有Bézout's Identity得到$1 = 0$, 这就没什么考虑的必要了.

    对特征$p$的交换环, 有一个特别的同态$F$称为Frobenius自同态,
    \[
        F: R \to R, \quad a \mapsto a^p
    \]
    我们说明这确实是一个同态.
    
    保持乘法是因为交换环, 不平凡的是保持加法.
    \[
        (a + b)^p = a^p + \binom{p}{1}a^{p - 1}b + \cdots + b^p.
    \]
    其中$1 \leqslant i \leqslant p - 1$时,
    \[
        \binom{p}{i} = \frac{p(p - 1) \cdots (p - i + 1)}{1 \cdot 2 \cdots i} 
    \]
    由于$p$是素数, $1, 2, \cdots i$都不整除$p$, 而$\binom{p}{i}$是整数, 因此只能是$i! \mid (p - 1) \cdots (p - i + 1)$. 所以$p \mid \binom{p}{i}$. 而$p = 0$, 故$(a + b)^p = a^p + b^p$.

    因此$\varphi: R \to R,\, x \mapsto x^{p^m}$也是自同态, $\varphi = F^m$, 这里$F^m$表示复合$m$次.
\end{proof}

\begin{remark}
    Frobenius一般在域中使用的多一些. 虽然对交换环Frobenius都是可以定义的, 但是整环才能保证Frobenius是单射. Frobenius一般不是满的, 但对有限域就是自同构了.
\end{remark}

\begin{problem}
    证明: 只有有限个元素的整环一定是一个域.
\end{problem}

\begin{proof}
    整环$R$有乘法消去律\ref{ex:1.1.1}, 而\ref{ex:1.3.9}告诉我们, 满足消去律的有限半群是群. 因此$(R \setminus \{0\}, \cdot)$是群, 即$R$是一个域.
\end{proof}

\begin{problem}\label{ex:2.1.4}
    证明: 只有有限个理想的整环是一个域.
\end{problem}

\begin{proof}
    事实上条件可以再减弱一点, 一个Artin整环一定是域.

    设$a \neq 0$, 考虑理想降链
    \[
        (a) \supseteq (a^2) \supseteq \cdots 
    \]
    因此$\exists n \in \mathbb{Z}_{>0},\, (a^n) = (a^{n + 1})$. 即有$a^n \in (a^{n + 1})$, 那么$\exists b \in R,\, a^n = a^{n + 1}b$, 从而$ab = 1_R$.
\end{proof}

\begin{remark}
    Artin环定义为任意理想降链稳定的环, i.e. 若有理想降链
    \[
        I_1 \supseteq I_2 \supseteq \cdots
    \]
    则存在$n \in \mathbb{Z}_{>0}$使得$\forall m > n,\, I_m = I_n$, 也就是说从某一个$n$开始就稳定了$I_n = I_{n + 1} = \cdots$. 这个条件称为descending chain condition(d.c.c.), 与之对应的是ascending chain condtion(a.c.c.), 满足a.c.c.的正是Noether环.
\end{remark}

\begin{problem}\label{ex:2.1.5}
    理想$P \,\red{\subsetneq}\, R$称为素理想, 如果: $ab \in P \Rightarrow a \in P$或$b \in P$. 试证明: $P \,\red{\subsetneq}\, R$是素理想当且仅当$R/P$没有零因子.
\end{problem}

\begin{proof}
    \begin{enumerate}[(1)]
        \item "$\implies$":
        \[
        \begin{gathered}
            \forall \overline{a}, \overline{b} \in R/P,\, \overline{a}\overline{b} = \overline{ab} = \overline{0} \implies ab \in P \implies a \in P \text{ or } b \in P \\ \implies \overline{a} = \overline{0} \text{ or } \overline{b} = \overline{0}.
        \end{gathered}
        \]
        \item "$\impliedby$":
        \[
            ab \in P \implies \overline{a}\overline{b} = \overline{ab} = \overline{0} \implies \overline{a} = 0 \text{ or } \overline{b} = 0 \implies a \in P \text{ or } b \in P.
        \]
    \end{enumerate}
\end{proof}

\begin{remark}
    \begin{enumerate}[1.]
        \item 零理想$(0)$是素理想.
        \item 一个交换环的(Krull) dimension定义为最长素理想链的长度, 其中, 若有素理想链
        \[
            \mathfrak{p}_0 \subsetneq \mathfrak{p}_1 \subsetneq \cdots \subsetneq \mathfrak{p}_n
        \]
        他的长度定义为$n$.(可参考\cite{atiyah1994introduction}p89, \cite{aluffi2009algebra}p153)
        
        交换Artin环(\ref{ex:2.1.4})是$0$维的Noether环. $0$维即意味着所有的素理想都是极大理想.
        \item 对交换环$R$, $\mathrm{Spec}(R) \defeq \{\mathfrak{p} \mid \mathfrak{p} \text{ 是 } R \text{ 的素理想}\}$称为$R$的素谱(spectrum). $\mathrm{Spec}(R)$上有一个Zariski拓扑.(这个就不给参考书了, 自行搜索或查阅代数几何相关书籍吧)
    \end{enumerate}
\end{remark}

\begin{problem}\label{ex:2.1.6}
    理想$m \,\red{\subsetneq}\, R$称为极大理想, 如果$R$中不存在真包含$m$的非平凡理想(即: 如果$I \,\red{\supsetneq}\, m$是$R$的理想, 则必有$I = R)$. 试证明: 当$R$是交换环时, $m \,\red{\subsetneq}\, R$是极大理想当且仅当$R/m$是一个域. 特别, 交换环中的极大理想必为素理想.
\end{problem}

\begin{proof}
    \begin{enumerate}[(1)]
        \item "$\implies$":
        \[
        \begin{aligned}
            \forall \overline{0} \neq \overline{a} \in R/m &\implies a \notin m \implies m \subsetneq m + (a) \implies m + (a) = R = (1)\\
            &\implies \exists x \in m, b \in R,\, x + ab = 1 \implies \overline{ab} = \overline{1 - x} = \overline{1}.
        \end{aligned}
        \]
        \item "$\impliedby$":
        \[
        \begin{aligned}
            m \subsetneq I \underset{\text{ideal}}{\subseteq} R &\implies \exists a \in I \setminus m \text{ i.e. } \overline{a} \neq 0 \implies \exists b \in R,\, \overline{a}\overline{b} = \overline{ab} = \overline{1}\\
            &\implies \exists x \in m \subsetneq I,\, ab = 1 + x \implies 1 = ab - x \in I \\
            &\implies I = (1) = R.
        \end{aligned}
        \]
        或者用同态基本定理, 包含$m$的理想和$R/m$的理想有一个一一对应, 而域的理想只有$\{0\}$和本身.
    \end{enumerate}
\end{proof}

\begin{remark}
    (1)中用到了理想的和. 若$I, J$都是$R$的理想, $I + J \defeq \{i + j \mid i \in I, j \in J\}$. 可以验证这确实是一个理想, 类似可以定义一族理想$\{I_\alpha\}_{\alpha \in A}$的和,
    \[
        \sum_{\alpha \in A} I_\alpha = \left\{\sum_{\alpha \in A} i_\alpha \Big| i_\alpha \in I_\alpha, \text{ 且只有有限个 } i_\alpha \neq 0 \right\}
    \]
    即考虑所有可能的有限和. 所谓子集$S \subseteq R$生成的理想, 是指理想
    \[
        (S) = \sum_{a \in S} (a).
    \]
    对一个理想$I$, 若存在有限子集$S$生成$I$, 则称$I$是有限生成的.
    
    另外$\bigcap_{\alpha \in A} I_\alpha$也是一个理想. 还有一个是理想的积, 相对要复杂一些,
    \[
    \begin{aligned}
        IJ &\defeq (\{ij \mid i \in I, j \in J\})\\
        &= \left\{\sum_{k = 1}^{n} i_kj_k \Big| \exists n \in \mathbb{N},\, 1 \leqslant k \leqslant n,\, i_k \in I, j_k \in J, \right\}
    \end{aligned}
    \]
    他是所有乘积$ij$生成的理想. 那么一族理想的乘积就是考虑所有可能的有限乘积生成的理想.
\end{remark}

\begin{problem}
    设 $I \,\red{\subsetneq}\, \mathbb{Z}$是整数环的非零理想, 证明下述结论等价
    \begin{enumerate}[(1)]
        \item $I$是极大理想;
        \item $I$是素理想;
        \item 存在素数$p$使得$I = (p)\mathbb{Z} = \{ap \mid \forall a \in \mathbb{Z}\}$.
    \end{enumerate}
\end{problem}

\begin{proof}
    \begin{enumerate}[1.]
        \item (1)$\implies$(2): 由于域一定是整环, 由\ref{ex:2.1.5}和\ref{ex:2.1.6}知极大理想是素理想.
        \item (2)$\implies$(3): 由于$\mathbb{Z}$是PID(带余除法可证), 故存在整数$p$使得$I = (p)$. 由于是素理想, 因此$ab \in (p) \implies a \in (p)$或$b \in (p)$. 即
        \[
            p \mid ab \implies p \mid a \text{ or } p \mid b
        \]
        则$p$是素数(若不然, $p = qr$, 取$a = q, b = r$即导出矛盾).
        \item (3)$\implies$(1): 设$I = (p) \subsetneq J$, 则存在$n \in J \setminus I$. 由于$p$是素数, 故有$(n, p) = 1$. 由Bézout's Identity, $\exists u, v \in \mathbb{Z}$使得$nu + pv = 1$, 从而$1 \in J,\, J = \mathbb{Z}$.(这和\ref{ex:2.1.6}的证明是类似的)
        
        或直接用$\mathbb{Z}/p\mathbb{Z}$是域.
    \end{enumerate}
\end{proof}

\begin{problem}\label{ex:2.1.8}
    设$p \in \mathbb{Z}$是素数, 证明$(p)\mathbb{Z}[x] = \{pf(x) \mid \forall f(x) \in \mathbb{Z}[x]\}$是整系数多项式环的素理想, 但不是$\mathbb{Z}[x]$的极大理想.
\end{problem}

\begin{proof}
    事实上若$I$是$R$的理想, 我们有
    \[
        \frac{R[x]}{IR[x]} \cong \frac{R}{I}[x]
    \]
    这是根据同态基本定理得到, 考虑同态
    \[
        \varphi: R[x] \to \frac{R}{I}[x],\quad a_0 + a_1x + \cdots + a_nx^n \mapsto \overline{a_0} + \overline{a_1}x + \cdots + \overline{a_n}x^n
    \]
    可以验证这确实是一个同态. 事实上, 它是$R \twoheadrightarrow R/I \hookrightarrow \frac{R}{I}[x]$的一个延拓.
    
    回到原题, 有
    \[
        \frac{\mathbb{Z}[x]}{(p)\mathbb{Z}[x]} \cong \mathbb{F}_p[x]
    \]
    这里$\mathbb{F}_p = \mathbb{Z}/p\mathbb{Z}$是域. 因此$\mathbb{F}_p[x]$是PID, 自然是整环, 但不是域($x$没有逆). 因此由\ref{ex:2.1.5}和\ref{ex:2.1.6}, $(p)\mathbb{Z}[x]$是素理想但不是极大理想.
\end{proof}

\begin{remark}
    给定环同态$R \overset{\varphi}\to S$, 其中$R$是交换环. 若$\varphi(R) \subseteq C(S)$(\ref{ex:2.1.11}), 根据我们之前\ref{ex:1.4.9}说过的, 首先$S$上有一个$R$-模结构. 其次有
    \[
        (r_1s_1)(r_2s_2) = \varphi(r_1)s_1\varphi(r_2)s_2 = \varphi(r_1)\varphi(r_2)s_1s_2 = \varphi(r_1r_2)s_1s_2 = (r_1r_2)(s_1s_2).
    \]
    即数乘和$S$本身的乘法是相容的. 这样的结构我们称为一个$R$-代数($R$-algebra), 这也是\ref{ex:2.1.12}介绍的东西. 因此一个$R$-代数就是带有加法, ($R$-)数乘, 乘法的一个代数结构.
    
    当$S$本身就是交换环时, 此时乘法是交换的, 且$C(S) = S$, 这样会变得简单很多. 这时$S$称为一个交换$R$-代数, 这也是交换代数会考虑的情形. 我们会把$S$看作一个有序对$(S, \varphi)$, 一个交换$R$-代数$S$也叫做一个$R$-(交换)环. 那么交换$R$-代数构成的范畴是交换环范畴的余切片范畴(coslice category).
    
    而这里提到的延拓其实是多项式环的泛性质(universal property), 或者说是自由交换$R$-代数的泛性质, 因为$R[x]$就是一个的自由交换$R$-代数.(可参考\cite{aluffi2009algebra}III.\S6.3)
\end{remark}

\begin{problem}\label{ex:2.1.9}
    映射$D:R[x] \longrightarrow R[x]$定义如下: $\forall f(x) = a_nx^n + \cdots + a_1x + a_0$,
    \[
        D(f) = na_nx^{n - 1} + (n - 1)a_{n - 1}x^{n - 2} + \cdots + 2a_2x + a_1.
    \]
    $\forall$ $a \in R$, $f, g \in R[x]$, 试证明:
    \begin{enumerate}[(1)]
        \item $D(f + g) = D(f) + D(g)$, $D(af) = aD(f)$;
        \item $D(f \cdot g) = D(f) \cdot g + f \cdot D(g)$.
    \end{enumerate}
    ($D(f)$称为$f(x)$的导数. 记为$f'(x) = D(f),\, f^{(m)}(x) = \overset{m}{\widehat{D \cdots D}}(f)$称为$f(x)$的$m$次导数).
\end{problem}

\begin{proof}
    按定义验证. 设$f = a_nx^n + \cdots + a_1x + a_0$, $g = b_mx^m + \cdots + b_1x + b_0$.
    \begin{enumerate}[(1)]
        \item 不妨设$n \geqslant m$, 且令$b_k = 0, k > m$.
        \[
        \begin{aligned}
            D(f + g) &= D\left(\sum_{k = 0}^{n} (a_k + b_k)x^k\right) = \sum_{k = 1}^{n} k(a_k + b_k)x^{k - 1}\\ 
            &= \sum_{k = 1}^{n} ka_kx^{k - 1} + \sum_{k = 1}^{m} kb_kx^{k - 1} = D(f) + D(g).
        \end{aligned}
        \]
        \[
            D(af) = D\left(\sum_{k = 0}^{n} aa_kx^k\right) = \sum_{k = 1}^{n} kaa_kx^{k - 1} = a\sum_{k = 1}^{n} ka_kx^{k - 1} = aD(f).
        \]
        这里能把$a$提出来是因为$k$作为$k1$(\ref{ex:1.2.1}的注记), 有$ka = ak$.
        \item 
        \[
        \begin{aligned}
            D(f \cdot g) &= D\left(\sum_{k = 0}^{n + m} \sum_{i + j = k} a_ib_jx^k\right) = \sum_{k = 1}^{n + m} \sum_{i + j = k} ka_ib_jx^{k - 1}\\
            &= \sum_{k = 1}^{n + m} \sum_{i + j = k} (i + j)a_ib_jx^{i + j - 1}\\
            &= \sum_{k = 1}^{n + m} \sum_{i + j = k} (ia_ix^{i - 1})b_jx^j + a_ix^i(jb_jx^{j - 1})\\
            &= \sum_{k = 0}^{n + m - 1} \sum_{(i - 1) + j = k} (ia_i)b_jx^{k} + a_i(jb_j)x^k\\
            &= D(f) \cdot g + f \cdot D(g).
        \end{aligned}
        \]
    \end{enumerate}
\end{proof}

\begin{problem}\label{ex:2.1.10}
    如果$F$是特征零的域, 则$f'(x) = 0 \Leftrightarrow \deg(f) = 0$或$f(x) = 0$(即常数); 如果$F$的特征是$p > 0$, 则$f'(x) = 0 \Leftrightarrow$存在$g(x) \in F[x]$使得$f(x) = g(x^p)$.
\end{problem}

\begin{proof}
    $\mathrm{Char}(F) = 0$, 即$\forall n \in \mathbb{Z}_{>0}, n \neq 0$(\ref{ex:1.2.1}的注记), 那么
    \[
        f'(x) = na^{n - 1} + \cdots + a_1 = 0 \implies 1 \leqslant k \leqslant n, ka_k = 0 \implies 1 \leqslant k \leqslant n, a_k = 0
    \]
    故$f(x) = a_0$, $\deg(f) = 0$或$f = 0$, 反过来是平凡的.

    若$\mathrm{Char}(F) = p$, 则$p = 0$, 那么设$\deg(f) = n = kp + r, 0 \leqslant r < p, k \in \mathbb{N}$,
    \[
    \begin{aligned}
        f &= a_0 + a_1x + \cdots a_px^p + \cdots + a_{2p}x^{2p} + \cdots + a_{kp}x^{kp} + a_nx^n.\\
        \implies f' &= a_1 + \cdots + pa_px^{p - 1} + \cdots + kpa_{kp}x^{kp - 1} + \cdots na_nx^{n - 1}\\
        &= a_1 + \cdots + (p - 1)a_{p - 1}x^{p - 2} + (p + 1)a_{p + 1}x^p + \cdots (kp - 1)a_{kp - 1}x^{kp - 2}\\
        &+ (kp + 1)a_{kp + 1}x^kp + \cdots + na_nx^{n - 1}.
    \end{aligned}
    \]
    此时$f' = 0$有$f = a_0 + a_px^p + \cdots + a_{kp}x^{kp} = g(x^p)$. 这里$g = a_0 + a_px + \cdots + a_{kp}x^k$. 反过来也是类似的.
\end{proof}

\begin{problem}\label{ex:2.1.11}
    设$R$是一个环, 子环$C(R) = \{a \in R \mid ab = ba\, \forall b \in R\}$称为$R$的中心. 试证明:
    \begin{enumerate}[(1)]
        \item 如果$R$是一个除环, 则$C(R)$是一个域;
        \item 令$\mathbb{H}$表示Hamilton四元数环, 则$C(\mathbb{H}) = \mathbb{R}$.
    \end{enumerate}
\end{problem}

\begin{proof}
    \begin{enumerate}[(1)]
        \item 除环的子环自然是除环, $C(R)$和$R$中所有元素交换, 故$C(R)$本身是交换环, 从而是域.
        \item 设$\alpha = a + ib + jc + kd \in C(\mathbb{H})$, 则有
        \[
        \begin{aligned}
            \alpha \cdot i &= i \cdot \alpha\\
            \alpha \cdot j &= j \cdot \alpha
        \end{aligned}
        \]
        得到$b = c = d = 0$, 即$\alpha \in \mathbb{R}$.
    \end{enumerate}
\end{proof}

\begin{problem}\label{ex:2.1.12}
    设$K$是一个域. 如果$C(R)$包含一个同构于$K$的子域, 则称环$R$为$K$-代数. 试证明: 加法群$(R, +)$通过$R$的乘法成为一个$K$-向量空间.
\end{problem}

\begin{proof}
    见\ref{ex:1.4.9}和\ref{ex:2.1.8}的注记. $C(R)$包含一个和$K$同构的子域, 等价地说就是有一个域同态$K \to R$.
\end{proof}

\begin{remark}
    $C(R)$包含一个同构于$K$的子域, 即存在同态$K \overset{\varphi}\to R$使得$\varphi(K) \subseteq C(R)$(这是因为域出发的同态一定是单的). 这和之前说的是一样的.
\end{remark}

\begin{problem}\label{ex:2.1.13}
    设$R$是一个$K$-代数, $\dim_K(R)$称为$R$的维数. 试证明: 
    \begin{enumerate}[(1)]
        \item 矩阵环$M_n(K)$是一个$n^2$维$K$-代数;
        \item 任意$n$维$K$-代数必同构于$M_n(K)$的子环;
        \item 如果$R$是一个有限除环, 则$R$是有限域上的有限维代数.
    \end{enumerate}
\end{problem}

\begin{proof}
    \begin{enumerate}[(1)]
        \item $M_n(K)$是$n^2$维$K$-线性空间, 按\ref{ex:2.1.8}注记, 只需验证
        \[
            k_1M_1k_2M_2 = k_1k_2M_1M_2, k_1, k_2 \in K, M_1, M_2 \in M_n(K).
        \]
        这可以根据$M_n(K)$的定义得到. 事实上$C(M_n(K)) = \{kI_n \mid k \in K\} \cong K$.
        \item 由教材例1.4.3, 对任意的环$R$, 我们用$\mathrm{End}_{\mathsf{Ab}}(R)$表示加法群的自同态环(关于加法和复合). 有一个自然的环同态,
        \[
            R \to \mathrm{End}_{\mathsf{Ab}}(R),\quad r \mapsto \lambda_r
        \]
        其中$\lambda_r: R \to R,\, a \mapsto ra$, 即左乘$r$这个自同态(这里换成右乘也是一样的). 这是一个单同态, 所以$R$同构于$\mathrm{End}_{\mathsf{Ab}}(R)$的一个子环.

        那么当$R$是$n$维$K$-代数时, $\lambda_r$还是$K$-线性映射. 因此有单射$R \hookrightarrow \mathrm{Hom}_K(R) \cong M_n(K)$.
        \item $R$是有限除环, 因此$C(R)$是有限域(\ref{ex:2.1.11}). 根据定义$R$是一个$C(R)$-代数, 且$R$有限, 故是有限维的($|R| = [R:C(R)]|C(R)|$).
    \end{enumerate}
\end{proof}

\begin{problem}
    设$K$是一个域, $R$是一个有限维$K$-代数. 试证明: 
    \begin{enumerate}[(1)]
        \item $\forall \alpha \in R$, 存在\red{非零}多项式$f(x) \in K[x]$使得$f(\alpha) = 0$;
        \item 如果$R$是除环, $\alpha \neq 0$, 则$\alpha$的极小多项式$\mu_\alpha(x) \in K[x]$不可约;
        \item 如果$R$是除环, $K$是代数闭域(即$K[x]$中次数大于零的多项式在$K$中必有根),则$R = K$.
    \end{enumerate}
    历史上, 有限维可除$K$-代数的分类是一个热门话题. 当$K$是实数域时, $R$必同构于实数域, 复数域或Hamilton四元数环之一(Frobenius 定理); 当$K$是有限域时, $R$必为交换环(Wedderburn 定理).
\end{problem}

\begin{remark}
    零多项式是平凡的, 因此(1)我做了修改. 在域扩张中, 这样的元素称为$K$上的代数元(algebraic element), 或者称$\alpha$在$K$上代数(algebraic over $K$). 给定域扩张$L/K$, 若$\forall \alpha \in L$都在$K$上代数, 则称该扩张是代数扩张.
\end{remark}

\begin{proof}
    \begin{enumerate}[(1)]
        \item 设$\mathrm{dim}_K R = n$. 则$1, \alpha, \alpha^2, \cdots, \alpha^n$线性相关. 或者考虑线性映射$r \mapsto \alpha r$. 那么它对应的矩阵的特征多项式满足条件(Cayley-Hamilton Theorem).
        \item 按定义, $\mu_\alpha$是满足$\alpha$的次数最小的(首一)多项式. 假设$\mu_\alpha$可约, 即$\mu_\alpha(x) = f(x)g(x),\, \deg(f),\, \deg(g)> 0$, 则$0 = \mu_\alpha(\alpha) = f(\alpha)g(\alpha)$. 由于除环无零因子, 故$\deg(\mu_\alpha) = \deg(f) + \deg(g)$, 且$f(\alpha) = 0$或$g(\alpha) = 0$. 不妨设$f(\alpha) = 0$, 但$\deg(f) < \deg (\mu_\alpha)$与极小矛盾.
        \item 代数闭域等价于任意多项式可分解成一次多项式的乘积. 这和代数基本定理是类似的. 此时$K[x]$中的不可约多项式即为所有一次多项式. 由(2), $\forall \alpha \in R$, 极小多项式$\mu_\alpha(x) = x - k_\alpha, k_\alpha \in K$. 因此$\alpha = k_\alpha \in K$. 即$R = K$.
    \end{enumerate}
\end{proof}

\begin{problem}
    证明: 集合
    \(
        \mathbb{F}_{3^2} =
        \left\{
            \begin{pmatrix}
                a & b\\
                -b & a
            \end{pmatrix}
        \bigg| a, b \in \mathbb{F}_3 = \mathbb{Z}/(3)
        \right\}
    \)
    关于矩阵的“加法”和“乘法”成为一个9元域. 若将定义中的$\mathbb{F}_3$换成$\mathbb{F}_5$, 上述集合是否是一个25元域, 为什么?
\end{problem}

\begin{proof}
    这个集合是$\mathbb{F}_3$上的$2$维线性空间.
    \[
        \mathbb{F}_{3^2} =
        \left\{
        a\lambda + b\xi \mid
        \lambda = I_2 =
        \begin{pmatrix}
            1 & 0\\
            0 & 1
        \end{pmatrix},
        \xi = 
        \begin{pmatrix}
            0 & 1\\
            -1 & 0
        \end{pmatrix},
        a, b \in \mathbb{F}_3
        \right\}
    \]
    其中$\xi$的特征多项式为$x^2 + 1$, 可以验证它是不可约的. 又因为$\mathbb{F}_3[x]$是PID($\mathbb{F}_3[x]$是域), 故$(x^2 + 1)$是极大理想, $x^2 + 1$就是$\xi$的极小多项式. 则$\mathbb{F}_{3^2} = \mathbb{F}_3[x]/(x^2 + 1)$是域.
    
    但$x^2 + 1$在$\mathbb{F}_5[x]$中是可约的: 在$\mathbb{F}_5[x]$中, 
    \[
        x^2 + 1 = x^2 - 4 = (x - 2)(x + 2).
    \]
\end{proof}
\subsection{教材p35-p36}

\begin{problem}
    设$m, n$是两个正整数, 证明它们在$\mathbb{Z}$中的最大公因数
和它们在$\mathbb{Z}[i]$中的最大公因数相同.
\end{problem}

\begin{solution}
    
\end{solution}

\begin{problem}
    设$R$是整环, $p \in R$称为一个素元如果它生成的理想
$P=(p)R$是素理想. 证明:$R$中素元必为不可约元.
\end{problem}

\begin{solution}
    
\end{solution}

\begin{problem}
    设$R$是一个主理想整环(PID), $0 \neq r \in R$.
证明:在$R$中仅有有限个理想包含$r$.
\end{problem}

\begin{solution}
    
\end{solution}

\begin{problem}[辗转相除法]
    设$R$是欧氏环, $a, b \in R$非零. 由带余除法得
\[
a = q_{1}b + r_{1},\,
b = q_{2}r_{1}+ r_{2},\,
r_{1} = q_{3}r_{2} + r_{3},\, \cdots,\,
r_{k- 2}= q_{k}r_{k- 1}+ r_{k}
\]
满足$\delta(r_k) < \delta(r_{k - 1}) < \cdots < \delta(r_2) < \delta(r_1) < \delta(b)$.
试证明:
\begin{enumerate}[(1)]
    \item 存在$k$使得$r_k + 1 = 0$;
    \item $r_k$是$a$, $b$的一个最大公因子;
    \item 求$u$, $v \in R$使得$r_k = ua + vb$.
\end{enumerate}
\end{problem}

\begin{solution}
    
\end{solution}

\begin{problem}
    设$R = \mathbb{Z}[\sqrt{-5}] = \{a + b\sqrt{-5} \mid \forall a, b \in\mathbb{Z}\} \subset \mathbb{C}$,
定义:$N(a + b\sqrt{-5}) = a^2 + 5b^2$. 试证明:
\begin{enumerate}[(1)]
    \item $U(R) = \{1, -1\}$;
    \item $R$中任意元素都有不可约分解;
    \item $3$, $2 + \sqrt{-5}$, $2 - \sqrt{-5} \in R$是不可约元;
    \item $9 = 3 \cdot 3= (2 + \sqrt{-5}) \cdot (2 - \sqrt{-5})$是$9$的两个不相同的不可约分解.
\end{enumerate}
\end{problem}

\begin{solution}
    
\end{solution}

\begin{problem}
    令$\mathbb{R}, \mathbb{C}$分别表示实数域和复数域, 试证明:
\begin{enumerate}[(1)]
    \item 若$R$是由关于$\cos t$和$\sin t$的实系数多项式组成的函数环, 
则$R \cong \mathbb{R}[x, y]/(x^2 + y^2 - 1)$;
    \item $\mathbb{C}[x, y]/(x^2 + y^2 - 1)$是唯一分解整环(提示:证明其为ED);
    \item $\mathbb{R}[x, y]/(x^2 + y^2 - 1)$不是唯一分解整环.
\end{enumerate}
\end{problem}

\begin{solution}
    
\end{solution}
\subsection{教材p41-p42}

\begin{problem}
    设$F$是一个域, $F[[x]]$是系数在$F$中的形式幂级数环, 试证明:
\begin{enumerate}[(1)]
    \item $f(x) = a_0 + a_1x + a_2x^2 + \cdots$在$F[[x]]$中可逆$\Leftrightarrow a_0 \neq 0$;
    \item $F[[x]]$中任意不可约元$p(x)$均与$x$相伴, 即$p(x) \sim x$;
    \item $F[[x]]$是主理想整环, 它是欧氏整环吗?如果是, 请写出一个欧氏映射.
\end{enumerate}
\end{problem}

\begin{solution}
    \begin{enumerate}[(1)]
        \item 按定义, 存在$g = \sum_{k = 0}^{\infty} b_kx^k$使得$fg = 1$. 则
        \[
        \begin{aligned}
            a_0b_0 &= 1,\\
            a_1b_0 + a_0b_1 &= 0,\\
            a_2b_0 + a_1b_1 + a_0b_2 &= 0,\\
            \vdots
        \end{aligned}
        \]
        因此$a_0 \neq 0$.
        
        反过来, 若$a_0 \neq 0$, 由$F$是域, $a_0$可逆. 即存在$b_0 \in F$, $a_0b_0 = 1$. 我们可以通过上面的无穷个方程组递归的解出$b_k(k \geqslant 1)$.
        \[
        \begin{aligned}
            b_1 &= -a_1b_0^2,\\
            b_2 &= -a_2b_0^2 - a_1b_1b_0,\\
            \vdots\\
            b_k &= -\sum_{i = 1}^{k} a_ib_{k - i}b_0,\\
            \vdots
        \end{aligned}
        \]
        从而存在$g = \sum_{k = 0}^{\infty} b_kx^k$为$f$的逆.
        \item 设$p(x) = \sum_{k = 0}^{\infty} a_kx^k$. $p$不可约则不是可逆元, 由(1), $a_0 = 0$, 因此$p(x) = xp_1(x)$. 又因为不可约, 且$x$不是可逆元, 因此$p_1$是可逆元, 故$p(x) \sim x$.
        \item 
    \end{enumerate}
\end{solution}

\begin{problem}
    设$F$是一个域, $p(x) \in F[x]$不可约, 令$I = p(x)F[x]$表示由$p(x)$生
成的理想, 试证明:商环$F[x]/I$是一个域, 且环同态
\[
    \varphi:F[x] \to F[x]/I,\quad f(x) \mapsto \overline{f(x)}
\]
诱导了域嵌入$\varphi|_F: F\hookrightarrow F[x]/I, a \mapsto \bar{a}$
(如果将$F$与它的像等同, 则$\bar{x} \in F[\bar{x}] \defeq F[x]/I$
是$p(x)$在扩域$F[\bar{x}]$中的一个根).
\end{problem}

\begin{solution}
    
\end{solution}

\begin{problem}
    设$F$是一个域, $K \subset F$是一个子域, $f(x), g(x) \in K[x]$.
试证明:$f(x)$, $g(x)$在$K[x]$中互素$\Leftrightarrow f(x)$,
$g(x)$在$F[x]$中互素.
\end{problem}

\begin{solution}
    
\end{solution}

\begin{problem}
    设$F$是特征零的域, $f(x) \in F[x]$不可约. 证明$f(x)$与$f'(x)$互素.
\end{problem}

\begin{solution}
    
\end{solution}

\begin{problem}
    设$\mathbb{F}_2 = \mathbb{Z}/(2) = \{\bar{0}, \bar{1}\}$
是一个二元域. 证明:
\[
    f(x) = x^n + a_1x^{n - 1} + \cdots + a_{n - 1}x + a_n \in \mathbb{F}_2[x]
\]
没有一次因子(即不被一次多项式整除)
\(
    \Leftrightarrow a_n\left(1 + \sum_{i = 1}^n a_i\right) \neq 0.
\)
写出$\mathbb{F}_2[x]$中所有次数不超过$3$的所有不可约多项式.
\end{problem}

\begin{solution}
    
\end{solution}

\begin{problem}
    设$p$是素数, $\mathbb{Z} \to \mathbb{F}_p = \mathbb{Z}/(p)\mathbb{Z},~a \mapsto \bar{a}$,
是商同态. 证明:
\begin{enumerate}[(1)]
    \item 映射
\[
    \phi_p:\mathbb{Z}[x] \to \mathbb{F}_p[x],\quad f(x) = \sum_{i = 1}^n a_ix^i \mapsto \bar{f}(x) = \sum_{i = 1}^n \bar{a}_ix^i
\]
是环同态;
    \item 对于首项系数为$1$的多项式$f(x) \in \mathbb{Z}[x]$,
如果存在素数$p$使$\bar{f}(x)$在$\mathbb{F}_p[x]$中不可约, 
则$f(x)$在$\mathbb{Z}[x]$中也不可约.
\end{enumerate}
\end{problem}

\begin{solution}
    
\end{solution}

\begin{problem}
    设$R, A$是两个环, $C(A) \subset A$是$A$的中心,
$\psi:R \to C(A)$是一个环同态. 证明:$\forall u \in A$,
存在唯一环同态$\psi_u:R[x] \to A$满足:
\[
    \psi_u(x) = u,\quad \psi_u(a) = \psi(a) \quad (\forall a \in R).
\]
所以, $\forall f(x) = a_nx^n + a_{n - 1}x^{n - 1} + \cdots + a_1x + a_0 \in R[x]$,
它在$\psi_u$下的像
\[
    \psi_u(f(x)) = \psi(a_n)u^n + \psi(a_{n - 1})u^{n - 1} + \cdots + \psi(a_1)u + \psi(a_0) \in A
\]
称为$f(x)$在$u \in A$的取值, 记为$f(u) \defeq \psi_u(f(x))$.
\end{problem}

\begin{solution}
    
\end{solution}

\begin{problem}
    设$R$是一个交换环, $f(x) \in R[x]$. 证明:$f(x)$是环$R[x]$
中的零因子当且仅当存在$0 \neq r \in R$使得$r \cdot f(x) = 0$.
\end{problem}

\begin{solution}
    
\end{solution}

\begin{problem}
    证明多项式$f(x) = x^4 - 10x^2 + 1$在$\mathbb{Z}[x]$中不可约, 
但是对任意的素数$p$, 它在$\mathbb{F}_p[x]$中总是可约的.
\end{problem}

\begin{solution}
    
\end{solution}

\begin{problem}
    设$f(x) \in \mathbb{R}(x)$是一个有理函数. 如果对任意
整数$m \in \mathbb{Z}$必有$f(m) \in \mathbb{Z}$,
试证明$f(x)$必为多项式. 这样的$f(x)$是否必为有理系数多项式?
请证明你的结论.
\end{problem}

\begin{solution}
    
\end{solution}
\subsection{教材p48-p49}

\begin{problem}
    设$F$是一个域, $R = F[x_1, x_2, \cdots, x_n]$, 令$R_m \subset R$表示所有$m$次齐次多项式的集合(并上零多项式). 证明: $R_m$是域$F$上的$\binom{m + n - 1}{m}$维向量空间.
\end{problem}

\begin{proof}
    设$f \in R_m$, 根据$R_m$的定义, $f$可以写成
    \[
        f = \sum_{i_1 + i_2 + \cdots + i_n = m} a_{i_1i_2 \cdots i_n}x_1^{i_1}x_2^{i_2} \cdots x_n^{i_n}
    \]
    $a_{i_1i_2 \cdots i_n} \in F$允许为$0$, $i_k \geqslant 0,\, 1 \leqslant k \leqslant n$. 那么$f$的表达式中共有$\binom{m + n - 1}{m}$项. 记$N = \binom{m + n - 1}{m}$, $I = \{(i_1, i_2, \cdots, i_n) \in \mathbb{N}^n \mid i_1 + i_2 + \cdots + i_n = m\}$. 因此映射
    \[
        R_m \to F^{N},\quad f \mapsto (a_{i_1i_2 \cdots i_n})_{(i_1, i_2, \cdots, i_n) \in I}
    \]
    是(线性)同构.
\end{proof}

\begin{problem}
    证明: $f(x_1, x_2, \cdots, x_n) \in F[x_1, x_2, \cdots, x_n]$是$m$次齐次多项式当且仅当$f(tx_1, tx_2, \cdots, tx_n) = t^mf(x_1, x_2, \cdots, x_n)$, ($t$是一个新的不定元).
\end{problem}

\begin{proof}
    "$\implies$"这个方向提出公因式$t^m$即可, 下证"$\impliedby$":

    由于$f$可以唯一表示成齐次多项式的和, 即
    \[
        f = f_0 + f_1 + \cdots + f_k
    \]
    其中$k$是$f$的最高次数. 那么有
    \[
        f(tx_1, tx_2, \cdots, tx_n) = f_0(x_1, \cdots, x_n) + tf_1(x_1, \cdots, x_n) + \cdots + t^kf_k(x_1, \cdots, x_n)
    \]
    这是一个$F[x_1, x_2, \cdots, x_n]$上的关于$t$的多项式. 若有$f(tx_1, tx_2, \cdots, tx_n) = t^mf(x_1, x_2, \cdots, x_n)$, 对比系数知$f = f_m$.
\end{proof}

\begin{problem}
    设$F$是一个域, $K \,\red{\supseteq}\, F$是$F$的一个扩域, 试证明: $a \in K$是多项式$f(x) \in F[x]$的重根$\Leftrightarrow f(a) = 0, f'(a) = 0$.
\end{problem}

\begin{proof}
    "$\implies$"的部分按定义直接验证, 下证"$\impliedby$":

    由$f(a) = 0$, 可以得到$f(x) = (x - a)f_1(x)$. 那么$f'(x) = f_1(x) + (x - a)f'_1(x)$. 由$f'(a) = f_1(a) = 0$, 得到$f_1(x) = (x - a)f_2(x)$. 因此$f(x) = (x - a)^2f_2(x)$, 即$a$是重根.
\end{proof}

\begin{problem}
    设$F$是一个无限域, $f(x_1, x_2, \cdots ,x_n) \in F[x_1, x_2, \cdots,x_n]$是一非零多项式. 试证明: 存在$a_1, a_2, \cdots, a_n \in F$, 使$f(a_1, a_2, \cdots, a_n) \neq 0$.
\end{problem}

\begin{proof}
    对$n$归纳.
    
    $n = 1$时, $f(x_1)$至多有$\deg(f)$个根, 由$F$无限, 存在$a_1 \in F$使得$f(a_1) \neq 0$. 现假设结论对$n$成立, 考虑多项式
    \[
        f(x_1, \cdots, x_n, x_{n + 1}) \in F[x_1, \cdots, x_n, x_{n + 1}] = F[x_1, \cdots, x_n][x_{n + 1}].
    \]
    从而
    \[
        f(x_1, \cdots, x_n, x_{n + 1}) = c_m(x_1, \cdots, c_n)x_{n + 1}^m + \cdots + c_0(x_1, \cdots, x_n)
    \]
    其中$c_m(x_1, \cdots, x_n) \neq 0$. 由归纳假设存在$(a_1, \cdots, a_n) \in F^n$使得$c_m(a_1, \cdots, a_n) \neq 0$, 那么对多项式$g(x_{n + 1}) = f(a_1, \cdots, a_n, x_{n + 1}) \in F[x_{n + 1}]$使用$n = 1$的结论即可.
\end{proof}

\begin{problem}\label{ex:2.4.5}
    设$\psi:R \to A$是环同态, $u = (u_1, u_2, \cdots, u_n) \in A^n$满足:
    \[
        u_iu_j = u_ju_i,\quad u_i\psi(a) = \psi(a)u_i \quad (\forall a \in R, 1 \leqslant i, j \leqslant n).
    \]
    请直接验证取值映射$\psi_u:R[x_1, x_2, \cdots, x_n] \to A$,
    \[
        f = \sum_{i_1i_2\cdots i_n} a_{i_1i_2\cdots i_n}x_1^{i_1}x_2^{i_2}\cdots x_n^{i_n} \mapsto \psi_u(f) \defeq \sum_{i_1i_2\cdots i_n} \psi(a_{i_1i_2\cdots i_n})u_1^{i_1}u_2^{i_2}\cdots u_n^{i_n},
    \]
    是一个环同态.
\end{problem}

\begin{proof}
    参考\ref{ex:2.3.7}, 实际上这题是把条件减到了最弱的情况, 取定的$n$个$A$中的$u_1, u_2, \cdots, u_n$, 只需要它们互相之间是交换的且和所有$\psi(a)$也是交换的(也就是说$\forall i,\, u_i \in C(\psi(R))$, 中心化子, 见\ref{ex:1.2.4}), 那么映射
    \[
        \psi_u:R[x_1, \cdots, x_n] \to A,\quad x_i \mapsto u_i,\,  \psi_u|_R = \psi
    \]
    就是环同态. 交换的条件是用在保持乘法上, 保$1$是平凡的, 保加法只需要分配律. 但要注意, 此时不能说$A$是$R$-代数, 因为按定义是要求任意给定$u_1, \cdots, u_n$, $\psi_u$都是同态, 才能说$A$是一个$R$-代数.
\end{proof}

\begin{problem}\label{ex:2.4.6}
    设$K$是一个域, $A = \{(a_{ij})_{n \times n}|a_{ij} \in K[\lambda]\}$是$n$阶$\lambda$-矩阵环, $u = \lambda \cdot I_n \in A$表示对角线上全为$\lambda$的矩阵. 试证明: 如果$R = M_n(K),\, \psi:R \to R$是恒等映射, 则取值映射$\psi_u:R[x] \to A$是一个环同构.
\end{problem}

\begin{proof}
    按定义$A = M_n(K[\lambda])$, 由于$K \subseteq K[\lambda]$, 所以自然有$R = M_n(K) \subseteq M_n(K[\lambda]) = A$. 但事实上$A = R[\lambda]$, 在\ref{ex:1.2.8}可以看到这一点. 但是对一个矩阵$B \in M_n(K)$, $B\lambda$和$B(\lambda \cdot I_n)$是一样的. 因此$A = R[\lambda \cdot I_n] = R[u]$. $u$和$\lambda$, $x$一样是和$R$无关的变量, 所以只是换了个字母而已, 那么$\psi_u$自然是是同构.
\end{proof}

\begin{problem}\label{ex:2.4.7}
    设$R$是一个无零因子的非交换环, $\psi:R \to R$是恒等映射. 证明存在$u \in R$使得$\psi_u:R[x] \to R$, $f(x) \mapsto f(u)$, 不是一个映射.
\end{problem}
    
\begin{proof}
    由非交换性知存在$u, v \in R$使得$uv \neq vu$. 而$R[x]$关于$x$是交换的, 所以可以取$f(x) = vx = xv$. 那么带入$u$, 有$uv$和$vu$两个值, 因此$\psi_u$在$f(x)$处不是良定义的, $\psi_u$不是一个映射.
\end{proof}

\begin{problem}\label{ex:2.4.8}
    设$K$是一个域, $M_m(K)$是$m$-阶矩阵环, $\psi:K \to M_m(K)$定义为$\psi(a) = a \cdot I_{m}$(对角线元素为$a$的数量矩阵). 令
    \[
        u = (A, B) \in M_m(K) \times M_m(K),\quad AB \neq BA,
    \]
    试证明$\psi_u:K[x_1, x_2] \to M_m(K),\, f(x_1, x_2) \mapsto f(A, B)$, 不是一个映射.
\end{problem}

\begin{proof}
    和上题是类似的, 用于赋值的$A$和$B$是非交换的, 但$x_1$和$x_2$是交换的. 所以取$f(x_1, x_2) = x_1x_2 = x_2x_1$即可.
\end{proof}

\begin{remark}
    现在把涉及到多项式环的题目放在一起看, \ref{ex:2.3.7}, \ref{ex:2.4.5}, \ref{ex:2.4.6}, \ref{ex:2.4.7}, \ref{ex:2.4.8}.
    
    我们希望“多项式”可以满足我们一直以来的直觉, 其中最重要的一条应该是可以赋值, 也就是说我们希望一个多项式同时也是一个多项式函数. “赋值”这个操作在\ref{ex:2.3.7}解释为由环同态$\psi:R \to A$诱导的唯一的同态$\psi_u:R[x] \to A$, $\psi_u$的含义就是代入$u$, 也就是说此时多项式$f(x)$确实是一个函数
    \[
        f:R \to A,\quad u \mapsto f(u) = \psi_u(f(x)).
    \]
    可以看到交换环的条件在多项式里是很重要的, 没有交换环, 多项式就不一定是函数了, \ref{ex:2.4.7}和\ref{ex:2.4.8}分别为一元和多元的反例.
    
    $R[x_1, x_2, \cdots, x_n]$要求所有未定元$x_1, x_2, \cdots, x_n$是两两交换, 以及所有$x_i$要和$R$中所有元素交换. 可以看到$R$是交换环等价于$R[x]$是交换环, 自然也等价于$R[x_1, x_2, \cdots, x_n]$是交换环. $R$是交换环的时候, $R[x_1, x_2, \cdots, x_n]$的结构已经很清楚了, 就是自由交换$R$-代数, 它满足和其他自由对象类似的泛性质, 正是这个泛性质保证了赋值的唯一性, 从而多项式函数才是一个良定义的东西.

    \ref{ex:2.4.5}虽然减弱了条件, 但也失去了一般性, 所以$R$非交换的时候, 就没有那么好的泛性质了. 这也是为什么在定义$R$-代数的时候需要要求$R$是一个交换环.

    而\ref{ex:2.4.6}其实有更深刻的含义. 泛性质有很多, 某个特定对象的泛性质都是“对任意的一些态射, 存在唯一的态射使得图表交换”这种形式. 这其实是和Yoneda Lemma, representable functor, limit等有关系. 这种对象其实上是某个新的范畴里的final object(或者initial object, 这俩是对偶的, 就差一个反范畴). 而final object的定义里就是该范畴里一个特殊的对象: 任意对象到final object存在唯一的态射. 正是这个存在唯一, 保证了final object在同构的意义下是唯一的. 所以当有两个东西满足同一个泛性质, 它们俩一定是同构的.
\end{remark}
\clearpage
\section{域扩张}
\subsection{教材p52-54}

\begin{problem}
    设$K$是特征零的域, $f(x) \in K[x]$是次数大于零的首项系数为$1$的多
项式, $d(x) = (f(x), f'(x))$是$f(x)$与$f'(x)$的最大公因子. 令
\[
    f(x) = d(x) \cdot g(x).
\]
证明:$g(x)$与$f(x)$有相同的根且$g(x)$没有重根.
\end{problem}

\begin{solution}
    
\end{solution}

\begin{problem}
    设$K \subset L$是域扩张, $\alpha \in L$是域$K$上的代数元.
令$K[x] \xrightarrow{\psi_{\alpha}}L$,
$f(x) \mapsto f(\alpha)$, 表示多项式在$x = \alpha$的取值映射.
试证明:
\begin{enumerate}[(1)]
    \item $\ker(\psi_\alpha)$由极小多项式$\mu_\alpha(x)$生成;
    \item $\psi_\alpha$诱导了域同构$\mathbb{K}[x]/(\mu_\alpha(x)) \cong K[\alpha]$.
\end{enumerate}
\end{problem}

\begin{solution}
    
\end{solution}

\begin{problem}
    设$E = \mathbb{Q}[u], u^3 - u^2 + u + 2 = 0$.
试将$(u^2 + u + 1)(u^2 - u)$和$(u - 1)^{-1}$表示成
$au^2 + bu + c (a, b, c \in \mathbb{Q})$的形式.
\end{problem}

\begin{solution}
    
\end{solution}

\begin{problem}
    求$[\mathbb{Q}[\sqrt2, \sqrt3]:\mathbb{Q}]$
(提示:证明$\left[\mathbb{Q}[\sqrt2, \sqrt3]:\mathbb{Q}[\sqrt3]\right] = 2)$.
\end{problem}

\begin{solution}
    
\end{solution}

\begin{problem}
    设$p$是一个素数, $z \in \mathbb{C}$满足$z^p = 1$
且$z \neq 1$, 试证明$[\mathbb{Q}[z]:\mathbb{Q}] = p - 1$.
\end{problem}

\begin{solution}
    
\end{solution}

\begin{problem}
    证明:
\begin{enumerate}[(1)]
    \item $U_n = \{z \in \mathbb{C} \mid z^n = 1\}$是一个循环群;
    \item $z = \cos \frac\pi6 + i\sin \frac\pi6$是$U_{12}$的一个生成元,
但$[\mathbb{Q}[z]:\mathbb{Q}] = 4$;
    \item 求$z = \cos \frac\pi6 + i\sin \frac\pi6$在$\mathbb{Q}$上的极小多项式.
\end{enumerate}
\end{problem}

\begin{solution}
    
\end{solution}

\begin{problem}
    设$E = K[u]$是一个代数扩张, 且$u$的极小多项式的次数是奇数. 证
明:$E = K[u^2]$.
\end{problem}

\begin{solution}
    
\end{solution}

\begin{problem}
    设$E_1, E_2$是域扩张$K \subset L$的中间域
(即:$K \subset E_i \subset L)$, 且$[E_i:K] < +\infty$.
令$E = K[E_1,E_2] \subset L$是由$E_1, E_2$生成的子域.
证明:
\[
    [E:K] \leqslant [E_1:K] \cdot [E_2:K].
\]
\end{problem}

\begin{solution}
    
\end{solution}

\begin{problem}
    设$K \subset L$是代数扩张, $E \subset L$是中间子环
(即:$K \subset E \subset L)$. 证明:$E\subset L$必为子域
(所以任何有限扩张$K \subset L$的中间子环必为域).
\end{problem}

\begin{solution}
    
\end{solution}

\begin{problem}
    设$L = K(u)$, $u$是$K$上的超越元, $E \neq K$
是$K\subset L$的中间域. 证明:$u$是$E$上的代数元.
\end{problem}

\begin{solution}
    
\end{solution}

\begin{problem}
    设$p$是素数, $K \subset L$是$p$次扩张. 证明:
$K \subset L$必为单纯扩张(即:存在$u \in L$, 使$L = K[u]$).
\end{problem}

\begin{solution}
    
\end{solution}

\begin{problem}
    设域扩张 $K \subset L$满足条件:
\begin{enumerate}[(1)]
    \item $[L:K] < +\infty$;
    \item 对任意两个中间域$K \subset E_1 \subset L,\, K \subset E_2 \subset L$,
必有$E_1 \subset E_2$或者$E_2 \subset E_1$.
\end{enumerate}
证明:$K \subset L$必为单纯扩张(即:存在$u \in L$,使$L = K[u])$.
\end{problem}

\begin{solution}
    
\end{solution}

\begin{problem}
    设$\alpha = 2 + \sqrt[3]{2} + \sqrt[3]{4}$,
给出一个首项系数为$1$的最低次数的多项式
$f(x) \in \mathbb{Q}[x]$使$f(\alpha) = 0$.
\end{problem}

\begin{solution}
    
\end{solution}

\begin{problem}
    设$K = \mathbb{Q}[\sqrt[3]{3}]$,证明:$x^5 - 5$在$K[x]$中不可约.
\end{problem}

\begin{solution}
    
\end{solution}

\begin{problem}
    设$k$是特征$p > 0$的域, $x, y$是$k$上的代数无关元.
令$K = k(x^{p}, y^{p})$, $L = k(x, y)$. 
试证明$[L:K] = p^{2}$.
\end{problem}

\begin{solution}
    
\end{solution}
\subsection{教材p59}

\begin{problem}
    解释说明$3^\circ$角可以由尺规作出, 但是$1^\circ$角不可作.
\end{problem}

\begin{proof}
    事实上正五边形是可以由尺规作图得到的, 因此$54^\circ$是可构造的, 又由于$60^\circ$是可构造的, 故$6^\circ$可构造, 进而$3^\circ$可构造.
\end{proof}

\begin{problem}
    设$\zeta_{17} = \cos(2\pi/17) + i\sin(2\pi/17)$, $L = \mathbb{Q}[\zeta_{17}]$. 请利用高斯关于$\cos(2\pi/17)$的公式写出$\mathbb{Q} \,\red{\subseteq}\, L$的中间域使$L = \mathbb{Q}[\zeta_{17}]$成为$\mathbb{Q}$上的一个二次根塔.
\end{problem}

\begin{proof}
    记$\alpha = \sqrt{17}$. 根据教材p59页的公式, 先做二次扩张$\mathbb{Q} \subseteq \mathbb{Q}[\alpha]$, 得到
    \[
        \cos(\frac{2\pi}{17}) = -\frac{1}{16} + \frac{1}{16}\alpha + \frac{1}{16}\sqrt{2\alpha^2 - 2\alpha} + \frac{1}{8}\sqrt{\alpha^2 + 3\alpha - \sqrt{2\alpha^2 - 2\alpha} - 2\sqrt{2\alpha^2 + 2\alpha}}
    \]
    那么令$\beta = \sqrt{2\alpha^2 - 2\alpha}$, 得到$\mathbb{Q} \subseteq \mathbb{Q}[\alpha] \subseteq \mathbb{Q}[\alpha, \beta]$. 注意到
    \[
        \frac{\alpha}{\beta} = \frac{\alpha}{\sqrt{2\alpha^2 - 2\alpha}} = \frac{\alpha\sqrt{2\alpha^2 + 2\alpha}}{\sqrt{4\alpha^4 - 4\alpha^2}} = \frac{\sqrt{2\alpha^2 + 2\alpha}}{2 \cdot \sqrt{\alpha^2 - 1}} = \frac{1}{8}\sqrt{2\alpha^2 + 2\alpha} 
    \]
    因此有
    \[
        \cos(\frac{2\pi}{17}) = -\frac{1}{16} + \frac{1}{16}\alpha + \frac{1}{16}\beta + \frac{1}{8}\sqrt{\alpha^2 + 3\alpha - \beta - 2 \cdot \frac{8\alpha}{\beta}}
    \]
    那么令$\gamma = \sqrt{\alpha^2 + 3\alpha - \beta - 2 \cdot \frac{8\alpha}{\beta}}$, 就有$\cos(\frac{2\pi}{17}) \in \mathbb{Q}[\alpha, \beta, \gamma]$. 最后只需构造$\sin(\frac{2\pi}{17}) = \sqrt{1 - \cos^2(\frac{2\pi}{17})}$, 因此令$\delta = \sqrt{1 - \cos^2(\frac{2\pi}{17})}$即可. 那么
    \[
        \mathbb{Q} \subseteq \mathbb{Q}[\alpha] \subseteq \mathbb{Q}[\alpha, \beta] \subseteq \mathbb{Q}[\alpha, \beta, \gamma] \subseteq \mathbb{Q}[\alpha, \beta, \gamma, \delta] = L
    \]
    是一个二次根塔, 因为由\ref{ex:3.1.5}, $L = \mathbb{Q}[\zeta_{17}] \implies [L:K] = 16 = 2^4$, 而$\zeta_{17} \in \mathbb{Q}[\alpha, \beta, \gamma, \delta], \left[\mathbb{Q}[\alpha, \beta, \gamma, \delta]:\mathbb{Q}\right]\leqslant 2^4$, 从而只能取等号.
\end{proof}
\subsection{教材p64}

\begin{problem}\label{ex:3.3.1}
    设$f(x) = x^2 + ax + b \in K[x]$不可约, $E = K[u_1]$(其中$f(u_1) = 0$) 证明: $E$必包含$f(x) = 0$的另一个根(所以$E$是$f(x)$的分裂域).
\end{problem}

\begin{proof}
    由于$u_1 \in E$是$f(x)$的根, 因此在$E[x]$中有分解$f(x) = (x - u_1)f_1(x)$. 而$\deg(f) = 2$, 故只能是$\deg(f_1) = 1$, 即$f_1 = x - u_2$, $u_2 \in E$自然是$f(x)$的另一个根.

    或设$f(x)$的分裂域是$E'$, 令$f$的另一个根为$u_2 \in E'$, 则有$u_1 + u_2 = a \in K \subseteq E$. 而$u_1 \in E$, 因此$u_2 \in E$, 即$E' = E$.
\end{proof}

\begin{remark}
    若要严谨一点, 则不能在$E$中直接使用韦达定理, 因为$u_2 \in E$是要证的结论. 韦达定理实际上是$f$在其分裂域可以分解成一次因式的乘积(即分裂), 再对比系数得到的结论. 而按分裂域的定义可知它是使得$f$分裂的最小扩域. 那么直接使用韦达定理是在用结论证结论. 不过由于代数闭包总存在且唯一(见\ref{ex:3.3.2}和\ref{ex:3.3.6}), 我们总能把任意多项式分解成一次多项式的乘积, 所以直接使用事实上是没问题的.

    这题也告诉我们, 二次扩张都是正规扩张.
\end{remark}

\begin{problem}\label{ex:3.3.2}
    设$f(x) = x^3 - 2 \in \mathbb{Q}[x], u_1 = \sqrt[3]{2}$. 证明: $E = \mathbb{Q}[u_1]$不包含$f(x) = 0$的其他两个根.
\end{problem}

\begin{proof}
    教材例3.3.4.

    由于$\mathbb{Q} \subseteq \mathbb{C}$, 而$\mathbb{C}$是代数闭域, 我们可以把所有根都明确的写出来. $x^3 - 2$的根为$\alpha_k = \sqrt[3]{2}e^\frac{2k\pi i}{3} = \sqrt[3]{2}\zeta_3^k$, $k = 0, 1, 2$, $u_1 = \alpha_0$. 而$\mathbb{Q}[u_1] \subseteq \mathbb{R}$, $\alpha_{1,2} \in \mathbb{C} \setminus \mathbb{R}$.
\end{proof}

\begin{remark}
    借此补充代数扩张的一个结论, 任何域在同构的意义下都有唯一的代数闭包. 这个结果的证明分为两部分, 一是存在性, 二是\ref{ex:3.3.6}提到的延拓.

    \begin{defstar}
        若域$K$满足任意次数大于1的多项式$f(x) \in K[x]$在$K$中都有根, 我们称$K$是一个代数闭域. 根据教材的定义2.4.2, $K$是代数闭域等价于$K[x]$中的不可约多项式都是一次多项式. 即$f(x)$总能分解成一次多项式的乘积.
    \end{defstar}
    由定义, $K$是代数闭域意味着$K$无法再做非平凡的代数扩张了. 若有代数扩张$K \subseteq L$且$[L:K] > 1$, 则存在$\alpha \in L \setminus K$在$K$上代数, 即存在非零多项式$f(x) \in K[x]$使得$f(\alpha) = 0$, 而$K$是代数闭域, $f(x)$的所有根都在$K$里, 这就矛盾了. 换句话说, 代数闭域做代数扩张只能得到它自己. 反过来也是对的, 若$K$没有非平凡代数扩张, 且有次数大于1的不可约多项式, 那根据\ref{ex:3.1.2}就能做真代数扩张, 矛盾.

    \begin{defstar}
        设域扩张$K \subseteq L$, 考虑所有的代数元
        \[
            E = \{\alpha \in L \mid \alpha \text{ 在}K \text{ 上代数}\}
        \]
        由教材的推论3.1.1可知$E$是一个中间域, 且$K \subseteq E$是代数扩张. $E$称为$K$在$L$中的(相对)代数闭包.
    \end{defstar}
    比如对于扩张$\mathbb{Q} \subseteq \mathbb{R}$, 这里的$E$就是所有的实代数数. 把$\mathbb{R}$换成$\mathbb{C}$, $E$就是所有的复代数数, 也就是$\mathbb{Q}$的代数闭包, 一般用$\overline{\mathbb{Q}}$表示.

    相对代数闭包$E$在$L$内没有非平凡代数扩张, 即若$E \subseteq E' \subseteq L$且$E \subseteq E'$是代数扩张, 则$E = E'$. 证明这个结论需要一个很基本的定理.

    \begin{thmstar}
        代数扩张的代数扩张仍是代数扩张, 即代数扩张是可以传递的. \cite{2009近世代数引论}p105, \cite{lang2012algebra}p228. 即对域扩张$K \subseteq E \subseteq L$, $L/K$是代数扩张$\iff E/K$和$L/E$都是代数扩张.
    \end{thmstar}

    那么$E'/E$代数, $E/K$代数, 就有$E'/K$代数. 但根据$E$的定义是所有$K$上代数元构成的中间域, 因此$E' \subseteq E$, 所以$E' = E$.

    \begin{defstar}
        设$K \subseteq L$是代数扩张, 若$L$是代数闭域, 称$L$是$K$的(绝对)代数闭包. 一般用记号$\overline{K}$表示. (所以教材定理3.3.2的记号容易引起误解)
    \end{defstar}
    一般说代数闭包默认指绝对代数闭包.

    \begin{propstar}
        设$K \subseteq L$是域扩张, $L$是代数闭域, $E$是$K$在$L$中的相对代数闭包, 则$E = \overline{K}$.
    \end{propstar}
    只需证明这样得到的相对代数闭包是一个代数闭域, 注意到次数大于1的多项式$f(x) \in E[x] \subseteq L[x]$, 而$L$是代数闭的, 因此$f(x) = (x - \alpha_1) \cdots (x - \alpha_n)$, $\alpha_1, \cdots, \alpha_n \in L$在$E$上代数, 自然就在$K$上代数, 按$E$的定义就有$\alpha_1, \cdots, \alpha_n \in E$. 从而$f(x) \in E[x]$总能在$E$上分解称一次多项式的乘积.

    \begin{propstar}
        对任意域$K$, 存在代数闭域$L$使得$K \subseteq L$. \cite{lang2012algebra}\S V.2 Theorem 2.5
    \end{propstar}
    若该命题成立, 那么根据上面的讨论, 任何域都存在代数闭包, 唯一性见\ref{ex:3.3.6}.

    这个命题的证明是构造性的, 构造方法属于Artin, 需要用到\ref{ex:2.1.6}注记里补充的命题. 基本的思路就是构造一个域扩张$K \subseteq K_1$使得$K[x]$中所有非常数多项式在$K_1$中都有至少一个根, 这个操作做可数次之后就能得到一个代数闭域. 类似\ref{ex:2.3.2}和\ref{ex:3.1.2}中说的那样, 令$S = \{X_f \mid f \in K[x] \setminus K\}$, 即用$K[x]$里的非常数多项式来编号, 得到一个无穷的未定元构成的集合, 然后考虑多项式环$K[S]$. 此时记$K[S]$的一个理想$I = (f(X_f))_{f \in K[x] \setminus K} = \sum_{f \in K[x] \setminus K} (f(X_f))$, 即所有这种形式的$K[S]$里的多项式生成的理想(\ref{ex:2.1.6}的注记), 如果商掉这个理想, 那么和\ref{ex:2.3.2}一样, $\overline{X_f}$就是$f$的一个根, 但$I$不一定是极大理想, 因此需要用到\ref{ex:2.1.6}注记里补充的命题(新增加的), 即考虑$I \subseteq \mathfrak{m}$, 其中$\mathfrak{m}$是极大理想. 不过需要先验证$I \neq (1)$, 这是容易的, 若$I = (1)$, 意味着$1 = \sum_{i = 1}^{n} a_if_i(X_{f_i})$, 而这是不可能的, 因为我们可以用$n$次\ref{ex:3.1.2}得到扩张$K \subseteq E$让这里的$f_i$都有根, 赋值(这里用的是多元的\ref{ex:2.4.5})之后就得到$1 = 0$, 矛盾. 因此这样我们得到了$K_1 = K[S]/\mathfrak{m}$. 然后做可数次$K \subseteq K_1 \subseteq \cdots \subseteq K_i \subseteq \cdots$. 最终得到的代数闭域就是$L = \bigcup_{i = 1}^{\infty} K_i$.
\end{remark}

\begin{problem}
    设$L$是$n$次多项式$f(x) \in K[x]$的分裂域, 证明: $[L:K] \leqslant n!$.
\end{problem}

\begin{proof}
    对$n$归纳.

    $n = 1$或$f$已经在$K$上分裂, 都有$[L:K] = 1$. 假设结论对$n$成立, 现考虑$\deg(f) = n + 1$, 且$f$在$K$上不分裂, 那么存在$f$的不可与因子$g$满足$\deg(g) > 1$(否则$f$在$K$上分裂). 设$u \in L$是$g(x)$的一个根, 由\ref{ex:3.1.2}, $K[u] \cong K[x]/(g(x))$是中间域, $[K[u]:K] = \deg(g) \leqslant \deg(f) = n + 1$, 且在$K[u][x]$上有分解$f(x) = (x - u)h(x)$, $\deg(h) = n$. 由归纳假设, 此时$L$是$n$次多项式$h(x) \in K[u][x]$的分裂域, 有$[L:K[u]] \leqslant n!$, 从而$[L:K] = [L:K[u]] \cdot [K[u]:K] \leqslant (n + 1)!$.
\end{proof}

\begin{problem}\label{ex:3.3.4}
    构造$x^5 - 2 \in \mathbb{Q}[x]$的一个分裂域$L$, 并求$[L:\mathbb{Q}]$.
\end{problem}

\begin{proof}
    仍使用\ref{ex:3.1.14}分析degree的方法, $x^5 - 2$的根为$\sqrt[5]{2}\zeta_5^k$, $k = 0, 1, 2, 3, 4$. $L = \mathbb{Q}\left[\sqrt[5]{2}\zeta_5^i\right] = \mathbb{Q}\left[\sqrt[5]{2}, \zeta_5\right]$. 借助中间域$\mathbb{Q}[\sqrt[5]{2}]$. 一方面, $\left[\mathbb{Q}[\sqrt[5]{2}]:\mathbb{Q}\right] = 5$(\ref{ex:3.1.14}); 另一方面, $\left[\mathbb{Q}[\zeta_5]:\mathbb{Q}\right] = 4$(\ref{ex:3.1.5}). 因此$5, 4 \mid [L:\mathbb{Q}]$, 由于$(4,5) = 1$, 因此$4 \cdot 5 = 20 \mid [L:\mathbb{Q}]$, 另一方面$x^5 - 2 \in \mathbb{Q}[\zeta_5][x]$仍是$\sqrt[5]{2}$的化零多项式, 又有$[L:\mathbb{Q}] \leqslant 20$, 故$[L:\mathbb{Q}] = 20$.
\end{proof}

\begin{problem}
    确定多项式$x^{p^n} - 1 \in \mathbb{F}_p[x]$在$\mathbb{F}_p$上的分裂域$(n \in \mathbb{N})$.
\end{problem}

\begin{proof}
    特征$p$的域的多项式环上Frobenius(\ref{ex:2.1.2})也是成立的, 故有$x^{p^n} - 1 = x^{p^n} - 1^{p^n} = (x - 1)^{p^n}$. 从而$x^{p^n}$在$\mathbb{F}_p$上分裂, 分裂域即$\mathbb{F}_p$.
\end{proof}

\begin{problem}\label{ex:3.3.6}
    设$L$是可分多项式$f(x) \in K[x]$的一个分裂域, $K \,\red{\subseteq}\, E \,\red{\subseteq}\, L$是任意中间域. 证明: 对任意单同态$\varphi:E \to L$,若$\varphi|_K = \mathrm{id}_K$, 则$\varphi$一定可以延拓成域同构$\overline\varphi:L \to L$.
\end{problem}

\begin{proof}
    按分裂域定义, 存在$\alpha_1, \alpha_2, \cdots, \alpha_m \in L$使得$f(x) = c(x - \alpha_1)(x - \alpha_2) \cdots (x - \alpha_m)$, 且$L = K[\alpha_1, \cdots, \alpha_m]$. 由于$K \subseteq E \subseteq L$, $f(x) \in K[x] \subseteq E[x]$, 将$f(x)$视为$E[x]$中的多项式. $f(x)$仍有这样的分解. 且$L = K[\alpha_1, \cdots, \alpha_m] \subseteq E[\alpha_1, \cdots, \alpha_m] \subseteq L$, 故$L$也是$f(x)$在$E$上的分裂域. 由于$\varphi|_K = \mathrm{id}_K$, 因此$\varphi(f(x)) = f(x)$, 那么$L$也是$f(x)$在$\varphi(E)$上的分裂域. 由教材的定理3.3.2, 由于$f(x)$是可分多项式, 因此$f(x)$在$\varphi(E)$中无重根, 有$|\{\text{域同构 } L \xrightarrow{\psi} L \mid \psi|_E = \varphi\}| = [L:E] > 0$, 即该集合非空, 那么存在同构$\psi:L \to L$使得$\psi|_E = \varphi$.
\end{proof}

\begin{remark}
    接\ref{ex:3.3.2}注记, 这种延拓对代数扩张是都成立的. 

    \begin{propstar}
        已知对任意域$K$存在代数闭域$L$使得$K \subseteq L$. 记$i_K:K \to L$是域嵌入, 那么对任意代数扩张$K \subseteq E$, 存在嵌入$i:E \to L$使得$i|_K = i_K$. 若$E$是代数闭域且$L/i_K(K)$是代数的, 那么$i$是同构. 因此代数闭包在同构的意义下一定唯一. 证明需要Zorn's Lemma. \cite{lang2012algebra} \S V.2 Theorem 2.8. 
    \end{propstar}
    另外也可以参考\cite{2009近世代数引论}p136 引理1, 这个是单代数扩张的版本, 相对简单一些, 如果只考虑有限扩张, 那么用这个版本就够了. 事实上这只是\ref{ex:3.1.2}的进一步解释, 在此基础上加了一个同构让他变成如下的交换图
    \[
        \begin{tikzcd}
            & E                                                              & E'                                          &                                               \\
{K[x]/(\mu_\alpha(x))} \arrow[r, "\sim"] & K(\alpha) \arrow[r, "\varphi"] \arrow[u, "\subseteq", phantom, sloped] & K'(\alpha') \arrow[u, "\subseteq", phantom, sloped] & {K'[x]/(\mu_{\alpha'}(x))} \arrow[l, "\sim"'] \\
            & K \arrow[r, "\eta"] \arrow[u, hook]                            & K' \arrow[u, hook]                          &                                              
        \end{tikzcd}
    \]
    其中$\eta$是域同构, 那么根据教材引理2.3.2, $\eta$可以延拓成同构$\tilde{\eta}:K[x] \xrightarrow{\sim} K'[x]$. 这个同态会把$\alpha$的极小多项式映到$\alpha'$的极小多项式. 这样就有
    \[
        \begin{tikzcd}
            & {K[x]} \arrow[d, "\pi", two heads] \arrow[r, "\tilde{\eta}"]       & {K'[x]} \arrow[d, "\pi'", two heads]                 &            \\
K(\alpha) \arrow[r, "\sim"] \arrow[rrr, "\varphi"', bend right] & {K[x]/(\mu_\alpha(x))} \arrow[r, "\overline{\tilde{\eta}}"] & {K'[x]/(\mu_{\alpha'}(x))} \arrow[r, "\sim"] & K'(\alpha')
        \end{tikzcd}
    \]
    注意到$(\mu_\alpha(x)) \subseteq \ker(\pi' \circ \tilde{\eta})$, 由quotient的泛性质(也就是同态基本定理的推广, \ref{ex:2.1.8}增加的命题)得到$\overline{\tilde{\eta}}$, 而这是域之间的满同态, 故只能是同构, 进而得到同构$\varphi$, 且$\varphi|_K = \eta$. 注意根据我们$\varphi$的构造一定是$\varphi(\alpha) = \alpha'$, 若这里$\eta = \mathrm{id}_K$, 那么$\varphi$就是把$\alpha$换成$\mu_\alpha(x)$的其中一个根.

    这也说明我们在同构的意义下考虑域扩张是可行的. 但对代数闭包而言, 不一定是有限扩张, 比如$\overline{\mathbb{Q}}/\mathbb{Q}$.

    有了代数闭包, 可分多项式的等价定义为: $f(x) \in K[x]$是可分的, 即$f(x)$的不可约因子在$\overline{K}$中(或者说在$f(x)$的分裂域中)无重根. 这也解释了\ref{ex:2.4.3}和\ref{ex:3.1.1}的关系, 特征零的不可约多项式可分, 从而特征零的代数扩张一定是可分扩张.
\end{remark}

\begin{problem}
    令$f(x) = (x^2 - 2)(x^2 - 3)$, $K = \mathbb{Q}[x]/(x^2 - 2) = \mathbb{Q}[u_1]$, 此处$u_1 = \overline{x} \in \mathbb{Q}[x]/(x^2 - 2)$. 试证明: 
    \begin{enumerate}[(1)]
        \item $K$是一个域, 且$x^2 - 3$在$K[x]$中不可约;
        \item $L = K[x]/(x^2 - 3) = K[u_2]$(此处$u_2 = \overline{x} \in K[x]/(x^2 - 3))$是$f(x) = (x^2 - 2)(x^2 - 3)$的分裂域, 且$[L:\mathbb{Q}] = 4$.
    \end{enumerate}
\end{problem}

\begin{proof}
    \begin{enumerate}[(1)]
        \item $K$是域是因为$(x^2 - 2)$是极大理想, 见\ref{ex:3.1.2}和\ref{ex:3.1.14}. $x^2 - 3$在$K$中不可约在\ref{ex:3.1.4}已证.
        \item 根据\ref{ex:3.1.2}和(1), $L = \mathbb{Q}[u_1, u_2]$是域. 且有分解$f(x) = (x - u_1)(x + u_1)(x - u_2)(x + u_2)$, 因此$L$是$f(x)$的分裂域(\ref{ex:3.3.1}). 且有$[L:\mathbb{Q}] = [L:K] \cdot [K:\mathbb{Q}] = 2 \cdot 2 = 4$.
    \end{enumerate}
\end{proof}

\begin{problem}
    设$p \in \mathbb{Z}$是一个素数, $F$是一个域, $c \in F$. 求证: $x^p - c$在$F[x]$中不可约当且仅当$x^p - c$在$F$中无根.
\end{problem}

\begin{proof}
    考虑$x^p - c$的分裂域$E$, 或者直接考虑$F$的代数闭包, 那么有分解$x^p - c = (x - \alpha_1)(x - \alpha_2) \cdots (x - \alpha_p)$. 我们证两次逆否.
    \begin{enumerate}
        \item["$\implies$"] 若$x^p - c$在$F$中有根, 根据教材定义2.4.2, $x^p - c$有一次因式, 可约.
        \item["$\impliedby$"] 若$x^p - c$可约, 按定义有$x^p - c = f(x)g(x)$, 那么不妨设$f(x) = (x - \alpha_1)(x - \alpha_2) \cdots (x - \alpha_n)$, 其中$0 < n < p$, 那么根据Bézout's Identity, 存在$u, v \in \mathbb{Z}$使得$nu + pv = 1$. 记$\alpha = \alpha_1\alpha_2 \cdots \alpha_n \in F$(韦达定理), 注意到$\alpha_i$都是$x^p - c$的根, $\alpha_i^p = c$. 那么$\alpha^p = \alpha_1^p\alpha_2^p \cdots \alpha_n^p = c^n$, 从而$\alpha^{pu} = c^{nu}$, 那么$(\alpha^uc^v)^p = c^{nu}c^{pv} = c$. 这样$\alpha^uc^v$是$x^p - c$的一个根, 且$\alpha \in F$, 因此$\alpha^uc^v \in F$.
    \end{enumerate}
\end{proof}

\begin{problem}
    设$f(x)$是$\mathbb{Q}[x]$中奇数次的不可约多项式, $\alpha$和$\beta$是$f(x)$在$\mathbb{C}$中的两个不同的根. 试证明$\alpha + \beta \notin \mathbb{Q}$且$\alpha\beta \notin \mathbb{Q}$.
\end{problem}

\begin{proof}
    
\end{proof}

\begin{problem}
    设$K = \mathbb{Q}[u]$, $u^3 + u^2 - 2u - 1 = 0$. 验证$\alpha = u^2 - 2$也是多项式$x^3 + x^2 - 2x - 1$的根. 试确定$\mathrm{Gal}(K/\mathbb{Q})$, 并证明: $K$是$\mathbb{Q}$的正规扩张.
\end{problem}

\begin{proof}
    
\end{proof}

\begin{remark}
    \cite{lang2012algebra}\S V.3 Theorem3.3给出了正规扩张的三个等价定义, 其中有一条是用到代数闭包的:

    $L/K$是正规扩张, 当且仅当延拓后的域嵌入$\varphi:L \to \overline{K}$是$L$的自同构, 即$\varphi(L) = L$.
\end{remark}

\begin{problem}
    证明: $\mathbb{Q}[\sqrt[4]{2}]$是$\mathbb{Q}[\sqrt{2}]$的正规扩张, 但不是$\mathbb{Q}$的正规扩张.
\end{problem}

\begin{proof}
    由\ref{ex:3.3.1}, $\mathbb{Q}[\sqrt[4]{2}]/\mathbb{Q}[\sqrt{2}]$是二次扩张, 从而是正规扩张. 另一方面, 和\ref{ex:3.3.2}类似, $\sqrt[4]{2}$在$\mathbb{Q}$上的的极小多项式$x^4 - 2$有非实数根$\sqrt[4]{2}i$的存在, 自然不是正规扩张.
\end{proof}

\begin{problem}
    设$f(x) \in K[x]$不可约, $\mathrm{Char}(K) = p > 0$. 证明: 存在不可约的可分多项式$g(x) \in K[x]$使得$f(x) = g(x^{p^n})$($n$是某个整数). 由此证明$f(x)$在分裂域中的每个根都是$p^n$重根.
\end{problem}

\begin{proof}
    
\end{proof}

\begin{problem}
    设$L = K[\alpha],~\alpha$是多项式$x^d - a \in K[x]$的根. 如果$\mathrm{Char}(K) = 0$, 且$K$包含全部$d$次单位根, 则$K \,\red{\subseteq}\, L$是正规扩张.
\end{problem}

\begin{proof}
    这是\ref{ex:3.3.4}的一般情况. 设$1 = \omega_0, \omega_1, \cdots \omega_{d - 1}$是$x^d - 1$的根, 根据题设, $\omega_i \in L = K[\alpha], 0 \leqslant i < d$. 而$(\omega_i\alpha)^d = \omega^d\alpha^d = 1 \cdot a = a$, 从而$\omega_i\alpha \in L$是$x^d - a$的$d$个根. 因此按定义$L$是$x^d - a$的分裂域. 由\ref{ex:3.3.14}的注记是Galois扩张, 自然是正规扩张.
\end{proof}

\begin{problem}[*]\label{ex:3.3.14}
    设$k$是特征$p > 0$的域, $x, y$是$k$上的代数无关元. 令$K = k(x^p, y^p)$, $L = k(x, y)$. 试证明: 
    \begin{enumerate}[(1)]
        \item $\mathrm{Gal}(L/K) = \{1\}$ (但$[L:K] = p^2)$;
        \item $K \,\red{\subseteq}\, L$有无穷多个中间域;
        \item $K \,\red{\subseteq}\, L$不是单扩张, 即不存在$\alpha \in L$使得$L = K[\alpha]$.
    \end{enumerate}
\end{problem}

\begin{proof}
    这题是\ref{ex:3.1.15}的延续.
    \begin{enumerate}[(1)]
        \item 设$\eta \in \mathrm{Gal}(L/K)$, 只需证$\eta = \mathrm{id}_L$. 由\ref{ex:3.3.6}, $x$是$K$上的代数元, 且$x$的极小多项式是$t^p - x^p$, 因此$\eta(x)$是多项式$t^p - x^p$的根, 而根据\ref{ex:3.1.15}, $t^p - x^p$只有一个$p$重根$x$, 因此$\eta(x) = x$. 同理$\eta(y) = y$, 从而$\eta = \mathrm{id}_L$.
        \item 设$E$是一个非平凡中间域, 由于$[L:K] = [L:E][E:K] = p^2$, 因此只能是$[L:E] = [E:K] = p$. 而形如$E_c = k(x + cy, y^p)$就是非平凡的中间域, 其中$c \in K$而$K$是无穷域. 且$c_1 \neq c_2 \implies E_{c_1} \neq E_{c_2}$. 因此有无穷多个中间域.
        \item 由Frobenius同态可知, $\forall \alpha \in L$, $\alpha^p \in K$, 则$t^p - \alpha^p \in K[t]$是$\alpha$的化零多项式. 从而$[K[\alpha]:K] \leqslant p < p^2$. 因此$K[\alpha] \neq L$.
    \end{enumerate}
\end{proof}

\begin{remark}
    这题教材答案的错误比较严重, $K$并不是完全域, $x^p$不是$K$中任何一个元素的$p$次方, 但$L/K$确实不是一个可分扩张, $x$在$K$上的极小多项式$t^p - x^p$在$K$上不是可分的. 事实上教材的定理3.3.4已经告诉我们完全域的代数扩张一定是可分扩张, 而$L/K$是有限扩张, 自然是代数扩张, 因此教材的答案是前后矛盾的.

    事实上, 完全域应该定义为任意代数扩张都是可分扩张的域, 因此包括所有特征零的域. 在完全域上取分裂域得到的扩张一定是Galois扩张(教材定理3.4.1), 这也是为什么要有完全域这个概念.
\end{remark}
\subsection{教材p67-68}

\begin{problem}
    设$p > 2$是素数, $\alpha \in \mathbb{C}$是
$f(x) = x^{p - 1} + x^{p - 2} + \cdots + x + 1 \in \mathbb{Q}[x]$
的根. 证明:域$L = \mathbb{Q}[\alpha]$的自同构群$G$
是一个$p - 1$阶的循环群.
\end{problem}

\begin{proof}
    
\end{proof}

\begin{problem}
    设$K = \mathbb{Q},\, L = K[\sqrt[3]{2}]$.
证明:$G = \mathrm{Gal}(L/K)=\{1\}$ (所以$L^G = L \neq K$).
如果令$\overline{L} = K[\sqrt[3]{2}, \sqrt{-3}]$,
试证明:$\mathrm{Gal}(\overline{L}/K) \cong S_3$.
并求出中间域$K \subset K[\sqrt{-3}] \subset \overline{L}$
对应的子群$H \subset \mathrm{Gal}(\overline L/K)$,
即:求$H \subset \mathrm{Gal}(\overline L/K)$使得
$\overline{L}^{H} = K[\sqrt{-3}]$.
(提示:$H = \mathrm{Gal}(\overline L/K[\sqrt{-3}]) \cong A_3$.)
\end{problem}

\begin{proof}
    
\end{proof}

\begin{problem}
    设$K \subset L$是有限, 可分, 正规扩张, $G = \mathrm{Gal}(L/K)$.
设
\[
    K = K_0 \subset K_1 \subset K_2 \subset \cdots \subset K_i \subset K_{i + 1} \subset \cdots \subset K_m = L
\]
是一个子域链, 令
\[
    \{1\} = G_m \subset G_{m - 1} \subset G_{m - 2} \subset \cdots \subset G_{i + 1} \subset G_i \subset \cdots \subset G_0 = G
\]
是其对应的子群链, 其中$G_i = \mathrm{Gal}(L/K_i)$. 证明:
\begin{enumerate}[(1)]
    \item $K_i \subset K_{i + 1}$是正规扩张
$\Leftrightarrow \forall \eta \in G_i,\, \eta(K_{i + 1}) = K_{i + 1}$
(提示:应用推论3.3.4).
    \item $\forall\, \eta \in G_i$, 则
$\eta \cdot G_{i + 1} \cdot \eta^{-1} \subset G_i$
是一个子群, 且
\[
    \eta(K_{i + 1}) = L^{\eta G_{i + 1}\eta^{-1}},
\]
此处$\eta \cdot G_{i + 1} \cdot \eta ^{- 1} \defeq \{\eta \cdot x \cdot \eta^{-1} \mid \forall\, x \in G_{i + 1}\}$.
    \item 如果$K_i \subset K_{i + 1}$是正规扩张, $\forall \eta \in G_i$,
令
\[
    \bar{\eta} = \eta|_{K_{i + 1}}:K_{i + 1} \to K_{i + 1},
\]
则$\bar{\eta} \in \mathrm{Gal}(K_{i + 1}/K_i)$,
映射$G_i \overset{\phi}\to \mathrm{Gal}(K_{i + 1}/K_i)$,
$\eta \mapsto \bar{\eta}$, 是满同态, 而且
$\ker(\phi) = G_{i + 1}$.
\end{enumerate}
\end{problem}

\begin{proof}
    
\end{proof}
\clearpage
\section{群论初步}
\subsection{教材p72}

\begin{problem}
    设$G$是一个群, 定义映射$G \xrightarrow\varphi G,\, x \mapsto x^{-1}$.
试证明: $\varphi$是$G$的自同构当且仅当$G$是阿贝尔群.
\end{problem}

\begin{proof}
    
\end{proof}

\begin{problem}
    证明: 子群$H \,\red{\subseteq}\, G$是正规子群当且仅当,
$\forall g \in G$, $gHg^{-1} \,\red{\subseteq}\, H$.
\end{problem}

\begin{proof}
    
\end{proof}

\begin{problem}
    设$G \xrightarrow\varphi G'$是群同态, $K = \ker(\varphi)$
是同态$\varphi$的核. 试证明: 
\begin{enumerate}[(1)]
    \item 对于任意子群$H' \,\red{\subseteq}\, G', H = \varphi^{-1}(H') \,\red{\subseteq}\, G$是子群, 且包含$K$.
    \item 当$\varphi$是满射时, $H' \mapsto \varphi^{-1}(H')$建立了集合
\[
    \Gamma'=\{H' \,\red{\subseteq}\, G' \mid H' \text{ 是子群}\},
\]
与集合$\Gamma = \{H \,\red{\subseteq}\, G \mid H \text{ 是 }G \text{ 的子群, 且 } H \supset K\}$
之间的双射, 此时$H' \,\red{\subseteq}\, G'$是正规子群当且仅当
$\varphi^{-1}(H') \,\red{\subseteq}\, G$是正规子群.
\end{enumerate}
\end{problem}

\begin{proof}
    
\end{proof}

\begin{problem}
    设$H, N$都是$G$的正规子群, 并且$N \subseteq H$.
令$\bar{H} = H/N, \bar{G} = G/N$.
\begin{enumerate}[(1)]
    \item 证明$\bar{H}$是$\bar{G}$的正规子群.
    \item 证明$G/H \cong \bar{G}/\bar{H}$.
\end{enumerate}
\end{problem}

\begin{proof}
    
\end{proof}

\begin{problem}
    设$H \,\red{\subseteq}\, G$是$G$的子群, $K \lhd G$, 试证明: 
\begin{enumerate}[(1)]
    \item $H \cdot K = \{hk \mid \forall h \in H, k \in K\}$
是$G$中包含$H$和$K$的子群;
    \item $H$在商同态$G \to G/K$, ($g \mapsto \bar{g}$)
下的像是$(H \cdot K)/K$;
    \item $\varphi:H \to (HK)/K$, ($\varphi(h) = \bar{h}$)
的核是$H \cap K$;
    \item $\varphi$诱导群同构$H/(H \cap K) \cong (HK)/K$.
\end{enumerate}
\end{problem}

\begin{proof}
    
\end{proof}
\subsection{教材p77}

\begin{problem}
    设群$G = AB$, 其中$A, B$都是$G$的Abel子群(即交换子群), 且$AB = BA$. 令$G^{(1)}$表示$G$的换位子群, 证明: 
    \begin{enumerate}[(1)]
        \item $\forall a, x \in A$, $b, y \in B$, 总有$[x^{-1}, y^{-1}][a, b][x^{-1}, y^{-1}]^{-1} = [a, b]$;
        \item $G^{(1)}$是Abel群.
    \end{enumerate}
\end{problem}

\begin{proof}
    
\end{proof}

\begin{problem}
    证明: 
    \begin{enumerate}[(1)]
        \item $S_{n} = \left\langle (1\:2),\: (1\:3),\: \cdots,\: (1\:n) \right\rangle$, 即$S_n$由对换$(1\:2),\: (1\:3),\: \cdots,\: (1\:n)$生成;
        \item $S_{n}$可由$(1\:2)$和$(1\:2\:3 \cdots n)$生成, 即
        \[
            S_n = \big\langle (1\:2),(1\:2\:3 \cdots n) \big\rangle.
        \]
    \end{enumerate}
\end{problem}

\begin{proof}
    \begin{enumerate}[(1)]
        \item 这是教材推论4.2.2的直接结果, 任意置换总能写成有限个对换的乘积, 而$(i\:j) = (1\:i)(1\:j)(1\:i)$.
        \item 由(1), 只需证明$(1\:3), \cdots, (1\:n)$都可以被$(1\:2)$和$(1\:2 \cdots n)$生成. 事实上$(1\:n) = (1\:2 \cdots n)^{-1}(1\:2)(1\:2 \cdots n)$, $(1\:i) = (1\:(i+1))(i\:(i + 1))(1\:(i + 1))$, 且$(i\:(i + 1)) = (1\:2 \cdots n)^{-(n - i + 1)}(1\:2)(1\:2 \cdots n)^{n - i + 1}, 2 < i < n$(其实就是把$i$和$i + 1$先移到$1$和$2$的位置上, 用$(1\:2)$对换, 再移回去).
    \end{enumerate}
\end{proof}

\begin{problem}
    证明: 循环$\pi = (1\:2 \cdots n) \in S_n$的$k$次幂$\pi^k$是$d$个互不相交的循环之积, 每个循环的长度为$q = \frac nd$, 其中$d = (n, k)$是$n$和$k$的最大公因子.
\end{problem}

\begin{proof}
    
\end{proof}

\begin{problem}
    设$A_n= \{\pi \in S_n \mid \varepsilon_\pi = 1\} \,\red{\subseteq}\, S_n$, 证明: 
    \begin{enumerate}[(1)]
        \item $A_n \lhd S_n$ (即$A_n$是$S_n$的正规子群);
        \item $A_n$由$3$-循环生成, 事实上, $A_n = \left\langle (1\:2\:3),\: (1\:2\:4),\: \cdots (1\:2\:n) \right\rangle$.(提示: 利用$(a\:b) \cdot (b\:c) = (a\:b\:c),\, (a\:b) \cdot (c\:d) = (a\:b) \cdot (b\:c) \cdot (b\:c) \cdot (c\:d)$.)
    \end{enumerate}
\end{problem}

\begin{proof}
    \begin{enumerate}[(1)]
        \item $\pi \mapsto \varepsilon_\pi$实际上是一个群同态$S_n \to \{1, -1\} \cong \mathbb{Z}/2\mathbb{Z}$. 而$A_n$恰好是这个同态的kernel.
        \item 根据提示有$(a\:b)(c\:d) = (abc)(bcd)$, 因此所有的$3$-循环能生成$A_n$, 只需说明任意$3$-循环在$\left\langle (1\:2\:3),\: (1\:2\:4),\: \cdots (1\:2\:n) \right\rangle$中. 我们可以做拆解, 对$i, j, k \neq 1, 2$, 反复用上面的等式凑出来$(1\:2\:m)$.
        \[
        \begin{aligned}
            (1\:j\:k) &= (k\:1)(1\:j) = (k\:1)(1\:2)(1\:2)(2\:k)(2\:k)(1\:j) = (1\:2\:k)(1\:2\:k)(1\:j)(2\:k)\\
            &= (1\:2\:k)(1\:2\:k)(1\:j)(1\:2)(1\:2)(2\:k) = (1\:2\:k)(1\:2\:k)(1\:2\:j)(1\:2\:k)
        \end{aligned}
        \]
        同样的可以凑出
        \[
            (i\:j\:k) = (i\:j)(1\:j)(1\:j)(j\:k) = (1\:i\:j)(1\:j\:k)
        \]
    \end{enumerate}
\end{proof}

\begin{problem}
    群$G$中的两个元素$x, y$称为在$G$中共轭, 如果存在$a \in G$, 使$axa^{-1} = y$. 试证明: 
    \begin{enumerate}[(1)]
        \item $\forall\, \pi \in S_n\, \alpha = (i_1\:i_2 \cdots i_r) \in S_n$有公式
        \[
            \pi \cdot \alpha \cdot \pi^{-1} = (\pi(i_1)\:\pi(i_2) \cdots \pi(i_r)).
        \]
        \item 所有$3$-循环在$S_n$中相互共轭. (所以$S_n$中包含$3$-循环的正规子群必包含$A_n$.)
        \item 如果$n \geqslant 5$, 则所有$3$-循环在$A_n$中相互共轭, 即对于任意$3$-循环$x,y \in A_n$, 存在$a \in A_n$, 使$axa^{-1} = y$.
    \end{enumerate}
\end{problem}

\begin{proof}
    \begin{enumerate}[(1)]
        \item 按定义验证, 若$\pi \cdot \alpha \cdot \pi^{-1}(i) = \pi(\alpha(\pi^{-1}(i)))$. 若$i \notin \{\pi(i_1), \pi(i_2), \cdots, \pi(i_r)\} \iff \pi^{-1}(i) \notin \{i_1, i_2, \cdots, i_r\}$, 则$\pi(\alpha(\pi^{-1}(i))) = \pi(\pi^{-1}(i)) = i$. 反之$i \in \{\pi(i_1), \pi(i_2), \cdots, \pi(i_r)\}$, 有$\pi(\alpha(\pi^{-1}(i))) = \pi(\alpha(i_k)) = (\pi(i_1)\:\pi(i_2) \cdots \pi(i_r))(i)$.
        \item (1)的推论. 若$\alpha$是$3$-循环, 任意的$\pi \in S_n$, $\pi\alpha\pi^{-1}$仍是$3$-循环. 具体来说, 对两个$3$-循环$\alpha_1 = (a_1\:b_1\:c_1)$和$\alpha_2 = (a_2\:b_2\:c_2)$, 则令
        \(
            \pi = 
            \begin{pmatrix} 
                a_1 & b_1 & c_1 & \cdots \\
                a_2 & b_2 & c_2 & \cdots 
            \end{pmatrix}
        \) 
        即可.
        \item 设$x = (i_1\:i_2\:i_3), y = (j_1\:j_2\:j_3)$. 当$n \geqslant 5$时, 由(2), 考虑
        \[
            a_1 =
            \begin{pmatrix} 
                i_1 & i_2 & i_3 & i_4 & i_5 & \cdots \\
                j_1 & j_2 & j_3 & j_4 & j_5 & \cdots 
            \end{pmatrix},
            a_2 =
            \begin{pmatrix} 
                i_1 & i_2 & i_3 & i_4 & i_5 & \cdots \\
                j_1 & j_2 & j_3 & j_5 & j_4 & \cdots 
            \end{pmatrix}
        \]
        则由(2)可知$k = 1, 2$都满足$a_kxa_k^{-1} = y$, 但是$a_2 = a_1(j_4, j_5)$, 即刚好差一个对换, 那么$a_1, a_2$必然一奇一偶, 因此存在$a \in A_n$使得$axa^{-1} = y$.
    \end{enumerate}
\end{proof}

\begin{problem}
    证明: 对任意给定整数$n > 0$, 在同构意义下仅有有限个$n$阶群. (提示: 任意$n$阶群均同构于$S_n$的一个子群.)
\end{problem}

\begin{proof}
    
\end{proof}

\begin{problem}\label{ex:4.2.7}
    证明: 所有$4$阶群$G$都是交换群. 在同构意义下, $G$要么是循环群, 要么同构于下述克莱因$4$元群: 
    \[
        V_4 = \{(1), (12)(34), (13)(24), (14)(23)\} \subseteq S_4.
    \]
    (提示: 如果$x^2 = 1$对$G$中所有元成立, 则$\forall a, b \in G$, 有$abab = 1 \implies ab = b^{-1}a^{-1} = b(b^{-1})^2 \cdot (a^{-1})^2a = ba.$)
\end{problem}

\begin{proof}
    对于这种阶很小的群, 我们可以直接分析$4$阶群的乘法表, 这其实和数独有点像. 乘法表的每一行或每一列是不能有相同元素的, 因为左乘映射是单的$ga = gb \implies a = b$, 右乘也一样.

    设$G = \{e, a, b, c\}$, $e$是单位元. 那么首先有
    \[
    \begin{array}{c|c c c c}
      & e & a & b & c \\
    \hline
    e & e & a & b & c \\
    a & a &  &  &  \\
    b & b &  &  &  \\
    c & c &  &  &  \\
    \end{array}
    \]
    由\ref{ex:1.3.10}, $G$有$2$阶元, 不妨设$a^2 = e$, 则$ab \neq e, a, b$, 只能是$c$, $ac, ba, ca$同理, 得到
    \[
    \begin{array}{c|c c c c}
      & e & a & b & c \\
    \hline
    e & e & a & b & c \\
    a & a & e & c & b \\
    b & b & c &  &  \\
    c & c & b &  &  \\
    \end{array}
    \]
    此时若$b^2 = e$, 则得到
    \[
    \begin{array}{c|c c c c}
      & e & a & b & c \\
    \hline
    e & e & a & b & c \\
    a & a & e & c & b \\
    b & b & c & e & a \\
    c & c & b & a & e\\
    \end{array}
    \]
    否则$b^2 = a$, 得到
    \[
    \begin{array}{c|c c c c}
      & e & a & b & c \\
    \hline
    e & e & a & b & c \\
    a & a & e & c & b \\
    b & b & c & a & e \\
    c & c & b & e & a\\
    \end{array}
    \]
    第二种实际上是$\mathbb{Z}/4\mathbb{Z}$是循环群, 生成元是$b$或者$c$, 自然是交换群. 第一种就是题干中的克莱因$4$元群.
\end{proof}

\begin{remark}
    对于阶很大的群这种方法便不适用了. 事实上若$|G| = p^2$, 则$G$一定是Abel群, 其中$p$是素数. 这是共轭作用得到的分类公式的直接推论. 即教材引理4.2.2证明的中间结果
    \[
        |G| = C(G) + \sum_{O(x) > 1} |O(x)|,\quad |O(x)| = [G:H_x]
    \]
    这里的$H_x$是在共轭作用$g \cdot x = gxg^{-1}$下的稳定子, 又称做$x$的中心化子(所有和$x$交换的元素构成的子群), 和\ref{ex:1.2.4}是类似, 一般记作$C(x)$. 这个等式的每一项都是$|G|$的因子, 因此若$G$不是Abel群, 即$C(G) \neq G$, 那么只能是$|C(G)| = p$. 但这是不可能的. 因为对$x \notin C(G)$, $|C(x)|$也是$p^2$的因子, 而按定义$C(G)$是严格包含于$C(x)$的, $C(G) \subsetneq C(x)$, 这意味着$|C(x)| > |C(G)| = p$, 那么$|C(x)| = p^2$, 这就矛盾了, 因为$x \notin C(G)$, 所以$C(x)$不可能等于$G$.

    对一般的群作用$G \times X \to X$, $X$是有限集, 教材的引理4.5.2事实上可以表示为分类公式(class formula)
    \[
        |S| = |Z| + \sum_{|O(x)| > 1} [G:\mathrm{stab}(x)] = |Z| + \sum_{|O(x)| > 1} |O(x)|.
    \]
    其中$Z$称为该群作用下的不动点集, $x \in Z \iff \mathrm{stab}(x) = G \iff O(x) = \{x\}$. 若$G$是$p$-群, 则有
    \[
        |Z| \equiv |S| \mod p
    \]
\end{remark}

\begin{problem}
    找出交错群$A_4$的所有子群.
\end{problem}

\begin{remark}
    这题是Sylow定理, 应该放到习题4.5.
\end{remark}

\begin{proof}
    
\end{proof}
\subsection{教材p80}

\begin{problem}
    设$G = \langle \alpha \rangle$是$n$阶循环群, 试证明:
\begin{enumerate}[(1)]
    \item $\alpha^m$是$G$的生成元 (即$G = \langle \alpha^m \rangle ) \Leftrightarrow (m, n) = 1$;
    \item 若$\mathbb{Z}_n$表示模$n$的剩余类环, $U(\mathbb{Z}_n)$是它的单位群, 则
\[
    \bar{m} \in U(\mathbb{Z}_n) \Leftrightarrow (m, n)=1;
\]
    \item 设$\mathrm{Aut}(G)$表示群$G$的自同构群, 则$\mathrm{Aut}(G) \cong U(\mathbb{Z}_n)$.
\end{enumerate}
\end{problem}

\begin{proof}
    
\end{proof}

\begin{problem}
    设$F$是一个域, $F^* = F \setminus \{0\}$, 证明乘法群$F^*$的任何有限子群都是
循环群.
\end{problem}

\begin{proof}
    
\end{proof}

\begin{problem}
    设$K$是特征零的域, $L$是多项式$x^n - 1 \in K[x]$的分裂域. 试证明:
$\mathrm{Gal}(L/K)$同构于$U(\mathbb{Z}_n)$的一个子群. 特别地, 
$\mathrm{Gal}(L/K)$总是交换群.
\end{problem}

\begin{proof}
    
\end{proof}
\subsection{教材p84}

\begin{problem}
    设$E$是$x^4-2$在$\mathbb{Q}$上的分裂域.
\begin{enumerate}[(1)]
    \item 试求出$E/\mathbb{Q}$的全部中间域;
    \item 试问哪些中间域是$\mathbb{Q}$的伽罗瓦扩张, 哪些域彼此共轭?
\end{enumerate}
\end{problem}

\begin{solution}
    
\end{solution}

\begin{problem}
    设$K \supset F$是伽罗瓦扩张, $f(x)$是$\alpha \in K$在$F$上的极小多项式, 
\(
    g(x) = \prod_{\sigma \in \mathrm{Gal}(K/F)} \big(x - \sigma(\alpha) \big).
\)
证明: $g(x) \in F[x]$并且存在正整数$n$使得$g = f^n$.
\end{problem}

\begin{solution}
    
\end{solution}

\begin{problem}
    设
$\xi = e^{\frac{2\pi i}{13}}, \alpha = \xi + \xi^4 + \xi^3 + \xi^{12} + \xi^9 + \xi^{10}$,
证明:
\begin{enumerate}[(1)]
    \item $\mathrm{Gal}\left(\mathbb{Q}[\xi]/\mathbb{Q}\right)$同构于乘法群
$\mathbb{F}_{13}^* = \mathbb{F}_{13} \setminus \{0\}$.
    \item $\bigl[\mathbb{Q}[\xi]:\mathbb{Q}[\alpha]\bigr] = 6$.
    \item 求$\alpha$在$\mathbb{Q}$上的极小多项式.
\end{enumerate}
\end{problem}

\begin{solution}
    
\end{solution}

\begin{problem}
    设$p > 2$是素数, $\xi_p = e^{\frac{2\pi i}p},\, \xi_{p^2}$为$p^2$次本原单位根. 
\begin{enumerate}[(1)]
    \item 求$\mathbb{Q}(\xi_p)/\mathbb{Q}$的扩张次数, 并证明
$\mathrm{Gal}(\mathbb{Q}(\xi_p)/\mathbb{Q}) \cong F_p^*$;
    \item 求$\mathbb{Q}(\xi_{p^2})/\mathbb{Q}$的扩张次数, 并确定
$\mathrm{Gal}(\mathbb{Q}(\xi_{p^2})/\mathbb{Q})$
(提示:该群是$(\mathbb{Z}/p^2\mathbb{Z})^*$);
    \item 试确定$\mathbb{Q}(\xi_{p^2})/\mathbb{Q}(\xi_p)$的扩张次数, 并证明这是一个伽罗瓦扩张.
\end{enumerate}
\end{problem}

\begin{solution}
    
\end{solution}

\begin{problem}
    设$\xi_n$是$n$次本原单位根 (即$\xi_n = e^{\frac{2\pi i}n})$.
\begin{enumerate}[(1)]
    \item 证明$\mathbb{Q}(\xi_n)/\mathbb{Q}$是伽罗瓦扩张;
    \item 当$n = 12$时, 求伽罗瓦群$\mathrm{Gal}(\mathbb{Q}[\xi_n]/\mathbb{Q})$;
    \item 设$n > 2$为奇数, 证明$\mathbb{Q}[\xi_n] \cap \mathbb{R} = \mathbb{Q}[\xi_n + \xi_n^{-1}]$.
\end{enumerate}
\end{problem}

\begin{solution}
    
\end{solution}
\subsection{教材p87-p88}
习题4.5中除了4.5.1和4.5.8, 其余都是Sylow定理的应用.

\begin{problem}\label{ex:4.5.1}
    设$\mathrm{Aut}(X)$表示集合$X$的自同构群. 试证明: 
    \begin{enumerate}[(1)]
        \item 若$G \times X \to X,\, (g, x) \mapsto g \cdot x$, 是群$G$在$X$上的一个作用, $\forall g \in G$, 定义映射$X \xrightarrow{\rho(g)} X$, $x \mapsto g \cdot x$. 则$\rho(g) \in \mathrm{Aut}(X)$且映射
        \[
            \rho:G \to \mathrm{Aut}(X),\, g \mapsto \rho(g)
        \]
        是群同态.
        \item 若$\rho:G \to \mathrm{Aut}(X)$是一个群同态, 则映射
        \[
            G \times X \to X,\: (g, x) \mapsto \rho(g)(x)
        \]
        是一个群作用.
    \end{enumerate}
\end{problem}

\begin{proof}
    \begin{enumerate}[(1)]
        \item 由于$G$是群, $g^{-1}$是存在的, 从而这里定义的左乘映射$\rho(g)$自然是一个双射. 只需验证$\rho$是保持乘法的. 对$\forall x \in X$
        \[
            \rho(g_1g_2)(x) = (g_1g_2) \cdot x = g_1 \cdot (g_2 \cdot x) = \rho(g_1)(\rho(g_2)(x)) = (\rho(g_1) \circ \rho(g_2))(x).
        \]
        因此有$\rho(g_1g_2) = \rho(g_1) \circ \rho(g_2)$保持乘法, $\rho$是群同态.
        \item 反过来, 若$\rho$是群同态, 则$\rho(1) = 1$, 即$\rho(1) = \mathrm{id}_X$. 那么对$\forall x \in X$自然有$1 \cdot x = \rho(1)(x) = \mathrm{id}_X(x) = x$. 另一方面,
        \[
            (g_1g_2) \cdot x = \rho(g_1g_2)(x) = (\rho(g_1) \circ \rho(g_2))(x) = \rho(g_1)(\rho(g_2)(x)) = g_1 \cdot (g_2 \cdot x)
        \]
        因此$G \times X \to X,\, (g, x) \mapsto \rho(g)(x)$是一个群作用.
    \end{enumerate}
\end{proof}

\begin{remark}
    这题是群作用的两种表述, 若看成一个群同态$\rho:G \to \mathrm{Aut}(X)$则更贴近表示论的观点. 比如同态
    \[
        \rho:G \to \mathrm{GL}(V),\quad g \mapsto \rho(g)
    \]
    称为群$G$的一个$k$-表示(a $k$-representation, or a representation over field $k$). 其中$V$是一个$k$-线性空间, $\mathrm{GL}(V)$为$V$上所有可逆线性变换构成的群.
    
    一般地, 对于范畴$\mathcal{C}$中的对象$X$, 群$G$在$X$上的作用是群同态
    \[
        \rho:G \to \mathrm{Aut}_{\mathcal{C}}(X)
    \]

    此观点在模的定义中也是类似的, 见\ref{ex:5.1.3}, 即一个$R$-模实际上是环$R$在Abel群$M$上的一个作用.
\end{remark}

\begin{problem}
    $20$阶群中共有多少个$5$阶元?
\end{problem}

\begin{proof}
    
\end{proof}

\begin{problem}
    证明$15$阶的群一定是循环群.
\end{problem}

\begin{proof}
    
\end{proof}

\begin{problem}
    证明$6$阶非Abel群一定同构于$S_3$.
\end{problem}

\begin{proof}
    
\end{proof}

\begin{problem}
    证明$12$阶群共有$5$个同构类, 即$12$阶群本质上只有$5$个.
\end{problem}

\begin{proof}
    
\end{proof}

\begin{problem}
    设$p, q$是两个不同的素数, 则$pq$或$p^2q$阶群一定不是单群. (事实上: $p^aq^b$阶群一定是可解群.)
\end{problem}

\begin{proof}
    
\end{proof}

\begin{problem}
    证明$200$阶群一定不是单群.
\end{problem}

\begin{proof}
    
\end{proof}

\begin{problem}
    设$H$为群$G$的有限子群.
    \begin{enumerate}[(1)]
        \item 证明: $(h_1, h_2)(x) = h_2xh_1^{-1}$定义了$H \times H$在群$G$上的作用;
        \item 证明: $H$为$G$的正规子群当且仅当上述作用的每条轨道都恰有$|H|$个.
    \end{enumerate}
\end{problem}

\begin{proof}
    
\end{proof}

\begin{problem}
    试证明若$|G| < 60$且$G$是一个单群, 那么$G$一定是素数阶的循环群.
\end{problem}

\begin{proof}
    
\end{proof}

\begin{problem}
    若$G$是$60$阶单群, 那么$G$一定同构于$A_5$, 从而得到阶数最小的非交换单群是$A_5$.
\end{problem}

\begin{proof}
    
\end{proof}
\clearpage
\section{模论初步}
\subsection{教材p91}

\begin{problem}
    设$R \xrightarrow{\varphi} R'$是环同态, $M$是一个$R'$-模.
证明: 
\[
    R \times M \to M,\quad (a, x) \mapsto \varphi(a)x,
\]
定义了$M$的一个$R$-模结构使得$M$成为一个$R$-模.
\end{problem}

\begin{proof}
    
\end{proof}

\begin{problem}
    设$M$是一个$R$-模, $\mathrm{Ann}(M) = \{a \in R \mid ax = 0, \forall x \in M\}$,
证明: 
\begin{enumerate}[(1)]
    \item $\mathrm{Ann}(M) \subset R$是理想;
    \item 对任意理想$I \subset R$, 若$I \subset \mathrm{Ann}(M)$,
则$R/I \times M \to M,\, (\bar{a},x) \mapsto ax$,
定义了$M$的一个$R/I$-模结构.
\end{enumerate}
\end{problem}

\begin{proof}
    
\end{proof}

\begin{problem}
    设$M = (M, +, 0)$是加法群, $\mathrm{End}(M) = \{M \xrightarrow{\varphi} M \mid \varphi \text{ 是群同态}\}$
    是$M$所有群自同态组成的环. 试证明: 
\begin{enumerate}[(1)]
    \item $\mathrm{End}(M) \times M \to M,\, (\varphi, x) \mapsto \varphi \cdot x \defeq \varphi(x)$,
是$M$的一个$\mathrm{End}(M)$-模结构. (因此, $M$是一个$\mathrm{End}(M)$-模.)
    \item 设$R$是一个环, 则$M$有一个$R$-模结构$R \times M \to M,\, (a, x) \mapsto ax$
的充要条件是存在环同态$R \xrightarrow{\eta} \mathrm{End}(M)$使得
$ax = \eta(a)(x)$对任意$a \in R, x \in M$成立.
\end{enumerate}
\end{problem}

\begin{proof}

\end{proof}

\begin{problem}
    设$M = (M, +, 0)$是任意加法群, 证明: $M$有唯一的$\mathbb{Z}$-模结构.
\end{problem}

\begin{proof}
    
\end{proof}

\begin{problem}
    设$R$-模$M$的模结构由环同态$R \xrightarrow{\eta} \mathrm{End}(M)$确定, 
$\varphi \in \mathrm{End}(M)$. 试证明: 
$M \xrightarrow{\varphi} M$是$R$-模同态当且仅当
$\varphi \circ \eta (a) = \eta (a) \circ \varphi$, $\forall a \in R$.
\end{problem}

\begin{proof}
    
\end{proof}

\begin{problem}
    $R$-模$M$称为不可约模, 如果$M \neq 0$且$M$没有非平凡子模.
证明: $R$-模$M$不可约当且仅当存在极大左理想$I \subset R$
使得$M \cong R/I$.
\end{problem}

\begin{proof}
    
\end{proof}

\begin{problem}[舒尔(Schur)引理]
    证明: 如果$M_1, M_2$是不可约$R$-模, 则任何非
零模同态$M_1 \to M_2$必为同构.
\end{problem}

\begin{proof}
    
\end{proof}

\begin{problem}[同态基本定理]
    设$\varphi:M\to M^{\prime}$是$R$-模同态.证明: $\varphi$的核
\[
    \ker(\varphi) = \{x \in M \mid \varphi(x) = 0\}
\]
和像 $\mathrm{Im}(\varphi) = \{\varphi(x) \mid \forall x \in M\}$必为子模, 
且$\varphi$的诱导映射
\[
    \overline{\varphi}:M/\ker(\varphi) \to \mathrm{Im}(\varphi),\: \overline{\varphi}(\bar{x}) = \varphi(x),
\]
必为同构.
\end{problem}

\begin{proof}
    
\end{proof}
\subsection{教材p95-p96}

\begin{problem}
    设$R$是任意环, 证明: $R^m \cong R^n$当且仅当存在
$A \in M_{m \times n}(R), B \in M_{n \times m}(R)$
使得$AB = I_m$, $BA = I_{n}$.
\end{problem}

\begin{proof}
    
\end{proof}

\begin{problem}
    设$R$是交换环, $\eta:R^n \to R^n$是满同态.
证明$\eta$必为双射. 如果$\eta$是单射, 它一定是满射吗?
\end{problem}

\begin{proof}
    
\end{proof}

\begin{problem}
    设$R$是交换整环, $e_1, e_2, \cdots, e_n \in R^n$是一组基.
令
\[
    (f_1, f_2, \cdots, f_n) = (e_1, e_2, \cdots, e_n)A,\quad A \in M_n(R).
\]
证明:
\begin{enumerate}[(1)]
    \item $f_1, f_2, \cdots, f_n$生成一个秩为$n$的子模
$K \subset R^n$的充要条件是$\det(A) \neq 0$;
    \item $\forall \bar{x} \in R^{n}/K$,则$\det(A) \cdot \bar{x} = 0$.
\end{enumerate}
\end{problem}

\begin{proof}
    
\end{proof}

\begin{problem}
    设$K \subset \mathbb{Q}[\lambda]^3$是由
$f_1 = (2\lambda - 1, \lambda, \lambda^2 + 3)$,
$f_2 = (\lambda, \lambda, \lambda^2)$,
$f_3 = (\lambda + 1, 2\lambda, 2\lambda^2 - 3)$
生成的$\mathbb{Q}[\lambda]$-子模. 试求$K$的一组基.
\end{problem}

\begin{proof}
    
\end{proof}

\begin{problem}
    设$R$是欧氏环$(\delta: R^* \to \mathbb{N}), A \in M_n(R)$
且$\det(A) \neq 0$. 证明: 存在可逆矩阵$P \in M_n(R)$使得
\[
    PA = 
    \begin{pmatrix}
        d_1 & b_{12} & b_{13} & \cdots & b_{1n}\\
        & d_2 & b_{23} & \cdots & b_{2n}\\
        & & d_3 & \cdots & b_{3n}\\
        & & & \ddots & \vdots\\
        & & & & d_n
    \end{pmatrix}
\]
是上三角矩阵且
$d_i \neq 0 \, (1 \leqslant i \leqslant n),\, \delta(b_{ji}) < \delta(d_i)$.
\end{problem}

\begin{proof}
    
\end{proof}
\subsection{教材p101}

\begin{problem}
    设$M$是主理想整环$R$上的挠模. 证明:$M$是不可约$R$-模当且仅当
$M = R \cdot z$, $\mathrm{ann}(z) = (p)$, $p \in R$不可约.
\end{problem}

\begin{proof}
    
\end{proof}

\begin{problem}
    设$M$是主理想整环$R$上的有限生成挠模, $M$称为不可分解模, 如果$M$
不能写成两个非零子模的直和. 证明:$M$不可分解当且仅当$M = R \cdot z$,
$\mathrm{ann}(z) = (p^e)$, $p \in R$不可约.
\end{problem}

\begin{proof}
    
\end{proof}
\clearpage

\nocite{*}
\bibliographystyle{alpha}
\bibliography{textbook}
\end{document}