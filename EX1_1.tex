\subsection{教材p8-p9}

\begin{problem}\label{ex:1.1.1}
    设$K$是一个域, 试证明下述结论: 
    
    \begin{enumerate}[(1)]
        \item 如果$a \cdot c = b \cdot c$, $c \neq 0_K$, 则$a = b$ (乘法消去律);
        \item $\forall \, a, b\in K$, 如果$a \cdot b = 0_K$, 则$a = 0_K$或$b = 0_K$;
        \item $(a^{-1})^{-1} = a \quad (\forall \, a \in K ,\, a \neq 0_K)$;
        \item $(a \cdot b)^{-1} = a^{-1} \cdot b^{-1} \quad (a \neq 0_K ,\, b \neq 0_K)$;
        \item $(-a)^{-1} = -a^{-1} \quad (\forall a \neq 0_K)$;
        \item $\forall a \neq 0_K ,\, m, n \in \mathbb{Z}$, 则$a^{m + n} = a^m \cdot a^n ,\, a^{mn} = (a^m)^n$;
    \end{enumerate}
\end{problem}

\begin{solution}
    \begin{enumerate}[(1)]
        \item 由于$c \neq 0$, 故可在原式左右同乘$c^{-1}$, 得 
        \[ 
        \begin{aligned} 
            a \cdot c \cdot c^{-1} &= b \cdot c \cdot c^{-1}\\ 
            \implies a &= b.
        \end{aligned} 
        \]
        这告诉我们逆元的存在性强于乘法消去律, 乘法消去律已经可以保证乘法逆运算是良定的。
        这对加法也是一样的道理, 见\ref{ex:1.2.1}的(1).

        也可以用分配律得到
        \[
            a \cdot c = b \cdot c \implies (a - b) \cdot c = 0_K.
        \]
        要得到$a = b$需要使用(2), 即域$K$是没有零因子(zero-divisor)的. 由于$c \neq 0_K$,
    则$a - b = 0_K$, 即$a = b$.

        注:无零因子的非零交换环称为整环(integral domain), 见教材2.1节p23.
        \item 只需证明当$a \neq 0_K$时有$b = 0_K$, 同(1), 在等式$a \cdot b = 0_K$
    两端左乘$a^{-1}$即可.
        
        这告诉我们域$\implies$整环. 结合(1)知一个环是整环的条件已经可以推出乘法消去律.
        \item 即要证明$a^{-1}$的逆元是$a$, 这是根据定义以及逆元的唯一性得到, 教材在域, 
    环, 群三处定义下的注记都有提及. 事实上只要$a$在某一个幺半群(monoid)中关于这个运算有
    逆元, 该结论都会成立, 如\ref{ex:1.2.1}的(3).
        \item 即要证明$a \cdot b$的逆元是$a^{-1} \cdot b^{-1}$. 此处需要交换律, 因此
    验证半边逆就够了.
    \[
        (a^{-1} \cdot b^{-1}) \cdot (a \cdot b) = (a \cdot a^{-1}) \cdot (b \cdot b^{-1}) = 1_K.
    \]
    非交换的情形为$(ab)^{-1} = b^{-1}a^{-1}$, 见\ref{ex:1.3.2}.
        \item 即要证明$-a$的逆元是$-a^{-1}$. 我们用一下\ref{ex:1.2.1}的(6)
    \[
        (-a)(-a^{-1}) = aa^{-1} = 1_K, \quad (-a^{-1})(-a) = a^{-1}a = 1_K.
    \]
    这样这一条对一个环中的单位都成立.
        \item 首先需要明确定义, 教材关于$a^n$的定义并不清晰, 包括后面\ref{ex:1.2.1}中的$na$也是.
    事实上, 这种和$\mathbb{Z}$有关的东西都应该由递归定义给出, 相对应的证明要用归纳法.
        
        严格来说, 这是定义了一个映射
    \[
        \mathbb{Z} \times K^* \to K^*, (n, a) \mapsto a^n,
    \]
        这里$K^* = K \setminus \{0_K\}$(见\ref{ex:1.3.2}), 自然数的部分应由递归定义给出, 
    \[
        a^0 \defeq 1_K,\, a^{n + 1} \defeq a^n \cdot a,\, n \in \mathbb{N},
    \]
        负整数的部分定义为
    \[
        a^n \defeq (a^{-1})^{-n},\, n < 0. 
    \]
    由该定义可以验证对任意整数$n \in \mathbb{Z}$均有$a^{n + 1} = a^n \cdot a$
    以及$a^{-n} = (a^{-1})^n$, 这样在使用这两个等式的时候不用再区分正负了.

    回到原题, 对任意的$m \in \mathbb{Z}$, 先用归纳法证明$n \in \mathbb{N}$
    的情形, 负整数的情形可以结合定义得到.

    $n = 0$时根据定义左右均为$a^m$, 假设对$n$有$a^{m + n} = a^m \cdot a^n$, 根据定义有
    \[
        a^{m + n + 1} = a^{m + n} \cdot a = a^m \cdot a^n \cdot a = a^m \cdot a^{n + 1}.
    \]
    由归纳法知
    \begin{equation}
        \forall m \in \mathbb{Z}, n \in \mathbb{N} ,\, a^{m + n} = a^m \cdot a^n.
        \tag{*}
        \label{eq:1.1.1.61}
    \end{equation} 
    当$n < 0$时, 则存在$k \in \mathbb{Z}_{>0}$使得$m + kn < 0$,
    则有
    \[
    \begin{aligned}
        a^{m + n} &= a^{m + kn + (-(k - 1)n)}\\
        &\overset{\eqref{eq:1.1.1.61}}= a^{m + kn} \cdot a^{-(k - 1)n}\\
        &= (a^{-1})^{-m - kn} \cdot a^{n - kn}\\
        &\overset{\eqref{eq:1.1.1.61}}= (a^{-1})^{-m} \cdot (a^{-1})^{-kn} \cdot a^n \cdot a^{-kn}\\
        &= a^m \cdot (a^{-1})^{-kn} \cdot (a^{-1})^{-n} \cdot a^{-kn}\\
        &\overset{\eqref{eq:1.1.1.61}}= a^m \cdot (a^{-1})^{-kn - n} \cdot a^{-kn}\\
        &= a^m \cdot a^{(k + 1)n} \cdot a^{-kn}\\
        &\overset{\eqref{eq:1.1.1.61}}= a^m \cdot a^{(k + 1)n - kn} = a^m \cdot a^n.
    \end{aligned}
    \]
    这里我避免使用了乘法交换律, 这样该结论对一般的环也成立.

    同样地, 由于$a^{m(n + 1)} = a^{mn + m} = a^{mn} \cdot a^m = (a^{m})^n \cdot a^m = (a^m)^{n + 1}$,
    对$n$归纳可得
    \begin{equation}
        \forall m \in \mathbb{Z}, n \in \mathbb{N},\, a^{mn} = (a^m)^n.
        \tag{**}
        \label{eq:1.1.1.62}
    \end{equation}
    当$n < 0$时, 
    \[
    \begin{aligned}
        a^{mn} &= a^{-(m \cdot (-n))}\\
        &= (a^{-1})^{m \cdot (-n)}\\
        &\overset{\eqref{eq:1.1.1.62}}= ((a^{-1})^m)^{-n}\\
        &= (a^{-m})^{-n} = ((a^{-m})^{-1})^n
    \end{aligned}
    \]
    由于$a^{-m} \cdot a^m \overset{\eqref{eq:1.1.1.61}}= a^0 = 1_K$, 即括号内确实为$a^m$,
    故上式等于$(a^m)^n$.
    \end{enumerate}
\end{solution}

\begin{problem}
    设$K$是一个域, 证明:$K$的任意一组子域(可以无限多个)的交集仍是子域.
如果$K_i \subset K \, (i \in \mathbb{N})$是满足条件
$K_i \subseteq K_{i+1} \, (i \in \mathbb{N})$的子域, 则它们的并集也是
$K$的子域.
\end{problem}

\begin{solution}
    令
\(
    F = \bigcap_i K_i
\)
    由子域定义, 需要验证
\[
\begin{aligned}
    &\forall a, b \in F ,\, a - b \in F\\
    &\forall a, b \in F^* ,\, ab^{-1} \in F^*,\, F^* = F \setminus \{0\}.
\end{aligned}
\]
由于$K_i$均为子域, 且$a, b \in F \subseteq K_i$, 故
\[
    \forall i \in \mathbb{N}, \, a - b \in K_i.
\]
因此
\[
    a - b \in \bigcap_i K_i = F.
\]
$F^*$的部分同理, 故$F$为子域.

若还满足$\forall i \in \mathbb{N} ,\, K_i \subseteq K_{i + 1}$,
令
\(
    L = \bigcup_i K_i
\),
如果$a, b \in L$, 则存在$K_i$和$K_j$使得$a \in K_i ,\, b \in K_j$
记$r = \max(i, j)$, 则$a, b \in K_r$. 由于$K_r$为子域, 可得
\[
    a - b \in K_r \subseteq L.
\]
$L^*$同理, 故$L$为子域.
\end{solution}

\begin{problem}
    令$\mathbb{Q}[\sqrt{2}, \sqrt{3}]$表示$\mathbb{C}$中包含
$\mathbb{Q}, \sqrt{2}, \sqrt{3}$的最小子域, 证明
$\mathbb{Q}[\sqrt{2}, \sqrt{3}] = \mathbb{Q}[\sqrt{2} + \sqrt{3}]$.
\end{problem}

\begin{solution}
    该题本应该是域扩张的题, 此处我们只用定义来证明。

    由于$\sqrt{2} + \sqrt{3} \in \mathbb{Q}[\sqrt{2}, \sqrt{3}]$,
我们有$\mathbb{Q}[\sqrt{2} + \sqrt{3}] \subseteq \mathbb{Q}[\sqrt{2}, \sqrt{3}]$.
反过来, 
\(
    \frac{1}{\sqrt{2} + \sqrt{3}} = \sqrt{3} - \sqrt{2} \in \mathbb{Q}[\sqrt{2} + \sqrt{3}]
\),
故有
\(
    \sqrt{2} = \frac{(\sqrt{2} + \sqrt{3}) - (\sqrt{3} - \sqrt{2})}{2} \in \mathbb{Q}[\sqrt{2} + \sqrt{3}]
\),
\(
    \sqrt{3} = \frac{(\sqrt{2} + \sqrt{3}) + (\sqrt{3} - \sqrt{2})}{2} \in \mathbb{Q}[\sqrt{2} + \sqrt{3}]
\).
因此
\(
    \mathbb{Q}[\sqrt{2}, \sqrt{3}] \subseteq \mathbb{Q}[\sqrt{2} + \sqrt{3}]
\).
故两者相等。
\end{solution}

\begin{problem}\label{ex:1.1.4}
    设$\mathbb{N}$是所有正整数的集合, $\mathbb{Q}$是有理数域.因$\mathbb{Q}$是可数集, 
故存在双射$f:\mathbb{N} \to \mathbb{Q}$. 令$f^{-1}:\mathbb{Q} \to \mathbb{N}$表示$f$的
逆映射, 利用有理数的加法和乘法, 可通过双射$f$定义$\mathbb{N}$上的运算如下:
$\forall n, m \in \mathbb{N}$, 
\[
    n \oplus m = f^{-1}(f(n) + f(m)), \quad n \star m = f^{-1}(f(n)f(m)),
\]
试证明:$\mathbb{N} = (\mathbb{N}, \oplus, \star)$是域, 并求它的零元和单位元.
\end{problem}

\begin{solution}
    验证域公理, 加法交换律和乘法交换律易得.

    结合律:$\forall n, m, l \in \mathbb{N}$,
\[
\begin{aligned}
    (n \oplus m) \oplus l &= f^{-1}\biggl(f\bigl(f^{-1}(f(n) + f(m))\bigr)+ f(l)\biggr)\\
    &= f^{-1}(f(n) + f(m) + f(l)) \\
    &\overset{!}= n \oplus (m \oplus l);\\
    (n \star m) \star l &= f^{-1}\biggl(f\bigl(f^{-1}(f(n)f(m))\bigr) \cdot f(l)\biggr)\\
    &= f^{-1}(f(n)f(m)f(l)) \\
    &= n \star (m \star l).
\end{aligned}
\]
其中!处是因为计算出来的结果关于$n, m, l$是轮换对称的, 后面同理.

    零元为$f^{-1}(0)$: $\forall n \in \mathbb{N}$,
\[
\begin{aligned}
    n \oplus f^{-1}(0) &= f^{-1}(f(n) + f(f^{-1}(0)))\\
    &= f^{-1}(f(n) + 0)\\
    &= f^{-1}(f(n)) = n.
\end{aligned}
\]
    $n$的负元为$f^{-1}(-f(n))$:
\[
\begin{aligned}
    n \oplus f^{-1}(-f(n)) &= f^{-1}\biggl(f(n) + f\bigl(f^{-1}(-f(n))\bigr)\biggr)\\
    &= f^{-1}(f(n) - f(n))\\
    &= f^{-1}(0).
\end{aligned}
\]
    单位元为$f^{-1}(1)$: $\forall n \in \mathbb{N}$,
\[
\begin{aligned}
    n \star f^{-1}(1) &= f^{-1}(f(n) \cdot f(f^{-1}(1)))\\
    &= f^{-1}(f(n)) = n.
\end{aligned}
\]
    $n$的逆元为$f^{-1}(\frac1{f(n)})$:
\[
\begin{aligned}
    n \star f^{-1}(\frac{1}{f(n)}) &= f^{-1}\biggl(f(n) \cdot f\bigl(f^{-1}(\frac{1}{f(n)})\bigr)\biggr)\\
    &= f^{-1}(f(n) \cdot \frac{1}{f(n)})\\
    &= f^{-1}(1).
\end{aligned}
\]
    分配律:$\forall n, m, l \in \mathbb{N}$,
\[
\begin{aligned}
    n \star (m \oplus l) &= f^{-1}\biggl(f(n) \cdot f\bigl(f^{-1}(f(m) + f(l))\bigr)\biggr)\\
    &= f^{-1}\bigl(f(n) \cdot (f(m) + f(l))\bigr)\\
    &= f^{-1}\bigl(f(n)f(m) + f(n)f(l)\bigr)\\
    &= f^{-1}(f(n)f(m)) \oplus f^{-1}(f(n)f(l))\\
    &= n \star m \oplus n \star l.
\end{aligned}
\]
\end{solution}

\begin{problem}\label{ex:1.1.5}
    证明:在域的定义中, 加法的交换律可以由其他条件推
出. 提示:按两种方式展开 $(1 + 1) \cdot (a + b)$.
\end{problem}

\begin{solution}
    一方面
\[
\begin{aligned}
    (1 + 1) \cdot (a + b) &= 1 \cdot (a + b) + 1 \cdot (a + b)
    &= a + b + a + b;
\end{aligned}    
\]
    另一方面
\[
\begin{aligned}
    (1 + 1) \cdot (a + b) &= (1 + 1) \cdot a + (1 + 1) \cdot b\\
    &= a + a + b + b.
\end{aligned}
\]
故有$a + b + a + b = a + a + b + b$.
消去两端的一个$a$和一个$b$即得加法交换律.
\end{solution}

\begin{problem}
    设$p > 2$是素数,
$\mathbb{F}_p = \{\bar{0}, \bar{1}, \bar{2}, \cdots, \overline{p-1}\}$
是$\mathbb{Z}$的模$p$剩余类域. 试计算:
\begin{enumerate}[(1)]
    \item $\bar{2}$在$\mathbb{F}_p$中的逆元$\bar{2}^{-1}$;
    \item $\overline{p - 1} \cdot \overline{p - 2}$;
    \item $\overline{p - 2}$在$\mathbb{F}_p$中的逆元$\overline{p-2}^{-1}$.
\end{enumerate}
\end{problem}

\begin{solution}
\begin{enumerate}[(1)]
    \item 只需找到能被$2$整除的$1 + kp(k \in \mathbb{Z})$.
由于素数$p > 2$, $p + 1$即可. i.e. $\overline{2}^{-1} = \overline{\frac12(p + 1)}$.
    \item $\overline{p - 1} \cdot \overline{p - 2} = \overline{-1} \cdot \overline{-2} = \overline{2}$.
    \item 由(1), $\overline{p - 2}^{-1} = \overline{-2}^{-1} = \overline{-\frac12(p + 1)} = \overline{\frac12(p - 1)}$.
\end{enumerate}
\end{solution}
    