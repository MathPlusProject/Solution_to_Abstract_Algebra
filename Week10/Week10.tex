\documentclass{../solutions-cn}

\begin{document}
\section*{第十周作业参考解答及补充}

\subsection*{作业}

\begin{exercise}[习题3.3.1]
    设$f(x) = x^2 + ax + b \in K[x]$不可约, $E = K[u_1]$(其中$f(u_1) = 0$) 证明: $E$必包含$f(x) = 0$的另一个根(所以$E$是$f(x)$的分裂域).
\end{exercise}

\begin{proof}
    由于$u_1 \in E$是$f(x)$的根, 因此在$E[x]$中有分解$f(x) = (x - u_1)f_1(x)$. 而$\deg(f) = 2$, 故只能是$\deg(f_1) = 1$, 即$f_1 = x - u_2$, $u_2 \in E$自然是$f(x)$的另一个根.

    或设$f(x)$的分裂域是$E'$, 令$f$的另一个根为$u_2 \in E'$, 则有$u_1 + u_2 = a \in K \subseteq E$. 而$u_1 \in E$, 因此$u_2 \in E$, 即$E' = E$.
\end{proof}

\begin{remark}
    若要严谨一点, 则不能在$E$中直接使用韦达定理, 因为$u_2 \in E$是要证的结论. 韦达定理实际上是$f$在其分裂域可以分解成一次因式的乘积(即分裂), 再对比系数得到的结论. 而按分裂域的定义可知它是使得$f$分裂的最小扩域. 那么直接使用韦达定理是在用结论证结论. 不过由于代数闭包总存在且唯一(见3.3.2和3.3.6), 我们总能把任意多项式分解成一次多项式的乘积, 所以直接使用事实上是没问题的.

    这题也告诉我们, 二次扩张都是正规扩张.
\end{remark}

\begin{exercise}[习题3.3.2]
    设$f(x) = x^3 - 2 \in \mathbb{Q}[x], u_1 = \sqrt[3]{2}$. 证明: $E = \mathbb{Q}[u_1]$不包含$f(x) = 0$的其他两个根.
\end{exercise}

\begin{proof}
    教材例3.3.4.

    由于$\mathbb{Q} \subseteq \mathbb{C}$, 而$\mathbb{C}$是代数闭域, 我们可以把所有根都明确的写出来. $x^3 - 2$的根为$\alpha_k = \sqrt[3]{2}e^\frac{2k\pi i}{3} = \sqrt[3]{2}\zeta_3^k$, $k = 0, 1, 2$, $u_1 = \alpha_0$. 而$\mathbb{Q}[u_1] \subseteq \mathbb{R}$, $\alpha_{1,2} \in \mathbb{C} \setminus \mathbb{R}$.
\end{proof}

\begin{remark}
    借此补充代数扩张的一个结论, 任何域在同构的意义下都有唯一的代数闭包. 这个结果的证明分为两部分, 一是存在性, 二是3.3.6提到的延拓.

    \begin{defstar}
        若域$K$满足任意次数大于1的多项式$f(x) \in K[x]$在$K$中都有根, 我们称$K$是一个代数闭域. 根据教材的定义2.4.2, $K$是代数闭域等价于$K[x]$中的不可约多项式都是一次多项式. 即$f(x)$总能分解成一次多项式的乘积.
    \end{defstar}
    由定义, $K$是代数闭域意味着$K$无法再做非平凡的代数扩张了. 若有代数扩张$K \subseteq L$且$[L:K] > 1$, 则存在$\alpha \in L \setminus K$在$K$上代数, 即存在非零多项式$f(x) \in K[x]$使得$f(\alpha) = 0$, 而$K$是代数闭域, $f(x)$的所有根都在$K$里, 这就矛盾了. 换句话说, 代数闭域做代数扩张只能得到它自己. 反过来也是对的, 若$K$没有非平凡代数扩张, 且有次数大于1的不可约多项式, 那根据3.1.2就能做真代数扩张, 矛盾.

    \begin{defstar}
        设域扩张$K \subseteq L$, 考虑所有的代数元
        \[
            E = \{\alpha \in L \mid \alpha \text{ 在}K \text{ 上代数}\}
        \]
        由教材的推论3.1.1可知$E$是一个中间域, 且$K \subseteq E$是代数扩张. $E$称为$K$在$L$中的(相对)代数闭包.
    \end{defstar}
    比如对于扩张$\mathbb{Q} \subseteq \mathbb{R}$, 这里的$E$就是所有的实代数数. 把$\mathbb{R}$换成$\mathbb{C}$, $E$就是所有的复代数数, 也就是$\mathbb{Q}$的代数闭包, 一般用$\overline{\mathbb{Q}}$表示.

    相对代数闭包$E$在$L$内没有非平凡代数扩张, 即若$E \subseteq E' \subseteq L$且$E \subseteq E'$是代数扩张, 则$E = E'$. 证明这个结论需要一个很基本的定理.

    \begin{thmstar}
        代数扩张的代数扩张仍是代数扩张, 即代数扩张是可以传递的. 《近世代数引论》p105, Serge Lang《Algebra》p228. 即对域扩张$K \subseteq E \subseteq L$, $L/K$是代数扩张$\iff E/K$和$L/E$都是代数扩张.
    \end{thmstar}

    那么$E'/E$代数, $E/K$代数, 就有$E'/K$代数. 但根据$E$的定义是所有$K$上代数元构成的中间域, 因此$E' \subseteq E$, 所以$E' = E$.

    \begin{defstar}
        设$K \subseteq L$是代数扩张, 若$L$是代数闭域, 称$L$是$K$的(绝对)代数闭包. 一般用记号$\overline{K}$表示. (所以教材定理3.3.2的记号容易引起误解)
    \end{defstar}
    一般说代数闭包默认指绝对代数闭包.

    \begin{propstar}
        设$K \subseteq L$是域扩张, $L$是代数闭域, $E$是$K$在$L$中的相对代数闭包, 则$E = \overline{K}$.
    \end{propstar}
    只需证明这样得到的相对代数闭包是一个代数闭域, 注意到次数大于1的多项式$f(x) \in E[x] \subseteq L[x]$, 而$L$是代数闭的, 因此$f(x) = (x - \alpha_1) \cdots (x - \alpha_n)$, $\alpha_1, \cdots, \alpha_n \in L$在$E$上代数, 自然就在$K$上代数, 按$E$的定义就有$\alpha_1, \cdots, \alpha_n \in E$. 从而$f(x) \in E[x]$总能在$E$上分解称一次多项式的乘积.

    \begin{propstar}
        对任意域$K$, 存在代数闭域$L$使得$K \subseteq L$. Serge Lang《Algebra》\S V.2 Theorem 2.5
    \end{propstar}
    若该命题成立, 那么根据上面的讨论, 任何域都存在代数闭包, 唯一性见3.3.6.

    这个命题的证明是构造性的, 构造方法属于Artin, 需要用到2.1.6注记里补充的命题. 基本的思路就是构造一个域扩张$K \subseteq K_1$使得$K[x]$中所有非常数多项式在$K_1$中都有至少一个根, 这个操作做可数次之后就能得到一个代数闭域. 类似2.3.2和3.1.2中说的那样, 令$S = \{X_f \mid f \in K[x] \setminus K\}$, 即用$K[x]$里的非常数多项式来编号, 得到一个无穷的未定元构成的集合, 然后考虑多项式环$K[S]$. 此时记$K[S]$的一个理想$I = (f(X_f))_{f \in K[x] \setminus K} = \sum_{f \in K[x] \setminus K} (f(X_f))$, 即所有这种形式的$K[S]$里的多项式生成的理想(2.1.6的注记), 如果商掉这个理想, 那么和2.3.2一样, $\overline{X_f}$就是$f$的一个根, 但$I$不一定是极大理想, 因此需要用到2.1.6注记里补充的命题(新增加的), 即考虑$I \subseteq \mathfrak{m}$, 其中$\mathfrak{m}$是极大理想. 不过需要先验证$I \neq (1)$, 这是容易的, 若$I = (1)$, 意味着$1 = \sum_{i = 1}^{n} a_if_i(X_{f_i})$, 而这是不可能的, 因为我们可以用$n$次3.1.2得到扩张$K \subseteq E$让这里的$f_i$都有根, 赋值(这里用的是多元的2.4.5)之后就得到$1 = 0$, 矛盾. 因此这样我们得到了$K_1 = K[S]/\mathfrak{m}$. 然后做可数次$K \subseteq K_1 \subseteq \cdots \subseteq K_i \subseteq \cdots$. 最终得到的代数闭域就是$L = \bigcup_{i = 1}^{\infty} K_i$.
\end{remark}

\begin{exercise}[习题3.3.3]
    设$L$是$n$次多项式$f(x) \in K[x]$的分裂域, 证明: $[L:K] \leqslant n!$.
\end{exercise}

\begin{proof}
    对$n$归纳.

    $n = 1$或$f$已经在$K$上分裂, 都有$[L:K] = 1$. 假设结论对$n$成立, 现考虑$\deg(f) = n + 1$, 且$f$在$K$上不分裂, 那么存在$f$的不可与因子$g$满足$\deg(g) > 1$(否则$f$在$K$上分裂). 设$u \in L$是$g(x)$的一个根, 由3.1.2, $K[u] \cong K[x]/(g(x))$是中间域, $[K[u]:K] = \deg(g) \leqslant \deg(f) = n + 1$, 且在$K[u][x]$上有分解$f(x) = (x - u)h(x)$, $\deg(h) = n$. 由归纳假设, 此时$L$是$n$次多项式$h(x) \in K[u][x]$的分裂域, 有$[L:K[u]] \leqslant n!$, 从而$[L:K] = [L:K[u]] \cdot [K[u]:K] \leqslant (n + 1)!$.
\end{proof}

\begin{exercise}[习题3.3.4]
    构造$x^5 - 2 \in \mathbb{Q}[x]$的一个分裂域$L$, 并求$[L:\mathbb{Q}]$.
\end{exercise}

\begin{proof}
    仍使用3.1.14分析degree的方法, $x^5 - 2$的根为$\sqrt[5]{2}\zeta_5^k$, $k = 0, 1, 2, 3, 4$. $L = \mathbb{Q}\left[\sqrt[5]{2}\zeta_5^i\right] = \mathbb{Q}\left[\sqrt[5]{2}, \zeta_5\right]$. 借助中间域$\mathbb{Q}[\sqrt[5]{2}]$. 一方面, $\left[\mathbb{Q}[\sqrt[5]{2}]:\mathbb{Q}\right] = 5$(3.1.14); 另一方面, $\left[\mathbb{Q}[\zeta_5]:\mathbb{Q}\right] = 4$(3.1.5). 因此$5, 4 \mid [L:\mathbb{Q}]$, 由于$(4,5) = 1$, 因此$4 \cdot 5 = 20 \mid [L:\mathbb{Q}]$, 另一方面$x^5 - 2 \in \mathbb{Q}[\zeta_5][x]$仍是$\sqrt[5]{2}$的化零多项式, 又有$[L:\mathbb{Q}] \leqslant 20$, 故$[L:\mathbb{Q}] = 20$.
\end{proof}

\begin{exercise}[习题3.3.5]
    确定多项式$x^{p^n} - 1 \in \mathbb{F}_p[x]$在$\mathbb{F}_p$上的分裂域$(n \in \mathbb{N})$.
\end{exercise}

\begin{proof}
    特征$p$的域的多项式环上Frobenius(2.1.2)也是成立的, 故有$x^{p^n} - 1 = x^{p^n} - 1^{p^n} = (x - 1)^{p^n}$. 从而$x^{p^n}$在$\mathbb{F}_p$上分裂, 分裂域即$\mathbb{F}_p$.
\end{proof}

\begin{exercise}[习题3.3.7]
    令$f(x) = (x^2 - 2)(x^2 - 3)$, $K = \mathbb{Q}[x]/(x^2 - 2) = \mathbb{Q}[u_1]$, 此处$u_1 = \overline{x} \in \mathbb{Q}[x]/(x^2 - 2)$. 试证明: 
    \begin{enumerate}[(1)]
        \item $K$是一个域, 且$x^2 - 3$在$K[x]$中不可约;
        \item $L = K[x]/(x^2 - 3) = K[u_2]$(此处$u_2 = \overline{x} \in K[x]/(x^2 - 3))$是$f(x) = (x^2 - 2)(x^2 - 3)$的分裂域, 且$[L:\mathbb{Q}] = 4$.
    \end{enumerate}
\end{exercise}

\begin{proof}
    \begin{enumerate}[(1)]
        \item $K$是域是因为$(x^2 - 2)$是极大理想, 见3.1.2和3.1.14. $x^2 - 3$在$K$中不可约在3.1.4已证.
        \item 根据3.1.2和(1), $L = \mathbb{Q}[u_1, u_2]$是域. 且有分解$f(x) = (x - u_1)(x + u_1)(x - u_2)(x + u_2)$, 因此$L$是$f(x)$的分裂域(3.3.1). 且有$[L:\mathbb{Q}] = [L:K] \cdot [K:\mathbb{Q}] = 2 \cdot 2 = 4$.
    \end{enumerate}
\end{proof}
\end{document}