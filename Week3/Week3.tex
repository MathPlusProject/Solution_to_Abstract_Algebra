\documentclass{../solutions-cn}

\begin{document}
\section*{第三周作业参考解答及补充}

教材的符号有不清晰的地方, 主要是素理想那一块, 我进行了修正, 用$\subseteq$表示子集, $\subsetneq$表示真子集. 其他有错误的地方也会修改, 用\red{红色}表示.

\subsection*{作业}
\begin{exercise}[习题2.1.1]
    设$R$是一个交换环, $I \,\red{\subsetneq}\, R$是一个理想. 证明
\[
    \sqrt{I} = \{r \in R \mid \exists m \in \red{\mathbb{N}} \text{ 使得 } r^m \in I\}
\]
也是$R$的理想 (称为理想$I$的根).
\end{exercise}

\begin{remark}
    这题的理想的根定义有误, 应是$\mathbb{N}$而不是$\mathbb{Z}$. 一旦出现负整数意味着有可逆元, 从而$\sqrt{I}$是单位理想了.
\end{remark}

\begin{proof}
    可以清楚地看出$I \subseteq \sqrt{I}$. 先验证加法子群,
    \[
    \begin{gathered}
        \forall a, b \in \sqrt{I},\, \exists m, n \in \mathbb{N},\, a^m, b^n \in I,\\
        \implies (a - b)^{m + n} \in I
    \end{gathered}
    \]
    这是因为单项$a^ib^j$的指数$i + j = m + n$, 故$i < m$和$j < n$不能同时成立, 即$i \geqslant m$或$j \geqslant n$, i.e. $a^i \in I$或$b^j \in I$.
    从而$(a - b)^{m + n} \in I$, $a - b \in \sqrt{I}$.

    再验证吸收律(交换验证单边即可),
    \[
        \forall a \in \sqrt{I}, r \in R,\, \exists m \in \mathbb{N},\, a^m \in I \implies (ar)^m = a^mr^m \in I
    \]
    因此$ar \in \sqrt{I}$.
\end{proof}

\begin{remark}
    零理想的根$\sqrt{\{0\}} = \{x \in R \mid \exists n \in \mathbb{N}, x^n = 0\}$是所有幂零元(nilpotent)组成的理想, 叫做$R$的幂零根(nilradical), 一般记作$\mathfrak{N}(R)$. 可以证明$\mathfrak{N}(R) = \bigcap_{\mathfrak{p} \text{ 是素理想}} \mathfrak{p}$.

    对任何的理想$I$可以清楚地看出$I \subseteq \sqrt{I}$. 若$\sqrt{I} = I$, 我们称$I$是一个根理想(radical ideal). 任何的素理想(2.1.5)都是根理想.
\end{remark}

\begin{exercise}[习题2.1.2]
    设$R$是一个交换环, $p > 0$是一个素数. 如果
$p \cdot x = 0 (\forall x\in R)$. 试证明: 
$(x + y)^{p^m} = x^{p^m}+y^{p^m} (\forall x, y \in R, m > 0)$
\end{exercise}

\begin{proof}
    事实上, 这个$p$就是环$R$的特征. 若$\mathrm{Char}(R) \neq p$, 则由$p1_R = 0_R$, $\mathrm{Char}(R) < p$.
    那么$(p, \mathrm{Char}(R)) = 1$, 有Bézout's Identity得到$1_R = 0_R$, 这就没什么考虑的必要了.

    对特征$p$的交换环, 有一个特别的同态$F$称为Frobenius自同态,
    \[
        F: R \to R, \quad a \mapsto a^p
    \]
    我们说明这确实是一个同态.
    
    保持乘法是因为交换环, 不平凡的是保持加法.
    \[
        (a + b)^p = a^p + \binom{p}{1}a^{p - 1}b + \cdots + b^p.
    \]
    其中$1 \leqslant i \leqslant p - 1$时,
    \[
        \binom{p}{i} = \frac{p(p - 1) \cdots (p - i + 1)}{1 \cdot 2 \cdots i} 
    \]
    由于$p$是素数, $1, 2, \cdots i$都不整除$p$, 而$\binom{p}{i}$是整数, 因此只能是$i! \mid (p - 1) \cdots (p - i + 1)$. 所以$p \mid \binom{p}{i}$. 而$px = 0,\, \forall x \in R$, 故$(a + b)^p = a^p + b^p$.

    因此$\varphi: R \to R,\, x \mapsto x^{p^m}$也是自同态, $\varphi = F^m$, 这里$F^m$表示复合$m$次.
\end{proof}

\begin{remark}
    Frobenius一般在域中使用的多一些. 虽然对交换环Frobenius都是可以定义的, 但是整环才能保证Frobenius是单射. Frobenius一般不是满的, 但对有限域就是自同构了.
\end{remark}

\begin{exercise}[习题2.1.5]
    理想\(P \,\red{\subsetneq}\, R\)称为素理想, 如果: $ab \in P \Rightarrow a \in P$
或$b \in P$. 试证明: 
\(P \,\red{\subsetneq}\, R\)是素理想当且仅当$R/P$没有零因子.
\end{exercise}

\begin{proof}
    \begin{enumerate}[(1)]
        \item "$\implies$":
        \[
        \begin{gathered}
            \forall \overline{a}, \overline{b} \in R/P,\, \overline{a}\overline{b} = \overline{ab} = \overline{0} \implies ab \in P \implies a \in P \text{ or } b \in P \\ \implies \overline{a} = \overline{0} \text{ or } \overline{b} = \overline{0}.
        \end{gathered}
        \]
        \item "$\impliedby$":
        \[
            ab \in P \implies \overline{a}\overline{b} = \overline{ab} = \overline{0} \implies \overline{a} = 0 \text{ or } \overline{b} = 0 \implies a \in P \text{ or } b \in P.
        \]
    \end{enumerate} 
\end{proof}

\begin{exercise}[习题2.1.6]
    理想$m \,\red{\subsetneq}\, R$称为极大理想, 如果$R$中不存在真包含$m$的非平凡理想
(即: 如果$I \,\red{\supsetneq}\, m$是$R$的理想, 则必有$I = R)$. 试证明: 当$R$是交换环时, 
$m \,\red{\subsetneq}\, R$是极大理想当且仅当$R/m$是一个域. 特别, 交换环中的极大理想必为素理想.
\end{exercise}

\begin{proof}
    \begin{enumerate}[(1)]
        \item "$\implies$":
        \[
        \begin{aligned}
            \forall \overline{0} \neq \overline{a} \in R/m &\implies a \notin m \implies m \subsetneq m + (a) \implies m + (a) = R = (1)\\
            &\implies \exists x \in m, b \in R,\, x + ab = 1 \implies \overline{ab} = \overline{1 - x} = \overline{1}.
        \end{aligned}
        \]
        \item "$\impliedby$":
        \[
        \begin{aligned}
            m \subsetneq I \underset{\text{ideal}}{\subseteq} R &\implies \exists a \in I \setminus m \text{ i.e. } \overline{a} \neq 0 \implies \exists b \in R,\, \overline{a}\overline{b} = \overline{ab} = \overline{1}\\
            &\implies \exists x \in m \subsetneq I,\, ab = 1 + x \implies 1 = ab - x \in I \\
            &\implies I = (1) = R.
        \end{aligned}
        \]
        或者用同态基本定理, 包含$m$的理想和$R/m$的理想有一个一一对应, 而域的理想只有$\{0\}$和本身.
    \end{enumerate}
\end{proof}

\begin{remark}
     (1)中用到了理想的和. 若$I, J$都是$R$的理想, $I + J \defeq \{i + j \mid i \in I, j \in J\}$. 可以验证这确实是一个理想, 类似可以定义一族理想$\{I_\alpha\}_{\alpha \in A}$的和,
    \[
        \sum_{\alpha \in A} I_\alpha = \left\{\sum_{\alpha \in A} i_\alpha \Big|i_\alpha \in I_\alpha, \text{ 且只有有限个 } i_\alpha \neq 0 \right\} 
    \]
    即考虑所有可能的有限和.
    
        另外$\bigcap_{\alpha \in A} I_\alpha$也是一个理想. 还有一个是理想的积, 相对要复杂一些,
    \[
    \begin{aligned}
        IJ &\defeq (\{ij \mid i \in I, j \in J\})\\
        &= \left\{\sum_{k = 1}^{n} i_kj_k \Big| \exists n \in \mathbb{N},\, 1 \leqslant k \leqslant n,\, i_k \in I, j_k \in J, \right\}
    \end{aligned}  
    \]
    他是所有乘积$ij$生成的理想. 那么一族理想的乘积就是考虑所有可能的有限乘积生成的理想.
\end{remark}

\begin{exercise}[习题2.1.7]
    设 $I \,\red{\subsetneq}\, \mathbb{Z}$是整数环的非零理想, 证明下述结论等价
\begin{enumerate}[(1)]
    \item $I$是极大理想;
    \item $I$是素理想;
    \item 存在素数$p$使得$I = (p)\mathbb{Z} = \{ap \mid \forall a \in \mathbb{Z}\}$.
\end{enumerate}
\end{exercise}

\begin{proof}
    \begin{enumerate}[1.]
        \item (1)$\implies$(2): 由于域一定是整环, 由2.1.5和2.1.6知极大理想是素理想.
        \item (2)$\implies$(3): 由于$\mathbb{Z}$是PID(带余除法可证), 故存在整数$p$使得$I = (p)$. 由于是素理想, 因此$ab \in (p) \implies a \in (p)$或$b \in (p)$. 即
        \[
            p \mid ab \implies p \mid a \text{ or } p \mid b
        \]
        则$p$是素数(若不然, $p = qr$, 取$a = q, b = r$即导出矛盾).
        \item (3)$\implies$(1): 设$I = (p) \subsetneq J$, 则存在$n \in J \setminus I$. 由于$p$是素数, 故有$(n, p) = 1$. 由Bézout's Identity, $\exists u, v \in \mathbb{Z}$使得$nu + pv = 1$, 从而$1 \in J,\, J = \mathbb{Z}$.(这和2.1.6的证明是类似的)
        
        或直接用$\mathbb{Z}/(p) = \mathbb{Z}/p\mathbb{Z}$是域.
    \end{enumerate}
\end{proof}

\begin{exercise}[习题2.1.8]
    设$p \in \mathbb{Z}$是素数, 证明
$(p)\mathbb{Z}[x] = \{pf(x) \mid \forall f(x) \in \mathbb{Z}[x]\}$是整系数多
项式环的素理想, 但不是$\mathbb{Z}[x]$的极大理想.
\end{exercise}

\begin{proof}
    事实上若$I$是$R$的理想, 我们有
\[
    \frac{R[x]}{IR[x]} \cong \frac{R}{I}[x]
\]
这是根据同态基本定理得到, 考虑同态
\[
    \varphi: R[x] \to \frac{R}{I}[x],\quad a_0 + a_1x + \cdots + a_nx^n \mapsto \overline{a_0} + \overline{a_1}x + \cdots + \overline{a_n}x^n
\]
可以验证这确实是一个同态.(事实上, 它是$R \twoheadrightarrow R/I \hookrightarrow \frac{R}{I}[x]$的一个延拓.)

回到原题, 有
\[
    \frac{\mathbb{Z}[x]}{(p)\mathbb{Z}[x]} \cong \mathbb{Z}_p[x]
\]
这里$\mathbb{Z}_p = \mathbb{Z}/p\mathbb{Z}$是域. 因此$\mathbb{Z}_p[x]$是PID, 自然是整环, 但不是域($x$没有逆). 因此由2.1.5和2.1.6, $(p)\mathbb{Z}[x]$是素理想但不是极大理想.
\end{proof}

\begin{remark}
    (这个比较超纲看不懂可以不看)给定环同态$R \overset{\varphi}\to S$, 其中$R$是交换环. 若$\varphi(R) \subseteq C(S)$(习题2.1.11), 根据之前习题1.4.9说过的, 首先$S$上有一个$R$-模结构. 其次有
\[
    (r_1s_1)(r_2s_2) = \varphi(r_1)s_1\varphi(r_2)s_2 = \varphi(r_1)\varphi(r_2)s_1s_2 = \varphi(r_1r_2)s_1s_2 = (r_1r_2)(s_1s_2).
\]
即数乘和$S$本身的乘法是相容的. 这样的结构我们称为一个$R$-代数($R$-algebra), 这也是习题2.1.12介绍的东西. 因此一个$R$-代数就是带有加法, ($R$-)数乘, 乘法的一个代数结构.

当$S$本身就是交换环时, 此时乘法是交换的, 且$C(S) = S$, 这样会变得简单很多. 这时$S$称为一个交换$R$-代数, 这也是交换代数会考虑的情形. 我们会把$S$看作一个有序对$(S, \varphi)$, 一个交换$R$-代数$S$也叫做一个$R$-(交换)环. 那么交换$R$-代数构成的范畴是交换环范畴的余切片范畴(coslice category).

而这里提到的延拓其实是多项式环的泛性质(universal property), 或者说是自由交换$R$-代数的泛性质, 因为$R[x]$就是一个的自由交换$R$-代数.
\end{remark}

\begin{exercise}[习题2.2.2]
    设$R$是整环, $p \in R$称为一个素元如果它生成的理想
$P=(p)R$是素理想. 证明: $R$中素元必为不可约元.
\end{exercise}

\begin{proof}
    由定义$(p) \neq (1)$, 因此$p$不可逆. 设$p = ab$, 则$ab \in (p)$, 由素理想知$a \in (p)$或$b \in (p)$, 不妨设$a \in (p)$, 则$(a) \subseteq (p)$. 另一方面$(p) \subseteq (a)$, 因此$(p) = (a)$, 从而$b$是单位.
\end{proof}

\begin{remark}
    $x \sim y \defequiv \exists u \in U(R),\, x = uy \iff (x) = (y) \iff x \mid y \text{ 且 } y \mid x$.
\end{remark}

\begin{exercise}[习题2.2.3]
    设$R$是一个主理想整环(PID), $0 \neq r \in R$.
证明: 在$R$中仅有有限个理想包含$r$.
\end{exercise}

\begin{proof}
    $R$是PID, 即对任意理想$I$, 存在$a \in R$, 理想$I = (a)$. 理想$I$包含$r$指$r \in I$, 它等价于$(r) \subseteq I = (a) \iff a \mid r$. 又因为PID是UFD, 因此又唯一分解$r = p_1p_2\cdots p_n$, 从而$r$因子个数在相伴的意义下(上题的注记)有限($\leqslant2^n$), 即包含$r$的理想有限.
\end{proof}

\begin{exercise}[习题2.2.4]
    (辗转相除法) 设$R$是欧氏环, $a, b \in R$非零. 由带余除法得
\[
a = q_{1}b + r_{1},\,
b = q_{2}r_{1}+ r_{2},\,
r_{1} = q_{3}r_{2} + r_{3},\, \cdots,\,
r_{k- 2}= q_{k}r_{k- 1}+ r_{k}
\]
满足$\delta(r_k) < \delta(r_{k - 1}) < \cdots < \delta(r_2) < \delta(r_1) < \delta(b)$.
试证明: 
\begin{enumerate}[(1)]
    \item 存在$k$使得$r_{k + 1} = 0$;
    \item $r_k$是$a$, $b$的一个最大公因子;
    \item 求$u$, $v \in R$使得$r_k = ua + vb$.
\end{enumerate}
\end{exercise}

\begin{proof}
\begin{enumerate}[(1)]
    \item 由于$\delta(b) < \infty$, 且$\delta(r_k)$是严格递减的自然数序列, 因此$\delta(k) \leqslant \delta(b) - k$, 取$k > \delta(b)$即可.
    \item 由(1)知最后一个等式为$r_{k - 1} = q_{k + 1}r_{k}$. 且
\[
    (a, b) = (bq_1 + r_1, b) = (b, r_1) = (q_2r_1 + r_2, r_1) = (r_1, r_2) = \cdots = (r_{k - 1}, r_k) = r_k.
\]
    \item 根据辗转相除法的算式反过来表示$r_k$.
\[
    \begin{aligned}
        r_k &= r_{k - 2} - q_kr_{k - 1} = u_1r_{k - 2} + v_1r_{k - 1},\quad u_1 = 1, v_1 = -q_k\\ 
        &= u_1r_{k - 2} + v_1(r_{k - 3} - q_{k - 1}r_{k - 2}),\quad (r_{k - 3} = q_{k - 1}r_{k - 2} + r_{k - 1})\\
        &= u_2r_{k - 3} + v_2r_{k - 2}, \quad u_2 = -q_k, v_2 = 1 + q_kq_{k - 1}\\
        &= \cdots \\
        &= u_ka + v_kb 
    \end{aligned}
\]
递归关系是$u_i = v_{i - 1}, v_i = u_{i - 1} - v_{i - 1}q_{k - i + 1}$.
\end{enumerate}
\end{proof}

\begin{remark}
    (1)是著名的无穷递降的思路, 即递归的得到一列对象且对应着一个严格递减的自然数序列, 根据自然数有下界$0$来得到矛盾或得出某个结论.

    另外(3)的题干表述可能有些问题, 这里并不需要把$u, v$具体表达出来.
\end{remark}

\begin{exercise}[习题2.2.5]
    设$R = \mathbb{Z}[\sqrt{-5}] = \{a + b\sqrt{-5} \mid \forall a, b \in\mathbb{Z}\} \subset \mathbb{C}$,
定义: $N(a + b\sqrt{-5}) = a^2 + 5b^2$. 试证明: 
\begin{enumerate}[(1)]
    \item $U(R) = \{1, -1\}$;
    \item $R$中任意元素都有不可约分解;
    \item $3$, $2 + \sqrt{-5}$, $2 - \sqrt{-5} \in R$是不可约元;
    \item $9 = 3 \cdot 3= (2 + \sqrt{-5}) \cdot (2 - \sqrt{-5})$是$9$的两个不相同的不可约分解.
\end{enumerate}
\end{exercise}

\begin{proof}
\begin{enumerate}[(1)]
    \item 验证$N$满足$N(\alpha\beta) = N(\alpha)N(\beta)$, 这和复数中$|z_1z_2| = |z_1||z_2|$是类似的, 且$N(\alpha) \in \mathbb{N}$. 那么若$\alpha$是单位, 则存在$\beta$使得$\alpha\beta = 1$, 故$N(\alpha\beta) = N(\alpha)N(\beta) = N(1) = 1$, 故只能有$N(\alpha) = N(\beta) = 1$, 解得$\alpha = \pm 1$.
    \item 因为$R$是Noether环.(但其实没那么好说明, 如果知道Hilbert's Basis Theorem就没问题)
    
    或者可以用$N(\alpha)$保持乘法的特性. 对任意$\alpha \in R$, 若它不可约, 则已经是一个分解了; 否则$\alpha = \beta\gamma$, 其中$\beta, \gamma$不是单位, 且有$N(\alpha) = N(\beta)N(\gamma)$. 因此$N(\beta), N(\gamma) < N(\alpha)$, 由于$N(\alpha) < \infty$, 因此这样分解是有限的, 这和Noether环$\implies$存在分解的过程是类似的.
    \item 由于$N(3) = N(2 + \sqrt{-5}) = N(2 - \sqrt{-5}) = 9 = 3^2$, 若它们可约, 则存在$\alpha$使得$N(\alpha) = 3$, 这是不可能的.
    
    另外, 若$N(\alpha)$是素数, 则一定不可约, 但是反过来不对, 比如这里$9$并不是素数.
    \item 由(3).
\end{enumerate}
\end{proof}

\begin{remark}
    \begin{enumerate}[1.]
    \item 这个$N$是范数(norm). 它其实是$\mathbb{Q}$-线性映射$\beta \mapsto \alpha\beta$所对应矩阵的行列式. 这个概念在模论和代数数论都有提及.
    \item (1)和(2)的结论是可以推广的, 对于一个代数数域$K/\mathbb{Q}$(即$\mathbb{Q}$的有限扩张), $\alpha \in \mathcal{O}_K$是单位当且仅当$N_{K/\mathbb{Q}}(\alpha) = 1$. 其中$\mathcal{O}_K$是对应的代数整数环. $\mathcal{O}_K$是存在不可约分解的环. 其证明方法和(2)几乎一模一样. 具体细节参考代数数论的教材.
\end{enumerate}
\end{remark}

\subsection*{课上的补充内容}
\begin{additional}[$\mathbb{Z}$是PID]
    若$I \subseteq \mathbb{Z}$是理想, 则$\exists n \in \mathbb{Z}_{>0}$使得$I = (n)$.
\end{additional}

\begin{additional}[集合论相关]
\begin{enumerate}[I.]
    \item 设$f: X \to Y$是映射, $X_1 \subseteq X, Y_1 \subseteq Y$,
    \begin{enumerate}[(1)]
        \item $X_1$的像(image)$f(X_1)$是一个$Y$的子集, $f(X_1) \defeq \{f(x) \in Y \mid x \in X_1\}$,
        \item $Y_1$的原像(preimage)$f^{-1}(Y)$是一个$X$的子集, $f^{-1} = \{x \in X \mid f(x) \in Y_1\}$. 当$Y_1$是单点集$\{y\}$时, $f^{-1}(Y_1)$可以简记为$f^{-1}(y)$.
    \end{enumerate}
    \item 映射$f: X \to Y$是
    \begin{enumerate}[(1)]
        \item 单的(injective), 如果$f(x) = f(y) \implies x = y$,
        \item 满的(surjective), 如果$f(X) = Y$.
        \item 双射, 如果$f$既单又满.
        \item 可逆的, 如果存在$g: Y \to X$使得$g \circ f = \mathrm{id}_X,\, f \circ g = \mathrm{id}_Y$.
    \end{enumerate}
    \item 若$f: X \to Y$, $g: Y \to X$, 则
    \[
        g \circ f = \mathrm{id}_X \implies f \text{ 单},\, g \text{ 满}
    \]
    因此可逆映射是双射.
    \item 若$f: X \to Y$满且有交换图表
    \[
        \begin{tikzcd}
            X \arrow[rr, "\varphi"] \arrow[rd, "f"', two heads] &                    & Z \\
                                                                & Y \arrow[ru, "g"'] &  
            \end{tikzcd}
    \]
    则$g$唯一. 换句话说, $g_1 \circ f = g_2 \circ f \implies g_1 = g_2$. 在一般的范畴中, 满足这条性质的态射叫一个满态射(epimorphism).

    类似的, 若$f$是单的, 且有交换图表
    \[
        \begin{tikzcd}
            Z \arrow[rr, "\varphi"] \arrow[rd, "g"'] &                          & Y \\
                                                     & X \arrow[ru, "f"', hook] &  
            \end{tikzcd}
    \]
    则$g$唯一. 即$f \circ g_1 = f \circ g_2 \implies g_1 = g_2$. 满足这条性质的态射叫一个单态射(monomorphism).
\end{enumerate}
\end{additional}
\end{document}