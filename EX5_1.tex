\subsection{教材p91}

\begin{problem}
    设$R \xrightarrow{\varphi} R'$是环同态, $M$是一个$R'$-模.
证明: 
\[
    R \times M \to M,\quad (a, x) \mapsto \varphi(a)x,
\]
定义了$M$的一个$R$-模结构使得$M$成为一个$R$-模.
\end{problem}

\begin{proof}
    
\end{proof}

\begin{problem}
    设$M$是一个$R$-模, $\mathrm{Ann}(M) = \{a \in R \mid ax = 0, \forall x \in M\}$,
证明: 
\begin{enumerate}[(1)]
    \item $\mathrm{Ann}(M) \,\red{\subseteq}\, R$是理想;
    \item 对任意理想$I \,\red{\subseteq}\, R$, 若$I \,\red{\subseteq}\, \mathrm{Ann}(M)$,
则$R/I \times M \to M,\, (\bar{a},x) \mapsto ax$,
定义了$M$的一个$R/I$-模结构.
\end{enumerate}
\end{problem}

\begin{proof}
    
\end{proof}

\begin{problem}
    设$M = (M, +, 0)$是加法群, $\mathrm{End}(M) = \{M \xrightarrow{\varphi} M \mid \varphi \text{ 是群同态}\}$
    是$M$所有群自同态组成的环. 试证明: 
\begin{enumerate}[(1)]
    \item $\mathrm{End}(M) \times M \to M,\, (\varphi, x) \mapsto \varphi \cdot x \defeq \varphi(x)$,
是$M$的一个$\mathrm{End}(M)$-模结构. (因此, $M$是一个$\mathrm{End}(M)$-模.)
    \item 设$R$是一个环, 则$M$有一个$R$-模结构$R \times M \to M,\, (a, x) \mapsto ax$
的充要条件是存在环同态$R \xrightarrow{\eta} \mathrm{End}(M)$使得
$ax = \eta(a)(x)$对任意$a \in R, x \in M$成立.
\end{enumerate}
\end{problem}

\begin{proof}

\end{proof}

\begin{problem}
    设$M = (M, +, 0)$是任意加法群, 证明: $M$有唯一的$\mathbb{Z}$-模结构.
\end{problem}

\begin{proof}
    
\end{proof}

\begin{problem}
    设$R$-模$M$的模结构由环同态$R \xrightarrow{\eta} \mathrm{End}(M)$确定, 
$\varphi \in \mathrm{End}(M)$. 试证明: 
$M \xrightarrow{\varphi} M$是$R$-模同态当且仅当
$\varphi \circ \eta (a) = \eta (a) \circ \varphi$, $\forall a \in R$.
\end{problem}

\begin{proof}
    
\end{proof}

\begin{problem}
    $R$-模$M$称为不可约模, 如果$M \neq 0$且$M$没有非平凡子模.
证明: $R$-模$M$不可约当且仅当存在极大左理想$I \,\red{\subseteq}\, R$
使得$M \cong R/I$.
\end{problem}

\begin{proof}
    
\end{proof}

\begin{problem}[舒尔(Schur)引理]
    证明: 如果$M_1, M_2$是不可约$R$-模, 则任何非
零模同态$M_1 \to M_2$必为同构.
\end{problem}

\begin{proof}
    
\end{proof}

\begin{problem}[同态基本定理]
    设$\varphi:M\to M^{\prime}$是$R$-模同态.证明: $\varphi$的核
\[
    \ker(\varphi) = \{x \in M \mid \varphi(x) = 0\}
\]
和像 $\mathrm{Im}(\varphi) = \{\varphi(x) \mid \forall x \in M\}$必为子模, 
且$\varphi$的诱导映射
\[
    \overline{\varphi}:M/\ker(\varphi) \to \mathrm{Im}(\varphi),\: \overline{\varphi}(\bar{x}) = \varphi(x),
\]
必为同构.
\end{problem}

\begin{proof}
    
\end{proof}