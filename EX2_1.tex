\subsection{教材p28-p29}

\begin{problem}
    设$R$是一个交换环, $I \,\red{\subsetneq}\, R$是一个理想. 证明
\[
    \sqrt{I} = \{r \in R \mid \exists m \in \red{\mathbb{N}} \text{ 使得 } r^m \in I\}
\]
也是$R$的理想 (称为理想$I$的根).
\end{problem}

注: 这题的根理想定义有误, 应是$\mathbb{N}$而不是$\mathbb{Z}$. 一旦出现负整数意味着有可逆元, 从而$\sqrt{I}$是单位理想了.

\begin{solution}
    可以清楚地看出$I \subseteq \sqrt{I}$. 先验证加法子群, 
    \[
    \begin{gathered}
        \forall a, b \in \sqrt{I},\, \exists m, n \in \mathbb{N},\, a^m, b^n \in I,\\
        \implies (a - b)^{m + n} \in I
    \end{gathered}
    \]
    这是因为单项$a^ib^j$的指数$i + j = m + n$, 故$i < m$和$j < n$不能同时成立, 即$i \geqslant m$或$j \geqslant n$, i.e. $a^i \in I$或$b^j \in I$.
    从而$(a - b)^{m + n} \in I$, $a - b \in \sqrt{I}$.

    再验证吸收律(交换验证单边即可),
    \[
        \forall a \in \sqrt{I}, r \in R,\, \exists m \in \mathbb{N},\, a^m \in I \implies (ar)^m = a^mr^m \in I
    \]
    因此$ar \in \sqrt{I}$.

注: 零理想的根$\sqrt{\{0\}} = \{x \in R \mid \exists n \in \mathbb{N}, x^n = 0\}$是所有幂零元(nilpotent)组成的理想, 叫做$R$的幂零根(nilradical), 一般记作$\mathfrak{N}(R)$. 可以证明$\mathfrak{N}(R) = \bigcap_{\mathfrak{p} \text{ 是素理想}} \mathfrak{p}$.(可以参考\cite{atiyah1994introduction}p5)
\end{solution}

\begin{problem}
    设$R$是一个交换环, $p > 0$是一个素数. 如果
$p \cdot x = 0 (\forall x\in R)$. 试证明:
$(x + y)^{p^m} = x^{p^m}+y^{p^m} (\forall x, y \in R, m > 0)$
\end{problem}

\begin{solution}
    事实上, 这个$p$就是环$R$的特征. 若$\mathrm{Char}(R) \neq p$, 则由$p = 0$, $\mathrm{Char}(R) < p$.
    那么$(p, \mathrm{Char}(R)) = 1$, 有Bezout's Theorem得到$1 = 0$, 这就没什么考虑的必要了.

    对特征$p$的交换环, 有一个特别的同态$F$称为Frobenius自同态,
    \[
        F: R \to R, \quad a \mapsto a^p
    \]
    我们说明这确实是一个同态.
    
    保持乘法是因为交换环, 不平凡的是保持加法.
    \[
        (a + b)^p = a^p + \binom{p}{1}a^{p - 1}b + \cdots + b^p.
    \]
    其中$1 \leqslant i \leqslant p - 1$时,
    \[
        \binom{p}{i} = \frac{p(p - 1) \cdots (p - i + 1)}{1 \cdot 2 \cdots i} 
    \]
    由于$p$是素数, $1, 2, \cdots i$都不整除$p$, 而$\binom{p}{i}$是整数, 因此只能是$i! \mid (p - 1) \cdots (p - i + 1)$. 所以$p \mid \binom{p}{i}$. 而$p = 0$, 故$(a + b)^p = a^p + b^p$.

    因此$\varphi: R \to R,\, x \mapsto x^{p^m}$也是自同态, $\varphi = F^m$, 这里$F^m$表示复合$m$次.

注: Frobenius一般在域中使用的多一些. 虽然对交换环Frobenius都是可以定义的, 但是整环才能保证Frobenius是单射. Frobenius一般不是满的, 但对有限域就是自同构了.
\end{solution}

\begin{problem}
    证明:只有有限个元素的整环一定是一个域.
\end{problem}

\begin{solution}
    整环$R$有乘法消去律\ref{ex:1.1.1}, 而\ref{ex:1.3.9}告诉我们, 满足消去律的有限半群是群. 因此$(R \setminus \{0\}, \cdot)$是群, 即$R$是一个域.
\end{solution}

\begin{problem}\label{ex:2.1.4}
    证明:只有有限个理想的整环是一个域.
\end{problem}

\begin{solution}
    事实上条件可以再减弱一点, 一个Artin整环一定是域.

    设$a \neq 0$, 考虑理想降链
    \[
        (a) \supseteq (a^2) \supseteq \cdots 
    \]
    因此$\exists n \in \mathbb{Z}_{>0},\, (a^n) = (a^{n + 1})$. 即有$a^n \in (a^{n + 1})$, 那么$\exists b \in R,\, a^n = a^{n + 1}b$, 从而$ab = 1_R$.

注: Artin环定义为任意理想降链稳定的环, i.e. 若有理想降链
\[
    I_1 \supseteq I_2 \supseteq \cdots 
\]
则存在$n \in \mathbb{Z}_{>0}$使得$\forall m > n,\, I_m = I_n$, 也就是说从某一个$n$开始就稳定了$I_n = I_{n + 1} = \cdots$.
这个条件称为descending chain condition(d.c.c.), 与之对应的是ascending chain condtion(a.c.c.), 满足a.c.c.的正是Noether环.
\end{solution}

\begin{problem}\label{ex:2.1.5}
    理想$P \,\red{\subsetneq}\, R$称为素理想, 如果:$ab \in P \Rightarrow a \in P$
或$b \in P$. 试证明:
$P \,\red{\subsetneq}\, R$是素理想当且仅当$R/P$没有零因子.
\end{problem}

\begin{solution}
    \begin{enumerate}[(1)]
        \item "$\implies$":
        \[
            \forall \overline{a}, \overline{b} \in R/P,\, \overline{a}\overline{b} = \overline{ab} = \overline{0} \implies ab \in P \implies a \in P \text{ or } b \in P \implies \overline{a} = \overline{0} \text{ or } \overline{b} = \overline{0}.
        \]
        \item "$\impliedby$":
        \[
            ab \in P \implies \overline{a}\overline{b} = \overline{ab} = \overline{0} \implies \overline{a} = 0 \text{ or } \overline{b} = 0 \implies a \in P \text{ or } b \in P.
        \]
    \end{enumerate}

注: \begin{enumerate}[1.]
    \item 零理想$(0)$是素理想.
    \item 一个交换环的(Krull) dimension定义为最长素理想链的长度, 其中, 若有素理想链
    \[
    \mathfrak{p}_0 \subsetneq \mathfrak{p}_1 \subsetneq \cdots \subsetneq \mathfrak{p}_n
    \]
    他的长度定义为$n$.(可参考\cite{atiyah1994introduction}p89, \cite{aluffi2009algebra}p153)

    交换Artin环(\ref{ex:2.1.4})是$0$维的Noether环. $0$维即意味着所有的素理想都是极大理想.
    \item 对交换环$R$, $\mathrm{Spec}(R) \defeq \{\mathfrak{p} \mid \mathfrak{p} \text{ 是 } R \text{ 的素理想}\}$称为$R$的素谱(spectrum). $\mathrm{Spec}(R)$上有一个Zariski拓扑.(这个就不给参考书了, 自行搜索或查阅代数几何相关书籍吧)
\end{enumerate} 
\end{solution}

\begin{problem}\label{ex:2.1.6}
    理想$m \,\red{\subsetneq}\, R$称为极大理想, 如果$R$中不存在真包含$m$的非平凡理想
(即:如果$I \,\red{\supsetneq}\, m$是$R$的理想, 则必有$I = R)$. 试证明:当$R$是交换环时, 
$m \,\red{\subsetneq}\, R$是极大理想当且仅当$R/m$是一个域. 特别, 交换环中的极大理想必为素理想.
\end{problem}

\begin{solution}
    \begin{enumerate}[(1)]
        \item "$\implies$":
        \[
        \begin{aligned}
            \forall \overline{0} \neq \overline{a} \in R/m &\implies a \notin m \implies m \subsetneq m + (a) \implies m + (a) = R = (1)\\
            &\implies \exists x \in m, b \in R,\, x + ab = 1 \implies \overline{ab} = \overline{1 - x} = \overline{1}.
        \end{aligned}
        \]
        \item "$\impliedby$":
        \[
        \begin{aligned}
            m \subsetneq I \underset{\text{ideal}}{\subseteq} R &\implies \exists a \in I \setminus m \text{ i.e. } \overline{a} \neq 0 \implies \exists b \in R,\, \overline{a}\overline{b} = \overline{ab} = \overline{1}\\
            &\implies \exists x \in m \subsetneq I,\, ab = 1 + x \implies 1 = ab - x \in I \implies I = (1) = R.
        \end{aligned}
        \]
        或者用同态基本定理, 包含$m$的理想和$R/m$的理想有一个一一对应, 而域的理想只有$\{0\}$和本身.
    \end{enumerate}
注: (1)中用到了理想的和. 若$I, J$都是$R$的理想, $I + J \defeq \{i + j \mid i \in I, j \in J\}$. 可以验证这确实是一个理想, 类似可以定义一族理想$\{I_\alpha\}_{\alpha \in A}$的和,
\[
    \sum_{\alpha \in A} I_\alpha = \left\{\sum_{\alpha \in A} i_\alpha \Big|i_\alpha \in I_\alpha, \text{ 且只有有限个 } i_\alpha \neq 0 \right\} 
\]
即考虑所有可能的有限和.

    另外$\bigcap_{\alpha \in A} I_\alpha$也是一个理想. 还有一个是理想的积, 相对要复杂一些,
\[
\begin{aligned}
    IJ &\defeq (\{ij \mid i \in I, j \in J\})\\
    &= \left\{\sum_{k = 1}^{n} i_kj_k \Big| \exists n \in \mathbb{N},\, 1 \leqslant k \leqslant n,\, i_k \in I, j_k \in J, \right\}
\end{aligned}  
\]
他是所有乘积$ij$生成的理想. 那么一族理想的乘积就是考虑所有可能的有限乘积生成的理想.
\end{solution}

\begin{problem}
    设 $I \,\red{\subsetneq}\, \mathbb{Z}$是整数环的非零理想, 证明下述结论等价
\begin{enumerate}[(1)]
    \item $I$是极大理想;
    \item $I$是素理想;
    \item 存在素数$p$使得$I = (p)\mathbb{Z} = \{ap \mid \forall a \in \mathbb{Z}\}$.
\end{enumerate}
\end{problem}

\begin{solution}
    \begin{enumerate}[1.]
        \item (1)$\implies$(2): 由于域一定是整环, 由\ref{ex:2.1.5}和\ref{ex:2.1.6}知极大理想是素理想.
        \item (2)$\implies$(3): 由于$\mathbb{Z}$是PID(带余除法可证), 故存在整数$p$使得$I = (p)$. 由于是素理想, 因此$ab \in (p) \implies a \in (p)$或$b \in (p)$. 即
        \[
            p \mid ab \implies p \mid a \text{ or } p \mid b
        \]
        则$p$是素数(若不然, $p = qr$, 取$a = q, b = r$即导出矛盾).
        \item (3)$\implies$(1): 设$I = (p) \subsetneq J$, 则存在$n \in J \setminus I$. 由于$p$是素数, 故有$(n, p) = 1$. 由Bezout's Theorem, $\exists u, v \in \mathbb{Z}$使得$nu + pv = 1$, 从而$1 \in J,\, J = \mathbb{Z}$.(这和\ref{ex:2.1.6}的证明是类似的)
        
        或直接用$\mathbb{Z}/(p) = \mathbb{Z}/p\mathbb{Z}$是域.
    \end{enumerate}
\end{solution}

\begin{problem}\label{ex:2.1.8}
    设$p \in \mathbb{Z}$是素数, 证明
$(p)\mathbb{Z}[x] = \{pf(x) \mid \forall f(x) \in \mathbb{Z}[x]\}$是整系数多
项式环的素理想, 但不是$\mathbb{Z}[x]$的极大理想.
\end{problem}

\begin{solution}
    事实上若$I$是$R$的理想, 我们有
\[
    \frac{R[x]}{IR[x]} \cong \frac{R}{I}[x]
\]
这是根据同态基本定理得到, 考虑同态
\[
    \varphi: R[x] \to \frac{R}{I}[x],\quad a_0 + a_1x + \cdots + a_nx^n \mapsto \overline{a_0} + \overline{a_1}x + \cdots + \overline{a_n}x^n
\]
可以验证这确实是一个同态. 事实上, 它是$R \twoheadrightarrow R/I \hookrightarrow \frac{R}{I}[x]$的一个延拓.

回到原题, 有
\[
    \frac{\mathbb{Z}[x]}{(p)\mathbb{Z}[x]} \cong \mathbb{Z}_p[x]
\]
这里$\mathbb{Z}_p = \mathbb{Z}/p\mathbb{Z}$是域. 因此$\mathbb{Z}_p[x]$是PID, 自然是整环, 但不是域($x$没有逆). 因此由\ref{ex:2.1.5}和\ref{ex:2.1.6}, $(p)\mathbb{Z}[x]$是素理想但不是极大理想.

注: 给定环同态$R \overset{\varphi}\to S$, 其中$R$是交换环. 若$\varphi(R) \subseteq C(S)$(\ref{ex:2.1.11}), 根据我们之前\ref{ex:1.4.9}说过的, 首先$S$上有一个$R$-模结构. 其次有
\[
    (r_1s_1)(r_2s_2) = \varphi(r_1)s_1\varphi(r_2)s_2 = \varphi(r_1)\varphi(r_2)s_1s_2 = \varphi(r_1r_2)s_1s_2 = (r_1r_2)(s_1s_2).
\]
即数乘和$S$本身的乘法是相容的. 这样的结构我们称为一个$R$-代数($R$-algebra), 这也是\ref{ex:2.1.12}介绍的东西. 因此一个$R$-代数就是带有加法, ($R$-)数乘, 乘法的一个代数结构.

当$S$本身就是交换环时, 此时乘法是交换的, 且$C(S) = S$, 这样会变得简单很多. 这时$S$称为一个交换$R$-代数, 这也是交换代数会考虑的情形. 我们会把$S$看作一个有序对$(S, \varphi)$, 一个交换$R$-代数$S$也叫做一个$R$-(交换)环. 那么交换$R$-代数构成的范畴是交换环范畴的余切片范畴(coslice category).

而这里提到的延拓其实是多项式环的泛性质(universal property), 或者说是自由交换$R$-代数的泛性质, 因为$R[x]$就是一个的自由交换$R$-代数.(可参考\cite{aluffi2009algebra}III.\S6.3)
\end{solution}

\begin{problem}\label{ex:2.1.9}
    映射$D:R[x] \longrightarrow R[x]$定义如下:
$\forall f(x) = a_nx^n + \cdots + a_1x + a_0$,
\[
    D(f) = na_nx^{n - 1} + (n - 1)a_{n - 1}x^{n - 2} + \cdots + 2a_2x + a_1.
\]
$\forall$ $a \in R$, $f, g \in R[x]$, 试证明:
\begin{enumerate}[(1)]
    \item $D(f + g) = D(f) + D(g)$, $D(af) = aD(f)$;
    \item $D(f \cdot g) = D(f) \cdot g + f \cdot D(g)$.
\end{enumerate}
($D(f)$称为$f(x)$的导数. 记为
$f'(x) = D(f),\, f^{(m)}(x) = \overset{m}{\widehat{D \cdots D}}(f)$
称为$f(x)$的$m$次导数).
\end{problem}

\begin{solution}
    按定义验证. 设$f = a_nx^n + \cdots + a_1x + a_0$, $g = b_mx^m + \cdots + b_1x + b_0$.
    \begin{enumerate}[(1)]
        \item 不妨设$n \geqslant m$, 且令$b_k = 0, k > m$.
        \[
        \begin{aligned}
            D(f + g) &= D\left(\sum_{k = 0}^{n} (a_k + b_k)x^k\right) = \sum_{k = 1}^{n} k(a_k + b_k)x^{k - 1}\\ 
            &= \sum_{k = 1}^{n} ka_kx^{k - 1} + \sum_{k = 1}^{m} kb_kx^{k - 1} = D(f) + D(g).
        \end{aligned}
        \]
        \item \[
            \begin{aligned}
                D(f \cdot g) &= D\left(\sum_{k = 0}^{n + m} \sum_{i + j = k} a_ib_jx^k\right) = \sum_{k = 1}^{n + m} \sum_{i + j = k} ka_ib_jx^{k - 1}\\
                &= \sum_{k = 1}^{n + m} \sum_{i + j = k} (i + j)a_ib_jx^{i + j - 1}\\
                &= \sum_{k = 1}^{n + m} \sum_{i + j = k} (ia_ix^{i - 1})b_jx^j + a_ix^i(jb_jx^{j - 1})\\
                &= \sum_{k = 0}^{n + m - 1} \sum_{(i - 1) + j = k} (ia_i)b_jx^{k} + a_i(jb_j)x^k\\
                &= D(f) \cdot g + f \cdot D(g).
            \end{aligned}
        \]
    \end{enumerate}
\end{solution}

\begin{problem}
    如果$F$是特征零的域, 则$f'(x) = 0 \Leftrightarrow \deg(f) = 0$
或$f(x) = 0$(即常数); 如果$F$的特征是$p > 0$, 则
$f'(x) = 0 \Leftrightarrow$存在$g(x) \in F[x]$使得$f(x) = g(x^p)$.
\end{problem}

\begin{solution}
    $\mathrm{Char}(F) = 0$, 即$\forall n \in \mathbb{Z}_{>0}, n \neq 0$(\ref{ex:1.2.1}的注记), 那么
\[
    f'(x) = na^{n - 1} + \cdots + a_1 = 0 \implies 1 \leqslant k \leqslant n, ka_k = 0 \implies 1 \leqslant k \leqslant n, a_k = 0
\]
故$f(x) = a_0$, $\deg(f) = 0$或$f = 0$, 反过来是平凡的.

    若$\mathrm{Char}(F) = p$, 则$p = 0$, 那么设$\deg(f) = n = kp + r, 0 \leqslant r < p, k \in \mathbb{N}$,
\[
\begin{aligned}
    f &= a_0 + a_1x + \cdots a_px^p + \cdots + a_{2p}x^{2p} + \cdots + a_{kp}x^{kp} + a_nx^n.\\
    \implies f' &= a_1 + \cdots + pa_px^{p - 1} + \cdots + kpa_{kp}x^{kp - 1} + \cdots na_nx^{n - 1}\\
    &= a_1 + \cdots + (p - 1)a_{p - 1}x^{p - 2} + (p + 1)a_{p + 1}x^p + \cdots (kp - 1)a_{kp - 1}x^{kp - 2}\\
    &+ (kp + 1)a_{kp + 1}x^kp + \cdots + na_nx^{n - 1}.
\end{aligned}
\]
此时$f' = 0$有$f = a_0 + a_px^p + \cdots + a_{kp}x^{kp} = g(x^p)$. 这里$g = a_0 + a_px + \cdots + a_{kp}x^k$. 反过来也是类似的.
\end{solution}

\begin{problem}\label{ex:2.1.11}
    设$R$是一个环, 子环
$C(R) = \{a \in R \mid ab = ba \forall b \in R\}$
称为$R$的中心. 试证明:
\begin{enumerate}[(1)]
    \item 如果$R$是一个除环, 则$C(R)$是一个域;
    \item 令$\mathbb{H}$表示Hamilton四元数环,则$C(\mathbb{H}) = \mathbb{R}$.
\end{enumerate}
\end{problem}

\begin{solution}
    \begin{enumerate}[(1)]
        \item 除环的子环自然是除环, $C(R)$和$R$中所有元素交换, 故$C(R)$本身是交换环, 从而是域.
        \item 设$\alpha = a + ib + jc + kd \in C(\mathbb{H})$, 则有
        \[
        \begin{aligned}
            \alpha \cdot i &= i \cdot \alpha\\
            \alpha \cdot j &= j \cdot \alpha
        \end{aligned}
        \]
        得到$b = c = d = 0$, 即$\alpha \in \mathbb{R}$.
    \end{enumerate}
\end{solution}

\begin{problem}\label{ex:2.1.12}
    设$K$是一个域. 如果$C(R)$包含一个同构于$K$的子域, 则称环$R$
为$K$-代数. 试证明:加法群$(R, +)$通过$R$的乘法成为一个$K$-向量空间.
\end{problem}

\begin{solution}
    见\ref{ex:1.4.9}和\ref{ex:2.1.8}的注记. 域上的模就是线性空间.

注: $C(R)$包含一个同构于$K$的子域, 即存在同态$K \overset{\varphi}\to R$使得$\varphi(K) \subseteq C(R)$(这是因为域出发的同态一定是单的). 这和之前说的是一样的.
\end{solution}

\begin{problem}
    设$R$是一个$K$-代数, $\dim_K(R)$称为$R$的维数.
试证明:
\begin{enumerate}[(1)]
    \item 矩阵环$M_n(K)$是一个$n^2$维$K$-代数;
    \item 任意$n$维$K$-代数必同构于$M_n(K)$的子环;
    \item 如果$R$是一个有限除环, 则$R$是有限域上的有限维代数.
\end{enumerate}
\end{problem}

\begin{solution}
    \begin{enumerate}[(1)]
        \item $M_n(K)$是$n^2$维$K$-线性空间, 只需验证
        \[
            k_1M_1k_2M_2 = k_1k_2M_1M_2, k_1, k_2 \in K, M_1, M_2 \in M_n(K).
        \]
        这是根据$M_n(K)$的定义. 事实上$C(M_n(K)) = \{kI_n \mid k \in K\} \cong K$.
        \item 由教材例1.4.3, 对任意的环$R$, 我们用$\mathrm{End}_{\mathsf{Ab}}(R)$表示加法群的自同态环(关于加法和复合). 有一个自然的环同态,
        \[
            R \to \mathrm{End}_{\mathsf{Ab}}(R),\quad r \mapsto \lambda_r
        \]
        其中$\lambda_r: R \to R,\, a \mapsto ra$, 即左乘$r$这个自同态(这里换成右乘也是一样的). 这是一个单同态, 所以$R$同构于$\mathrm{End}_{\mathsf{Ab}}(R)$的一个子环.

        那么当$R$是$n$维$K$-代数时, $\lambda_r$还是$K$-线性映射. 因此有单射$R \hookrightarrow \mathrm{Hom}_K(R) \cong M_n(K)$.
        \item $R$是有限除环, 因此$C(R)$是有限域(\ref{ex:2.1.11}). 根据定义$R$是一个$C(R)$-代数, 且$R$有限, 故是有限维的($|R| = [R:C(R)]|C(R)|$).
    \end{enumerate}
\end{solution}

\begin{problem}
    设$K$是一个域, $R$是一个有限维$K$-代数.试证明:
\begin{enumerate}[(1)]
    \item $\forall \alpha \in R$, 存在\red{非零}多项式$f(x) \in K[x]$使得$f(\alpha) = 0$;
    \item 如果$R$是除环, $\alpha \neq 0$, 则$\alpha$的极小多项式$\mu_\alpha(x) \in K[x]$不可约;
    \item 如果$R$是除环, $K$是代数闭域(即$K[x]$中次数大于零的多项式在$K$中
必有根),则$R = K$.
\end{enumerate}
历史上, 有限维可除$K$-代数的分类是一个热门话题. 当$K$是实数域时, $R$必同构于实数域,
复数域或Hamilton四元数环之一(Frobenius 定理);
当$K$是有限域时, $R$必为交换环(Wedderburn 定理).
\end{problem}

零多项式是平凡的, 因此(1)我做了修改. 在域扩张中, 这样的元素称为$K$上的代数元(algebraic element), 或者称$\alpha$在$K$上代数(algebraic over $K$). 给定域扩张$L/K$, 若$\forall \alpha \in L$都在$K$上代数, 则称该扩张是代数扩张.

\begin{solution}
    \begin{enumerate}[(1)]
        \item 设$\mathrm{dim}_K R = n$. 则$1, \alpha, \alpha^2, \cdots, \alpha^n$线性相关. 或者考虑线性映射$r \mapsto \alpha r$. 那么它对应的矩阵的特征多项式满足条件(Cayley-Hamilton Theorem).
        \item 按定义, $\mu_\alpha$是满足$\alpha$的次数最小的(首一)多项式. 假设$\mu_\alpha$可约, 即$\mu_\alpha(x) = f(x)g(x),\, \deg(f),\, \deg(g)> 0$, 则$0 = \mu_\alpha(\alpha) = f(\alpha)g(\alpha)$. 由于除环无零因子, 故$\deg(\mu_\alpha) = \deg(f) + \deg(g)$, 且$f(\alpha) = 0$或$g(\alpha) = 0$. 不妨设$f(\alpha) = 0$, 但$\deg(f) < \deg (\mu_\alpha)$与极小矛盾.
        \item 代数闭域等价于任意多项式可分解成一次多项式的乘积. 这和代数基本定理是类似的. 此时$K[x]$中的不可约多项式即为所有一次多项式. 由(2), $\forall \alpha \in R$, 极小多项式$\mu_\alpha(x) = x - k_\alpha, k_\alpha \in K$. 因此$\alpha = k_\alpha \in K$. 即$R = K$.
    \end{enumerate}
\end{solution}

\begin{problem}
    证明:集合
\(
    \mathbb{F}_{3^2} =
    \left\{
        \begin{pmatrix}
            a & b\\
            -b & a
        \end{pmatrix} 
    \bigg| a, b \in \mathbb{F}_3 = \mathbb{Z}/(3)
    \right\}
\)
关于矩阵的“加法”和“乘法”成为一个9元域.
若将定义中的$\mathbb{F}_3$换成$\mathbb{F}_5$,
上述集合是否是一个25元域, 为什么?
\end{problem}

\begin{solution}
    这个集合是$\mathbb{F}_3$上的$2$维线性空间.
\[
    \mathbb{F}_{3^2} = \left\{ a\lambda + b\xi \mid
    \lambda = I_2 = 
    \begin{pmatrix}
        1 & 0\\
        0 & 1
    \end{pmatrix},
    \xi = 
    \begin{pmatrix} 
        0 & 1\\
        -1 & 0
    \end{pmatrix}, a, b \in \mathbb{F}_3 \right\}
\]
其中$\xi$的特征多项式为$x^2 + 1$, 可以验证它是不可约的. 又因为$\mathbb{F}_3[x]$是PID($\mathbb{F}_3[x]$是域), 故$(x^2 + 1)$是极大理想, $x^2 + 1$就是$\xi$的极小多项式. 则$\mathbb{F}_{3^2} = \mathbb{F}_3[x]/(x^2 + 1)$是域.

但$x^2 + 1$在$\mathbb{F}_5[x]$中是可约的: 在$\mathbb{F}_5[x]$中, 
\[
    x^2 + 1 = x^2 - 4 = (x - 2)(x + 2).
\]
\end{solution}