\subsection{教材p28-p29}

\begin{problem}
    设$R$是一个交换环, $I \subset R$是一个理想. 证明
\[
    \sqrt{I} = \{r \in R \mid \exists m \in \mathbb{Z} \text{ 使得 } r^m \in I\}
\]
也是$R$的理想 (称为理想$I$的根).
\end{problem}

\begin{solution}
    
\end{solution}

\begin{problem}
    设$R$是一个交换环, $p > 0$是一个素数. 如果
$p \cdot x = 0 (\forall x\in R)$. 试证明:
$(x + y)^{p^m} = x^{p^m}+y^{p^m} (\forall x, y \in R, m > 0)$
\end{problem}

\begin{solution}
    
\end{solution}

\begin{problem}
    证明:只有有限个元素的整环一定是一个域.
\end{problem}

\begin{solution}
    
\end{solution}

\begin{problem}
    证明:只有有限个理想的整环是一个域.
\end{problem}

\begin{solution}
    
\end{solution}

\begin{problem}
    理想$P \subset R$称为素理想, 如果:$ab \in P \Rightarrow a \in P$
或$b \in P$. 试证明:
$P \subset R$是素理想当且仅当$R/P$没有零因子.
\end{problem}

\begin{solution}
    
\end{solution}

\begin{problem}
    理想$m \subset R$称为极大理想, 如果$R$中不存在真包含$m$的非平凡理想
(即:如果$I \supsetneq m$是$R$的理想, 则必有$I = R)$. 试证明:当$R$是交换环时, 
$m \subset R$是极大理想当且仅当$R/m$是一个域. 特别, 交换环中的极大理想必为素理想.
\end{problem}

\begin{solution}
    
\end{solution}

\begin{problem}
    设 $I \subset \mathbb{Z}$是整数环的非零理想, 证明下述结论等价
\begin{enumerate}[(1)]
    \item $I$是极大理想;
    \item $I$是素理想;
    \item 存在素数$p$使得$I = (p)\mathbb{Z} = \{ap \mid \forall a \in \mathbb{Z}\}$.
\end{enumerate}
\end{problem}

\begin{solution}
    
\end{solution}

\begin{problem}
    设$p \in \mathbb{Z}$是素数, 证明
$(p)\mathbb{Z}[x]=\{pf(x) \mid \forall f(x) \in \mathbb{Z}[x]\}$是整系数多
项式环的素理想, 但不是$\mathbb{Z}[x]$的极大理想.
\end{problem}

\begin{solution}
    
\end{solution}

\begin{problem}
    映射$D:R[x] \longrightarrow R[x]$定义如下:
$\forall f(x) = a_nx^n + \cdots + a_1x + a_0$,
\[
    D(f) = na_nx^{n - 1} + (n - 1)a_{n - 1}x^{n - 2} + \cdots + 2a_2x + a_1.
\]
$\forall$ $a\in R$, $f, g\in$ $R[ x]$, 试证明:
\begin{enumerate}[(1)]
    \item $D(f + g) = D(f) + D(g)$, $D(af) = aD(f)$;
    \item $D(f \cdot g) = D(f) \cdot g + f \cdot D(g)$.
\end{enumerate}
($D(f)$称为$f(x)$的导数. 记为
$f'(x) = D(f),\, f^{(m)}(x) = \overset{m}{\widehat{D \cdots D}}(f)$
称为$f(x)$的$m$次导数).
\end{problem}

\begin{solution}
    
\end{solution}

\begin{problem}
    如果$F$是特征零的域, 则$f'(x) = 0 \Leftrightarrow \deg(f) = 0$
或$f(x) = 0$(即常数); 如果$F$的特征是$p > 0$, 则
$f'(x) = 0 \Leftrightarrow$存在$g(x) \in F[x]$使得$f(x) = g(x^p)$.
\end{problem}

\begin{solution}
    
\end{solution}

\begin{problem}
    设$R$是一个环, 子环
$C(R) = \{a \in R \mid ab = ba \forall b \in R\}$
称为$R$的中心. 试证明:
\begin{enumerate}[(1)]
    \item 如果$R$是一个除环, 则$C(R)$是一个域;
    \item 令$\mathbb{H}$表示Hamilton四元数环,则$C(\mathbb{H}) = \mathbb{R}$.
\end{enumerate}
\end{problem}

\begin{solution}
    
\end{solution}

\begin{problem}
    设$K$是一个域. 如果$C(R)$包含一个同构于$K$的子域, 则称环$R$
为$K$-代数. 试证明:加法群$(R, +)$通过$R$的乘法成为一个$K$-向量空间.
\end{problem}

\begin{solution}
    
\end{solution}

\begin{problem}
    设$R$是一个$K$-代数, $\dim_K(R)$称为$R$的维数.
试证明:
\begin{enumerate}[(1)]
    \item 矩阵环$M_n(K)$是一个$n^2$维$K$-代数;
    \item 任意$n$维$K$-代数必同构于$M_n(K)$的子环;
    \item 如果$R$是一个有限除环, 则$R$是有限域上的有限维代数.
\end{enumerate}
\end{problem}

\begin{solution}
    
\end{solution}

\begin{problem}
    设$K$是一个域, $R$是一个有限维$K$-代数.试证明:
\begin{enumerate}[(1)]
    \item $\forall \alpha \in R$, 存在多项式$f(x) \in K[x]$使得$f(\alpha) = 0$;
    \item 如果$R$是除环, $\alpha \neq 0$, 则$\alpha$的极小多项式$\mu_\alpha(x) \in K[x]$不可约;
    \item 如果$R$是除环, $K$是代数闭域(即$K[x]$中次数大于零的多项式在$K$中
必有根),则$R = K$.
\end{enumerate}
历史上, 有限维可除$K$-代数的分类是一个热门话题. 当$K$是实数域时, $R$必同构于实数域,
复数域或Hamilton四元数环之一(Frobenius 定理);
当$K$是有限域时, $R$必为交换环(Wedderburn 定理).
\end{problem}

\begin{solution}
    
\end{solution}

\begin{problem}
    证明:集合
\(
    \mathbb{F}_{3^2} =
    \left\{
        \left(
        \begin{pmatrix}
            a & b\\
            -b & a
        \end{pmatrix}
        \right) 
    \bigg| a, b \in \mathbb{F}_3 = \mathbb{Z}/(3)
    \right\}
\)
关于矩阵的“加法”和“乘法”成为一个9元域.
若将定义中的$\mathbb{F}_3$换成$\mathbb{F}_5$,
上述集合是否是一个25元域, 为什么?
\end{problem}

\begin{solution}
    
\end{solution}