\subsection{教材p64}

\begin{problem}
    设$f(x) = x^2 + ax + b \in K[x]$不可约, $E = K[u_1]$
(其中$f(u_1) = 0$) 证明: $E$必包含$f(x) = 0$的另一个根
(所以$E$是$f(x)$的分裂域).
\end{problem}

\begin{proof}
    
\end{proof}

\begin{problem}
    设$f(x) = x^3 - 2 \in \mathbb{Q}[x], u_1 = \sqrt[3]{2}$.
证明: $E = \mathbb{Q}[u_1]$不包含$f(x) = 0$的其他两个根.
\end{problem}

\begin{proof}
    
\end{proof}

\begin{problem}
    设$L$是$n$次多项式$f(x) \in K[x]$的分裂域, 证明: 
$[L:K] \leqslant n!$.
\end{problem}

\begin{proof}
    
\end{proof}

\begin{problem}
    构造$x^5 - 2 \in \mathbb{Q}[x]$的一个分裂域$L$,
并求$[L:\mathbb{Q}]$.
\end{problem}

\begin{proof}
    
\end{proof}

\begin{problem}
    确定多项式$x^{p^n} - 1 \in \mathbb{F}_p[x]$在$\mathbb{F}_p$
上的分裂域$(n \in \mathbb{N})$.
\end{problem}

\begin{proof}
    
\end{proof}

\begin{problem}
    设$L$是可分多项式$f(x) \in K[x]$的一个分裂域, $K \subset E \subset L$是任意中间域.
证明: 对任意单同态$\varphi:E \to L$,若$\varphi|_K = \mathrm{id}_K$,
则$\varphi$一定可以延拓成域同构$\overline\varphi:L \to L$.
\end{problem}

\begin{proof}
    
\end{proof}

\begin{problem}
    令$f(x) = (x^2 - 2)(x^2 - 3)$,
$K = \mathbb{Q}[x]/(x^2 - 2) = \mathbb{Q}[u_1]$,
此处$u_1 = \bar{x} \in \mathbb{Q}[x]/(x^2 - 2)$.
试证明: 
\begin{enumerate}[(1)]
    \item $K$是一个域, 且$x^2 - 3$在$K[x]$中不可约;
    \item $L = K[x]/(x^2 - 3) = K[u_2]$
(此处$u_2 = \bar{x} \in K[x]/(x^2 - 3))$是
$f(x) = (x^2 - 2)(x^2 - 3)$的分裂域, 且
$[L:\mathbb{Q}] = 4$.
\end{enumerate}
\end{problem}

\begin{proof}
    
\end{proof}

\begin{problem}
    设$p \in \mathbb{Z}$是一个素数, $F$是一个域,
$c\in F$. 求证: $x^p - c$在$F[x]$中不
可约当且仅当$x^p - c$在$F$中无根.
\end{problem}

\begin{proof}
    
\end{proof}

\begin{problem}
    设$f(x)$是$\mathbb{Q}[x]$中奇数次的不可约多项式,
$\alpha$和$\beta$是$f(x)$在$\mathbb{C}$中的
两个不同的根. 试证明$\alpha + \beta \notin \mathbb{Q}$
且$\alpha\beta \notin \mathbb{Q}$.
\end{problem}

\begin{proof}
    
\end{proof}

\begin{problem}
    设$K = \mathbb{Q}[u]$, $u^3 + u^2 - 2u - 1 = 0$.
验证$\alpha = u^2 - 2$也是多项式$x^3 + x^2 - 2x - 1$的根.
试确定$\mathrm{Gal}(K/\mathbb{Q})$,
并证明: $K$是$\mathbb{Q}$的正规扩张.
\end{problem}

\begin{proof}
    
\end{proof}

\begin{problem}
    证明: $\mathbb{Q}[\sqrt[4]{2}]$是$\mathbb{Q}[\sqrt{2}]$
的正规扩张, 但不是$\mathbb{Q}$的正规扩张.
\end{problem}

\begin{proof}
    
\end{proof}

\begin{problem}
    设$f(x) \in K[x]$不可约, $\mathrm{Char}(K) = p > 0$.
证明: 存在不可约的可分多项式$g(x) \in K[x]$使得$f(x) = g(x^{p^n})$
($n$是某个整数). 由此证明$f(x)$在分裂域中的每个根都是$p^n$重根.
\end{problem}

\begin{proof}
    
\end{proof}

\begin{problem}
    设$L = K[\alpha],~\alpha$是多项式$x^d - a \in K[x]$的根.
如果$\mathrm{Char}(K) = 0$,
且$K$包含全部$d$次单位根, 则$K \subset L$是正规扩张.
\end{problem}

\begin{proof}
    
\end{proof}

\begin{problem}
    设$k$是特征$p > 0$的域, $x, y$是$k$上的代数无关元.
令$K = k(x^p, y^p)$, $L = k(x, y)$. 试证明: 
\begin{enumerate}[(1)]
    \item $\mathrm{Gal}(L/K) = \{1\}$ (但$[L:K] = p^2)$;
    \item $K \subset L$有无穷多个中间域;
    \item $K \subset L$不是单扩张, 即不存在$\alpha \in L$
使得$L = K[\alpha]$.
\end{enumerate}
\end{problem}

\begin{proof}
    
\end{proof}