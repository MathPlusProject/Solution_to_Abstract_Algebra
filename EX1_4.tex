\subsection{教材p21-p22}

\begin{problem}\label{ex:1.4.1}
    设$\varphi:G \to G'$是群同态, 试证明:
    \begin{enumerate}[(1)]
        \item $\ker(\varphi) \defeq \{g \in G \mid \varphi(g) = e'\}$ $(e' \in G'$表示的单位元)是$G$的子群(称为群同态$\varphi$的核);
        \item
        \[
            \varphi(G) = \{\varphi(g) \mid \forall g \in G\} \subset G'
        \]
        是$G'$的子群(称为群同态$\varphi$的像).
    \end{enumerate}
\end{problem}

\begin{proof}
    教材命题1.4.1的(1)(5)直接使用.
    \begin{enumerate}[(1)]
        \item $e \in \ker(\varphi)$非空, 直接验证
        \[
        \begin{gathered}
            \forall a, b \in \ker(\varphi),\, \varphi(ab^{-1}) = \varphi(a)\varphi(b)^{-1} = e'e' = e'\\
            \implies ab^{-1} \in \ker(\varphi).
        \end{gathered}
        \]
        \item $e' \in \varphi(G)$非空, 直接验证
        \[
        \begin{gathered}
            \forall x, y \in \varphi(G),\, \exists a, b \in G,\, x = \varphi(a), y = \varphi(b)\\
            \implies xy^{-1} = \varphi(a)(\varphi(b))^{-1} = \varphi(a)\varphi(b^{-1}) = \varphi(ab^{-1}) \in \varphi(G).
        \end{gathered}
        \]
    \end{enumerate}
\end{proof}

\begin{problem}
    令$G$是函数$f(x) = \frac1x, g(x) = \frac{x-1}x$关于函数的合成生成的一个群(即群乘法为函数合成), 证明$G$同构于$S_3$.
\end{problem}

\begin{proof}
    由\ref{ex:1.3.5}的注记, 只需验证$f^2 = \mathrm{id}, g^3 = \mathrm{id}, fgfg = \mathrm{id}$.
    \[
    \begin{aligned}
        f^2(x) &= f(f(x)) = \frac{1}{\frac{1}{x}} = x.\\
        g^2(x) &= g(g(x)) = 1 - \frac{1}{(1 - \frac{1}{x})} = -\frac{1}{x - 1}\\
        g^3(x) &= g(g^2(x)) = 1 - (\frac{1}{-\frac{1}{x - 1}}) = 1 + x - 1 = x.\\
        (fg)(x) &= f(g(x)) = \frac{x}{x - 1},\\
        (fgfg)(x) &= (fg)^2(x) = 1 + \frac{1}{\frac{x}{x - 1} - 1} = 1 + x - 1 = x.
    \end{aligned}
    \]
\end{proof}

\begin{problem}
    设$R \overset{\varphi}\to R'$是环同态, 证明集合$ker(\varphi) = \{x \in R \mid \varphi(x) = 0_{R'}\}$满足:
    \begin{enumerate}[(1)]
        \item $\ker(\varphi)$是$(R, +)$的子群;
        \item $\forall a \in \ker (\varphi), x \in R$有$ax \in \ker(\varphi)$, $xa \in \ker (\varphi)$. ($\ker(\varphi)$称为环同态$\varphi$的核)
    \end{enumerate}
\end{problem}

\begin{proof}
    \begin{enumerate}[(1)]
        \item 即\ref{ex:1.4.1}(1);
        \item 直接验证
        \[
            \forall a \in \ker(\varphi),\, x \in R,\, \varphi(xa) = \varphi(x)\varphi(a) = \varphi(x)0_{R'} = 0_{R'}
        \]
        另一半同理.
    \end{enumerate}
\end{proof}

\begin{remark}
    满足(1)(2)的$R$的子集称为$R$的一个理想(ideal), 教材p25定义2.1.4.
\end{remark}

\begin{problem}
    设$K$是一个域, $\phi:K[x] \to K[x]$是$K$的多项式环之间的环自同态. 如果对于任意的$k \in K, \phi(k) = k$, 试证明: $\phi$是满同态的充分必要条件是存在$a, b \in K(a \neq 0)$使得$\phi(x) = ax + b$.
\end{problem}

\begin{proof}
    \begin{enumerate}[(1)]
        \item "$\implies$": 记$f(x) = \phi(x)$, 若$\phi$是满的, 则存在$g(x) \in K[x]$使得$\phi(g(x)) = x$, 则$x = \phi(g(x)) \overset{!}= g(\phi(x)) = g(f(x))$, !处是根据环同态的定义以及$\phi(k) = k,\, \forall k \in K$得到. 考查次数$1 = \deg(g(f(x))) = \deg(g) \cdot \deg(f)$(域没有零因子). 因此$\deg(f) = \deg(g) = 1$, i.e. $\phi(x) = f(x) = ax + b,\, \exists a \neq 0, b \in K$.
        \item "$\impliedby$": 若存在$a \neq 0, b \in K$使得$\phi(x) = ax + b$, 则令$y = ax + b$得到$x = a^{-1}(y - b)$. 那么对任意的$f(x) \in K[x]$, 存在$g(x) = f(a^{-1}(x - b)) \in K[x]$使得$\phi(g(x)) = g(\phi(x)) = g(y) = f(a^{-1}(y - b)) = f(x)$.
    \end{enumerate}
\end{proof}

\begin{problem}
    证明实数的加法群$(\mathbb{R}, +)$和正实数的乘法群$(\mathbb{R}_{>0}, \cdot)$同构.
\end{problem}

\begin{proof}
    注意到$f: \mathbb{R} \to \mathbb{R}_{>0},\, x \mapsto e^x$是同构. $f^{-1}(x) = \ln x$.
\end{proof}

\begin{remark}
    事实上, 由$f(x + y) = f(x)f(y)$并利用归纳法和同态定义可以直接推出$f(x) = a^x,\, a = f(1),\, x \in \mathbb{Q}$, 若有连续性则可以延拓到$\mathbb{R}$上.
\end{remark}

\begin{problem}
    证明有理数的加法群$(\mathbb{Q}, +)$和正有理数的乘法群$(\mathbb{Q}_{>0}, \cdot)$不同构.
\end{problem}

\begin{proof}
    反证, 假设存在同构$f: \mathbb{Q} \to \mathbb{Q}_{>0}$, 则设$2 = f(a) = f(\frac{a}{2} + \frac{a}{2}) = f(\frac{a}{2}) \cdot f(\frac{a}{2}) = f(\frac{a}{2})^2$矛盾.
\end{proof}

\begin{problem}\label{ex:1.4.7}
    证明有理数域$\mathbb{Q}$和实数域$\mathbb{R}$的自同构都只有恒等映射.
\end{problem}

\begin{proof}
    不妨设$\sigma: \mathbb{Q} \to \mathbb{Q}$是同构, 根据定义, 有$\sigma(0) = 0, \sigma(1) = 1, \sigma(-a) = -\sigma(a), sigma(a^{-1}) = (\sigma(a))^{-1}$. 因此先用归纳法得到$\sigma|_{\mathbb{N}} = \mathrm{id}_{\mathbb{N}}$, 用负元延拓到$\mathbb{Z}$, 再用逆元延拓到$\mathbb{Q}$得$\sigma = \mathrm{id}_{\mathbb{Q}}$. 事实上, 这个推导对于任何特征$0$的域都是对的, 即$\mathbb{Q}$是特征$0$最小域(环的特征见教材2.1节p27定义2.1.5).

    对$\mathbb{R}$, 首先若$\phi: \mathbb{R} \to \mathbb{R}$是同构, 有上面可知$\phi|_{\mathbb{Q}} = \mathrm{id}_{\mathbb{Q}}$. 另外, 可以证明$\phi$保序结构, 即$x \geqslant 0 \implies \phi(x) \geqslant 0$. 这是因为对$x > 0$总有$\phi(x) = \phi(\sqrt{x} \cdot \sqrt{x}) = \phi(\sqrt{x})^2 > 0$. 保序则保极限, 即对单调有界有理数列$\{q_n\}_{n \in \mathbb{N}}$有$\lim_{n \to \infty} \phi(q_n) = \lim_{n \to \infty} q_n$(实际上保序就可以保持$\mathbb{R}$上的拓扑结构, $\phi$是连续的). 由于$\mathbb{Q}$在$R$中稠密, 从而$\phi = \mathrm{id}_{\mathbb{R}}$.

    一般情况下子域的自同构是不一定能延拓到扩域上, 比如考虑$\mathbb{Q}(\sqrt{2})$的共轭自同构(类似复共轭, $\sqrt{2} \mapsto -\sqrt{2}$), 它不能延拓到$\mathbb{R}$上.

    综上可得, $\mathrm{Aut}_{\mathsf{Ring}}(\mathbb{R}) = \mathrm{Aut}_{\mathbb{Q}}(\mathbb{R})$是平凡群.(由于$\mathbb{R}/\mathbb{Q}$并不是Galois扩张, 因此没有用符号$\mathrm{Gal}(\mathbb{R}/\mathbb{Q})$, $\mathsf{Ring}$表示环范畴)
\end{proof}

\begin{problem}
    证明: $\mathbb{Q}[\sqrt 2] = \{a + b\sqrt 2 \mid a, b \in \mathbb{Q}\}$, $\mathbb{Q}[\sqrt 5] = \{a + b\sqrt 5 \mid a, b \in \mathbb{Q}\}$都是$\mathbb{R}$的子域. 它们是同构的域吗?
\end{problem}

\begin{proof}
    由教材命题1.4.1的(9), 两个域若存在同态则一定是单同态, 即只有两种可能, 一个域为另一个域的扩张或两者同构. 我们断言这两个域之间不存在同态.
    
    假设存在同态$\varphi: \mathbb{Q}[\sqrt{2}] \to \mathbb{Q}[\sqrt{5}]$, 则设$\varphi(\sqrt{2}) = a + b\sqrt{5},\, a, b \in \mathbb{Q}$. 注意到由同态定义有$\varphi(2) = 2$, 立刻有
    \[
        2 = \varphi(2) = \varphi(\sqrt{2})^2 = (a + b\sqrt{5})^2 = a^2 + 5b^2 + 2ab\sqrt{5}
    \]
    这要求$a^2 + 5b^2 = 2$且$ab = 0$, 这是不可能的, 矛盾.
\end{proof}

\begin{problem}\label{ex:1.4.9}
    设$K, L$是两个域, 如果$L$是$K$的子域, 则$K$称为$L$的扩域, $K \supset L$称为域扩张, 试证明:
    \begin{enumerate}[(1)]
        \item 域的加法和乘法使得$K$是一个$L$-向量空间$([K:L] = \dim_L(K)$称为域扩张$K \supset L$ 的次数);
        \item 如果$K \supset \mathbb{R}$是一个二次扩张(即$[K:\mathbb{R}] = 2)$, 则$K$必同构于复数域$\mathbb{C}$.
    \end{enumerate}
\end{problem}

\begin{proof}
    \begin{enumerate}[(1)]
        \item $(K, +)$是一个Abel群, 这一点无需再说明. 乘法在这里可能有些歧义, 此处是要验证乘法限制在$L \times K$上, 即
        \[
            \cdot: L \times K \to K, \quad (l, k) \mapsto lk
        \]
        是数乘. 即要验证
        \[
        \begin{gathered}
            (l_1l_2)k = l_1(l_2k),\\
            (l_1 + l_2)k = l_1k + l_2k,\\
            l(k_1 + k_2) = lk_1 + lk_2,\\
            1k = k = k1.
        \end{gathered}
        \]
        这些都由域的定义得到.
    
        这也说明若同态$K_1 \to K_2$保持$L$($K_1, K_2$为$L$的两个扩域), 则一定是$L$-线性映射.
        \item 由(1), 扩域$\mathbb{C}/\mathbb{R}$的自同构一定是$\mathbb{R}$-线性的. 设同构$f: \mathbb{C} \to \mathbb{C}$, 则有$f(x + yi) = x + yf(i),\, x, y \in \mathbb{R}$, 且保持乘法, 即
        \[
        \begin{gathered}
            f\left((x_1 + iy_1) \cdot (x_2 + iy_2)\right) = f(x_1 + iy_1) \cdot f(x_1 + iy_1)\\
            = (x_1 + y_1f(i)) \cdot (x_2 + y_2f(i))\\
            \implies f\left(x_1x_2 - y_1y_2 + (x_1y_2 +x_2y_1)i\right)\\
            = x_1x_2 + y_1y_2f(i) \cdot f(i) + (x_1y_2 +x_2y_1)f(i)\\
            \implies x_1x_2 - y_1y_2 + (x_1y_2 +x_2y_1)f(i)\\
            = x_1x_2 + y_1y_2f(i) \cdot f(i) + (x_1y_2 +x_2y_1)f(i)\\
            \implies f(i) \cdot f(i) = -1.
        \end{gathered}
        \]
        因此$f(i) = \pm i$. 也就是说$\mathbb{C}/\mathbb{R}$的自同构都只有恒等映射和共轭, 即$\mathrm{Gal}(\mathbb{C}/\mathbb{R}) = \mathbb{Z}/2\mathbb{Z}$.
        
        由线性代数的结论, 可以直接得到$K$和$\mathbb{C}$是作为线性空间同构, 但这是不够的, 只有上述两个线性映射是域同构, 需要做基变换转为恒等或共轭才能保持乘法. 事实上只要存在一个基变换就能变回恒等映射, 恒等映射总是同构, 但前提是承载集合(underlying set)要一样. 比如$\mathbb{Q}(\sqrt{2})$和$\mathbb{Q}(\sqrt{3})$作为$\mathbb{Q}$-线性空间也是同构的, 但他们之间没有域同态.
        
        可取$K$的一组基为$1, \alpha$, 其中$\alpha \in \mathbb{C} \setminus \mathbb{R}$. 不可避免地要考虑$\alpha^2$的结果, 由于$1, \alpha$是基, 因此$\alpha^2$可以被线性表出, 即$\alpha^2 = x + y\alpha$. 由于$\alpha \notin \mathbb{R}$, 有$y^2 + 4x < 0$, 解二次方程得到$\alpha = \frac{y \pm i\sqrt{|y^2 + 4x|}}{2}$. 故映射
        \[
            f: K \to \mathbb{C},\, u + v\alpha \mapsto u + v\frac{y \pm i\sqrt{|y^2 + 4x|}}{2}
        \]
        是域同构.
    \end{enumerate}
\end{proof}

\begin{remark}
    事实上, 若有环同态$R \overset{\varphi}\to S$, 则$S$上自动有一个$R$-模结构
    \[
        R \times S \to S, \quad (r, s) \mapsto rs = \varphi(r)s
    \]
    $rs$是数乘, $\varphi(r)s$是$S$中的乘法. 域上的模就是线性空间.
    
    (1)对应的同态其实就是包含(inclusion)$L \overset{i}\hookrightarrow K$.
\end{remark}

\begin{problem}
    设$d$是一个非零整数, 且$\sqrt d \notin \mathbb{Q}$. 证明:
    \[
        \mathbb{Q}[\sqrt{d}] = \{a + b\sqrt{d} \mid a, b \in \mathbb{Q}\} \supset \mathbb{Q}
    \]
    是一个二次扩张($d < 0$时, $\mathbb{Q}[\sqrt{d}]$称为虚二次域, $d > 0$时称为实二次域).
\end{problem}

\begin{proof}
    只需验证$\mathbb{Q}[\sqrt{d}] = \{a + b\sqrt{d} \mid a, b \in \mathbb{Q}\}$确实是一个域. 这样它自动就是一个$2$维的$\mathbb{Q}$-线性空间.
    
    加法:
    \[
        (a_1 + b_1\sqrt{d}) + (a_2 + b_2\sqrt{d}) = (a_1 + a_2) + (b_1 + b_2)\sqrt{d}
    \]
    容易验证结合律, $0 = 0 + 0\sqrt{d},\, -(a + b\sqrt{d}) = (-a) + (-b)\sqrt{d}$.
    
    乘法:
    \[
        (a_1 + b_1\sqrt{d}) \cdot (a_2 + b_2\sqrt{d}) = a_1a_2 + b_1b_2d + (a_1b_2 + a_2b_1)\sqrt{d}
    \]
    其中$1 = 1 + 0\sqrt{d}$, 逆元做一次分母有理化
    \[
        (a + b\sqrt{d})^{-1} = \frac{1}{a + b\sqrt{d}} = \frac{a - b\sqrt{d}}{a^2 + b^2d} = \frac{a}{a^2 + b^2d} + \frac{- b}{a^2 + b^2d}\sqrt{d}
    \],
    
    结合律是容易验证的(计算出的结果是轮换对称的, 参考\ref{ex:1.1.4}和\ref{ex:1.2.3}).
\end{proof}

\begin{problem}
    设$L \supset K$是一个域扩张, 证明: 下述集合
    \[
        \mathrm{Gal}(L/K) = \left\{L \xrightarrow{\sigma} L \mid \sigma\text{ 是域同构},\text{ 且 } \sigma(a) = a \text{ 对任意 } a \in K \text{ 成立}\right\}
    \]
    关于映射的合成是一个群(称为域扩张$L\supset K$的伽罗瓦群).
\end{problem}

\begin{proof}
    $\mathrm{Gal}(L/K) \subseteq \mathrm{Aut}(L)$, 只需说明$\mathrm{Gal}(L/K)$是子群.

    $\forall \varphi, \psi \in \mathrm{Gal}(L/K)$, 由于$\psi|_K = \mathrm{id}_K$, 因此$\psi^{-1}|_K = \mathrm{id}_K$, 故$(\varphi \circ \psi^{-1})|_K = \mathrm{id}_K$, 即$\varphi \circ \psi^{-1} \in \mathrm{Gal}(L/K)$.
\end{proof}

\begin{problem}
    求$\mathrm{Gal}\left(\mathbb{Q}[\sqrt{d}]/\mathbb{Q}\right)$, 此处$d \in \mathbb{Z}, \sqrt{d} \notin \mathbb{Q}$.
\end{problem}

\begin{proof}
    同\ref{ex:1.4.9}, 若$\sigma \in \mathrm{Gal}\left(\mathbb{Q}[\sqrt{d}]/\mathbb{Q}\right)$, 则$\sigma \in \mathrm{Aut}_{\mathbb{Q}-\mathsf{Vect}}\left(\mathbb{Q}[\sqrt{d}]\right)$, 因此$\sigma(a + b\sqrt{d}) = a + b\sigma(\sqrt{d})$. 然后由于保持乘法得到$\sigma(\sqrt{d}) \cdot \sigma(\sqrt{d}) = d$, 得到$\sigma(\sqrt{d}) = \pm\sqrt{d}$.

    因此$\mathrm{Gal}\left(\mathbb{Q}[\sqrt{d}]/\mathbb{Q}\right) = \mathbb{Z}/2\mathbb{Z}$, 两个同构分别为$\mathrm{id}$和共轭.
\end{proof}

\begin{problem}\label{ex:1.4.13}
    设$V = (V, +)$ 是一个加法群, $\mathrm{Hom}(V)$表示它的自同态环. 对任意域$K$, 如果存在一个数乘运算$K \times V \to V,\, (\lambda,v) \mapsto \lambda \cdot v$, 使得加法群$V = (V, +)$成为一个$K$-线性空间, 则称该数乘运算是加法群$V = (V, +)$上的一个$K$-线性空间结构. 试证明:
    \begin{enumerate}[(1)]
        \item 如果存在一个环同态$\varphi:K \to \mathrm{Hom}(V)$,则数乘运算
        \[
            K \times V \to V,\quad (\lambda, v) \mapsto \lambda \cdot v \defeq \varphi(\lambda)(v)
        \]
        是$V$上的一个$K$-线性空间结构;
        \item 如果在$V$上存在$K$-线性空间结构$\phi:K \times V \to V$,则映射
        \[
            \varphi:K \to \mathrm{Hom}(V),\quad \lambda \mapsto \phi(\lambda, \cdot)
        \]
        是一个环同态, 其中$\phi(\lambda, \cdot):V \to V$定义为$v \mapsto \phi(\lambda, v) \defeq \lambda \cdot v$;
        \item 对任意域$K$, 整数加法群$\mathbb{Z} = (\mathbb{Z}, +)$上不存在$K$-线性空间结构.
    \end{enumerate}
\end{problem}

\begin{proof}
    \begin{enumerate}[(1)]
        \item 验证数乘的四条:
        \begin{enumerate}[(i)]
            \item 由于$\varphi$是环同态, 因此$\varphi(1) = 1_{\mathrm{Hom(V)}} = \mathrm{id}_V$. 故$\forall v \in V$有$1v = \varphi(1)(v) = \mathrm{id}_V(v) = v$.
            \item 
            \(
                \forall a, b \in K,\, v \in V\, (a + b)v = (\varphi(a + b))(v) = (\varphi(a) + \varphi(b))(v) = \varphi(a)(v) + \varphi(b)(v) = av + bv.
            \)
            \item 
            \(
                \forall a \in K,\, v, w \in V\, a(v + w) = \varphi(a)(v + w) = \varphi(a)(v) + \varphi(a)(w) = av + aw.
            \)
            \item 
            \(
                \forall a, b \in K,\, v \in V\, (ab)v = \varphi(ab)(v) = (\varphi(a) \circ \varphi(b))(v) = \varphi(a)(\varphi(b)(v)) = a(bv).
            \)
        \end{enumerate}
        \item (1)的反向.
        \begin{enumerate}[(i)]
            \item 
            \(
                \forall a \in K,\, v, w \in V\, \varphi(a)(v + w) = \phi(a, v + w) = a(v + w) = av + aw = \phi(a, v) + \phi(a, w) = \varphi(a)(v) + \varphi(a)(w)
            \)
            这说明$\varphi(a)$保持加法.
            \item 
            \(
                \forall a \in K,\, v \in V\, \varphi(a)(kv) = \phi(a, kv) = a(kv) = (ka)v = \phi(ka, a) = \varphi(ka)(v)
            \)
            这说明$\varphi(a)$保持数乘.
            
            由(i)(ii)知$\varphi$是良定义的(well-defined). 同时(ii)也说明$\varphi(ab) = \varphi(a) \circ \varphi(b)$.
            \item 由$\phi$是数乘, 即$1v = v,\, \forall v \in V$. 也就是说$\phi(1, \cdot) = \mathrm{id}_V$.
            \item 
            \(
                \forall a, b \in K,\, v \in V,\, \varphi(a + b)(v) = \phi(a + b, v) = (a + b)v = av + bv = \phi(a, v) + \phi(b, v) = \varphi(a) + \varphi(b)
            \)
        \end{enumerate}
        \item 用反证法, 假设$(\mathbb{Z}, +)$上存在一个$K$-线性空间结构, 即存在一个环同态$\varphi: K \to \mathrm{Hom}(\mathbb{Z})$.
        
        但是$\mathrm{Hom}(\mathbb{Z})$和整数环$\mathbb{Z}$是同构的(教材例1.4.4). 我们又知道域出发的环同态一定是单的(教材命题1.4.1的(9)), 也就是说存在一个域$K$到$\mathbb{Z}$的单同态, 这是不可能的. 由\ref{ex:1.4.7}, 一定有$n \mapsto n,\, \forall n \in \mathbb{Z}$, 而同态一定会把单位映到单位, 但$\mathbb{Z}$中只有$\pm1$是单位.
        
        \ref{ex:1.4.7}也说明了环同态是保特征的, 因此$\mathrm{Char}(K) = \mathrm{Char}(\mathbb{Z}) = 0$, 从而$\mathbb{Q} \subseteq K$. 这样也可以看出矛盾.
    \end{enumerate}
\end{proof}

\begin{remark}
    (1)(2)即一个模结构的两种等价表述, 在群作用(教材4.5节)也会看到类似的定义.
\end{remark}

\begin{problem}
    证明: 在整数集合$\mathbb{Z}$上存在运算$\mathbb{Z} \times \mathbb{Z} \to \mathbb{Z}, (a,b) \mapsto a \oplus b$, 使得$(\mathbb{Z}, \oplus)$是一个交换群, 但它与整数加法群$(\mathbb{Z}, +)$不同构. 提示: 利用$\mathbb{Q}$是可数集和上题中的问题(3).
\end{problem}

\begin{proof}
    类似\ref{ex:1.1.4}, 存在一个可数集之间的双射$f: \mathbb{Z} \to \mathbb{Q}$, 由$\mathbb{Q}$的环结构导出$(\mathbb{Z}, \oplus, \star)$.
    
    则同态
    \[
        f^{-1}: \mathbb{Q} \to (\mathbb{Z}, \oplus, \star)
    \]
    会自然诱导出一个$\mathbb{Q}$-线性空间结构(\ref{ex:1.4.9}的(1)). 由\ref{ex:1.4.13}的(3), $(\mathbb{Z}, \oplus)$和$(\mathbb{Z}, +)$不同构.
\end{proof}

\begin{remark}
    对于$\mathbb{Z}$还有一个重要的结论, $\mathbb{Z}$的(含幺)环结构是唯一的. 更严格来说, 在$(\mathbb{Z}, +)$上添加乘法, 那么只能得到唯一的环结构.(可参考\cite{aluffi2009algebra}III.2.15, 2.16)
\end{remark}