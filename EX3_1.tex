\subsection{教材p52-54}

\begin{problem}
    设$K$是特征零的域, $f(x) \in K[x]$是次数大于零的首项系数为$1$的多
项式, $d(x) = (f(x), f'(x))$是$f(x)$与$f'(x)$的最大公因子. 令
\[
    f(x) = d(x) \cdot g(x).
\]
证明: $g(x)$与$f(x)$有相同的根且$g(x)$没有重根.
\end{problem}

\begin{proof}
    
\end{proof}

\begin{problem}
    设$K \subset L$是域扩张, $\alpha \in L$是域$K$上的代数元.
令$K[x] \xrightarrow{\psi_{\alpha}}L$,
$f(x) \mapsto f(\alpha)$, 表示多项式在$x = \alpha$的取值映射.
试证明: 
\begin{enumerate}[(1)]
    \item $\ker(\psi_\alpha)$由极小多项式$\mu_\alpha(x)$生成;
    \item $\psi_\alpha$诱导了域同构$\mathbb{K}[x]/(\mu_\alpha(x)) \cong K[\alpha]$.
\end{enumerate}
\end{problem}

\begin{proof}
    
\end{proof}

\begin{problem}
    设$E = \mathbb{Q}[u], u^3 - u^2 + u + 2 = 0$.
试将$(u^2 + u + 1)(u^2 - u)$和$(u - 1)^{-1}$表示成
$au^2 + bu + c (a, b, c \in \mathbb{Q})$的形式.
\end{problem}

\begin{proof}
    
\end{proof}

\begin{problem}
    求$[\mathbb{Q}[\sqrt2, \sqrt3]:\mathbb{Q}]$
(提示: 证明$\left[\mathbb{Q}[\sqrt2, \sqrt3]:\mathbb{Q}[\sqrt3]\right] = 2)$.
\end{problem}

\begin{proof}
    
\end{proof}

\begin{problem}
    设$p$是一个素数, $z \in \mathbb{C}$满足$z^p = 1$
且$z \neq 1$, 试证明$[\mathbb{Q}[z]:\mathbb{Q}] = p - 1$.
\end{problem}

\begin{proof}
    
\end{proof}

\begin{problem}
    证明: 
\begin{enumerate}[(1)]
    \item $U_n = \{z \in \mathbb{C} \mid z^n = 1\}$是一个循环群;
    \item $z = \cos \frac\pi6 + i\sin \frac\pi6$是$U_{12}$的一个生成元,
但$[\mathbb{Q}[z]:\mathbb{Q}] = 4$;
    \item 求$z = \cos \frac\pi6 + i\sin \frac\pi6$在$\mathbb{Q}$上的极小多项式.
\end{enumerate}
\end{problem}

\begin{proof}
    
\end{proof}

\begin{problem}
    设$E = K[u]$是一个代数扩张, 且$u$的极小多项式的次数是奇数. 证
明: $E = K[u^2]$.
\end{problem}

\begin{proof}
    
\end{proof}

\begin{problem}
    设$E_1, E_2$是域扩张$K \subset L$的中间域
(即: $K \subset E_i \subset L)$, 且$[E_i:K] < +\infty$.
令$E = K[E_1,E_2] \subset L$是由$E_1, E_2$生成的子域.
证明: 
\[
    [E:K] \leqslant [E_1:K] \cdot [E_2:K].
\]
\end{problem}

\begin{proof}
    
\end{proof}

\begin{problem}
    设$K \subset L$是代数扩张, $E \subset L$是中间子环
(即: $K \subset E \subset L)$. 证明: $E\subset L$必为子域
(所以任何有限扩张$K \subset L$的中间子环必为域).
\end{problem}

\begin{proof}
    
\end{proof}

\begin{problem}
    设$L = K(u)$, $u$是$K$上的超越元, $E \neq K$
是$K\subset L$的中间域. 证明: $u$是$E$上的代数元.
\end{problem}

\begin{proof}
    
\end{proof}

\begin{problem}
    设$p$是素数, $K \subset L$是$p$次扩张. 证明: 
$K \subset L$必为单纯扩张(即: 存在$u \in L$, 使$L = K[u]$).
\end{problem}

\begin{proof}
    
\end{proof}

\begin{problem}
    设域扩张 $K \subset L$满足条件: 
\begin{enumerate}[(1)]
    \item $[L:K] < +\infty$;
    \item 对任意两个中间域$K \subset E_1 \subset L,\, K \subset E_2 \subset L$,
必有$E_1 \subset E_2$或者$E_2 \subset E_1$.
\end{enumerate}
证明: $K \subset L$必为单纯扩张(即: 存在$u \in L$,使$L = K[u])$.
\end{problem}

\begin{proof}
    
\end{proof}

\begin{problem}
    设$\alpha = 2 + \sqrt[3]{2} + \sqrt[3]{4}$,
给出一个首项系数为$1$的最低次数的多项式
$f(x) \in \mathbb{Q}[x]$使$f(\alpha) = 0$.
\end{problem}

\begin{proof}
    
\end{proof}

\begin{problem}
    设$K = \mathbb{Q}[\sqrt[3]{3}]$,证明: $x^5 - 5$在$K[x]$中不可约.
\end{problem}

\begin{proof}
    
\end{proof}

\begin{problem}
    设$k$是特征$p > 0$的域, $x, y$是$k$上的代数无关元.
令$K = k(x^{p}, y^{p})$, $L = k(x, y)$. 
试证明$[L:K] = p^{2}$.
\end{problem}

\begin{proof}
    
\end{proof}