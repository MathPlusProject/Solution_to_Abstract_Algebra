\subsection{教材p35-p36}

\begin{problem}
    设$m, n$是两个正整数, 证明它们在$\mathbb{Z}$中的最大公因数和它们在$\mathbb{Z}[i]$中的最大公因数相同.
\end{problem}

注意这里的相同指的在相伴的意义下相同.

\begin{proof}
    由于$U(\mathbb{Z}[i]) = \{\pm 1, \pm i\}$, 在相伴的意义下, 可以假设$(m, n)$在$\mathbb{Z}$和$\mathbb{Z}[i]$中都是正整数, 分别记为$d$和$d'$.

    那么PID上Bézout's Identity成立, 有
    \[
        d = mu + nv,\quad d' = m\alpha + n\beta.
    \]
    其中$u, v \in \mathbb{Z},\, \alpha, \beta \in \mathbb{Z}[i]$. 设$\alpha = a_1 + ia_2,\, \beta = b_1 + ib_2$, 由于我们假设的是$d' \in \mathbb{Z}_{>0}$, 故$d' = ma_1 + nb_1$, 从而$d \mid m, d \mid n \implies d \mid d'$. 反过来也有$d' \mid d$, 所以$d = d'$.
\end{proof}

\begin{problem}\label{ex:2.2.2}
    设$R$是整环, $p \in R$称为一个素元如果它生成的理想$P = (p)R$是素理想. 证明: $R$中素元必为不可约元.
\end{problem}

\begin{proof}
    由定义$(p) \neq (1)$, 因此$p$不可逆. 设$p = ab$, 则$ab \in (p)$, 由素理想知$a \in (p)$或$b \in (p)$, 不妨设$a \in (p)$, 则$(a) \subseteq (p)$. 另一方面$(p) \subseteq (a)$, 因此$(p) = (a)$, 从而$b$是单位.
\end{proof}

\begin{remark}
    $x \sim y \defequiv \exists u \in U(R),\, x = uy \iff (x) = (y) \iff x \mid y \text{ 且 } y \mid x$.
\end{remark}

\begin{problem}
    设$R$是一个主理想整环(PID), $0 \neq r \in R$. 证明: 在$R$中仅有有限个理想包含$r$.
\end{problem}

\begin{proof}
    $R$是PID, 即对任意理想$I$, 存在$a \in R$, 理想$I = (a)$. 理想$I$包含$r$指$r \in I$, 它等价于$(r) \subseteq I = (a) \iff a \mid r$. 又因为PID是UFD, 因此又唯一分解$r = p_1p_2\cdots p_n$, 从而$r$因子个数在相伴的意义下(\ref{ex:2.2.2}的注记)有限($\leqslant2^n$), 即包含$r$的理想有限.
\end{proof}

\begin{problem}[辗转相除法]
    设$R$是欧氏环, $a, b \in R$非零. 由带余除法得
    \[
        a = q_{1}b + r_{1},\, b = q_{2}r_{1} + r_{2},\, r_{1} = q_{3}r_{2} + r_{3},\, \cdots,\, r_{k- 2}= q_{k}r_{k- 1} + r_{k}
    \]
    满足$\delta(r_k) < \delta(r_{k - 1}) < \cdots < \delta(r_2) < \delta(r_1) < \delta(b)$. 试证明:
    \begin{enumerate}[(1)]
        \item 存在$k$使得$r_{k + 1} = 0$;
        \item $r_k$是$a$, $b$的一个最大公因子;
        \item 求$u$, $v \in R$使得$r_k = ua + vb$.
    \end{enumerate}
\end{problem}

\begin{proof}
    \begin{enumerate}[(1)]
        \item 由于$\delta(b) < \infty$, 且$\delta(r_k)$是严格递减的自然数序列, 因此$\delta(k) \leqslant \delta(b) - k$, 取$k > \delta(b)$即可.
        \item 由(1)知最后一个等式为$r_{k - 1} = q_{k + 1}r_{k}$. 且
        \[
            (a, b) = (bq_1 + r_1, b) = (b, r_1) = (r_1, r_2) = \cdots = (r_{k - 1}, r_k) = r_k.
        \]
        \item 根据辗转相除法的算式反过来表示$r_k$.
        \[
        \begin{aligned}
            r_k &= r_{k - 2} - q_kr_{k - 1} = u_1r_{k - 2} + v_1r_{k - 1},\quad u_1 = 1, v_1 = -q_k\\ 
            &= u_1r_{k - 2} + v_1(r_{k - 3} - q_{k - 1}r_{k - 2}),\quad (r_{k - 3} = q_{k - 1}r_{k - 2} + r_{k - 1})\\
            &= u_2r_{k - 3} + v_2r_{k - 2}, \quad u_2 = -q_k, v_2 = 1 + q_kq_{k - 1}\\
            &= \cdots \\
            &= u_ka + v_kb 
        \end{aligned}
        \]
        递归关系是$u_i = v_{i - 1}, v_i = u_{i - 1} - v_{i - 1}q_{k - i + 1}$.
    \end{enumerate}
\end{proof}

\begin{remark}
    (1)是著名的无穷递降的思路, 即递归的得到一列对象且对应着一个严格递减的自然数序列, 根据自然数有下界$0$来得到矛盾或得出某个结论.

    另外(3)的题干表述可能有些问题, 这里并不需要把$u, v$具体表达出来.
\end{remark}

\begin{problem}
    设$R = \mathbb{Z}[\sqrt{-5}] = \{a + b\sqrt{-5} \mid \forall a, b \in\mathbb{Z}\} \subset \mathbb{C}$, 定义: $N(a + b\sqrt{-5}) = a^2 + 5b^2$. 试证明:
    \begin{enumerate}[(1)]
        \item $U(R) = \{1, -1\}$;
        \item $R$中任意元素都有不可约分解;
        \item $3$, $2 + \sqrt{-5}$, $2 - \sqrt{-5} \in R$是不可约元;
        \item $9 = 3 \cdot 3= (2 + \sqrt{-5}) \cdot (2 - \sqrt{-5})$是$9$的两个不相同的不可约分解.
    \end{enumerate}
\end{problem}

\begin{proof}
    \begin{enumerate}[(1)]
        \item 验证$N$满足$N(\alpha\beta) = N(\alpha)N(\beta)$, 这和复数中$|z_1z_2| = |z_1||z_2|$是类似的, 且$N(\alpha) \in \mathbb{N}$. 那么若$\alpha$是单位, 则存在$\beta$使得$\alpha\beta = 1$, 故$N(\alpha\beta) = N(\alpha)N(\beta) = N(1) = 1$, 故只能有$N(\alpha) = N(\beta) = 1$, 解得$\alpha = \pm 1$.
        \item 因为$R$是Noether环.(由Hilbert's Basis Theorem)
        
        或者可以用$N(\alpha)$保持乘法的特性. 对任意$\alpha \in R$, 若它不可约, 则已经是一个分解了; 否则$\alpha = \beta\gamma$, 其中$\beta, \gamma$不是单位, 且有$N(\alpha) = N(\beta)N(\gamma)$. 因此$N(\beta), N(\gamma) < N(\alpha)$, 由于$N(\alpha) < \infty$, 因此这样分解是有限的, 这和Noether环$\implies$存在分解的过程是类似的.
        \item 由于$N(3) = N(2 + \sqrt{-5}) = N(2 - \sqrt{-5}) = 9 = 3^2$, 若它们可约, 则存在$\alpha$使得$N(\alpha) = 3$, 这是不可能的.
    
        另外, 若$N(\alpha)$是素数, 则一定不可约, 但是反过来不对, 比如这里$9$并不是素数.
        \item 由(3).
    \end{enumerate}
\end{proof}

\begin{remark}
    \begin{enumerate}[1.]
        \item 这个$N$是范数(norm). 它其实是$\mathbb{Q}$-线性映射$\beta \mapsto \alpha\beta$所对应矩阵的行列式. 这个概念在模论和代数数论都有提及.
        \item (1)和(2)的结论是可以推广的, 对于一个代数数域$K/\mathbb{Q}$(即$\mathbb{Q}$的有限扩张), $\alpha \in \mathcal{O}_K$是单位当且仅当$N_{K/\mathbb{Q}}(\alpha) = \pm1$. 其中$\mathcal{O}_K$是对应的代数整数环. $\mathcal{O}_K$是存在不可约分解的环. 其证明方法和(2)几乎一模一样. 具体细节参考代数数论的教材.
    \end{enumerate}
\end{remark}

\begin{problem}\label{ex:2.2.6}
    令$\mathbb{R}, \mathbb{C}$分别表示实数域和复数域, 试证明:
    \begin{enumerate}[(1)]
        \item 若$R$是由关于$\cos t$和$\sin t$的实系数多项式组成的函数环, 则$R \cong \mathbb{R}[x, y]/(x^2 + y^2 - 1)$;
        \item $\mathbb{C}[x, y]/(x^2 + y^2 - 1)$是唯一分解整环(提示: 证明其为ED);
        \item $\mathbb{R}[x, y]/(x^2 + y^2 - 1)$不是唯一分解整环.
    \end{enumerate}
\end{problem}

\begin{proof}
    \begin{enumerate}[(1)]
        \item 考虑同态
        \[
            \varphi: \mathbb{R}[x, y] \to R = \mathbb{R}[\cos t, \sin t],\, x \mapsto \cos t, y \mapsto \sin t,
        \]
        这自然是一个满同态, 由同态基本定理, 关键在于证明
        \[
            \ker(\varphi) = (x^2 + y^2 - 1)
        \]
        若多项式$f(x, y)$满足$\varphi(f) = f(\cos t, \sin t) = 0$, 将$f$看成是关于$y$的多项式
        \[
            f(x, y) = a_0(x) + a_1(x)y + \cdots + a_n(x)y^n,\, a_i(x) \in \mathbb{R}[x],\, 0 \leqslant i \leqslant n
        \]
        由于$x^2 + y^2 - 1$关于$y$是首一的, 因此可以做带余除法, 得$f = gq + r$, 其中$r(x, y) = r_0(x) + r_1(x)y$. 带入$x = \cos t, y = \sin t$得$r(\cos t, \sin t) = 0$, 即
        \[
            r_0(\cos t) + r_1(\cos t)\sin t = 0
        \]
        做代换$t \mapsto -t$, 得
        \[
            r_0(\cos t) - r_1(\cos t)\sin t = 0
        \]
        两式相加得$r_0 = 0$, 相减得$r_1 = 0$, 从而$r = 0$. 因此$f \in (x^2 + y^2 - 1)$, 即$\ker(\varphi) \subseteq (x^2 + y^2 - 1)$. 另一方面$x^2 + y^2 - 1 \in \ker(\varphi)$, 故$\ker(\varphi) = (x^2 + y^2 - 1)$.
        \item 做基变换$u = x + iy, v = x - iy$, 他有逆变换$x = \frac{u + v}{2}, y = \frac{u - v}{2i}$. 因此有同构$\mathbb{C}[u, v] \cong \mathbb{C}[x, y]$. 从而
        \[
            \mathbb{C}[x, y]/(x^2 + y^2 - 1) \cong \mathbb{C}[u, v]/(uv - 1)
        \]
        而同态
        \[
            \mathbb{C}[u, v] \to \mathbb{C}[u, u^{-1}],\, u \mapsto u, v \mapsto u^{-1}
        \]
        是满的, 且kernel是$(uv - 1)$, 证明类似于(1). 因此
        \[
            \mathbb{C}[u, v]/(uv - 1) \cong \mathbb{C}[u, u^{-1}]
        \]
        这个环称为Laurent多项式环, 这个环上可以做带余除法, 非零多项式的次数定义为最高次数$-$最低次数. 即$f = a_nu^n + a_{n + 1}u^{n + 1} + \cdots + a_mu^m, n, m \in \mathbb{Z}, n < m$的次数为$\deg(f) = m - n$. 因此这是一个ED, 从而是UFD.
        \item 由(2), $\mathbb{C}[\cos t, \sin t]$是UFD, 用待定系数, 假设
        \[
            \cos t = (a_1\cos t + a_2\sin t + a_3)(b_1\cos t + b_2\sin t + b_3)
        \]
        其中$a_i, b_i \in \mathbb{C}, i = 1, 2, 3$. 我们要忽略掉$a_1 = b_3 = 1$其余都是$0$这种平凡的情况, 左右展开得到
        \[
        \begin{aligned}
            a_1b_1 - a_2b_2 &= 0,\\
            a_1b_2 + a_2b_1 &= 0,\\
            a_1b_1 + a_3b_3 &= 0,\\
            a_1b_3 + a_3b_1 &= 1,\\
            a_2b_3 + a_3b_2 &= 0.
        \end{aligned}
        \]
        由第一个式子得$b_1 = \frac{a_2}{a_1}b_2$, 带入第二个式子得$a_2 = \pm ia_1$, 从而$b_1 = \pm ib_2$.
        
        由一, 三又能得到$a_2b_2 = -a_3b_3$, 类似地, 带入第五个式子, 有$a_3 = \pm a_2$, $b_2 = \pm b_3$.
        
        再用四, 五得$a_1b_3 = a_3b_1 = \frac{1}{2}$.
        
        把上述关系带入
        \[
        \begin{aligned}
            \cos t &= a_1b_3(\cos t \pm i\sin t \pm i)(\pm i\cos t \pm \sin t + 1)\\
            &= \frac{1}{2}(\cos t \pm i\sin t \pm i)(\pm i\cos t \pm \sin t + 1)
        \end{aligned}
        \]
        检查正负号, 得到结果
        \[
            \cos t = \frac{1}{2}(\cos t + i\sin t - i)(i\cos t + \sin t + 1)\, 
        \]
        类似有
        \[
            1 - \sin t = \frac{1}{2}(\cos t + i\sin t - i)(\cos t - i\sin t + i).
        \]
        带入$-t$就是$1 + \sin t$的分解.
        
        但这种方法比较难检查等式右边的因式确实为不可约元, 我们可以利用同构$\mathbb{C}[x, y]/(x^2 + y^2 - 1) \cong \mathbb{C}[u, u^{-1}]$, 那么等式变为
        \[
            x = \frac{1}{2}(u + u^{-1}) = \frac{u^{-1}}{2}(u - i)(u + i)
        \]
        注意到$U(\mathbb{C}[u, u^{-1}]) = \mathbb{C} \cup \{u^n \mid n \in \mathbb{Z}\}$. 右边为两个都是一次的且常数项不为$0$, 容易验证不可逆(注意这里$x = \frac{1}{2}(u + u^{-1})$次数为$2$). 对$1 - \sin t$同理.
        
        因此$\cos t$和$1 \pm \sin t$无法在$\mathbb{R}[\cos t, \sin t]$中分解(分解出的系数中一定带$i$). 这样就有$\cos^2 t = \cos t\cos t = (1 - \sin t)(1 + \sin t)$. 因此不是UFD.
    \end{enumerate}
\end{proof}

\begin{remark}
    (2)中若允许正次数到无穷的话, 则该环称为Laurent形式级数域(可以验证确实是一个域).

    另外, 可以说$x^2 + y^2 - 1$是单位圆的"极小多项式". 但这种说法是有些不合理的, 因为这样$a_{ij}x^iy^j$次数将定义成$i + j$, $f(x, y)$的次数定义成单项次数的最大值, 一旦这么定义就无法做带余除法, 就无法得到满足某个点集(一般是代数集, 即某些多项式的共同零点)的多项式是其极小多项式的倍数.
    
    一般地设$k$是一个域, $S \subseteq k[x_1, x_2, \cdots, x_n]$, 那么可以定义$S$中所有多项式的公共零点集
    \[
        Z(S) = \{(a_1, a_2, \cdots, a_n) \in k^n \mid \forall f \in S,\, f(a_1, a_2, \cdots a_n) = 0\}
    \]
    按定义有$S \subseteq S' \implies Z(S') \subseteq Z(S)$. 考虑$S$生成的理想$I = (S)$(见\ref{ex:2.1.6}的注记), 则有$Z(I) \subseteq Z(S)$. 另一方面, 根据$I = \left\{\sum f_ig_i \Big| f_i \in k[x_1, x_2, \cdots, x_n], g_i \in S\right\}$, 立刻得到$Z(S) \subseteq Z(I)$. 从而$Z(S) = Z(I)$. 我们称$Z(S)$这种点集为代数集(algebraic set), 用$\mathbb{A}_k^n$代替的$k^n$表示将它看作一个代数集(因为按定义$Z(\varnothing) = k^n$), 而Hilbert's Basis Theorem告诉我们$k[x_1, x_2, \cdots, x_n]$是Noether环, 所以理想都是有限生成的, 那么总有$Z(I) = Z(f_1, f_2, \cdots, f_r)$.
    
    反过来, 对$X \subseteq \mathbb{A}_k^n$, 定义
    \[
        \mathscr{I}(X) = \{f \in k[x_1, x_2, \cdots x_n] \mid \forall (a_1, a_2, \cdots a_n) \in X, f(a_1, a_2, \cdots a_n) = 0\}
    \]
    可以验证$\mathscr{I}(X)$是根理想(\ref{ex:2.1.1}的注记). 当$k$是代数闭域(algebraic closed field)时, 有一一对应
    \[
        \begin{tikzcd}
            \{\mathbb{A}_k^n \text{ 的代数集}\}  \arrow[r, "\mathscr{I}", shift left] & {\{k[x_1, x_2, \cdots x_n] \text{ 的根理想}\}} \arrow[l, "Z", shift left]
        \end{tikzcd}
    \]
    这就是Strong Nullstellensatz.

    那么(1)中$I = (x^2 + y^2 - 1)$, $Z(I) = \{(x, y) \in \mathbb{R}^2 \mid \forall f \in I,\, f(x, y) = 0\} = \{(x, y) \in \mathbb{R}^2 \mid x^2 + y^2 = 1\}$, 恰好是单位圆. 那么(1)的关键在于说明$\mathscr{I}(Z(I)) = I$. 可惜的是一般情况下这个并不成立, 比如还是在$\mathbb{R}[x, y]$上考虑, 记$J = (x^2 + y^2)$, 那么$Z(J) = {(0,0)}$, $\mathscr{I}(Z(J)) = (x, y) \neq J$. 这里$x^2 + y^2$是不可约的, 所以即使是单独一个不可约多项式也不一定可以有这个等式, $x^2 + y^2 - 1$这个不可约多项式还是比较特殊的.

    域扩张中的极小多项式和不可约是一样的, 这是由于$K[x]$是一个PID, 不可约元对应极大理想, 从而对应极小多项式.
\end{remark}