\subsection{教材p35-p36}

\begin{problem}
    设$m, n$是两个正整数, 证明它们在$\mathbb{Z}$中的最大公因数
和它们在$\mathbb{Z}[i]$中的最大公因数相同.
\end{problem}

\begin{solution}
    
\end{solution}

\begin{problem}
    设$R$是整环, $p \in R$称为一个素元如果它生成的理想
$P=(p)R$是素理想. 证明:$R$中素元必为不可约元.
\end{problem}

\begin{solution}
    
\end{solution}

\begin{problem}
    设$R$是一个主理想整环(PID), $0 \neq r \in R$.
证明:在$R$中仅有有限个理想包含$r$.
\end{problem}

\begin{solution}
    
\end{solution}

\begin{problem}[辗转相除法]
    设$R$是欧氏环, $a, b \in R$非零. 由带余除法得
\[
a = q_{1}b + r_{1},\,
b = q_{2}r_{1}+ r_{2},\,
r_{1} = q_{3}r_{2} + r_{3},\, \cdots,\,
r_{k- 2}= q_{k}r_{k- 1}+ r_{k}
\]
满足$\delta(r_k) < \delta(r_{k - 1}) < \cdots < \delta(r_2) < \delta(r_1) < \delta(b)$.
试证明:
\begin{enumerate}[(1)]
    \item 存在$k$使得$r_k + 1 = 0$;
    \item $r_k$是$a$, $b$的一个最大公因子;
    \item 求$u$, $v \in R$使得$r_k = ua + vb$.
\end{enumerate}
\end{problem}

\begin{solution}
    
\end{solution}

\begin{problem}
    设$R = \mathbb{Z}[\sqrt{-5}] = \{a + b\sqrt{-5} \mid \forall a, b \in\mathbb{Z}\} \subset \mathbb{C}$,
定义:$N(a + b\sqrt{-5}) = a^2 + 5b^2$. 试证明:
\begin{enumerate}[(1)]
    \item $U(R) = \{1, -1\}$;
    \item $R$中任意元素都有不可约分解;
    \item $3$, $2 + \sqrt{-5}$, $2 - \sqrt{-5} \in R$是不可约元;
    \item $9 = 3 \cdot 3= (2 + \sqrt{-5}) \cdot (2 - \sqrt{-5})$是$9$的两个不相同的不可约分解.
\end{enumerate}
\end{problem}

\begin{solution}
    
\end{solution}

\begin{problem}
    令$\mathbb{R}, \mathbb{C}$分别表示实数域和复数域, 试证明:
\begin{enumerate}[(1)]
    \item 若$R$是由关于$\cos t$和$\sin t$的实系数多项式组成的函数环, 
则$R \cong \mathbb{R}[x, y]/(x^2 + y^2 - 1)$;
    \item $\mathbb{C}[x, y]/(x^2 + y^2 - 1)$是唯一分解整环(提示:证明其为ED);
    \item $\mathbb{R}[x, y]/(x^2 + y^2 - 1)$不是唯一分解整环.
\end{enumerate}
\end{problem}

\begin{solution}
    
\end{solution}