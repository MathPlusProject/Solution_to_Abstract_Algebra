\subsection{教材p13-p14}

\begin{problem}\label{ex:1.2.1}
    设$R$是一个环, 试证明下述结论:
\begin{enumerate}[(1)]
    \item (加法消去律)\quad 如果$a + c = b + c$, 则$a = b$;
    \item $\forall a\in R$, 有$a \cdot 0_R = 0_R$;
    \item $-(-a) = a,\quad a(b - c) = ab - ac \quad (\forall a, b, c \in R)$;
    \item $-(a + b) = (-a) + (-b) \quad (\forall a, b \in R)$;
    \item $a(-b) = (-a)b = -(ab) \quad (\forall a, b \in R)$;
    \item $(-a)(-b) = ab \quad (\forall a, b \in R)$;
    \item $\forall a \in R, m, n \in \mathbb{Z}$, 有$(m + n)a = ma + na, (mn)a = m(na)$;
    \item $\forall a, b \in R, n \in \mathbb{Z}$, 有$n(a + b) = na + nb, n(ab) = a(nb)$;
    \item $\forall a, b \in R, m, n \in \mathbb{Z}$, 有$(ma) \cdot (nb) = mn(a \cdot b) = (mna) \cdot b$;
    \item (二项式定理)\quad $\forall a, b \in R$,设$ab = ba$, $n$是正整数, 则
    \[
        (a + b)^n = \sum_{i = 0}^n \binom{n}{i} a^{n - i}b^i.
    \]
\end{enumerate}
\end{problem}

\begin{solution}
    \begin{enumerate}[(1)]
        \item 两边同加$-c$.
        \item 由于
        \[
            a \cdot 0_R = a \cdot (0_R + 0_R) = a \cdot 0_R + a \cdot 0_R.
        \]
        再用一下负元消去即可, $0_R \cdot a = 0_R$同理.(需要用到: 分配律, 零元定义, 
    负元存在. 与之对比, $0_R \cdot 0_R = 0_R$只需要用到分配律, 零元和单位元, 
    因此在半环(semiring)中(2)是不成立的, 这里半环要求$0$和$1$存在)
        \item 前一个为负元定义(教材p9的注记); 后一个先由分配律, 
    \[
        a(b - c) = ab + a(-c),
    \]
    又由于
    \[
        a(-c) + ac = a(c + (-c)) = a \cdot 0_R \overset{(2)}= 0
    \]
    得$a(-c) = -ac$, 这也是(5)的证明.
    这里要注意仅使用$-a \overset{(*)}= -1_R \cdot a$也无法将负号提到前面, 需要$R$是交换环
    或者说明$-1_R \cdot a = a \cdot (-1_R) = -a$.
    $(*)$的证明如下
    \[
        -1_R \cdot a + a = -1_R \cdot a + 1_R \cdot a = (-1_R + 1_R) \cdot a = 0_R \cdot a \overset{(2)}= 0_R.
    \]
    右乘$-1_R$同理.
        \item 利用$-a = -1_R \cdot a$和分配律展开即可.
        \item 见(3).
        \item (3)和(5)的推论.
        \item 参考\ref{ex:1.1.1}的(6), 明确定义:
        \[
            0a \defeq 0_R,\, (n + 1)a \defeq na + a,\, n \in \mathbb{N}
        \]
        以及
        \[
            na \defeq -((-n)a),\, n < 0.
        \]
        一样的, 可以验证对任意整数$n \in \mathbb{Z}$都有
        $(n + 1)a = na + a$和$na = -((-n)a)$.
        先对$n$归纳得
    \begin{equation}
        (m + n)a = ma + na, \quad \forall m \in \mathbb{Z}, n \in \mathbb{N}
        \tag{i}
        \label{eq:1.2.1.7}
    \end{equation}
        
    然后$n < 0$, 存在$k\in \mathbb{Z}_{>0}$使得$m + kn < 0$,
    \[
    \begin{aligned}
        (m + n)a &= (m + kn - (k - 1)n)a\\
        &\overset{\eqref{eq:1.2.1.7}}= (m + kn)a + (-(k - 1)n)a\\
        &= -(-m - kn)a + (n - kn)a\\
        &\overset{\eqref{eq:1.2.1.7}}= -((-m)a + (-kn)a) + na + (-kn)a\\
        &\overset{(4)}= ma + (kn)a + na + (-kn)a = ma + na.
    \end{aligned}   
    \]
        第二个式子可直接利用第一个证明, 
    $m = 0$根据定义左右均为$0_R$, $m > 0$有, 
    \[
    \begin{aligned}
        (mn)a &= \left(\sum_{i = 1}^{m} n\right)a\\
        &= \sum_{i = 1}^{m} (na)\\
        &= m(na).
    \end{aligned}
    \]
    $m < 0$利用$mn = (-m)(-n)$, 做同样的操作.
        \item 对$n$归纳, 由于加法有交换律,
    \[
    \begin{aligned}
        (n + 1)(a + b) &= n(a + b) + a + b\\
        &= na + nb + a + b\\
        &= (n + 1)a + (n + 1)b.
    \end{aligned}
    \]
    得
    \[
        n(a + b) = na + nb, \quad \forall n \in \mathbb{N}
    \]
    当$n < 0$有
    \[
        n(a + b) = -(-n(a + b)) = -((-n)a + (-n)b) \overset{(4)}= na + nb.
    \]
    第二个等式使用分配律, $n = 0$根据定义左右均为$0_R$, $n > 0$,
    \[
        n(ab) = \sum_{i = 1}^{n} ab = a\sum_{i = 1}^{n} b = a(nb).
    \]
    $n < 0$, 用$n = -(-n)$, $n(ab) = -a((-n))b \overset{(5)}= a(nb)$.
    同样的也会有$n(ab) = (na)b$.
        \item (7)和(8)的推论, 
    \[
    \begin{aligned}
        (ma) \cdot (nb) &\overset{(8)}= m(a \cdot (nb))\\
        &\overset{(8)}= m(n(ab))\\
        &\overset{(7)}= mn(ab)\\
        &\overset{(8)}= (mna) \cdot b.
    \end{aligned}
    \]
        \item 对$n$归纳,
    \[
    \begin{aligned}
        (a + b)^n \cdot (a + b) &= \left(\sum_{i = 0}^{n} \binom{n}{i} a^{n - i}b^i\right) \cdot (a + b)\\
        &= \sum_{i = 0}^{n} \binom{n}{i} a^{n - i}b^ia + \sum_{i = 0}^{n} \binom{n}{i} a^{n - i}b^{i + 1}\\
        &\overset{ab = ba}= \sum_{i = 0}^{n} \binom{n}{i} a^{n - i + 1}b^i + \sum_{i = 0}^{n} \binom{n}{i} a^{n - i}b^{i + 1}\\
        &= a^{n + 1} + \sum_{i = 1}^{n} \left(\binom{n}{i} + \binom{n}{i - 1}\right) a^{n - i + 1}b^{i} + b^{n + 1}\\
        &= \sum_{i = 0}^{n + 1} \binom{n + 1}{i} a^{n + 1 - i}b^i.
    \end{aligned}
    \]
    \end{enumerate}
(7)-(9)中实际上需要用归纳法证明的只有
\[
\begin{aligned}
    n(a + b) &= na + nb,\\
    (m + n)a &= ma + na,\\
    (mn)a &= m(na),\\
\end{aligned}
\]
这三条加上$1a = a$, 是在说任何一个Abel群都是$\mathbb{Z}$-模(见教材5.1节).
\end{solution}

\begin{problem}
    假设集合$R$上有两个运算, 除加法的交换律外满足环的所有其他公
理. 利用分配律证明:加法是交换的 (从而$R$是环).
\end{problem}

\begin{solution}
    这和\ref{ex:1.1.5}是一道题.
\end{solution}

\begin{problem}
    设$X$是集合, $P(X)$表示$X$的所有子集形成的集合, 在$P(X)$上
定义“加法”和“乘法”: $A + B = A \cup B - A \cap B$,
$A \cdot B = A \cap B$. 证明:在这些运算下$P(X)$是一个环, 
且$2A = 0 (\forall A \in P(X))$.
\end{problem}

\begin{solution}
    这里$A + B$为对称差, $A + B = A \cup B - A \cap B = (A - B) \cup (B - A)$.
用$A^c$表示$A$的补集. 那么,
\[
    A + B = (A \cap B^c) \cup (A^c \cap B).
\]
\begin{enumerate}[(i)]
    \item $(P(X), +)$是Abel群. 交换律由定义是显然的.
    
    结合律:
    \[
    \begin{aligned}
        (A + B) + C &= (((A \cap B^c) \cup (A^c \cap B)) \cap C^c) \cup (((A \cap B^c) \cup (A^c \cap B))^c \cap C)\\
        &= (A \cap B^c \cap C^c) \cup (A^c \cap B \cap C^c) \cup (A^c \cap B^c \cap C) \cup (A \cap B \cap C)\\
        &= A + (B + C). \quad \text{(轮换对称, 见\ref{ex:1.1.4}的结合律证明)}
    \end{aligned} 
    \]
    零元为$\varnothing$,
    \[
        A + \varnothing = \varnothing + A = A \cup \varnothing - A \cap \varnothing = A.
    \]
    负元为$A$本身,
    \[
        A + A = A \cup A - A \cap A = A - A = \varnothing.
    \]
    即$2A = 0$.
    \item $(P(X), \cdot)$是(交换)幺半群, 单位元是$X$. 由于$\cdot$就是交集$\cap$,
因此这一点是显然的.
    \item 分配律:
    \[
    \begin{aligned}
        (A + B) \cdot C &= ((A \cap B^c) \cup (A^c \cap B)) \cap C\\
        &= (A \cap B^c \cap C) \cup (A^c \cap B \cap C)\\
        A \cdot C + B \cdot C &= (A \cap C \cap (B \cap C)^c) \cup ((A \cap C)^c \cap B \cap C)\\
        &= (A \cap B^c \cap C) \cup (A^c \cap B \cap C).
    \end{aligned}  
    \]
    故有$(A + B) \cdot C = A \cdot C + B \cdot C$.
    另一部分证明类似.
\end{enumerate}
因此$(P(X), +, \cdot)$为一个(交换)环.
\end{solution}

\begin{problem}
    设$R$是一个环, $S \subset R$是一个非空子集合. 试证明
\[
    C(S) \defeq \{a \in R \mid ax = xa, \forall x \in S \}
\]
是$R$的一个子环.
\end{problem}

\begin{solution}
    该子环称为子集$S$的中心化子(centralizer).

    $\forall a, b \in C(S)$, 需要验证
    \[
        a - b \in C(S), \quad ab \in C(S), \quad 1 \in C(S).
    \]\
    其中$1 \in C(S)$是显然的.
    对$\forall x \in S$
    \[
    \begin{gathered}
        (a - b)x = ax + bx = xa + xb = x(a - b),\\
        (ab)x = a(bx) = a(xb) = (ax)b = (xa)b = x(ab).\\
    \end{gathered}
    \]
    因此$a - b, ab \in C(S)$, $C(S)$是子环.
\end{solution}

\begin{problem}\label{ex:1.2.5}
    证明:如果在环$R$中$1 - ab$可逆, 则$1 - ba$也可逆.
\end{problem}

\begin{solution}
    设$1 - ab$的逆为$c$, 则$ab = 1 - c^{-1}$.

    考虑形式级数
    \[
        (1 - x)^{-1} = \sum_{i = 0}^{+\infty} x^i
    \]
    则有
    \[
    \begin{aligned}
        (1 - ba)^{-1} &= \sum_{i = 0}^{+\infty} (ba)^i\\
        &= 1 + b\left(\sum_{i = 0}^{+\infty} (ab)^i\right)a\\
        &= 1 + b(1 - ab)^{-1}\\
        &= 1 + bca.
    \end{aligned} 
    \]
    验证$1 + bca$确实是$1 - ba$的逆:
    \[
    \begin{aligned}
        (1 - ba)(1 + bca) &= 1 - ba + bca - b(abc)a\\
        &= 1 - ba + bca -b(c - 1)a\\
        &= 1 - ba + bca -bca + ba = 1\\
        (1 + bca)(1 - ba) &= 1 + bca - ba - b(cab)a\\
        &= 1 + bca - ba - b(c - 1)a\\
        &= 1.
    \end{aligned}
    \]
\end{solution}

\begin{problem}\label{ex:1.2.6}
    如果环$R$满足条件:$\forall x \in R,\quad x^2 = x$,证明$R$是交换环.
\end{problem}

\begin{solution}
    条件$x^2 = x$称为乘法是幂等(idempotent)的. 考虑
    \[
        (x + 1)^2 = x^2 + 2x + 1 = x + 1,
    \]
    或者直接带入$-x$,
    得
    \[
        -x = x^2 = x.
    \]
    再考虑
    \[
        (x + y)^2 = x^2 + xy + yx + y^2 = x + xy + yx + y = x + y,
    \]
    得
    \[
        xy = -yx = yx.
    \]
\end{solution}

\begin{problem}[华罗庚恒等式]
    设$a, b$是环$R$中的元素. 如果$a, b, ab - 1$可逆, 
证明$a - b^{-1}$, $(a - b^{-1})^{-1} - a^{-1}$也可逆, 且有下列恒等式:
\[
    \left((a - b^{-1})^{-1} - a^{-1}\right)^{-1} = aba - a.
\]
\end{problem}

\begin{solution}
    由于$a, b, ab - 1$均可逆, 即$a, b, ab - 1 \in U(R)$.
$U(R)$为环$R$的单位群(\ref{ex:1.3.2}).
故
    \[
        a - b^{-1} = (ab - 1)b^{-1} \in U(R),
    \]
    那么只需证明华罗庚恒等式. 直接验证即可
    由\ref{ex:1.2.5}, $(ba - 1)^{-1} = b(ab - 1)^{-1}a - 1$以及
    \ref{ex:1.3.2}证明的(1).
    \[
    \begin{aligned}
        \left((a - b^{-1})^{-1} - a^{-1}\right)^{-1} &= \left(((ab - 1)b^{-1})^{-1} - a^{-1}\right)^{-1}\\
        &= (b(ab - 1)^{-1} - a^{-1})^{-1}\\
        &= \left((b(ab - 1)^{-1}a - 1)a^{-1}\right)^{-1}\\
        &= a(b(ab - 1)^{-1}a - 1)^{-1}\\
        &= a(ba - 1)\\
        &= aba - a.
    \end{aligned}
    \]
\end{solution}

\begin{problem}[多项式矩阵的带余除法]
    设$A \in M_n(K)$是一个给定的$n$阶矩阵.
对任意多项式矩阵$A(x) \in M_{n \times m}(K[x])$, 证明存在唯一的
$B(x) \in M_{n \times m}(K[x])$, $R \in M_{n \times m}(K)$使得
$A(x) = (xI_n - A)B(x) + R$.
\end{problem}

\begin{solution}
    先证唯一性, 若存在$B'(x) \in M_{n \times m}(K[x])$
和$R' \in M_{n \times m}(K)$也满足条件, 则有
    \[
        (xI_n - A)(B(x) - B'(x)) = R' - R \in M_{n \times m}(K).
    \]
    设
\[
    B(x) - B'(x) = B_0 + B_1x + B_2x^2 + \cdots + B_kx^k, \quad B_i \in M_{n \times m}(K), 0 \leqslant i \leqslant k.
\]
    将左边展开得
\[
\begin{aligned}
    B_k &= 0,\\
    -AB_k + B_{k - 1} &= 0 \implies B_{k - 1} = 0,\\
    -AB_{k - 1} + B_{k - 2} &= 0 \implies B_{k - 2} = 0,\\
    \vdots\\
    -AB_1 + B_0 &= 0 \implies B_0 = 0,\\
    -AB_0 &= R' - R = 0.
\end{aligned}
\]
    再证存在性, 将$A(x)$写成多项式的形式,
\[
    A(x) = A_0 + A_1x + A_2x^2 + \cdots + A_kx^k, \quad A_i \in M_{n \times m}(K), 0 \leqslant i \leqslant k.
\]
我们对$k$归纳, $k = 0$时, $A(x) = A_0$为常数矩阵, 取$B(x) = O_{n \times m}$(零矩阵), 
$R = A_0$即可.

假设对任意$k$次多项式$A(x)$有$B(x) \in M_{n \times m}(K[x])$, $R \in M_{n \times m}(K)$使得
$A(x) = (xI_n - A)B(x) + R$. 考查$k + 1$的情形:
\[
\begin{aligned}
    A(x) &= A_0 + x(A_1 + A_2x + \cdots + A_{k + 1}x^k)\\
    &= A_0 + x((xI_n - A)\tilde{B}(x) + \tilde{R})\\
    &= (xI_n - A)x\tilde{B}(x) + xI_n\tilde{R} - A\tilde{R} + A\tilde{R}+ A_0\\
    &= (xI_n - A)(x\tilde{B}(x) + \tilde{R}) + A\tilde{R} + A_0.
\end{aligned}
\]
取$B(x) = x\tilde{B}(x) + \tilde{R} \in M_{n \times m}(K[x])$, 
$R = A\tilde{R} + A_0$即可.
\end{solution}

\begin{problem}
    设$m > 0$是任意整数, $\mathbb{Z}_m = \{\bar{0}, \bar{1}, \cdots, \overline{m-1}\}$
是$\mathbb{Z}$的模$m$剩余类环. 试证明:$\bar{a} \in \mathbb{Z}_m$可逆当且仅当$(a, m) = 1$
(即:$a$与$m$互素).
\end{problem}

\begin{solution}
    $\overline{a} \in \mathbb{Z}_m$可逆, 
\[
\begin{aligned}
    &\iff \exists \overline{b} \in \mathbb{Z}, \quad \overline{a}\overline{b} = \overline{1}\\
    &\iff ab = 1 + km, \quad k \in \mathbb{Z},\\
    &\iff (a, m) = 1. \quad \text{(Bezout's Theorem)}
\end{aligned}
\]
注:一般用记号$\mathbb{Z}/m\mathbb{Z}$表示模$m$剩余类环.(理想和商环, 教材2.1节p25)
若$(a, m) = 1$, 则$\overline{a}$是加法群$(\mathbb{Z}/m\mathbb{Z}, +)$
的生成元, 即$\overline{a}$(在加法群)的阶(教材1.3节, p17)是$m$.
\end{solution}

\begin{problem}
    设$R$是仅有$n$个元素的环, 试证明对任意$a \in R$有
\[
    na \defeq \underbrace{a + a + \cdots + a}_n = 0.
\]
\end{problem}
    
\begin{solution}
    该题的证明归结为一句话, 加法群的阶$(R, +)$为$n$, 故$na = 0$.

注:有限群$G$内任一元素$a$, 有$|a| \Big| |G|$(教材4.1节p70推论4.1.3),
因此必有$a^{|G|} = e$, 在这道题就是$na = 0$.

另外, $n$称为环$R$的特征(characteristic), 见教材p27.
\end{solution}

\begin{problem}
    环$R$中非零元$x$称为幂零元(nilpotent), 若存在$n > 0$使$x^n = 0$. 证明:
\begin{enumerate}[(1)]
    \item 如果$x$是幂零元, 则$1 - x$是可逆元;
    \item $\mathbb{Z}_m$有幂零元当且仅当$m$可以被一个大于$1$的整数的平方整除.
\end{enumerate}
\end{problem}

\begin{solution}
\begin{enumerate}[(1)]
    \item 注意到
\[
    1 = 1 - x^n = (1 - x)(1 + x + x^2 + \cdots + x^{n - 1})
\]
    \item "$\Rightarrow$": 若$\mathbb{Z}_m$有幂零元$\overline{a}$,
    则存在$n > 1(a \neq 0)$使得$\overline{a}^n = \overline{a^n} = \overline{0}$.
    即$m \mid a^n$. 取素数$p \mid m$, 则$p \mid a^n$, 故$p \mid a$.
    因此, 若$m$的素因数分解为$m = p_1^{e_1}p_2^{e_2}\cdots p_r^{e_r}$,
    其中$p_1, p_2, \cdots, p_r$为互异的素数, $e_1, e_2, \cdots, e_r \geqslant 1$,
    则$p_i \mid a,\, 1 \leqslant i \leqslant r$, 故有$p_1p_2\cdots p_r \mid a$.
    因此有$p_1p_2\cdots p_r \leqslant a < m$, 故必有某个$e_i > 1$, 即
    $\exists 1 \leqslant i \leqslant r$, $e_i \geqslant 2$, 这样$p_i^2 \mid m$.

    "$\Leftarrow$": 反过来, 若$m$可以被某个大于$1$的平方整除, 则上述$e_i$中必有一个
    大于$1$, 此时取$a = p_1p_2\cdots p_r$, $\overline{a}$为$\mathbb{Z}_m$的幂零元.
\end{enumerate}
注:有关初等数论整除相关的部分请自行查阅书籍.
\end{solution}

\begin{problem}
    设$R$是一个环, 如果$(xy)^2 = x^2y^2 (\forall x, y \in R)$, 则
$R$是交换环.
\end{problem}

\begin{solution}
    先考虑
    \[
        ((x + 1)y)^2 = (x + 1)^2y^2 \implies\, xy^2 = yxy,
    \]
    再将上式中$y$换成$y + 1$,
    \[
        x(y + 1)^2 = (y + 1)x(y + 1) \implies xy = yx.
    \]
\end{solution}

\begin{problem}
    如果环$R$满足条件:$x^6 = x (\forall x \in R)$. 证明:
\begin{enumerate}[(1)]
    \item $x^2 = x (\forall x \in R)$;
    \item $R$是一个交换环.
\end{enumerate}
\end{problem}

\begin{solution}
\begin{enumerate}[(1)]
    \item 先带入$-x$,
\[
    -x = (-x)^6 = x^6 = x \implies 2x = 0.
\]
    考虑$(x + 1)^6$,
\[
    (x + 1)^6 = x^6 + 6x^5 + 15x^4 + 20x^3 + 15x^2 + 6x + 1 = x + 1,
\]
得到
\[
    6x^5 + 15x^4 + 20x^3 + 15x^2 + 6x = 0.
\]
利用$2x = 0$消去含$2x$的项得
\[
    x^4 + x^2 = 0.
\]
两边乘$x^2$得
\[
    x + x^4 = 0.
\]
再相减得$x^2 = x$.
    \item 由(1)和\ref{ex:1.2.6}.
\end{enumerate}
\end{solution}