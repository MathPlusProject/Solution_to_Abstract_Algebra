\subsection{教材p17-p18}

\begin{problem}\label{ex:1.3.1}
    设$G$是一个群, 对于任意的$a, b \in G$, 证明$ab$的阶和$ba$的阶相等.
\end{problem}

\begin{proof}
    若$|ab| = n < \infty$, 则
    \[
        (ba)^n = b \cdot (ab)^n \cdot b^{-1} = bb^{-1} = e.
    \]
    且对$1 \leqslant k < n$, $(ba)^k = b(ab)^kb^{-1} \neq e$. 因此$|ba| = n$. 反之亦然.
    
    若$|ab| = \infty$, 则
    \[
        \forall n \in \mathbb{Z}_{\geqslant 1}, \quad (ba)^n = b(ab)^nb^{-1} \neq e.
    \]
    故$|ba| = \infty$. 反之亦然.
\end{proof}

\begin{remark}
    事实上, 群$G$内$g$和$h = aga^{-1}$阶相等. $h$称为$g$的一个共轭(conjugate, 教材p77).
    \[
        \sigma_a: G \to G, \quad g \mapsto aga^{-1}
    \]
    是群$G$的一个自同构. 而对一般的群同态$\varphi: G \to G'$, $|g| < \infty \implies |\varphi(g)| < \infty$且$|\varphi(g)| \big| |g|$. 因此若$\varphi$为同构, 则$|g| = |\varphi(g)|$(包括左右为无穷的情况).
\end{remark}

\begin{problem}\label{ex:1.3.2}
    设$R$是一个环, $U(R)$表示$R$中所有可逆元集合, 试证明: $U(R)$关于环$R$的乘法是一个群(称为$R$的单位群).
\end{problem}

\begin{proof}
    \begin{enumerate}[(1)]
        \item 这里首先需要验证运算的封闭性, $\forall a, b \in U(R)$, 有$(b^{-1}a^{-1})(ab) = b^{-1}(a^{-1}a)b = 1$, 故$ab \in U(R)$且$(ab)^{-1} = b^{-1}a^{-1}$.
        \item $1 \in U(R)$, 因为$1 \cdot 1 = 1$的确可逆;
        \item 由于乘法是$R$上的乘法, 故结合律成立;
        \item 若$a \in U(R)$, 则由\ref{ex:1.1.1}的(3), $a^{-1} \in U(R)$且$(a^{-1})^{-1} = a$;
    \end{enumerate}
\end{proof}

\begin{remark}
    一般$U(R)$也记作$R^\times$, 比如$K$是域时, $K^\times = K^* =  K \setminus \{0\}$.
\end{remark}

\begin{problem}
    证明除了单位元之外所有元素的阶都是$2$的群一定是交换群.
\end{problem}

\begin{proof}
    由于任意$a^2 = e$, 故$a = a^{-1}$.

    考虑
    \[
        (ab)^2 = e \implies ab = b^{-1}a^{-1} = ba.
    \]
    或直接验证
    \[
        ab = ab \cdot (ba)^2 = abbaba = ba
    \]
\end{proof}

\begin{problem}
    令$C(\mathbb{R} ) = \left\{\text{所有连续函数: } \mathbb{R} \overset{f}\to \mathbb{R} \right\}$, $\forall f ,\, g \in C(\mathbb{R})$,
    \[
        f + g \in C(\mathbb{R}),\quad f \cdot g \in C(\mathbb{R})
    \]
    定义: $\forall x \in \mathbb{R}, (f + g)(x) = f(x) + g(x), (f \cdot g)(x) = f(g(x))$, 证明$(C(\mathbb{R}), +)$是交换群. $(C(\mathbb{R}), +, \cdot)$是否为环?
\end{problem}

\begin{proof}
    $(C(\mathbb{R}), +)$的零元为零函数$\mathbf{0}: \mathbb{R} \to \mathbb{R},\, x \mapsto 0$, $(f + \mathbf{0})(x) = f(x) + 0 = f(x) = 0 + f(x) = (\mathbf{0} + f)(x),\, \forall x \in \mathbb{R}$.

    $f \in C(\mathbb{R})$的负元为$-f: \mathbb{R} \to \mathbb{R},\, x \mapsto -f(x)$, $(f + (-f))(x) = ((-f) + f)(x) = f(x) - f(x) = 0 = \mathbf{0}(x)$.

    由于$f + g$为逐点定义, 故交换律和结合律依赖于$\mathbb{R}$的加法, 是平凡的. 故$(C(\mathbb{R}), +)$是Abel群.

    若$f$不是$\mathbb{R}$-线性函数, 如$f(x) = x^2$, 则$(f \cdot (g + h))(x) = f((g + h)(x)) = f(g(x) + h(x)) \neq f(g(x)) + f(h(x))$. 故$C(\mathbb{R}, +, \cdot)$不是环.
\end{proof}

\begin{problem}\label{ex:1.3.5}
    写出对称群$S_3$的乘法表.
\end{problem}

\begin{proof}
    记$\mathrm{id}_{S_3} = e$, 令$a = (1\:2)$, $b = (1\:2\:3)$, 有$a^2 = e$, $b^3 = e$, $abab = e \iff ba = ab^2$. 乘法表如下:
    \[
    \begin{array}{c|cccccc}
             & e    & a    & b   & b^2  & ab   & ab^2 \\
        \hline
        e    & e    & a    & b   & b^2  & ab   & ab^2 \\
        a    & a    & e    & ab  & ab^2 & b^2  & b \\
        b    & b    & ab^2 & b^2 & e    & a    & ab \\
        b^2  & b^2  & ab   & e   & b    & ab^2 & a \\
        ab   & ab   & b^2  & a   & ab^2 & e    & b \\
        ab^2 & ab^2 & b    & ab  & a    & b^2  & e \\
    \end{array}
    \]
\end{proof}

\begin{remark}
    可以看到$S_3$, 若取$a = (1\:2),\, b = (1\:2\:3)$, 则$S_3$可以由$a, b$生成, 即考虑所有可能的乘积, 一般可以表示为$S_3 = \langle a, b \rangle,\, a = (1\:2),\, b = (1\:2\:3)$.
    
    若不给$a, b$加任何限制, 便得到一个自由群(free group)$F(\{a, b\})$. 一般地, 任意一个集合$A$都可以生成一个自由群$F(A)$, $A$就是生成元组成的集合. 可以证明任何一个群都同构于某个自由群的商群, 而对应的正规子群便是由生成元满足的某些关系确定(将$A$看成字母表, $\Sigma_A$表示单词的集合, 这些关系可以表示为一些满足$w = e$单词$w \in \Sigma_A$). 把这些$w$组成的集合记为$\mathscr{R}$, $A$和$\mathscr{R}$将唯一确定一个群$G$, $(A \mid \mathscr{R})$称为$G$的一个展示(presentation). 以$S_3$为例, $S_3$的一个展示为$(\{a, b\} \mid a^2, b^3, abab)$. 另外有二面体群(Dihedral Groups)$D_{2n} = (a, b \mid a^2, b^n, abab)$
    
    由于这本教材没有讲自由群, 所以想要了解的话需要查阅别的教材.(可参考\cite{aluffi2009algebra}II.\S5和II.\S8.2)
    
    BTW, 这本教材和很多教材一样, 会把集合$A$对称群$S_A$上的乘法写成$f \cdot g \defeq f \circ g$, 这个其实会有一点不舒服. 正常我们习惯于说: 映射$f:X \to Y$和$g:Y \to Z$的复合是$g \circ f$. 这在范畴的定义也是习惯于这样, 复合会写成这样:
    \[
        \mathrm{Hom}_{\mathcal{C}}(X, Y) \times \mathrm{Hom}_{\mathcal{C}}(Y, Z) \to \mathrm{Hom}_{\mathcal{C}}(X, Z),\, (f, g) \mapsto g \circ f.
    \]
    这样说的好处在于一眼能感觉出这个运算是不交换的. 当然这只是个人感觉, 也有可能是我先入为主了, 因为我最开始接触到的范畴里的复合是这样写的. 如果引入范畴的记号, $S_A$会记作$\mathrm{Aut}_{\mathsf{Set}}(A)$, 其中$\mathsf{Set}$表示集合范畴. 那么$S_A$上的乘法按范畴的定义来写应该是:
    \[
        S_A \times S_A \to S_A,\quad (f, g) \mapsto f \cdot g \defeq g \circ f
    \]
    可以看到和$f \cdot g \defeq f \circ g$刚好是反过来的. 没有使用范畴语言的话就还好, 不会出现前后不自洽的问题, 但如果介绍了范畴语言, 那应该注意$S_A$上乘法的定义要和范畴定义不能冲突, 这一点\cite{lang2012algebra}和\cite{hungerford2003algebra}就做的很好. 它的范畴定义故意反了过来, 它写成$\mathrm{Hom}_{\mathcal{C}}(Y, Z) \times \mathrm{Hom}_{\mathcal{C}}(X, Y) \to \mathrm{Hom}_{\mathcal{C}}(X, Z)$.
    
    那么哪一个才对呢, 事实上都是对的, 你总能验证$S_A$确实时一个群. 原因在于, 当你只考虑所有的同构时, 就得到一个子范畴, 这是一个群胚(groupoid), 它是一个自反范畴, 所以顺序就没区别了. 但我个人认为还是统一一下比较好, 主要是复合是非交换的, $f \circ g$和$g \circ f$一般不等. 为了方便还是按照教材为准吧, 使用$f \cdot g = f \circ g$.(尽管我是有点不习惯的)
\end{remark}

\begin{problem}
    证明: 一个群$G$不会是两个真子群(不等于$G$的子群)的并.
\end{problem}

\begin{proof}
    反证, 假设$H_1, H_2 \lvertneqq G$且$G = H_1 \cup H_2$, 则$\exists h_1 \in G \setminus H_2 \subseteq H_1, h_2 \in G \setminus H_1 \subseteq H_2$, 有$h_1h_2 \in G = H_1 \cup H_2$, 矛盾. (不妨设$h_1h_2 \in H_1 \implies h_2 \in H_1$)
\end{proof}

\ref{ex:1.3.7}-\ref{ex:1.3.9}为群的其他三种定义.

\begin{problem}\label{ex:1.3.7}
    一个非空集合$G$带有满足结合律的“乘法”运算, 我们称之为半群. 如果$G$是一个半群, 且满足如下性质:
    \begin{enumerate}[(1)]
        \item $G$含有右单位元$1_r$(即: $a \cdot 1_r = a$, $\forall a \in G)$;
        \item $G$中的每个元素$a$有右逆(即: 存在$b \in G$, 使得$a \cdot b = 1_r)$.
    \end{enumerate}
    试证明: $G$是一个群.
\end{problem}

\begin{proof}
    先证右逆为逆,
    \[
    \begin{gathered}
        \forall a \in G \, \exists b \in G, ab = 1_r,\\
        \implies \exists c \in G, bc = 1_r,\\
        \implies ba = (ba)1_r = (ba)(bc) = b(ab)c = b1_rc = bc = 1_r.
    \end{gathered}
    \]
    再证右单位为单位,
    \[
        1_ra = (ab)a = a(ba) = a1_r = a.
    \]
\end{proof}

\begin{problem}\label{ex:1.3.8}
    证明: 半群$G$是群的充要条件是: $\forall a, b \in G$, $ax = b$和$ya = b$都有(唯一)解.
\end{problem}

\begin{proof}
\begin{enumerate}[(1)]
    \item "$\impliedby$": 取定一个$a \in G$, 方程$ax = a$的解设为$e_a$. 对$\forall b \in G$, 方程$ya = b$有解$y_b$, 则有
    \[
        be_a = (y_ba)e_a = y_b(ae_a) = y_ba = b.
    \]
    即$e_a$是$G$的右单位, 记为$1_r$, 又因为$\forall a \in G$, 方程$ax = 1_r$有解, 即$a$有右逆, 由\ref{ex:1.3.7}知$G$是群.
    \item "$\implies$": 若$G$是群, 则方程$ax = b$的唯一解为$a^{-1}b$, 方程$ya = b$的唯一解为$ba^{-1}$.
\end{enumerate}
    
\end{proof}

\begin{problem}\label{ex:1.3.9}
    证明:
    \begin{enumerate}[(1)]
        \item 在群中左右消去律都成立: 如果$ax = ay$, 则$x = y$; 如果$xa = ya$,则$x = y$.
        \item 左右消去律都成立的有限半群一定是群.
    \end{enumerate}
\end{problem}

\begin{proof}
    设$G = \{a_1, a_2, \cdots a_n\}$. 对$\forall 1 \leqslant i, j \leqslant n$,
    \[
        a_ia_1, a_ia_2, \cdots, a_ia_n
    \]
    互异, 否则存在$a_k \neq a_l$使得$a_ia_k = a_ia_l$, 由消去律得$a_k = a_l$矛盾. 因此$\exists 1 \leqslant t \leqslant n$, $a_ia_t = a_j$, 即方程$a_ix = a_j$有解. 同理方程$ya_i = a_j$也有解, 由\ref{ex:1.3.8}, $G$是群.
\end{proof}

\begin{problem}\label{ex:1.3.10}
    证明:偶数阶有限群$G$中必有$2$阶元.
\end{problem}

\begin{proof}
    设$|G| = 2n$. 对$e \neq g \in G$, $|g| = 2 \iff g = g^{-1}$. 定义$G$上的一个等价关系
    \[
        g \sim g' \iff g = g' \lor g' = g^{-1}.
    \]
    考虑商集$G/\sim = \{\overline{g} \mid g \in G\}$, 用$\#S$表示集合$S$的元素个数(基数)防止混淆. 若$|g| = 2$或$g = e$, 则$\#\, \overline{g} = 1$, 否则$\#\, \overline{g} = 2$. 因此若$m$为$G$中阶为$2$的元素的个数, 则$2n = m + 1 + 2(\#\, (G/\sim) - m - 1)$, 故$2n - m - 1$为偶数, 因此$m > 0$.
\end{proof}

\begin{remark}
    当然可以用Sylow定理一步到位.
\end{remark}

\begin{problem}
    证明:$GL_2(\mathbb{R})$中的元素
    \(
        x = \begin{pmatrix}
            0 & 1\\
            -1 & 0
        \end{pmatrix},
        y = \begin{pmatrix}
            0 & 1\\
            -1 & -1
        \end{pmatrix}
    \)
    的阶分别是$4$和$3$. 但$xy$是无限阶元.
\end{problem}

\begin{proof}
    用$I_n$表示$n$阶单位阵, 计算可得
    \[
        x^2 = \begin{pmatrix}
            -1 & 0 \\
            0 & -1
        \end{pmatrix},
        x^3 = \begin{pmatrix}
            0 & -1 \\
            1 & 0
        \end{pmatrix},
        x^4 = \begin{pmatrix}
            1 & 0 \\
            0 & 1
        \end{pmatrix} = I_2.
    \]
    故$|x| = 4$, 同理,
    \[
        y^2 = \begin{pmatrix}
            -1 & -1 \\
            1 & 0
        \end{pmatrix},
        y^3 = \begin{pmatrix}
            1 & 0 \\
            0 & 1
        \end{pmatrix} = I_2.
    \]
    $|y| = 3$. 最后是$xy$,
    \[
        xy = \begin{pmatrix}
            -1 & -1 \\
            0 & -1
        \end{pmatrix},
        (xy)^2 = \begin{pmatrix}
            1 & 2 \\
            0 & 1
        \end{pmatrix},
        (xy)^3 = \begin{pmatrix}
            -1 & -3 \\
            0 & -1
        \end{pmatrix}, \cdots
    \]
    可以用归纳法证明
    \[
        (xy)^n = (-1)^n
        \begin{pmatrix}
            1 & n \\
            0 & 1
        \end{pmatrix}
        \neq I_2, \forall n \in \mathbb{Z}_{\geqslant 1}.
    \]
    故$|xy| = \infty$.
\end{proof}
    
\begin{problem}\label{ex:1.3.12}
    证明群的任意多个子群的交仍是子群.
\end{problem}

\begin{proof}
    设$G$是群, 记$I$为指标集, $H_i < G,\, \forall \in I$. 验证\(H = \displaystyle\bigcap_{i \in I} H_i < G\): 首先$e_G \in H$, $H \neq \varnothing$,
    \[
    \begin{gathered}
        \forall a, b \in H = \bigcap_{i \in I} H_i \implies \forall i \in I,\,a, b \in H_i\\
        \implies ab^{-1} \in H_i, \quad \forall i \in I\\
        \implies ab^{-1} \in \bigcap_{i \in I} H_i = H.
    \end{gathered}
    \]
\end{proof}

\begin{remark}
    教材中并未提及这个判断子群的命题, 但其实是最常用的.

    \begin{propstar}[子群的判定]
        设$G$是一个群, $\varnothing \neq S \subseteq G$, 则$S < G$($S$是$G$的子群的记号)当且仅当
        \[
            \forall a, b \in S \iff ab^{-1} \in S.
        \]
    \end{propstar}
    证明可参考\cite{aluffi2009algebra}p79.
\end{remark}