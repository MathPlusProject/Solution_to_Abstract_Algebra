\documentclass{../solutions-cn}

\begin{document}
\section*{第八, 九周作业参考解答及补充}

\subsection*{作业}

\begin{exercise}[习题3.1.1]
    设$K$是特征零的域, $f(x) \in K[x]$是次数大于零的首项系数为$1$的多项式, $d(x) = (f(x), f'(x))$是$f(x)$与$f'(x)$的最大公因子. 令
    \[
        f(x) = d(x) \cdot g(x).
    \]
    证明: $g(x)$与$f(x)$有相同的根且$g(x)$没有重根.
\end{exercise}

\begin{proof}
    由2.4.3可知, $a$是$f(x)$和$f'(x)$的公共根当且仅当$a$是$f(x)$的重根. 而$f(x)$和$f'(x)$的公共根当且仅当是$d(x)$的根.
\end{proof}

\begin{exercise}[习题3.1.2]
    设$K \,\red{\subseteq}\, L$是域扩张, $\alpha \in L$是域$K$上的代数元. 令$K[x] \xrightarrow{\psi_{\alpha}}L$, $f(x) \mapsto f(\alpha)$, 表示多项式在$x = \alpha$的取值映射. 试证明:
    \begin{enumerate}[(1)]
        \item $\ker(\psi_\alpha)$由极小多项式$\mu_\alpha(x)$生成;
        \item $\psi_\alpha$诱导了域同构$\mathbb{K}[x]/(\mu_\alpha(x)) \cong K[\alpha]$.
    \end{enumerate}
\end{exercise}

\begin{proof}
    \begin{enumerate}[(1)]
        \item 回忆2.2.6和2.3.2,
        \[
            \ker(\psi_\alpha) = \{f \in K[x] \mid f(\alpha) = 0\}
        \]
        是一个理想. 类似2.2.6, 有$(\mu_\alpha(x)) \subseteq \ker(\psi_\alpha)$且$(\mu_\alpha(x))$是极大理想(由$K[x]$是PID, 2.3.2), 且$\ker(\psi_\alpha) \neq K[x]$, 因此只能是$\ker(\psi_\alpha) = (\mu_\alpha(x))$.
        \item 由2.3.2, $\overline{x}$为$\mu_\alpha$在扩域$K[x]/(\mu_\alpha(x))$中的一个根. 那么映射
        \[
            \varphi: K[x]/(\mu_\alpha(x)) \to K[\alpha],\quad \overline{x} \mapsto \alpha
        \]
        是同构, 这是因为$\psi_\alpha(K[x]) = K[\alpha] = K(\alpha)$(见注记), 那么由同态基本定理就得到同构.
    \end{enumerate}
\end{proof}

\begin{remark}
    教材出现了有限生成扩张(p7-8)但并未单独列出这个的定义, 这里需要用一下所以先把这个定义提一下.

    \begin{definitionstar}
        设$K \subseteq L$是一个域扩张, $S \subseteq L$是一个子集, $K(S)$称为由$F$和$S$生成的子域, 即包含$F$和$S$的最小域:
        \[
            K(S) \defeq \cap \{E \subseteq L \mid F \cup S \subseteq E\} 
        \]
        若$T \subseteq L$是另一个子集, 则定义$K(S)(T) \defeq K(S \cup T)$.

        若存在有限子集$S$使得$L = K(S)$, 即存在有限个$u_1, u_2, \cdots, u_n \in L$使得$L = K(u_1, u_2, \cdots, u_n)$, 则称该扩张是有限生成的(finitely generated).
        \[
            K(u_1, u_2, \cdots, u_n) \defeq \cap \{E \subseteq L \mid K \subseteq E,\, u_1, u_2, \cdots, u_n \in E\}
        \]
        $n = 1$时称为单扩张(simple extension).
    \end{definitionstar}
    下面要验证的事实是, 当$u_1, u_2, \cdots, u_n$都是$K$上代数元的时候,\par
    $K(u_1, u_2, \cdots, u_n) = K[u_1, u_2, \cdots, u_n]$, 且此时该扩张是有限扩张, 否则是无限扩张. 即对域扩张而言:
    \[
        \text{algebraic} + \text{finitely generated} = \text{finite}
    \]
    反过来是非常简单的, 有限扩张$\implies$代数扩张, 利用$1, \alpha, \cdots, \alpha^n$线性相关即可, $n = \mathrm{dim}_K L$. 有限扩张$\implies$有限生成, 取一组基就行.

    对任意的单扩张$K \subseteq K(\alpha)$, 很自然的想法就是用2.3.7的赋值映射来讨论. 对包含$K \overset{i}\hookrightarrow K(\alpha)$用泛性质, 存在唯一的同态 
    \[
        \psi_\alpha:K[x] \to K(\alpha),\quad f(x) \mapsto f(\alpha).
    \]
    由同态基本定理, $K[x]/\ker(\psi_\alpha) \cong \psi_\alpha(K[x]) = K[\alpha] \subseteq K(\alpha)$. 由于$K(\alpha)$按定义是域, 因此是整环. 而$K[\alpha]$是子环, 从而也是整环, 因此$\ker(\psi_\alpha)$是素理想. 而$K[x]$是一个PID, 因此只有两种情况.

    \begin{enumerate}[\bfseries \text{Case}~1]
        \item $\ker(\psi_\alpha) = \{0\}$, 此时$\psi_\alpha$是单射. 这意味着$\{f(x) \in K[x] \mid f(\alpha) = 0\} = \{0\}$, 即零多项式是唯一使得$f(\alpha) = 0$的多项式, 换言之$\alpha$是$K$上的超越元. 注意$K(x)$是$K[x]$的分式域, 由分式域(或者说localization)的泛性质(单独放在后面), 存在唯一的同态使得图表交换:
        \[
            \begin{tikzcd}
                K(x) \arrow[rr, "\varphi_\alpha"] &                                                    & K(\alpha) \\
                                                  & {K[x]} \arrow[ru, "\psi_\alpha"'] \arrow[lu, hook] &          
            \end{tikzcd}
        \]
        即
        \[
            \varphi_\alpha:K(x) \to K(\alpha),\quad \frac{p(x)}{q(x)} \mapsto \frac{p(\alpha)}{q(\alpha)}
        \]
        这是一个单射(因为$K(x)$是域). 考虑$\varphi_\alpha$的像, 它一定是一个域, 且包含$K$和$\alpha$, 因此按定义就有$K(\alpha) \cong \varphi_\alpha(K(x)) \cong K(x)$, 那么$K \subseteq K(\alpha)$扩张次数为无穷($1, \alpha, \alpha^2, \cdots$线性无关). 如$\mathbb{Q}(\pi) \cong \mathbb{Q}(x)$.
        \item $\ker(\psi_\alpha) = (p(x))$, $p(x)$是一个不可约多项式. 注意到$\alpha = \psi_\alpha(x)$, 此时$K[\alpha] = \psi_\alpha(K[x]) \cong K[x]/(p(x))$是包含$\alpha$的域, 而根据$\psi_\alpha$的构造, $K[\alpha] \subseteq K(\alpha)$, 由$K(\alpha)$的定义, 只能是$K[\alpha] = K(\alpha) \cong K[x]/(p(x))$. 此时是有限扩张, 扩张次数为$\deg(p(x))$, 且$p(x)$和$\alpha$的极小多项式$\mu_\alpha(x)$是相伴的($(p(x)) = (\mu_\alpha(x))$, 即就差一个常数, $p(x)$除掉首项系数就是$\mu_\alpha(x)$). 在这个域同构下$\overline{x} \in K[x]/(p(x))$的像就是$\alpha$, 这就是为什么2.3.2在最后提到, $\overline{x}$是多项式$p(x)$在扩域$K[x]/(p(x))$上的根. 
    \end{enumerate}
    多元的情形由归纳法即可,
    \[
    \begin{aligned}
        K(u_1, u_2, \cdots, u_n) &= K(u_1, u_2, \cdots, u_{n - 1})(u_n) = K[u_1, u_2, \cdots, u_{n - 1}](u_n)\\
        &= K[u_1, u_2, \cdots, u_{n - 1}][u_n] = K[u_1, u_2, \cdots, u_n]
    \end{aligned} 
    \],
    且
    \[
    \begin{multlined}
        \left[K[u_1, u_2, \cdots, u_n]:K\right]\\
        = \left[K[u_1, u_2, \cdots, u_n]: K[u_1, u_2, \cdots, u_{n - 1}]\right] \cdot \left[K[u_1, u_2, \cdots, u_{n - 1}]:K\right] < \infty.
    \end{multlined}
    \]
    值得注意的是多元的时候, 这里只是说明可以由$u_1, u_2, \cdots, u_n$通过多项式生成, 但仍有可能由更少的代数元生成, 如1.1.3, $\mathbb{Q}$加上$\sqrt{2}$和$\sqrt{3}$后仍是一个单扩张. 又比如$\mathbb{Q}[\sqrt{2}, \sqrt[4]{2}] = \mathbb{Q}[\sqrt[4]{2}]$.

    上面提到的localization可以理解成分式域的推广. 设$A$是交换环, $S \subseteq A$是一个乘法封闭子集(multiplicatively closed subset), 即$\forall s, t \in S \implies st \in S$并要求$1 \in S$. 换句话说, $S$是$(A, \cdot)$的一个子幺半群. 则$A \times S$上有一个等价关系:
    \[
        (a, s) \sim (a', s') \iff \exists u \in S,\, (as' - a's)u = 0
    \]
    记商集$A \times S / \sim$为$S^{-1}A$, 可以验证这是一个环, $A$是它的子环, 称为$A$对$S$的分式环. 它是包含$A$的并使得$S$中元素都可逆的“最小”的环. 这个构造通常称为环的局部化(localization). 当$A$是整环且取$S = A \setminus \{0\}$时就是分式域, 如$\mathbb{Q}$, $K(x)$. 若$0 \in S$则得到零环, 因此一般尽量排除这种平凡的情况. 当$\mathfrak{p} \subseteq A$是素理想时, 按素理想的定义可以验证$S = A \setminus \mathfrak{p}$是乘法封闭的(事实上反过来也是对的). 此时$S^{-1}A$一般记作$A_{\mathfrak{p}}$, 这是一个局部环(2.3.1), 如$\mathbb{Z}_p$($p$-adic integers).
    
    上述的“最小”对应它的泛性质:
    
    $S^{-1}A$的元素记为$\frac{a}{s}$, 我们有一个自然的同态
    \[
        f_S:A \to S^{-1}A,\quad a \mapsto \frac{a}{1}
    \]
    $f(s) = \frac{s}{1}$在$S^{-1}A$中都是可逆的, 逆是$\frac{1}{s}$. 若环同态$\varphi:A \to B$满足对任意的$s \in S$, $\varphi(s)$在$B$中可逆, 那么存在唯一的环同态$\varphi_S:S^{-1}A \to B$使得图表交换:
    \[
        \begin{tikzcd}
            S^{-1}A \arrow[rr, "\varphi_S"] &                                            & B \\
                                            & A \arrow[ru, "\varphi"'] \arrow[lu, "f_S"] &  
        \end{tikzcd}
    \]
    $\varphi_S(\frac{a}{s}) = \varphi(a)\varphi(s)^{-1}$. 注意一般情况下$f_S$不一定是单的, 事实上是非整环的情形有零因子导致的, 因此$A$是整环的时候且不考虑$0 \in S$, 等价关系$\sim$简化为和分式域的情况一样, 即
    \[
        (a, s) \sim (a', s') \iff as' = a's
    \]
\end{remark}

\begin{exercise}[习题3.1.3]
    设$E = \mathbb{Q}[u], u^3 - u^2 + u + 2 = 0$. 试将$(u^2 + u + 1)(u^2 - u)$和$(u - 1)^{-1}$表示成$au^2 + bu + c (a, b, c \in \mathbb{Q})$的形式.
\end{exercise}

\begin{proof}
    利用$u^3 = u^2 - u - 2$消去次数大于$2$的项.     
    \[
        (u^2 + u + 1)(u^2 - u) = (u^3 - 1)u = (u^2 - u - 3)u = u^2 - u - 2 - u^2 - 3u = -4u - 2.
    \]
    第二个可以用形式级数处理
    \[
    \begin{aligned}
        (u - 1)^{-1} &= \frac{1}{u - 1} = -(1 + u + u^2 + u^3(1 + u + u^2 + \cdots))\\
        &= -(1 + u + u^2 + \frac{u^2 - u - 2}{1 - u})\\
        &= -(1 + u^2 - \frac{2}{1 - u})
    \end{aligned}
    \]
    因此$\frac{1}{u - 1} = -\frac{1}{3}(1 + u^2)$.
\end{proof}

\begin{exercise}[习题3.1.4]
    求$\left[\mathbb{Q}[\sqrt2, \sqrt3]:\mathbb{Q}\right]$(提示: 证明$\left[\mathbb{Q}[\sqrt2, \sqrt3]:\mathbb{Q}[\sqrt3]\right] = 2)$.
\end{exercise}

\begin{proof}
    参考1.4.8, $\sqrt{3} \notin \mathbb{Q}[\sqrt{2}]$, 即不存在$\mathbb{Q}[\sqrt{2}]$上的一次多项式使得$\sqrt{3}$是根, 因此$x^2 - 3$是$\mathbb{Q}[\sqrt{2}]$上的不可约多项式, 且$\sqrt{3}$是它的根, 因此是$\sqrt{3}$对于$\mathbb{Q}[\sqrt{2}]$的极小多项式. 从而有
    \[
        \left[\mathbb{Q}[\sqrt{2}, \sqrt{3}]:\mathbb{Q}[\sqrt{2}]\right] = \left[\mathbb{Q}[\sqrt{2}][\sqrt{3}]:\mathbb{Q}[\sqrt{2}]\right] = \deg(x^2 - 3) = 2.
    \]
    故
    \[
        \left[\mathbb{Q}[\sqrt{2}, \sqrt{3}]:\mathbb{Q}\right] = \left[\mathbb{Q}[\sqrt{2}, \sqrt{3}]:\mathbb{Q}[\sqrt{2}]\right] \cdot \left[\mathbb{Q}[\sqrt{2}]:\mathbb{Q}\right] = 4
    \]
\end{proof}

\begin{exercise}[习题3.1.5]
    设$p$是一个素数, $z \in \mathbb{C}$满足$z^p = 1$且$z \neq 1$, 试证明$[\mathbb{Q}[z]:\mathbb{Q}] = p - 1$.
\end{exercise}

\begin{proof}
    注意到$x^p - 1 = (x - 1)(x^{p - 1} + \cdots + x + 1)$. 由于$z \neq 1$, 因此$z$是多项式$\Phi_p(x) = x^{p - 1} + \cdots + x + 1$的根, 这是一个不可约多项式(教材例2.3.4), 从而是$z$的极小多项式. 因此$[\mathbb{Q}[z]:\mathbb{Q}] = \deg(\Phi_p) = p - 1$.
\end{proof}

\begin{remark}
    这是分圆多项式中$n$为素数的情况.
\end{remark}

\begin{exercise}[习题3.1.6]
    证明:
    \begin{enumerate}[(1)]
        \item $U_n = \{z \in \mathbb{C} \mid z^n = 1\}$是一个循环群;
        \item $z = \cos \frac\pi6 + i\sin \frac\pi6$是$U_{12}$的一个生成元, 但$[\mathbb{Q}[z]:\mathbb{Q}] = 4$;
        \item 求$z = \cos \frac\pi6 + i\sin \frac\pi6$在$\mathbb{Q}$上的极小多项式.
    \end{enumerate}
\end{exercise}

\begin{proof}
    \begin{enumerate}[(1)]
        \item $\zeta_n = e^{\frac{2\pi i}{n}}$是$U_n$的生成元.
        \item $z = \cos \frac\pi6 + i\sin \frac\pi6 = e^\frac{2\pi i}{12} = \zeta_{12}$, 即(1)中提到的生成元. 而
        \[
        \begin{aligned}
            x^{12} - 1 &= (x^4 - 1)(x^8 + x^4 + 1)\\
            &= (x^2 - 1)(x^2 + 1)(x^4 + x^2 + 1)(x^4 - x^2 + 1)\\
            &= (x - 1)(x + 1)(x^2 + 1)(x^2 - x + 1)(x^2 + x + 1)(x^4 - x^2 + 1)
        \end{aligned}
        \]
        是不可约分解, 其中$x^4 - x^2 + 1$是$\Phi_{12}(x)$, 它的根是$12$次本原单位根$\zeta_{12}, \zeta_{12}^5, \zeta_{12}^7, \zeta_{12}^{11}$, $x - 1$为$\Phi_1(x)$, 根是$1 = \zeta_{12}^0$; $x + 1$为$\Phi_2(x)$, 根是$-1 = \zeta_{12}^6$; $x^2 + 1$为$\Phi_4(x)$, 根是$i = \zeta_{12}^3, -i = \zeta_{12}^9$; $x^2 + x + 1$为$\Phi_3(x)$, 根是$\zeta_{12}^4, \zeta_{12}^8$; $x^2 - x + 1$为$\Phi_6(x)$, 根是$\zeta_{12}^2, \zeta_{12}^{10}$. 故$x^4 - x^2 + 1$是$\zeta_{12}$的极小多项式, $[\mathbb{Q}[z]:\mathbb{Q}] = \deg(\Phi_{12}(x)) = 4$. 
        \item 见(2).
    \end{enumerate}
\end{proof}

\begin{remark}
    \begin{enumerate}[1.]
        \item 教材循环群的定义为由一个元素生成的(自由)群$\langle a \rangle = \{a^n \mid n \in \mathbb{Z}\}$, 等价的说就是和$\mathbb{Z}/n\mathbb{Z}$同构的群($n = 0$时为$\mathbb{Z}$本身). 对于$\mathbb{Z}/n\mathbb{Z}$有一个和初等数论有关的结论就是Euler函数$\phi(n) = |(\mathbb{Z}/n\mathbb{Z})^\times|$, 其中$(\mathbb{Z}/n\mathbb{Z})^\times$是单位群$U(\mathbb{Z}/n\mathbb{Z}) = \{\overline{m} \in \mathbb{Z}/n\mathbb{Z} \mid (m, n) = 1\}$. 即$\phi(n)$是$0$到$n - 1$中和$n$互素的元素个数. Fermat小定理的推广便是
        \[
            (a, n) = 1 \implies a^{\phi(n)} \equiv 1 \quad (\mathrm{mod}~n)
        \]
        且有恒等式
        \[
            n = \sum_{\substack{d \mid n\\d > 0}} \phi(d)
        \]
        \item 易见$U_n \cong \mathbb{Z}/n\mathbb{Z}$, $U_n$中的$1$对应$\mathbb{Z}/n\mathbb{Z}$中的$0$, $\zeta_n$对应$1$, 若$(k, n) = 1$, $k$就是生成元, 对应$U_n$中的$\zeta_n^k = e^{\frac{2k\pi i}{n}}$. $U_n$的生成元称为$n$次本原单位根. 分圆多项式$\Phi_n(x)$是$\zeta_n$的极小多项式, 事实上
        \[
            \Phi_n(x) = \prod_{0 \leqslant k < n, (n, k) = 1} (x - \zeta_n^k)
        \]
        且有恒等式
        \[
            x^n - 1 = \prod_{\substack{d \mid n\\d > 0}} \Phi_d(x)
        \]
        因此可以递归的计算出$\Phi_n(x)$,
        \[
            \Phi_n(x) = \frac{x^n - 1}{\displaystyle\prod_{\substack{d \mid n\\0 < d < n}} \Phi_d(x)} 
        \]
    \end{enumerate}    
\end{remark}

\begin{exercise}[习题3.1.7]
    设$E = K[u]$是一个代数扩张, 且$u$的极小多项式的次数是奇数. 证明: $E = K[u^2]$.
\end{exercise}

\begin{proof}
    由于$K[u^2] \subseteq K[u]$, 即$K[u^2]$是中间域, 设$\mu_u(x) \in K[x]$是$u$的极小多项式, 则
    \[
        \deg(\mu_u(x)) = \left[K[u]:K\right] = \left[K[u]:K[u^2]\right] \cdot \left[K[u^2]:K\right]
    \]
    是奇数. 而$u$是多项式$x^2 - u^2 \in K[u^2][x]$的根, 因此$\left[K[u]:K[u^2]\right] \leqslant 2$. 但由于奇数不可能有因子$2$, 故$\left[K[u]:K[u^2]\right] = 1$, 即$K[u^2] = K[u]$.
\end{proof}

\begin{exercise}[习题3.1.8]
    设$E_1, E_2$是域扩张$K \,\red{\subseteq}\, L$的中间域(即: $K \,\red{\subseteq}\, E_i \,\red{\subseteq}\, L)$, 且$[E_i:K] < +\infty$. 令$E = K\red{(}E_1,E_2\red{)} \,\red{\subseteq}\, L$是由$E_1, E_2$生成的子域. 证明:
    \[
        [E:K] \leqslant [E_1:K] \cdot [E_2:K].
    \]
\end{exercise}

\begin{remark}
    按正确的记号应该是用圆括号表示生成, 见3.1.2的注记.
\end{remark}

\begin{proof}
    设$\{\alpha_1, \alpha_2, \cdots, \alpha_n\}$是$E_1$的一组$K$-基, $\{\beta_1, \beta_2, \cdots, \beta_m\}$是$E_2$的一组$K$-基, 并要求$\alpha_1 = \beta_1 = 1$(总是可以乘上一个$\alpha_1^{-1}$或$\beta_1^{-1}$, 而这一组元素仍是基). 只需说明$S = \{\alpha_i\beta_j\}$可以生成$E = K(E_1, E_2) = K(E_1 \cup E_2)$. 按定义, 若$e_1 \in E_1, e_2 \in E_2$, $e_1 = \sum_{i = 1}^{n} k_i\alpha_i$, $e_2 = \sum_{j = 1}^{m} l_j\beta_j$. 由于取$\alpha_1 = \beta_1 = 1$, 因此$\alpha_i, \beta_j \in S$, 那么$e_1 \pm e_2, e_1e_2, e_i^{-1}$(若不为零)都可以由$S$生成.
\end{proof}

\begin{exercise}[习题3.1.9]
    设$K \,\red{\subseteq}\, L$是代数扩张, $E \,\red{\subseteq}\, L$是中间子环(即: $K \,\red{\subseteq}\, E \,\red{\subseteq}\, L)$. 证明: $E \,\red{\subseteq}\, L$必为子域(所以任何有限扩张$K \,\red{\subseteq}\, L$的中间子环必为域).
\end{exercise}

\begin{proof}
    按定义说明$E$中非零元可逆即可. 设$0 \neq u \in E \subseteq L$, 则$u$在$L$上代数, 那么$K(u) = K[u] \subseteq E$是域(包含关系由泛性质得到, 2.3.7), 则$u \in K[u]$可逆.
\end{proof}

\begin{exercise}[习题3.1.10]
    设$L = K(u)$, $u$是$K$上的超越元, $E \neq K$是$K \,\red{\subseteq}\, L$的中间域. 证明: $u$是$E$上的代数元.
\end{exercise}

\begin{proof}
    任取$v \in E \setminus K$, 那么按定义$v = \frac{p(u)}{q(u)}$, 则$p(u) - vq(u) = 0$, 即$u$是多项式$p(x) - vq(x) \in E[x]$的根.
\end{proof}

\begin{exercise}[习题3.1.11]
    设$p$是素数, $K \,\red{\subseteq}\, L$是$p$次扩张. 证明: $K \,\red{\subseteq}\, L$必为\red{单扩张}(即: 存在$u \in L$, 使$L = K[u]$).
\end{exercise}

\begin{remark}
    单扩张见3.1.2的注记, 由于3.3.14又写成单扩张了, 干脆把这里的“单纯”也改成“单”.
\end{remark}

\begin{proof}
    任取$u \in L \setminus K$, 和3.1.7的讨论类似,
    \[
        p = [L:K] = [L:K[u]] \cdot [K[u]:K]
    \]
    由$p$是素数, 且$[K[u]:K] \neq 1$(因为$u \notin K$), 可知$[L:K[u]] = 1, [k[u]:K] = p$, 即$L = K[u]$.
\end{proof}

\begin{exercise}[习题3.1.12]
    设域扩张$K \,\red{\subseteq}\, L$满足条件:
    \begin{enumerate}[(1)]
        \item $[L:K] < +\infty$;
        \item 对任意两个中间域$K \,\red{\subseteq}\, E_1 \,\red{\subseteq}\, L,\, K \,\red{\subseteq}\, E_2 \,\red{\subseteq}\, L$, 必有$E_1 \,\red{\subseteq}\, E_2$或者$E_2 \,\red{\subseteq}\, E_1$.
    \end{enumerate}
    证明: $K \,\red{\subseteq}\, L$必为\red{单扩张}(即: 存在$u \in L$,使$L = K[u])$.
\end{exercise}

\begin{proof}
    由3.1.2的注记, 有限扩张是有限生成代数扩张, 故存在代数元$u_1, u_2, \cdots, u_n$, $L = K[u_1, \cdots, u_n]$, 那么$K[u_i]$都是中间域, 若$K[u_i] \subseteq K[u_j]$, 则$K[u_i, u_j] = K[u_j][u_i] = K[u_j]$. 那么存在某个$u \in \{u_1, \cdots, u_n\}$使得$K[u] = \bigcup_{i = 1}^{n} K[u_i] = K[u_1, \cdots, u_n] = L$.
\end{proof}

\begin{exercise}[习题3.1.14]
    设$K = \mathbb{Q}[\sqrt[3]{3}]$, 证明: $x^5 - 5$在$K[x]$中不可约.
\end{exercise}

\begin{proof}
    只需说明$x^5 - 5$是$\sqrt[5]{5}$在$K$上的极小多项式. 我们证明$\left[\mathbb{Q}[\sqrt[3]{3}, \sqrt[5]{5}]:\mathbb{Q}[\sqrt[3]{3}]\right] = 5$即可.

    由Eisenstein判别法容易说明$x^3 - 3$和$x^5 - 5$在$\mathbb{Q}$上不可约, 从而有$\left[\mathbb{Q}[\sqrt[3]{3}]:\mathbb{Q}\right] = 3$, $\left[\mathbb{Q}[\sqrt[5]{5}]:\mathbb{Q}\right] = 5$. 由于
    \[
        \left[\mathbb{Q}[\sqrt[3]{3}, \sqrt[5]{5}]:\mathbb{Q}\right] = \left[\mathbb{Q}[\sqrt[3]{3}, \sqrt[5]{5}]:\mathbb{Q}[\sqrt[3]{3}]\right] \cdot \left[\mathbb{Q}[\sqrt[3]{3}]:\mathbb{Q}\right]
    \]
    那么只需说明$\left[\mathbb{Q}[\sqrt[3]{3}, \sqrt[5]{5}]:\mathbb{Q}\right] = 15$. 记$\alpha = \sqrt[3]{3}$, $\beta = \sqrt[5]{5}$. 则$\mathbb{Q}[\alpha]$的基为$\{1, \alpha, \alpha^2\}$, $\mathbb{Q}[\beta]$的基为$\{1, \beta, \beta^2, \beta^3, \beta^4\}$. 由3.1.8, 说明$\{\alpha^i\beta^j \mid 0 \leqslant i \leqslant 2, 0 \leqslant j \leqslant 4\}$是$\mathbb{Q}$-线性无关的即可, 这是比较容易看出来的(虽然很明显但是没想到优雅的证明先挖个坑).
\end{proof}

\begin{remark}
    3.1.4是直接证不可约得到扩张次数, 这里是用扩张次数来得到不可约.
\end{remark}

\begin{exercise}[习题3.1.15]
    设$k$是特征$p > 0$的域, $x, y$是$k$上的代数无关元. 令$K = k(x^{p}, y^{p})$, $L = k(x, y)$. 试证明$[L:K] = p^{2}$.
\end{exercise}

\begin{remark}
    代数无关是多元的超越, $a_1, a_2, \cdots, a_n$代数无关指不存在满足它们的代数方程, 即不存在多项式$f(x_1, x_2, \cdots, x_n) \in k[x_1, x_2, \cdots, x_n]$使得$f(a_1, a_2, \cdots, a_n) = 0$.
\end{remark}

\begin{proof}
    $x, y$代数无关, 按定义$x, y$就是超越元, 根据3.1.2的注记, $K$和$L$视为有理函数域处理即可. 考虑中间域$k(x, y^p)$, 由Eisenstein判别法, $x^p$是$k[x^p, y^p]$的不可约元, 则$t^p - x^p \in k[x^p, y^p][t]$是不可约多项式, 而$K$是$k[x^p, y^p]$的分式域, 从而在$K[t]$内也不可约(教材推论2.3.1). 那么$t^p - x^p$是$x \in k(x, y^p)$在$K$上的极小多项式. $[k(x, y^p):K] = \deg(t^p - x^p) = p$. 同理$[L:k(x, y^p)] = p$. 因此$[L:K] = [L:k(x, y^p)] \cdot [k(x, y^p):K] = p^2$.
\end{proof}

\subsection*{课上的补充内容}
基本上都在3.1.2里.

\end{document}