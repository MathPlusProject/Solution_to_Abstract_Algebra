\subsection{教材p87-p88}

\begin{problem}
    设$\mathrm{Aut}(X)$表示集合$X$的自同构群. 试证明:
\begin{enumerate}[(1)]
    \item 若$G \times X \to X,\, (g, x) \mapsto g \cdot x$,
是群$G$在$X$上的一个作用, $\forall g \in G$, 定义映射
$X \xrightarrow{\rho(g)} X$, $x \mapsto g \cdot x$.
则$\rho(g) \in \mathrm{Aut}(X)$且映射
\[
    \rho:G \to \mathrm{Aut}(X),\, g \mapsto \rho(g)
\]
是群同态.
    \item 若$\rho:G \to \mathrm{Aut}(X)$是一个群同态, 则映射
\[
    G \times X \to X,\: (g, x) \mapsto \rho(g)(x)
\]
是一个群作用.
\end{enumerate}
\end{problem}

\begin{solution}
    
\end{solution}

\begin{problem}
    $20$阶群中共有多少个$5$阶元?
\end{problem}

\begin{solution}
    
\end{solution}

\begin{problem}
    证明$15$阶的群一定是循环群.
\end{problem}

\begin{solution}
    
\end{solution}

\begin{problem}
    证明$6$阶非Abel群一定同构于$S_3$.
\end{problem}

\begin{solution}
    
\end{solution}

\begin{problem}
    证明$12$阶群共有$5$个同构类, 即$12$阶群本质上只有$5$个.
\end{problem}

\begin{solution}
    
\end{solution}

\begin{problem}
    设$p, q$是两个不同的素数, 则$pq$或$p^2q$阶群一定不是单群.
(事实上:$p^aq^b$阶群一定是可解群.)
\end{problem}

\begin{solution}
    
\end{solution}

\begin{problem}
    证明$200$阶群一定不是单群.
\end{problem}

\begin{solution}
    
\end{solution}

\begin{problem}
    设$H$为群$G$的有限子群.
\begin{enumerate}[(1)]
    \item 证明:$(h_1, h_2)(x) = h_2xh_1^{-1}$定义了$H \times H$在群$G$上的作用;
    \item 证明:$H$为$G$的正规子群当且仅当上述作用的每条轨道都恰有$|H|$个.
\end{enumerate}
\end{problem}

\begin{solution}
    
\end{solution}

\begin{problem}
    试证明若$|G| < 60$且$G$是一个单群, 那么$G$一定是素数阶的循环群.
\end{problem}

\begin{solution}
    
\end{solution}

\begin{problem}
    若$G$是$60$阶单群, 那么$G$一定同构于$A_5$,
从而得到阶数最小的非交换单群是$A_5$.
\end{problem}

\begin{solution}
    
\end{solution}