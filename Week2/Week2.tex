\documentclass{../solutions-cn}

\begin{document}
\section*{第二周作业参考解答及补充}

\subsection*{作业}

\begin{exercise}[习题1.4.1]
    设$\varphi:G\to G^\prime$是群同态, 试证明:
    \begin{enumerate}[(1)]
        \item $\ker(\varphi) \defeq \{g \in G \mid \varphi(g) = e'\}$ $(e' \in G'$表示的单位元)是$G$的子群(称为群同态$\varphi$的核);
        \item 
        \[
            \varphi(G) = \{\varphi(g) \mid \forall g \in G\} \,\red{\subseteq}\, G'
        \]
        是$G'$的子群(称为群同态$\varphi$的像).
    \end{enumerate}
\end{exercise}

\begin{proof}
    教材命题1.4.1的(1)(5)直接使用.
\begin{enumerate}[(1)]
    \item $e \in \ker(\varphi)$非空, 直接验证
    \[
    \begin{gathered}
        \forall a, b \in \ker(\varphi),\, \varphi(ab^{-1}) = \varphi(a)\varphi(b)^{-1} = e'e' = e'\\
        \implies ab^{-1} \in \ker(\varphi).
    \end{gathered}
    \]
    \item $e' \in \varphi(G)$非空, 直接验证
    \[
    \begin{gathered}
        \forall x, y \in \varphi(G),\, \exists a, b \in G,\, x = \varphi(a), y = \varphi(b)\\
        \implies xy^{-1} = \varphi(a)(\varphi(b))^{-1} = \varphi(a)\varphi(b^{-1}) = \varphi(ab^{-1}) \in \varphi(G).
    \end{gathered}
    \]
\end{enumerate}
\end{proof}

\begin{exercise}[习题1.4.3]
    设$R \overset{\varphi}\to R'$是环同态, 证明集合$ker(\varphi) = \{x \in R \mid \varphi(x) = 0_{R'}\}$满足:
    \begin{enumerate}[(1)]
        \item $\ker(\varphi)$是$(R, +)$的子群;
        \item $\forall a \in \ker (\varphi), x \in R$有$ax \in \ker(\varphi)$, $xa \in \ker (\varphi)$. ($\ker(\varphi)$称为环同态$\varphi$的核)
    \end{enumerate}
\end{exercise}

\begin{proof}
    \begin{enumerate}[(1)]
        \item 即习题1.4.1(1);
        \item 直接验证
        \[
            \forall a \in \ker(\varphi),\, x \in R,\, \varphi(xa) = \varphi(x)\varphi(a) = \varphi(x)0_{R'} = 0_{R'}
        \]
        另一半同理.
    \end{enumerate}
\end{proof}

\begin{exercise}[习题1.4.5]
    证明实数的加法群$(\mathbb{R}, +)$和正实数的乘法群$(\mathbb{R}_{>0}, \cdot)$同构.
\end{exercise}

\begin{proof}
    注意到$f: \mathbb{R} \to \mathbb{R}_{>0},\, x \mapsto e^x$是同构. $f^{-1}(x) = \ln x$.
\end{proof}

\begin{remark}
    事实上, 由$f(x + y) = f(x)f(y)$可以先直接推出$f(x) = a^x,\, a = f(1),\, x \in \mathbb{Q}$, 再根据连续性延拓到$\mathbb{R}$上.
\end{remark}

\begin{exercise}[习题1.4.6]
    证明有理数的加法群$(\mathbb{Q}, +)$和正有理数的乘法群$(\mathbb{Q}_{>0}, \cdot)$不同构.
\end{exercise}

\begin{proof}
    反证, 假设存在同构$f: \mathbb{Q} \to \mathbb{Q}_{>0}$, 则设$2 = f(a) = f(\frac{a}{2} + \frac{a}{2}) = f(\frac{a}{2}) \cdot f(\frac{a}{2}) = f(\frac{a}{2})^2$矛盾.
\end{proof}

\begin{exercise}[习题1.4.9]
    设$K, L$是两个域, 如果$L$是$K$的子域, 则$K$称为$L$的扩域, $K \supset L$称为域扩张, 试证明:
    \begin{enumerate}[(1)]
        \item 域的加法和乘法使得$K$是一个$L$-向量空间$([K:L] = \dim_L(K)$称为域扩张$K \supset L$ 的次数);
        \item 如果$K \supset \mathbb{R}$是一个二次扩张(即$[K:\mathbb{R}] = 2)$, 则$K$必同构于复数域$\mathbb{C}$.
    \end{enumerate}
\end{exercise}

\begin{proof}
    \begin{enumerate}[(1)]
        \item $(K, +)$是一个Abel群, 这一点无需再说明. 乘法在这里可能有些歧义, 此处是要验证乘法限制在$L \times K$上, 即
        \[
            \cdot: L \times K \to K, \quad (l, k) \mapsto lk
        \]
        是数乘. 即要验证
        \[
        \begin{gathered}
            (l_1l_2)k = l_1(l_2k),\\
            (l_1 + l_2)k = l_1k + l_2k,\\
            l(k_1 + k_2) = lk_1 + lk_2,\\
            1k = k = k1.
        \end{gathered}
        \]
        这些都由域的定义得到.
        
        这也说明若同态$K_1 \to K_2$保持$L$($K_1, K_2$为$L$的两个扩域), 则一定是$L$-线性映射.
        \item 由(1), 扩域$\mathbb{C}/\mathbb{R}$的自同构一定是$\mathbb{R}$-线性的. 设同构$f: \mathbb{C} \to \mathbb{C}$, 则有$f(x + yi) = x + yf(i),\, x, y \in \mathbb{R}$, 且保持乘法, 即
        \[
        \begin{gathered}
            f\left((x_1 + iy_1) \cdot (x_2 + iy_2)\right) = f(x_1 + iy_1) \cdot f(x_1 + iy_1)\\
            = (x_1 + y_1f(i)) \cdot (x_2 + y_2f(i))\\
            \implies f\left(x_1x_2 - y_1y_2 + (x_1y_2 +x_2y_1)i\right) \\= x_1x_2 + y_1y_2f(i) \cdot f(i) + (x_1y_2 +x_2y_1)f(i)\\
            \implies x_1x_2 - y_1y_2 + (x_1y_2 +x_2y_1)f(i)\\
            = x_1x_2 + y_1y_2f(i) \cdot f(i) + (x_1y_2 +x_2y_1)f(i)\\
            \implies f(i) \cdot f(i) = -1.
        \end{gathered}
        \]
        因此$f(i) = \pm i$. 也就是说$\mathbb{C}/\mathbb{R}$的自同构都只有恒等映射和共轭, 即$\mathrm{Gal}(\mathbb{C}/\mathbb{R}) = \mathbb{Z}/2\mathbb{Z}$.
        
        由线性代数的结论, 可以直接得到$K$和$\mathbb{C}$是作为线性空间同构, 但这是不够的, 只有上述两个线性映射是域同构, 需要做基变换转为恒等或共轭才能保持乘法. 事实上只要存在一个基变换就能变回恒等映射, 恒等映射总是同构, 但前提是承载集合(underlying set)要一样. 比如$\mathbb{Q}(\sqrt{2})$和$\mathbb{Q}(\sqrt{3})$作为$\mathbb{Q}$-线性空间也是同构的, 但他们之间没有域同态.
        
        可取$K$的一组基为$1, \alpha$, 其中$\alpha \in \mathbb{C} \setminus \mathbb{R}$. 不可避免地要考虑$\alpha^2$的结果, 由于$1, \alpha$是基, 因此$\alpha^2$可以被线性表出, 即$\alpha^2 = x + y\alpha$. 由于$\alpha \notin \mathbb{R}$, 有$y^2 + 4x < 0$, 解二次方程得到$\alpha = \frac{y \pm i\sqrt{|y^2 + 4x|}}{2}$. 故映射
        \[
            f: K \to \mathbb{C},\, u + v\alpha \mapsto u + v\frac{y \pm i\sqrt{|y^2 + 4x|}}{2}
        \]
        是域同构.
        
        注意$K$是域, 也就是说$K$的乘法是已知的, 无法把$K$先看作$\mathbb{R}$-线性空间然后重定义向量的乘法, 这在逻辑上是不对的.
    \end{enumerate}
\end{proof}

\begin{remark}
    事实上, 若有环同态$R \overset{\varphi}\to S$, 则$S$上自动有一个$R$-模结构
    \[
        R \times S \to S, \quad (r, s) \mapsto rs = \varphi(r)s
    \]
    $rs$是数乘, $\varphi(r)s$是$S$中的乘法. 域上的模就是线性空间.
    
    (1)对应的同态其实就是包含(inclusion)$L \overset{i}\hookrightarrow K$.
\end{remark}

\begin{exercise}[习题1.4.11]
    设$L \supset K$是一个域扩张, 证明: 下述集合
    \[
        \mathrm{Gal}(L/K) = \left\{L \xrightarrow{\sigma} L \mid \sigma\text{ 是域同构},\text{ 且 } \sigma(a) = a \text{ 对任意 } a \in K \text{ 成立}\right\}
    \]
    关于映射的合成是一个群(称为域扩张$L\supset K$的伽罗瓦群).
\end{exercise}

\begin{proof}
    $\mathrm{Gal}(L/K) \subseteq \mathrm{Aut}(L)$, 只需说明$\mathrm{Gal}(L/K)$是子群.

    $\forall \varphi, \psi \in \mathrm{Gal}(L/K)$, 由于$\psi|_K = \mathrm{id}_K$, 因此$\psi^{-1}|_K = \mathrm{id}_K$, 故$(\varphi \circ \psi^{-1})|_K = \mathrm{id}_K$, 即$\varphi \circ \psi^{-1} \in \mathrm{Gal}(L/K)$.
\end{proof}

\subsection*{课上的补充内容}
好像没有....
\end{document}