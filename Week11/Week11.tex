\documentclass{../solutions-cn}

\begin{document}
\section*{第十一周作业参考解答及补充}

\subsection*{作业}

\begin{exercise}[习题3.3.6]
    设$L$是可分多项式$f(x) \in K[x]$的一个分裂域, $K \,\red{\subseteq}\, E \,\red{\subseteq}\, L$是任意中间域. 证明: 对任意单同态$\varphi:E \to L$,若$\varphi|_K = \mathrm{id}_K$, 则$\varphi$一定可以延拓成域同构$\overline\varphi:L \to L$.
\end{exercise}

\begin{proof}
    按分裂域定义, 存在$\alpha_1, \alpha_2, \cdots, \alpha_m \in L$使得$f(x) = c(x - \alpha_1)(x - \alpha_2) \cdots (x - \alpha_m)$, 且$L = K[\alpha_1, \cdots, \alpha_m]$. 由于$K \subseteq E \subseteq L$, $f(x) \in K[x] \subseteq E[x]$, 将$f(x)$视为$E[x]$中的多项式. $f(x)$仍有这样的分解. 且$L = K[\alpha_1, \cdots, \alpha_m] \subseteq E[\alpha_1, \cdots, \alpha_m] \subseteq L$, 故$L$也是$f(x)$在$E$上的分裂域. 由于$\varphi|_K = \mathrm{id}_K$, 因此$\varphi(f(x)) = f(x)$, 那么$L$也是$f(x)$在$\varphi(E)$上的分裂域. 由教材的定理3.3.2, 由于$f(x)$是可分多项式, 因此$f(x)$在$\varphi(E)$中无重根, 有$|\{\text{域同构 } L \xrightarrow{\psi} L \mid \psi|_E = \varphi\}| = [L:E] > 0$, 即该集合非空, 那么存在同构$\psi:L \to L$使得$\psi|_E = \varphi$.
\end{proof}

\begin{remark}
    接3.3.2注记, 这种延拓对代数扩张是都成立的. 

    \begin{propstar}
        已知对任意域$K$存在代数闭域$L$使得$K \subseteq L$. 记$i_K:K \to L$是域嵌入, 那么对任意代数扩张$K \subseteq E$, 存在嵌入$i:E \to L$使得$i|_K = i_K$. 若$E$是代数闭域且$L/i_K(K)$是代数的, 那么$i$是同构. 因此代数闭包在同构的意义下一定唯一. 证明需要Zorn's Lemma. Serge Lang《Algebra》\S V.2 Theorem 2.8. 
    \end{propstar}
    另外也可以参考《近世代数引论》p136 引理1, 这个是单代数扩张的版本, 相对简单一些, 如果只考虑有限扩张, 那么用这个版本就够了. 事实上这只是3.1.2的进一步解释, 在此基础上加了一个同构让他变成如下的交换图
    \[
        \begin{tikzcd}
            & E                                                              & E'                                          &                                               \\
{K[x]/(\mu_\alpha(x))} \arrow[r, "\sim"] & K(\alpha) \arrow[r, "\varphi"] \arrow[u, "\subseteq", phantom, sloped] & K'(\alpha') \arrow[u, "\subseteq", phantom, sloped] & {K'[x]/(\mu_{\alpha'}(x))} \arrow[l, "\sim"'] \\
            & K \arrow[r, "\eta"] \arrow[u, hook]                            & K' \arrow[u, hook]                          &                                              
        \end{tikzcd}
    \]
    其中$\eta$是域同构, 那么根据教材引理2.3.2, $\eta$可以延拓成同构$\tilde{\eta}:K[x] \xrightarrow{\sim} K'[x]$. 这个同态会把$\alpha$的极小多项式映到$\alpha'$的极小多项式. 这样就有
    \[
        \begin{tikzcd}
            & {K[x]} \arrow[d, "\pi", two heads] \arrow[r, "\tilde{\eta}"]       & {K'[x]} \arrow[d, "\pi'", two heads]                 &            \\
K(\alpha) \arrow[r, "\sim"] \arrow[rrr, "\varphi"', bend right] & {K[x]/(\mu_\alpha(x))} \arrow[r, "\overline{\tilde{\eta}}"] & {K'[x]/(\mu_{\alpha'}(x))} \arrow[r, "\sim"] & K'(\alpha')
        \end{tikzcd}
    \]
    注意到$(\mu_\alpha(x)) \subseteq \ker(\pi' \circ \tilde{\eta})$, 由quotient的泛性质(也就是同态基本定理的推广, 2.1.8增加的命题)得到$\overline{\tilde{\eta}}$, 而这是域之间的满同态, 故只能是同构, 进而得到同构$\varphi$, 且$\varphi|_K = \eta$. 注意根据我们$\varphi$的构造一定是$\varphi(\alpha) = \alpha'$, 若这里$\eta = \mathrm{id}_K$, 那么$\varphi$就是把$\alpha$换成$\mu_\alpha(x)$的其中一个根.

    这也说明我们在同构的意义下考虑域扩张是可行的. 但对代数闭包而言, 不一定是有限扩张, 比如$\overline{\mathbb{Q}}/\mathbb{Q}$.

    有了代数闭包, 可分多项式的等价定义为: $f(x) \in K[x]$是可分的, 即$f(x)$的不可约因子在$\overline{K}$中(或者说在$f(x)$的分裂域中)无重根. 这也解释了2.4.3和3.1.1的关系, 特征零的不可约多项式可分, 从而特征零的代数扩张一定是可分扩张.
\end{remark}

\begin{exercise}[习题3.3.8]
    设$p \in \mathbb{Z}$是一个素数, $F$是一个域, $c \in F$. 求证: $x^p - c$在$F[x]$中不可约当且仅当$x^p - c$在$F$中无根.
\end{exercise}

\begin{proof}
    考虑$x^p - c$的分裂域$E$, 或者直接考虑$F$的代数闭包, 那么有分解$x^p - c = (x - \alpha_1)(x - \alpha_2) \cdots (x - \alpha_p)$. 我们证两次逆否.
    \begin{enumerate}
        \item["$\implies$"] 若$x^p - c$在$F$中有根, 根据教材定义2.4.2, $x^p - c$有一次因式, 可约.
        \item["$\impliedby$"] 若$x^p - c$可约, 按定义有$x^p - c = f(x)g(x)$, 那么不妨设$f(x) = (x - \alpha_1)(x - \alpha_2) \cdots (x - \alpha_n)$, 其中$0 < n < p$, 那么根据Bézout's Identity, 存在$u, v \in \mathbb{Z}$使得$nu + pv = 1$. 记$\alpha = \alpha_1\alpha_2 \cdots \alpha_n \in F$(韦达定理), 注意到$\alpha_i$都是$x^p - c$的根, $\alpha_i^p = c$. 那么$\alpha^p = \alpha_1^p\alpha_2^p \cdots \alpha_n^p = c^n$, 从而$\alpha^{pu} = c^{nu}$, 那么$(\alpha^uc^v)^p = c^{nu}c^{pv} = c$. 这样$\alpha^uc^v$是$x^p - c$的一个根, 且$\alpha \in F$, 因此$\alpha^uc^v \in F$.
    \end{enumerate}
\end{proof}

\begin{exercise}[习题3.3.11]
    证明: $\mathbb{Q}[\sqrt[4]{2}]$是$\mathbb{Q}[\sqrt{2}]$的正规扩张, 但不是$\mathbb{Q}$的正规扩张.
\end{exercise}

\begin{proof}
    由3.3.1, $\mathbb{Q}[\sqrt[4]{2}]/\mathbb{Q}[\sqrt{2}]$是二次扩张, 从而是正规扩张. 另一方面, 和3.3.2类似, $\sqrt[4]{2}$在$\mathbb{Q}$上的的极小多项式$x^4 - 2$有非实数根$\sqrt[4]{2}i$的存在, 自然不是正规扩张.
\end{proof}

\begin{exercise}[习题3.3.13]
    设$L = K[\alpha],~\alpha$是多项式$x^d - a \in K[x]$的根. 如果$\mathrm{Char}(K) = 0$, 且$K$包含全部$d$次单位根, 则$K \,\red{\subseteq}\, L$是正规扩张.
\end{exercise}

\begin{proof}
    这是3.3.4的一般情况. 设$1 = \omega_0, \omega_1, \cdots \omega_{d - 1}$是$x^d - 1$的根, 根据题设, $\omega_i \in L = K[\alpha], 0 \leqslant i < d$. 而$(\omega_i\alpha)^d = \omega^d\alpha^d = 1 \cdot a = a$, 从而$\omega_i\alpha \in L$是$x^d - a$的$d$个根. 因此按定义$L$是$x^d - a$的分裂域. 由3.3.14的注记是Galois扩张, 自然是正规扩张.
\end{proof}

\begin{exercise}[习题3.3.14*]
    设$k$是特征$p > 0$的域, $x, y$是$k$上的代数无关元. 令$K = k(x^p, y^p)$, $L = k(x, y)$. 试证明: 
    \begin{enumerate}[(1)]
        \item $\mathrm{Gal}(L/K) = \{1\}$ (但$[L:K] = p^2)$;
        \item $K \,\red{\subseteq}\, L$有无穷多个中间域;
        \item $K \,\red{\subseteq}\, L$不是单扩张, 即不存在$\alpha \in L$使得$L = K[\alpha]$.
    \end{enumerate}
\end{exercise}

\begin{proof}
    这题是3.1.15的延续.
    \begin{enumerate}[(1)]
        \item 设$\eta \in \mathrm{Gal}(L/K)$, 只需证$\eta = \mathrm{id}_L$. 由3.3.6, $x$是$K$上的代数元, 且$x$的极小多项式是$t^p - x^p$, 因此$\eta(x)$是多项式$t^p - x^p$的根, 而根据3.1.15, $t^p - x^p$只有一个$p$重根$x$, 因此$\eta(x) = x$. 同理$\eta(y) = y$, 从而$\eta = \mathrm{id}_L$.
        \item 设$E$是一个非平凡中间域, 由于$[L:K] = [L:E][E:K] = p^2$, 因此只能是$[L:E] = [E:K] = p$. 而形如$E_c = k(x + cy, y^p)$就是非平凡的中间域, 其中$c \in K$而$K$是无穷域. 且$c_1 \neq c_2 \implies E_{c_1} \neq E_{c_2}$. 因此有无穷多个中间域.
        \item 由Frobenius同态可知, $\forall \alpha \in L$, $\alpha^p \in K$, 则$t^p - \alpha^p \in K[t]$是$\alpha$的化零多项式. 从而$[K[\alpha]:K] \leqslant p < p^2$. 因此$K[\alpha] \neq L$.
    \end{enumerate}
\end{proof}

\begin{remark}
    这题教材答案的错误比较严重, $K$并不是完全域, $x^p$不是$K$中任何一个元素的$p$次方, 但$L/K$确实不是一个可分扩张, $x$在$K$上的极小多项式$t^p - x^p$在$K$上不是可分的. 事实上教材的定理3.3.4已经告诉我们完全域的代数扩张一定是可分扩张, 而$L/K$是有限扩张, 自然是代数扩张, 因此教材的答案是前后矛盾的.

    事实上, 完全域应该定义为任意代数扩张都是可分扩张的域, 因此包括所有特征零的域. 在完全域上取分裂域得到的扩张一定是Galois扩张(教材定理3.4.1), 这也是为什么要有完全域这个概念.
\end{remark}

\subsection*{课上的补充内容}
    Serge Lang 《Algebra》\S V.3 Theorem3.3给出了正规扩张的三个等价定义, 其中有一条是用到代数闭包的:

    $L/K$是正规扩张, 当且仅当延拓后的域嵌入$\varphi:L \to \overline{K}$是$L$的自同构, 即$\varphi(L) = L$.
\end{document}