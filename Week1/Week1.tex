\documentclass[UTF8,fontset=windows]{ctexart}
\usepackage{amssymb,mathtools,mathrsfs,tikz-cd}
\usepackage{enumerate,tcolorbox,xcolor}
\usepackage[top=25mm, bottom=20mm, left=30mm, right=30mm, a4paper]{geometry}
\tcbuselibrary{breakable}
\everymath{\displaystyle}

\newcommand{\defeq}{\mathrel{\coloneqq}}
\newcommand{\eqdef}{\mathrel{\eqqcolon}}
\newcommand{\defequiv}{\mathrel{\vcentcolon\Leftrightarrow}}

\newcounter{problem}
\newenvironment{problem}[1][]
{
    \refstepcounter{problem}
    \noindent\textbf{\theproblem.}
    \ifx\relax#1\relax
    \else
        (#1)
    \fi
    \par\vspace{0.5em}
}
{\vspace{1em}}

\newcounter{additional}
\newenvironment{additional}[1][]
{
    \refstepcounter{additional}
    \noindent\textbf{\theadditional.}
    \ifx\relax#1\relax
    \else
        (#1)
    \fi
    \par\vspace{0.5em}}
{\vspace{1em}}

\newenvironment{solution}
{\begin{tcolorbox}[colback=blue!10, colframe=blue!50, title=\textit{proof}, breakable]}
{\end{tcolorbox}}

\begin{document}
\section*{第一周作业参考解答及补充}

\subsection*{作业}

\begin{problem}[习题1.1.6]
    设$p > 2$是素数,
$\mathbb{F}_p = \{\bar{0}, \bar{1}, \bar{2}, \cdots, \overline{p-1}\}$
是$\mathbb{Z}$的模$p$剩余类域. 试计算:
\begin{enumerate}[(1)]
    \item $\bar{2}$在$\mathbb{F}_p$中的逆元$\bar{2}^{-1}$;
    \item $\overline{p - 1} \cdot \overline{p - 2}$;
    \item $\overline{p - 2}$在$\mathbb{F}_p$中的逆元$\overline{p-2}^{-1}$.
\end{enumerate}
\end{problem}

\begin{solution}
\begin{enumerate}[(1)]
    \item 只需找到能被$2$整除的$1 + kp(k \in \mathbb{Z})$.
由于素数$p > 2$, $p + 1$即可. i.e. $\overline{2}^{-1} = \overline{\frac12(p + 1)}$.
    \item $\overline{p - 1} \cdot \overline{p - 2} = \overline{-1} \cdot \overline{-2} = \overline{2}$.
    \item 由(1), $\overline{p - 2}^{-1} = \overline{-2}^{-1} = \overline{-\frac12(p + 1)} = \overline{\frac12(p - 1)}$.
\end{enumerate}
\end{solution}

\begin{problem}[习题1.2.1]
    设$R$是一个环, 试证明下述结论:
\begin{enumerate}[(1)]
    \item (加法消去律)\quad 如果$a + c = b + c$, 则$a = b$;
    \item $\forall a\in R$, 有$a \cdot 0_R = 0_R$;
    \item $-(-a) = a,\quad a(b - c) = ab - ac \quad (\forall a, b, c \in R)$;
    \item $-(a + b) = (-a) + (-b) \quad (\forall a, b \in R)$;
    \item $a(-b) = (-a)b = -(ab) \quad (\forall a, b \in R)$;
    \item $(-a)(-b) = ab \quad (\forall a, b \in R)$;
    \item $\forall a \in R, m, n \in \mathbb{Z}$, 有$(m + n)a = ma + na, (mn)a = m(na)$;
    \item $\forall a, b \in R, n \in \mathbb{Z}$, 有$n(a + b) = na + nb, n(ab) = a(nb)$;
    \item $\forall a, b \in R, m, n \in \mathbb{Z}$, 有$(ma) \cdot (nb) = mn(a \cdot b) = (mna) \cdot b$;
    \item (二项式定理)\quad $\forall a, b \in R$,设$ab = ba$, $n$是正整数, 则
    \[
        (a+b)^n = \sum_{i = 0}^n \binom{n}{i} a^{n - i}b^i.
    \]
\end{enumerate}
\end{problem}

\begin{solution}
    \begin{enumerate}[(1)]
        \item 两边同加$-c$.
        \item 由于
        \[
            a \cdot 0_R = a \cdot (0_R + 0_R) = a \cdot 0_R + a \cdot 0_R.
        \]
        再用一下负元消去即可, $0_R \cdot a = 0_R$同理.
        \item 前一个为负元定义(教材p9的注记); 后一个先由分配律, 
    \[
        a(b - c) = ab + a(-c),
    \]
    又由于
    \[
        a(-c) + ac = a(c + (-c)) = a \cdot 0_R \overset{(2)}= 0
    \]
    得$a(-c) = -ac$, 这也是(5)的证明.
    这里要注意仅使用$-a \overset{(*)}= -1_R \cdot a$也无法将负号提到前面, 需要$R$是交换环
    或者说明$-1_R \cdot a = a \cdot (-1_R) = -a$.
    
    $(*)$的证明如下
    \[
        -1_R \cdot a + a = -1_R \cdot a + 1_R \cdot a = (-1_R + 1_R) \cdot a = 0_R \cdot a \overset{(2)}= 0_R.
    \]
    右乘$-1_R$同理.
        \item 利用$-a = -1_R \cdot a$和分配律展开即可.
        \item 见(3).
        \item (3)和(5)的推论.
        \item (7)-(9)和习题1.1.1的(6)类似, 首先需要明确定义,
        教材在这里并没有强调递归定义, 事实上, 这种和$\mathbb{Z}$
        有关的东西都应该由递归定义给出, 相对应的证明要用归纳法.
        (就算不用归纳法, 至少要把正负整数分开证, 很多同学直接用一行证明, 这是不行的)

        严格来说, 这是定义了一个映射
        \[
            \mathbb{Z} \times R \to R, (n, a) \mapsto na,
        \]
        其中
        \[
            0a \defeq 0_R,\, (n + 1)a \defeq na + a,\, n \in \mathbb{N},
        \]
        以及
        \[
            na \defeq -((-n)a),\, n < 0.
        \]
        (注意$na$不是$R$上的乘法, 有同学甚至写了$\mathbb{Z} \subseteq R$然后直接乘法分配律,
        $R$中是否有整数是不知道的)

        由该定义可以验证对任意整数$n \in \mathbb{Z}$均有$(n + 1)a = na + a$
        以及$na = -((-n)a)$, 这样在使用这两个等式的时候不用再区分正负了.

        回到原题, 对任意的$m \in \mathbb{Z}$, 先用归纳法证明$n \in \mathbb{N}$的情形, 
        负整数的情形可以结合定义得到.

        $n = 0$根据定义左右均为$ma$, 假设对$n$有$(m + n)a = ma + na$, 根据定义有
        \[
            (m + n + 1)a = (m + n)a + a = ma + na + a = ma + (n + 1)a.
        \]
        由归纳法知
        \begin{equation}
            (m + n)a = ma + na, \quad \forall m \in \mathbb{Z}, n \in \mathbb{N}
            \tag{i}
            \label{eq:1.2.1.7}
        \end{equation}
    当$n < 0$时, 存在$k \in \mathbb{Z}_{>0}$使得$m + kn < 0$,
    \[
    \begin{aligned}
        (m + n)a &= (m + kn - (k - 1)n)a\\
        &\overset{\eqref{eq:1.2.1.7}}= (m + kn)a + (-(k - 1)n)a\\
        &= -(-m - kn)a + (n - kn)a\\
        &\overset{\eqref{eq:1.2.1.7}}= -((-m)a + (-kn)a) + na + (-kn)a\\
        &\overset{(4)}= ma + (kn)a + na + (-kn)a = ma + na.
    \end{aligned}   
    \]
        第二个式子可直接利用第一个证明, 
    $m = 0$根据定义左右均为$0_R$, $m > 0$有, 
    \[
    \begin{aligned}
        (mn)a &= \left(\sum_{i = 1}^{m} n\right)a\\
        &= \sum_{i = 1}^{m} (na)\\
        &= m(na).
    \end{aligned}
    \]
    $m < 0$利用$mn = (-m)(-n)$, 做同样的操作.
        \item 对$n$归纳, 由于加法有交换律,
    \[
    \begin{aligned}
        (n + 1)(a + b) &= n(a + b) + a + b\\
        &= na + nb + a + b\\
        &= (n + 1)a + (n + 1)b.
    \end{aligned}
    \]
    得
    \[
        n(a + b) = na + nb, \quad \forall n \in \mathbb{N}
    \]
    当$n < 0$有
    \[
        n(a + b) = -(-n(a + b)) = -((-n)a + (-n)b) \overset{(4)}= na + nb.
    \]
    第二个等式使用分配律, $n = 0$根据定义左右均为$0_R$, $n > 0$,
    \[
        n(ab) = \sum_{i = 1}^{n} ab = a\sum_{i = 1}^{n} b = a(nb).
    \]
    $n < 0$, 用$n = -(-n)$, $n(ab) = -a((-n))b \overset{(5)}= a(nb)$.
    同样的也会有$n(ab) = (na)b$.
        \item (7)和(8)的推论, 
    \[
    \begin{aligned}
        (ma) \cdot (nb) &\overset{(8)}= m(a \cdot (nb))\\
        &\overset{(8)}= m(n(ab))\\
        &\overset{(7)}= mn(ab)\\
        &\overset{(8)}= (mna) \cdot b.
    \end{aligned}
    \]
        \item 对$n$归纳,
    \[
    \begin{aligned}
        (a + b)^n \cdot (a + b) &= \left(\sum_{i = 0}^{n} \binom{n}{i} a^{n - i}b^i\right) \cdot (a + b)\\
        &= \sum_{i = 0}^{n} \binom{n}{i} a^{n - i}b^ia + \sum_{i = 0}^{n} \binom{n}{i} a^{n - i}b^{i + 1}\\
        &\overset{ab = ba}= \sum_{i = 0}^{n} \binom{n}{i} a^{n - i + 1}b^i + \sum_{i = 0}^{n} \binom{n}{i} a^{n - i}b^{i + 1}\\
        &= a^{n + 1} + \sum_{i = 1}^{n} \left(\binom{n}{i} + \binom{n}{i - 1}\right) a^{n - i + 1}b^{i} + b^{n + 1}\\
        &= \sum_{i = 0}^{n + 1} \binom{n + 1}{i} a^{n + 1 - i}b^i.
    \end{aligned}
    \]
    \end{enumerate}
(7)-(9)中实际上需要用归纳法证明的只有
\[
    \begin{aligned}
        n(a + b) &= na + nb,\\
        (m + n)a &= ma + na,\\
        (mn)a &= m(na),\\
    \end{aligned}
\]
这三条加上$1a = a$, 是在说任何一个Abel群都是$\mathbb{Z}$-模(见教材5.1节).
\end{solution}

\begin{problem}[习题1.2.9]
    设$m > 0$是任意整数, $\mathbb{Z}_m = \{\bar{0}, \bar{1}, \cdots, \overline{m-1}\}$
是$\mathbb{Z}$的模$m$剩余类环. 试证明:$\bar{a} \in \mathbb{Z}_m$可逆当且仅当$(a, m) = 1$
(即:$a$与$m$互素).
\end{problem}

\begin{solution}
    $\overline{a} \in \mathbb{Z}_m$可逆, 
    \[
    \begin{aligned}
        &\iff \exists \overline{b} \in \mathbb{Z}, \quad \overline{a}\overline{b} = \overline{1}\\
        &\iff ab = 1 + km, \quad k \in \mathbb{Z},\\
        &\iff (a, m) = 1. \quad \text{(Bezout's Theorem)}
    \end{aligned}
    \]
    注:一般用记号$\mathbb{Z}/m\mathbb{Z}$表示模$m$剩余类环.(理想和商环, 教材2.1节p25)
    
    若$(a, m) = 1$, 则$\overline{a}$是加法群$(\mathbb{Z}/m\mathbb{Z}, +)$
    的生成元, 即$\overline{a}$(在加法群)的阶是$m$.    
\end{solution}

\begin{problem}[习题1.2.10]
    设$R$是仅有$n$个元素的环, 试证明对任意$a \in R$有
\[
    na \defeq \underbrace{a + a + \cdots + a}_n = 0.
\]
\end{problem}

\begin{solution}
    该题的证明归结为一句话:加法群的阶$(R, +)$为$n$, 故$na = 0$.

注:有限群$G$内任一元素$a$, 有$|a| \Big| |G|$(Lagrange定理, 教材4.1节p70推论4.1.3),
因此必有$a^{|G|} = e$, 在这道题就是$na = 0$.

有些同学没有说明使用了Lagrange定理, 也没有证明这些子集
$b + \langle a \rangle = \{b + ma \mid m \in \mathbb{Z}\},\, b \in R$
(这里的$\langle a \rangle$是对于加法群$(R, +)$而言, 由$a$生成的子群, $\langle a \rangle = \{ma \mid m \in \mathbb{Z}\}$)
确实构成了$R$的一个分划(partition), 而是直接使用/推出这些结论,
我觉得这样的证明是不完整的. 至少要说明一下
\[
    b + \langle a \rangle \cap b' + \langle a \rangle \neq \varnothing
    \implies b + \langle a \rangle = b' + \langle a \rangle
\]
这个证明不难, 由交集非空
\[
\begin{gathered}
    \exists x \in b + \langle a \rangle \cap b' + \langle a \rangle\\
    \implies \exists m_1, m_2 \in \mathbb{Z},\, x = b + m_1a = b' + m_2a\\
    \implies b = b' + (m_2 - m_1)a \in b' + \langle a \rangle\\
    \implies \forall z \in b + \langle a \rangle,\, z = b + ma = b' + (m + m_2 - m_1)a \in b' + \langle a \rangle\\
    \implies b + \langle a \rangle \subseteq b' + \langle a \rangle
\end{gathered} 
\]
同理$b' + \langle a \rangle \subseteq b + \langle a \rangle$,
因此$b + \langle a \rangle = b' + \langle a \rangle$.

另外, "设$r$是使得$ra = 0_R$的最小正整数"是需要说明的(这个$r$就是$a$的阶).

先要说明$\exists k \in \mathbb{Z}_{>0}$使得$ka = 0_r$, 也就是说$|a| < \infty$.
用反证法, 假设这样的正整数不存在, 则
\[
    \langle a \rangle = \{ma \mid m \in \mathbb{Z}\} 
\]
中必两两互不相等(否则不妨设$i < j$使得$ia = ja$, 即$(j - i)a = 0_R$, 矛盾).

而$R$中只有$n$个元素, 这样就已经矛盾了. 因此存在$k \in \mathbb{N}$使得$ka = 0_R$.
也就是说$\{m \in \mathbb{N} \mid ma = 0_R\} \subseteq \mathbb{N}$非空.
由$\mathbb{N}$的良序性, 存在一个最小的$r$使得$ra = 0_R$.
这样$\langle a \rangle = \{0, a, 2a, \cdots, (r - 1)a\}$恰有$r$个元素
(如果还不放心这里面是否有相等元素, 在用一次反证就可以了, 和上面证两两不等类似).
\end{solution}

\begin{problem}[习题1.3.2]
    设$R$是一个环, $U(R)$表示$R$中所有可逆元集合, 试证明:$U(R)$关
于环$R$的乘法是一个群(称为$R$的单位群).
\end{problem}

\begin{solution}
    \begin{enumerate}[(1)]
        \item 这里首先需要验证运算的封闭性, $\forall a, b \in U(R)$, 有
    $(b^{-1}a^{-1})(ab) = b^{-1}(a^{-1}a)b = 1$, 故$ab \in U(R)$且
    $(ab)^{-1} = b^{-1}a^{-1}$.

        很多同学漏了这一条, 这里的乘法是$R$上的乘法限制在$U(R)$上, 即
        \[
            \cdot: U(R) \times U(R) \to R,\, (a, b) \mapsto ab \in R,
        \]
        \item $1 \in U(R)$, 因为$1 \cdot 1 = 1$的确可逆;
        \item 由于乘法是$R$上的乘法, 故结合律成立;
        \item 若$a \in U(R)$, 则由习题1.1.1的(3), $a^{-1} \in U(R)$且$(a^{-1})^{-1} = a$;
    \end{enumerate}
\end{solution}

\begin{problem}[习题1.3.5]
    写出对称群$S_3$的乘法表.
\end{problem}

\begin{solution}
    记$\mathrm{id}_{S_3} = e$, 令$a = (1\:2)$, $b = (1\:2\:3)$, 有
$a^2 = e$, $b^3 = e$, $abab = e \iff ba = ab^2$. 乘法表如下:
\[
\begin{array}{c|cccccc}
  & e   & a   & b   & b^2 & ab  & ab^2 \\
\hline
e  & e   & a   & b   & b^2 & ab  & ab^2 \\
a  & a   & e   & ab  & ab^2 & b^2 & b \\
b  & b   & ab^2 & b^2 & e   & a   & ab \\
b^2 & b^2 & ab  & e   & b   & ab^2 & a \\
ab & ab  & b^2 & a   & ab^2 & e   & b \\
ab^2 & ab^2 & b   & ab  & a   & b^2 & e \\
\end{array}
\]
注: 可以看到$S_3$, 若取$a = (1\:2),\, b = (1\:2\:3)$,
则$S_3$可以由$a, b$生成, 即考虑所有可能的乘积, 一般可以表示
为$S_3 = \langle a, b \rangle,\, a = (1\:2),\, b = (1\:2\:3)$.

若不给$a, b$加任何限制, 便得到一个自由群(free group)$F(\{a, b\})$.
一般地, 任意一个集合$A$都可以生成一个自由群$F(A)$, $A$就是生成元组成
的集合. 可以证明任何一个群都同构于某个自由群的商群, 而对应的正规子群便是由
生成元满足的某些关系确定(将$A$看成字母表, $\Sigma_A$表示单词的集合,
这些关系可以表示为一些满足$w = e$单词$w \in \Sigma_A$).
把这些$w$组成的集合记为$\mathscr{R}$,
$A$和$\mathscr{R}$将唯一确定一个群$G$, $(A \mid \mathscr{R})$
称为$G$的一个展示(presentation). 以$S_3$为例, $S_3$的一个展示
为$(\{a, b\} \mid a^2, b^3, abab)$.

由于这本教材没有讲自由群, 所以想要了解的话需要查阅别的教材.
\end{solution}

\begin{problem}[习题1.3.11]
    证明:$GL_2(\mathbb{R})$中的元素
\(
x=\begin{pmatrix}
        0 & 1\\
        -1 & 0
\end{pmatrix},
y=\begin{pmatrix}
        0 & 1\\
        -1 & -1
\end{pmatrix}
\)
的阶分别是$4$和$3$. 但$xy$是无限阶元.
\end{problem}

\begin{solution}
    用$I_n$表示$n$阶单位阵, 计算可得
\[
x^2 = \begin{pmatrix} 
    -1 & 0 \\
    0 & -1
\end{pmatrix},
x^3 = \begin{pmatrix}
    0 & -1 \\
    1 & 0
\end{pmatrix},
x^4 = \begin{pmatrix}
    1 & 0 \\
    0 & 1
\end{pmatrix} = I_2.
\]
故$|x| = 4$, 同理,
\[
y^2 = \begin{pmatrix}
    -1 & -1 \\
    1 & 0
\end{pmatrix},
y^3 = \begin{pmatrix}
    1 & 0 \\
    0 & 1
\end{pmatrix} = I_2.
\]
$|y| = 3$.
最后是$xy$,
\[
xy = \begin{pmatrix}
    -1 & -1 \\
    0 & -1
\end{pmatrix},
(xy)^2 = \begin{pmatrix}
    1 & 2 \\
    0 & 1
\end{pmatrix},
(xy)^3 = \begin{pmatrix}
    -1 & -3 \\
    0 & -1
\end{pmatrix}, \cdots
\]
可以用归纳法证明
\[
    (xy)^n = (-1)^n\begin{pmatrix}
        1 & n \\
        0 & 1
    \end{pmatrix} \neq I_2, \forall n \in \mathbb{Z}_{\geqslant 1}.
\]
故$|xy| = \infty$.
\end{solution}

\begin{problem}[习题1.3.12]
    证明群的任意多个子群的交仍是子群.
\end{problem}

\begin{solution}
    设$G$是群, 记$I$为指标集, $H_i < G,\, \forall \in I$.
验证\(H = \bigcap_{i \in I} H_i < G\):
\[
\begin{gathered}
    \forall a, b \in H = \bigcap_{i \in I} H_i \implies \forall i \in I,\,a, b \in H_i\\
    \implies ab^{-1} \in H_i, \quad \forall i \in I\\
    \implies ab^{-1} \in \bigcap_{i \in I} H_i = H.
\end{gathered}
\]
注:很多同学认为“任意多”是有限个, 即只考虑\(H = \bigcap_{i = 1}^{n} H_i\),
也有同学考虑了\(H = \bigcap_{i = 1}^{\infty} H_i\), 这也是不够的, 这里是允许
不可数无穷的.
\end{solution}

\subsection*{课上的补充内容}

\begin{additional}[子群的判定]
    设$G$是一个群, $\varnothing \neq S \subseteq G$, 则
$S < G$($S$是$G$的子群的记号)当且仅当
\[
    \forall a, b \in S \iff ab^{-1} \in S.
\]
\end{additional}

\begin{additional}[Bezout's Theorem]
    对$m, n \in \mathbb{Z}$,
\[
    (m, n) = 1 \iff \exists u, v \in \mathbb{Z} \quad mu + nv = 1.
\]
\end{additional}
\end{document}
