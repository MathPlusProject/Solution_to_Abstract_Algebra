\documentclass{../solutions-cn}

\begin{document}
\section*{第十五周作业参考解答及补充}

\subsection*{作业}

\begin{exercise}[习题4.3.1]
    设$G = \langle \alpha \rangle$是$n$阶循环群, 试证明: 
    \begin{enumerate}[(1)]
        \item $\alpha^m$是$G$的生成元 (即$G = \langle \alpha^m \rangle ) \Leftrightarrow (m, n) = 1$;
        \item 若$\red{\mathbb{Z}/n\mathbb{Z}}$表示模$n$的剩余类环, $U(\red{\mathbb{Z}/n\mathbb{Z}})$是它的单位群, 则
        \[
            \overline{m} \in U(\red{\mathbb{Z}/n\mathbb{Z}}) \Leftrightarrow (m, n)=1;
        \]
        \item 设$\mathrm{Aut}(G)$表示群$G$的自同构群, 则$\mathrm{Aut}(G) \cong U(\red{\mathbb{Z}/n\mathbb{Z}})$.
    \end{enumerate}
\end{exercise}

\begin{proof}
    该题在3.1.6的注记有提及. 已经指出$G$是$n$阶循环群即$G \cong \mathbb{Z}/n\mathbb{Z}$, 因此只需证(2), 而(2)是1.2.9. 因此只证(3).

    对$\sigma \in \mathrm{Aut}(G)$, 注意到$\sigma(\overline{m}) = m\sigma(\overline{1})$, 故可以验证映射
    \[
        \mathrm{Aut}(G) \to U(\mathbb{Z}/n\mathbb{Z}),\quad \sigma \mapsto \sigma(\overline{1})
    \]
    是一个群同构. 事实上只需要验证$(\sigma_1 \circ \sigma_2)(\overline{1}) = \sigma_1(\overline{1})\cdot \sigma_2(\overline{1})$, 这由群同态的定义得到. 而$\sigma \in \mathrm{Aut}(G)$可逆, 因此$\sigma(\overline{1}) \in \mathbb{Z}/n\mathbb{Z}$关于乘法可逆, 必须有$\sigma(\overline{1}) \in U(\mathbb{Z}/n\mathbb{Z})$.
\end{proof}

\begin{exercise}[习题4.3.2]
    设$F$是一个域, $F^* = F \setminus \{0\}$, 证明乘法群$F^*$的任何有限子群都是循环群.
\end{exercise}

\begin{proof}
    任意$F^*$的有限子群$G$, 设$|G| = n$, 那么$\forall \alpha \in G$, 有$\alpha^n = 1$. 因此$G$是$U_n(F)$的一个子群, 而循环群的子群一定是循环群.
\end{proof}

\begin{remark}
    $\mathbb{Z}$的子群一定是$n\mathbb{Z}$, $n$为该子群中最小的自然数. 循环群是$\mathbb{Z}$的一个商群, 因此对同态$\mathbb{Z} \to \mathbb{Z}/n\mathbb{Z}$用4.1.3, 那么$\mathbb{Z}/n\mathbb{Z}$的子群是某个$d\mathbb{Z}$的像, 这里要求$n\mathbb{Z} \subseteq d\mathbb{Z}$, 即$d \mid n$. 从而该子群就是$\langle \overline{d} \rangle$.
\end{remark}

\begin{exercise}[习题4.3.3]
    设$K$是特征零的域, $L$是多项式$x^n - 1 \in K[x]$的分裂域. 试证明: $\mathrm{Gal}(L/K)$同构于$U(\red{\mathbb{Z}/n\mathbb{Z}})$的一个子群. 特别地, $\mathrm{Gal}(L/K)$总是交换群.
\end{exercise}

\begin{proof}
    这是分圆域的一般情形(3.1.6), 设$\theta \in L$是$n$次本原单位根, 即$x^n - 1$的根, $\sigma \in \mathrm{Gal}(L/K)$. 那么$x^n - 1 = (x - 1)(x - \theta) \cdots (x - \theta^{n - 1})$. 对等式两边以$\sigma$作用, 就有$x^n - 1 = (x - 1)(x - \sigma(\theta)) \cdots (x - \sigma(\theta)^{n - 1})$. 因此$\sigma(\theta)$也是本原单位根, 则有同态
    \[
        \mathrm{Gal}(L/K) \to U(\mathbb{Z}/n\mathbb{Z}) = \langle \theta \rangle, \sigma \mapsto \sigma(\theta)
    \]
    因此$\mathrm{Gal}(L/K)$同构于$U(\mathbb{Z}/n\mathbb{Z})$的一个子群, 即它的像. 而交换群的子群自然是交换的.
\end{proof}

\begin{remark}
    若$K$特征零或特征$p$满足$(p, n) = 1$, 则$x^n - 1$是可分多项式, 因此无重根, 此时单位根群$U_n(K) \cong \mathbb{Z}/n\mathbb{Z}$. 这时有
    \[
        \mathrm{Gal}(L/K) =
        \begin{cases}
            U(\mathbb{Z}/n\mathbb{Z}) & \theta \notin K,\\
            \{\mathrm{id}\} & \theta \in K.
        \end{cases} 
    \]
\end{remark}

\begin{exercise}[习题4.5.1]
    设$\mathrm{Aut}(X)$表示集合$X$的自同构群. 试证明: 
    \begin{enumerate}[(1)]
        \item 若$G \times X \to X,\, (g, x) \mapsto g \cdot x$, 是群$G$在$X$上的一个作用, $\forall g \in G$, 定义映射$X \xrightarrow{\rho(g)} X$, $x \mapsto g \cdot x$. 则$\rho(g) \in \mathrm{Aut}(X)$且映射
        \[
            \rho:G \to \mathrm{Aut}(X),\, g \mapsto \rho(g)
        \]
        是群同态.
        \item 若$\rho:G \to \mathrm{Aut}(X)$是一个群同态, 则映射
        \[
            G \times X \to X,\: (g, x) \mapsto \rho(g)(x)
        \]
        是一个群作用.
    \end{enumerate}
\end{exercise}

\begin{proof}
    \begin{enumerate}[(1)]
        \item 由于$G$是群, $g^{-1}$是存在的, 从而这里定义的左乘映射$\rho(g)$自然是一个双射. 只需验证$\rho$是保持乘法的. 对$\forall x \in X$
        \[
            \rho(g_1g_2)(x) = (g_1g_2) \cdot x = g_1 \cdot (g_2 \cdot x) = \rho(g_1)(\rho(g_2)(x)) = (\rho(g_1) \circ \rho(g_2))(x).
        \]
        因此有$\rho(g_1g_2) = \rho(g_1) \circ \rho(g_2)$保持乘法, $\rho$是群同态.
        \item 反过来, 若$\rho$是群同态, 则$\rho(1) = 1$, 即$\rho(1) = \mathrm{id}_X$. 那么对$\forall x \in X$自然有$1 \cdot x = \rho(1)(x) = \mathrm{id}_X(x) = x$. 另一方面,
        \[
            (g_1g_2) \cdot x = \rho(g_1g_2)(x) = (\rho(g_1) \circ \rho(g_2))(x) = \rho(g_1)(\rho(g_2)(x)) = g_1 \cdot (g_2 \cdot x)
        \]
        因此$G \times X \to X,\, (g, x) \mapsto \rho(g)(x)$是一个群作用.
    \end{enumerate}
\end{proof}

\begin{remark}
    这题是群作用的两种表述, 若看成一个群同态$\rho:G \to \mathrm{Aut}(X)$则更贴近表示论的观点. 比如同态
    \[
        \rho:G \to \mathrm{GL}(V),\quad g \mapsto \rho(g)
    \]
    称为群$G$的一个$k$-表示(a $k$-representation, or a representation over field $k$). 其中$V$是一个$k$-线性空间, $\mathrm{GL}(V)$为$V$上所有可逆线性变换构成的群.
    
    一般地, 对于范畴$\mathcal{C}$中的对象$X$, 群$G$在$X$上的作用是群同态
    \[
        \rho:G \to \mathrm{Aut}_{\mathcal{C}}(X)
    \]

    此观点在模的定义中也是类似的, 见5.1.3, 即一个$R$-模实际上是环$R$在Abel群$M$上的一个作用.
\end{remark}

\end{document}