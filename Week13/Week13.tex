\documentclass{../solutions-cn}

\begin{document}
\section*{第十三周作业参考解答及补充}

\subsection*{作业}

\begin{exercise}[习题4.1.1]
    设$G$是一个群, 定义映射$G \xrightarrow\varphi G,\, x \mapsto x^{-1}$. 试证明: $\varphi$是$G$的自同构当且仅当$G$是阿贝尔群.
\end{exercise}

\begin{proof}
    \begin{enumerate}[(1)]
        \item 若$\varphi$是同态, 对$\forall g, h \in G$, 有$g^{-1}h^{-1}gh = (gh)^{-1}gh = 1$, 而$g^{-1}h^{-1} = (hg)^{-1}$(1.3.2), 从而有$gh = hg$.
        \item "$\impliedby$", $G$是Abel群则有$(gh)^{-1} = h^{-1}g^{-1} = g^{-1}h^{-1}$. 因此$\varphi$是同态.
    \end{enumerate}
\end{proof}

\begin{exercise}[习题4.1.2]
    证明: 子群$H \,\red{\subseteq}\, G$是正规子群当且仅当, $\forall g \in G$, $gHg^{-1} \,\red{\subseteq}\, H$.
\end{exercise}

\begin{proof}
    即教材的推论4.1.6.

    由定义, $H \lhd G \iff \forall g \in G, gHg^{-1} = gg^{-1}H = H$. 那么我们实际上要证明的是$\forall g \in G, gHg^{-1} \subseteq H \implies gHg^{-1} = H$. 即只需说明此时也有反向的包含. 对$\forall h \in H, g \in G$, 考虑$g^{-1} \in G$, 则有$g^{-1}Hg \subseteq H$. 则$g^{-1}hg \in H$, 即$g^{-1}hg = h' \implies h = gh'g^{-1} \in gHg^{-1}$. 因此$H \subseteq gHg^{-1}$.
\end{proof}

\begin{exercise}[习题4.1.3]
    设$G \xrightarrow\varphi G'$是群同态, $K = \ker(\varphi)$是同态$\varphi$的核. 试证明: \begin{enumerate}[(1)]
        \item 对于任意子群$H' \,\red{\subseteq}\, G', H = \varphi^{-1}(H') \,\red{\subseteq}\, G$是子群, 且包含$K$.
        \item 当$\varphi$是满射时, $H' \mapsto \varphi^{-1}(H')$建立了集合
        \[
            \Gamma' = \{H' \,\red{\subseteq}\, G' \mid H' \text{ 是子群}\},
        \]
        与集合$\Gamma = \{H \,\red{\subseteq}\, G \mid H \text{ 是 }G \text{ 的子群, 且 } H \,\red{\supseteq}\, K\}$之间的双射, 此时$H' \,\red{\subseteq}\, G'$是正规子群当且仅当$\varphi^{-1}(H') \,\red{\subseteq}\, G$是正规子群.
    \end{enumerate}
\end{exercise}

\begin{proof}
    类似定理2.1.2.

    \begin{enumerate}[(1)]
        \item 注意到$1 \in H'$而按定义$K = \ker(\varphi) = \varphi^{-1}(1)$. 故有$K \subseteq \varphi^{-1}(H')$. 验证是一个子群用1.3.12后提到的命题即可, $\forall a, b \in \varphi^{-1}(H'),$
        \[
            \varphi(a), \varphi(b) \in H' \implies \varphi(ab^{-1}) = \varphi(a)\varphi(b)^{-1} \in H' \implies ab^{-1} \in \varphi^{-1}(H')
        \]
        而反过来, 若$H \subseteq G$是子群, $\varphi(H) \subseteq G'$也是子群.

        \begin{remark}
            需要注意对于环同态$R \xrightarrow{\varphi} R'$来说, $\varphi^{-1}(J)$是理想, 但$\varphi(I)$只有当$\varphi$是满射时才是理想, 其中$I \subseteq R, J \subseteq R'$是理想.
        \end{remark}
        \item $\varphi$是满射时, 由同态基本定理, $G' \cong G/K$, 我们把$\varphi$看成商映射$g \mapsto gK$. 由(1), $H' \in \Gamma'$, $\varphi^{-1}(H') \in \Gamma$, 且$\varphi^{-1}(H') = \{g \in G \mid gK \in H'\}$; $H \in \Gamma$, $\varphi(H) \in \Gamma'$, 且$\varphi(H) = \{hK \mid h \in H\} = H/K$(把$\varphi$限制在$H$上, 而且$\ker(\varphi) = K \subseteq H$, 因此按定义$\ker(\varphi|_H)$仍是$K$, 因此$K \lhd H$, 商群$H/K$在这里是合理的). 我们只需说明$H \mapsto \varphi(H)$和$H' \mapsto \varphi^{-1}(H')$互逆, 按定义这是很简单的:
        \[
        \begin{aligned}
            \varphi^{-1}(\varphi(H)) &= \varphi^{-1}(H/K) = \{g \in G \mid gK \in H/K\} = H,\\
            \varphi(\varphi^{-1}(H')) &= \varphi(\{g \in G \mid gK \in H'\}) = \{hK \mid h \in \{g \in G \mid gK \in H'\}\} = H'
        \end{aligned}
        \]
        剩下的部分使用注记中的定理来说明. 根据上面的双射, 我们设$H' = \varphi(H) = H/K$, $H \in \Gamma$.
        \begin{enumerate}
            \item["$\implies$"] 当$H/K \lhd G' = G/K$时, 我们考虑商映射$\pi':G/K \to \frac{G/K}{H/K}$和$\varphi$的复合$f$, 即
            \[
                \begin{tikzcd}
                    G \arrow[r, "\varphi"', two heads] \arrow[rr, "f", bend left] & G/K \arrow[r, "\pi'"', two heads] & \frac{G/K}{H/K}
                \end{tikzcd}
            \]
            注意到$\ker(f) = \{g \in G \mid gK \in H/K\} = H$, 从而$H \lhd G$.
            \item["$\impliedby$"] 反过来若$H \lhd G$, 则考虑商映射$\pi:G \to G/H$. 注意到$K \subseteq \ker(\pi) = H$, 由注记中的商群的泛性质, 存在唯一的同态
            \[
                \overline{\pi}:G/K \to G/H
            \]
            且$\ker(\overline{\pi}) = H/K = H'$, 则有$H' \lhd G/K = G'$.
        \end{enumerate}
    \end{enumerate}
\end{proof}

\begin{remark}
    和2.1.8一样, 群同态基本定理也要推广为商群的泛性质:
    \begin{thmstar}
        设$G$是群, $H \lhd G$, 对任意的同态$G \stackrel{f}{\to} G'$, 若$H \subseteq \ker(f)$, 则存在唯一的同态$\overline{f}:G/H \to G'$使得图表交换:
        \[
            \begin{tikzcd}
                G \arrow[rr, "f"] \arrow[rd, "\pi"', two heads] &                                 & G' \\
                                                                & G/H \arrow[ru, "\exists!\overline{f}"'] &   
                \end{tikzcd}
        \]
    \end{thmstar}
    同样的有$\ker(\overline{f}) = \ker(f)/H$, $H = \ker(f)$时就是同态基本定理.
\end{remark}

\begin{exercise}[习题4.1.4]
    设$H, N$都是$G$的正规子群, 并且$N \subseteq H$. 令$\overline{H} = H/N, \overline{G} = G/N$.
    \begin{enumerate}[(1)]
        \item 证明$\overline{H}$是$\overline{G}$的正规子群.
        \item 证明$G/H \cong \overline{G}/\overline{H}$.
    \end{enumerate}
\end{exercise}

\begin{proof}
    这题实际上是上题的推论, 此处我们考虑的同态是商同态$\pi:G \to G/N$.
    \begin{enumerate}[(1)]
        \item 根据4.1.3, $\overline{H} = \pi(H)$, $\overline{G} = \pi(G)$. 由于$H \lhd G$, 所以有$\overline{H} = \pi(H) \lhd G/N = \overline{G}$.
        \item 这里的同构实际上是在4.1.3"$\impliedby$"的部分最后再用一下同态基本定理. 记商同态$f:G \to G/H$, 则有$N \subseteq H = \ker(f)$, 由商群的泛性质, 存在唯一的同态$\overline{f}:G/N \to G/H$, 且$\ker(\overline{f}) = H/N$, 因此有同构$\overline{G}/\overline{H} = \frac{G/N}{H/N} \cong G/H$.
    \end{enumerate}
\end{proof}

\begin{exercise}[习题4.1.5]
    设$H \,\red{\subseteq}\, G$是$G$的子群, $K \lhd G$, 试证明: 
    \begin{enumerate}[(1)]
        \item $H \cdot K = \{hk \mid \forall h \in H, k \in K\}$是$G$中包含$H$和$K$的子群;
        \item $H$在商同态$G \to G/K$, ($g \mapsto \overline{g}$)下的像是$(H \cdot K)/K$;
        \item $\varphi:H \to (HK)/K$, ($\varphi(h) = \overline{h}$)的核是$H \cap K$;
        \item $\varphi$诱导群同构$H/(H \cap K) \cong (HK)/K$.
    \end{enumerate}
\end{exercise}

\begin{proof}
    \begin{enumerate}[(1)]
        \item 考虑商同态$\pi:G \to G/K$. 注意这里前两题不一样, $H$和$K$不一定有包含关系, 
        \[
            \pi(H) = \{hK \mid h \in H\} \implies \pi^{-1}(\pi(H)) = \bigcup_{h \in H} hK = HK.
        \]
        因此$HK$是$G$的子群(4.1.3的(1)).
        \item 见(1).
        \item 由(1), 把$\pi$限制在$H$上, 就得到
        \[
            \varphi:H \to (HK)/K
        \]
        因此
        \[
            \ker(\varphi) = \{\overline{h} = hK = \overline{1} = K \mid h \in H\} = \{h \in K \mid h \in H\} = H \cap K
        \]
        \item 对(3)中的$\varphi$用同态基本定理.
    \end{enumerate}
\end{proof}

\end{document}